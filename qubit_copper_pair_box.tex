\section{Single cooper pair box system\label{sec:cooper_pair_box}}
\begin{figure}[h]
  \centering \includegraphics[height=5cm]{energy_ratios_for_qubit}
  \caption{\small General overview of energy parameters\label{fig:energy_ratios_for_qubit}}
\end{figure}

\begin{enumerate}
\item Starting with \autoref{fig:cooper_pair_box_1_no_gate} and remembering Faraday's  law ($V = -\dot{\Phi}$), we read the
  \textbf{charging kinetic part}, that comes as a result of the time-varying flux on the cooper pair island.

  \begin{equation}
    T = \frac{C_g}{2}\dot{\Phi_J}^2 + \frac{C_J}{2}\dot{\Phi_J}^2 = \frac{C_\Sigma}{2}\dot{\Phi_J}^2.
  \end{equation}

  \begin{figure}[h]
    \centering \inkfig{6cm}{cooper_pair_box_1_no_gate}
    \caption{\small  The  Cooper  pairs   are  trapped  on  an  island  between  a  gated   capacitor  and  a  Josephson
      junction.\label{fig:cooper_pair_box_1_no_gate}}
  \end{figure}

\item \textbf{The potential  energy part} comes from the  JJs and the energy  of the gate voltage acting  on the induced
  charge on the capacitor.

  \begin{equation}
    \left\{
      \begin{aligned}
        & E_J = 1 - E_J\cos(\frac{2\pi}{\Phi_0}\Phi_J)\\
        & \begin{aligned} E_\text{gate} & = V_g \times Q_g\\b Q_g & = C_g\times -\dot{\Phi_J}b
        \end{aligned}
      \end{aligned}\right.  \rightarrow U =
    -E_J\cos(\frac{2\pi}{\Phi_0}\Phi_J)
    -
    V_gC_g\dot{\Phi_J}.
  \end{equation}

  \begin{figure}[h]
    \centering \inkfig{8cm}{cooper_pair_box_2_with_gate}
    \caption{\small Addition of bias voltage\label{fig:cooper_pair_box_2_with_gate}}
  \end{figure}

\item The full \textbf{Lagrangian} now reads
  \begin{equation}
    \mathcal{L} = T - U = \frac{C_\Sigma}{2}\dot{\Phi}_J^2 + E_J\cos\left( \frac{2\pi}{\Phi_0}\Phi_J \right) + V_gC_g\dot{\Phi_J}.
  \end{equation}

\item The \textbf{conjugate momentum}, which gives useful set of generalised coordinates to work with,

  \begin{equation}
    Q_J = \frac{d\mathcal{L}}{d\dot{\Phi}_J} = \slate{C_\Sigma\dot{\Phi}_J} + \gold{V_gC_g},
  \end{equation}
  \noindent turns out  to be the \slate{\textbf{induced  charge on the capacitor due  to the JJ voltage}}  offset by the
  \gold{\textbf{charge induced by the gate voltage $V_g$}}.

\item So we arrive at the following set of variables

  \begin{equation}
    { \textcolor{blue}{\mathbf{x\leftrightarrow \Phi \leftrightarrow \phi} \text{ (position/flux) }}\qquad \textcolor{red}{\mathbf{p\leftrightarrow Q \leftrightarrow N \text{ (momentum/electrons) }}}}
  \end{equation}

  \noindent with the commutation relations:

  \begin{align}
    \left[\blue{x},\red{p}\right] & =i\hbar & \left[\blue{\Phi},\red{Q}\right] & = i\hbar & \left[\phi,N\right] & = \frac{2\pi}{\frac{h}{2e}}\left[\Phi,Q\right]\frac{1}{2e} = i\\
    \red{\hat{p}} & = -i\hbar\ipartial{}{\blue{x}} & \hat{\red{Q}} & =-i\hbar\ipartial{}{\blue{\Phi}} & \hat{\red{N}} & =-i\ipartial{}{\blue{\phi}}
  \end{align}

\item\

\begin{framed}\noindent
  The Hamiltonian is then:

  \begin{equation}\label{eqn:cpbox_final}
    \begin{aligned}
      \mathcal{H} & = Q_J\dot{\Phi}_J - \mathcal{L}\\
      & = \frac{(Q_J - C_gV_g)^2}{2C_\Sigma} - E_J\cos(\frac{2\pi}{\Phi_0}\Phi_J)\\
      &    =    \mathbf{\red{4E_C{\left(\hat{N}-N_\text{ext}\right)^2}-     E_J\cos\left(\phi\right)}}\qquad    E_c    =
      \frac{e^2}{2C_\Sigma}
    \end{aligned}
  \end{equation}

  \noindent We will call $ N_\text{ext} = \frac{C_g V_g}{2e} $ the effective offset charge and $C_{\Sigma} = C_g+C_J$.
\end{framed}
\end{enumerate}

\subsection{Adding a parallel JJ\label{subsec:cpb_2}}
\begin{framed}\noindent
  A parallel JJ will allow one to tune the energy of the system by varying the biasing magnetic field.
\end{framed}

\begin{figure}[h]
  \centering \inkfig{8cm}{cooper_pair_box_3_tuneable}
  \caption{\small Adding a sencond JJ will allow one to control  the effective energy $E_{J}$ and thus the energy of the
    system.\label{fig:cooper_pair_box_3_tuneable}}
\end{figure}

\begin{itemize}
\item    The   two    currents   through    the   JJ    will    give   a    total   current    \begin{equation}   I    =
    I_{c1}\sin(\phi_1)+I_{c2}\sin(\phi_2),
  \end{equation}

\item  The phase  quantisation ($  \phi_1-\phi_2 =  \phi_\text{ext} +  2\pi m,  m \in  \mathbb{Z} $)  and symmetric  JJs
  ($I_{c1}=I_{c2}=I_{c0}$) results in

  \begin{equation}
    \begin{aligned}
      I & = 2I_{c0}\sin(\frac{\phi_1+\phi_2}{2})\cos(\frac{\phi_1-\phi_2}{2})) \\
      & = 2I_{c0}\cos(2\pi\frac{\Phi_{\text{ext}}}{2\Phi_0})\sin(\frac{\phi_1+\phi_2}{2})
    \end{aligned}
  \end{equation}

\begin{framed}\noindent
  In effect, the two JJ acts as a single JJ with current
  \begin{equation}
    I = \slate{I_{c}}\sin(\gold{\phi})
  \end{equation}
  \noindent where the critical current, $I_c$ is a magnetic-field-controlled parameter.
  \begin{equation}
    \slate{I_{c}(\Phi_\text{ext}) = 2I_{c0}|\cos(\pi\Phi_\text{ext}/\Phi_0)|} \qquad \gold{\phi = (\phi_1+\phi_2)/2}
  \end{equation}

\end{framed}

\item The potential energy $U(\Phi_{\text{ext}})$ is now a field-controlled parameter:

  \begin{equation}
    \left\{\begin{aligned}
        I &= \slate{I_c(\Phi_\text{ext})}\sin(\phi)\\
        V&=\dot{\Phi} \equiv \frac{\dot{\phi}}{2\pi}\Phi_0
      \end{aligned}\right.  \Rightarrow U =
    \left\{
      \begin{aligned}
        \int_{0}^{t}IV dt & = \int_{0}^{t}\slate{I_c(\Phi_\text{ext})}\sin(\phi)\frac{\Phi_0}{2\pi}\frac{d\phi}{dt}dt\\
        & = \int_{0}^{\phi} E_J\sin(\phi)d\phi\\
        &      =      \red{E_J}(1-\cos(\phi))      \qquad       \red{E_J      =      \ballblue{\frac{\Phi_0I_{c0}}{2\pi}}      \times
          2|\cos(\pi\Phi_\text{ext}/\Phi_0)|}.
      \end{aligned}\right.
  \end{equation}
  \begin{framed}\noindent
    Thus the Josephson energy from the JJ elements acquires an additional factor which can now be controlled:
    \begin{equation}
      E_J = \ballblue{E_{J0}} \times 2|\cos(\pi\Phi_\text{ext}/\Phi_0)|
    \end{equation}
  \end{framed}
\end{itemize}

\subsection{Transmon}
\label{sec:transmon}

A transmon extends  the cooper pair box,  by boosting the ratio  $\frac{E_{J}}{E_{C}}$.  This is done  by decreasing the
charging energy $E_c= \frac{e^2}{2C_\Sigma}$ by introducing a large shunt capacitance in parallel to the JJs.

\begin{framed}\noindent
  \begin{equation}
    C_{\Sigma} =  \gold{\mathbf{C_\text{transmon}}} + C_g  + C_J.
  \end{equation}
\end{framed}

\begin{figure}[h]
  \centering \inkfig{7cm}{cooper_pair_box_4_transmon}
  \caption{\small Transmon suppresses $E_{C}$ with a large shunt capacitance. \label{fig:cooper_pair_box_4_transmon}}
\end{figure}

\newpage
\begin{framed}\noindent
  \textbf{Summarising up to now}:
  \begin{equation}\label{eq:cooper-pair-box-summary}
    \begin{aligned}
      \mathcal{H} & = 4E_C{\left(\hat{N}-N_\text{ext}\right)^2} -
      E_J(\Phi_{\text{ext}})\cos\left(\phi\right)\\
      E_c& = \frac{e^2}{2C_\Sigma}\\
      N_\text{ext} & = \frac{C_g V_g}{2e} && \text{Induced charge from the gate}\\
      C_{\Sigma} & = \blue{C_\text{transmon} + C_g}+ \green{C_J}\\
      E_J(\Phi_{\text{ext}}) & = \ballblue{E_{J0}} \times 2|\cos(\pi\Phi_\text{ext}/\Phi_0)| && \text{controlled by the applied external flux}.\\
    \end{aligned}
  \end{equation}

  \subsubsection{Capacitance through parallel structures}
  \label{sec:capac-thro-parall}
  As stated in Sec.~\ref{sec:coupling-capacitance}, each \iunit{10}{$\mu$m} of parallel structures adds on \iunit{1}{fF}
  of capacitance ($10^{-10}\,$F/m).
  \begin{equation}
    \blue{C_{g} \text{ or } C_{\text{transmon}} = \iunit{1}{fF} \times \text{per 10$\mu$m of interface}}.
  \end{equation}

\begin{center}
  \inkfig{8cm}{capacitances_through_parallel_structures}
\end{center}

\subsubsection{Capacitances from JJ}
\label{sec:capacitances-from-jj}
\begin{itemize}
\item The JJ (see below) act like capacitors due to the Al0$_x$ oxide layer between them:
  \begin{equation}
    \green{C_J = \frac{\varepsilon\varepsilon_0A_{JJ}}{d}}
  \end{equation}
  \noindent where $A_{JJ}$ is the contact area of the JJ, \iunit{d=2}{nm}, $\varepsilon = 10$.

  \red{\textbf{This contribution will be relatively small for the Transmon.}}
\end{itemize}

\subsection{The critical current and $E_{J0}$}
\label{sec:crit-curr-josephs}

  \begin{equation}
    I_cR_n      =
    \frac{\pi\Delta(T)}{2e}\tanh\big(\frac{\Delta(T)}{2k_bT}\big)
  \end{equation}

  \noindent   can  derived   from   BCS  theory   (see   Eq.\eqref{introducing-qed-operators-critical-current})  for   a
  superconducting  energy  gap  of  $  \Delta(T)  $  and  normal  resistance   $  R_n  $  of  the  JJ,  which  in  the  limit
  $T \rightarrow 0$ read $ I_cR_n = \frac{\pi\Delta(0)}{2e}$ so that
  \begin{equation}
    \ballblue{E_{J0} = \Phi_0I_c\frac{1}{2\pi} = \frac{h}{2e}\frac{\pi\Delta(0)}{2eR_n}\frac{1}{2\pi} \equiv \frac{R_q}{R_{\square}/N_{sq}}\frac{\Delta(0)}{2}}
  \end{equation}

  \noindent where

  \begin{itemize}
  \item $R_{q} = \frac{h}{(2e)^{2}}$;
  \item  $R_n   =  R_{\square}/  N_{\text{sq}}$  is   the  resistance  of  the   JJ  oxide  layer.   The   more  \iunit{100  \times
      100}{nm$^{2}$} squares ($N_{\text{sq}}$), the lower the resistance.

     \begin{equation}
       R_{\square} = 3.57\,\text{k}\Omega (1.84\,\text{k}\Omega before) \text{ for } 100 \times 100\,\text{nm}^2.
     \end{equation}

     \red{\textbf{At room temperature, the resistances are 10\% lower than at cryogenic temperatures.}}

    \begin{center}
      \inkfig{9cm}{jj_oxide_square}%
      \includegraphics[height=4cm]{oxidation}
    \end{center}

  \item BCS theory says that at zero temperature, there is a universal value
    \begin{equation}
      \Delta(0) = 1.73 \times  k_{b} \times T_{c},
    \end{equation}

    \noindent that depends on the critical temperature of the metal.  For aluminium it is 1.3\,K.
  \end{itemize}
\end{framed}

\subsection{Quantising the Hamiltonian}
\subsubsection{\red{Charge energy dominates}, $ E_C/E_J >> 1$}
If charge is  the important variable, then we chall  work with the charge basis $  \lbrace N \rbrace $.  We use  the number of phase
operators derived in \autoref{sec:charge_basis}.

\begin{framed}\noindent

  \begin{equation}
    \begin{aligned}
      e^{\pm i\hat{\blue{\phi}}} & = \sum_{n}\iketbra{n\pm 1}{n}\\
      \hat{\red{N}} & = \sum_{n}n\iketbra{n}{n}.
    \end{aligned}
  \end{equation}
\end{framed}

\noindent and exponentiating \verb|cos|, Eq.\eqref{eq:cooper-pair-box-summary} becomes

\begin{equation}
  \begin{aligned}
    \mathcal{H} & = 4E_C{\left(\hat{\red{N}}-N_\text{ext}\right)^2}- E_J\cos\left(\hat{\blue{\phi}}\right)\\
    & = 4E_C{\left(\hat{\red{N}}-N_\text{ext}\right)^2}- E_J \frac{1}{2}\left( e^{+i\hat{\blue{\phi}}} + e^{-i\hat{\blue{\phi}}} \right)\\
    &                          =                          \sum_n\bigg[4E_C{\left(n-N_\text{ext}\right)^2}\iketbra{n}{n}-
    \frac{E_J}{2}\bigg(\iketbra{n+1}{n}+\iketbra{n-1}{n}\bigg)\bigg]
  \end{aligned}.
  \label{l2-subbed2}
\end{equation}

\noindent which takes on the matrix form:
\begin{equation}\label{eq:cooper-pair-box-matrix-representation}
  \begin{pmatrix}
    4E_C(-2-N_\text{ext})^2 & -E_J/2 & 0 & 0 & 0\\
    -E_J/2 & \red{4E_C(-1-N_\text{ext})^2} & \red{-E_J/2} & 0 & 0\\
    0 & \red{-E_J/2} & \red{4E_C(N_\text{ext})^2} & -E_J/2 & 0\\
    0 & 0 & -E_J/2 & 4E_C(1-N_\text{ext})^2 & -E_J/2\\
    0 & 0& 0 & -E_J/2 & 4E_C(2-N_\text{ext})^2\\
  \end{pmatrix}
\end{equation}

\begin{figure}[h]
  \centering \includegraphics[height=7.5cm]{cooper-pair-box-as-a-function-of-next}
  \caption{\small     Plot      of     energies     and      ground-eigenstate     contributions     -      taken     at
    $\Phi_{\text{ext}}=0$\label{fig:cp_box_energy_charge}}
\end{figure}

\noindent In  \autoref{fig:cp_box_energy_charge} we  demonstrate the  energy spectrum and  contributions of  the various
charge states (\iket{i}) to the ground state $ \Psi_g$:

\begin{equation}
  \iket{\Psi_{g}} = \sum_i^{\text{charge states}}\alpha_i \iket{i}.
\end{equation}

\begin{framed}\noindent
  The    stronger   the    coupling   ($E_{J0}$)    the    greater   the    splitting   at    the   degeneracy    points
  ($N_{\text{ext}} = \frac{2n + 1}{2}, n \in \mathbb{Z}$).
\end{framed}

\noindent Furthermore, we can run the simulation at a fixed gate charge ($N_{\text{ext}} = 0$) but varying external flux
($\Phi_{\text{ext}}$)     (as    done     in    an     experiment    in     which    we     sweep    the     field)    -
\autoref{fig:cooper-pair-box-as-a-function-of-phi-ext}.

\begin{figure}[h]
  \centering \includegraphics[height=7cm]{cooper-pair-box-as-a-function-of-phi-ext}
  \caption{\small      Fixed     charge      but     varied      flux     (what      we     would      typically     see
    experimentally).\label{fig:cooper-pair-box-as-a-function-of-phi-ext}}
\end{figure}

\subsection{Deriving energy from geometry}
\label{sec:back-transmon}

In \autoref{fig:cp_box_energy_charge}  and \autoref{fig:cooper-pair-box-as-a-function-of-phi-ext}  we hard-coded  in the
charging    energy    ($E_C=\iunit{70}{GHz}$)    and    Josephson   energy    ($E_{J0}=\iunit{10}{GHz}$),    but    from
\autoref{eq:cooper-pair-box-summary}  these  parameters  are  determined  by  the geometry  of  the  qubit  depicted  in
\autoref{fig:cooper_pair_box_5_geometry}.

\begin{figure}[h]
  \centering \inkfig{12cm}{cooper_pair_box_5_geometry}
  \caption{\small Energy defined by geometry of transmon\label{fig:cooper_pair_box_5_geometry}}
\end{figure}

Reading off the diagram
\begin{equation}
  \left\{
    \begin{aligned}
      E_c& = \frac{e^2}{2C_\Sigma}\\
      C_{\Sigma} & = \blue{C_\text{transmon} + C_g}+ \green{C_J}\\
      \blue{C_{g}} & = L_{g} \times 10^{-10}\\
      \blue{C_{\text{transmon}}} & = 4\times\left( L_{t} - 2S_{\text{transmon}} \right) \times 10^{-10}\\
      \green{C_J} & = \frac{\varepsilon\varepsilon_0N_{sq} \times A_{100\times100nm^2}}{d} \quad
      \text{where \iunit{d=2}{nm}, $\varepsilon = 10$}\\
      \red{E_{J0}} & = \frac{R_q}{R_{\square}/N_{sq}}\frac{\Delta(0)}{2}\\
      R_{\square} & = 3.57\,\text{k}\Omega (1.84\,\text{k}\Omega before) \text{ for }100 \times 100\,\text{nm}^2 \qquad (1.5\,\text{k}\Omega \text{ at room temperature})\\
      \Delta(0) & = 1.73 \times k_{b} \times T_{c} \quad \text{For aluminium $T_{c}$ is \iunit{1.3}{K}}.
    \end{aligned}\right.
\end{equation}

\Autoref{fig:transmon-with-parameters-from-geometry} shows the  energy and transition spectrum  for typical experimental
geometry values.

\begin{figure}[h]
  \centering \includegraphics[height=7.5cm]{transmon-simulations/transmon-with-parameters-from-geometry}
  \caption{\small  Parameters used:  $L_{g}  = 15\,\mu  m$,  $L_{t} =  150\mu m$,  $2S_{\text{transmon}}  = 10\,\mu  m$,
    $N_{sq}=2$, $d = \iunit{2}{nm}$, $\varepsilon = 10$. \label{fig:transmon-with-parameters-from-geometry}}
\end{figure}

\subsubsection{Effect of $N_{\text{ext}}$ is negligible in transmon}
\label{sec:effect-n_textext}

The ratio  $E_{C}/E_{J0}<<0$ due to the  shunting capacitance that suppresses  the charging energy, meaning  that charge
will not  strongly affect  the energy  spectrum.  In \autoref{fig:cp_box_energy_charge}  we can  see much  flatter (with
respect to $N_{ext}$) energies when $E_{J0}$ is much great than $E_C$.

This can be confirmed, by varying N$_{ext}$ in \autoref{fig:transmon-sweep-N_ext}.

\begin{figure}[h]
  \centering \includegraphics[height=6cm]{transmon-simulations/transmon-sweep-N_ext}
  \caption{\small Sweeping  the gate-induced charge, $N_{\text{ext}}  = \frac{V_{g}C_{g}}{2e}$ has little  effect on the
    energy    of     the    system,    since    any     variations    are    suppressed    by     the    big    shunting
    capacitance. \label{fig:transmon-sweep-N_ext}}
\end{figure}

\subsubsection{Using sufficient number of charge states for simulation}
\label{sec:note-simulation}

It   is  always   best  to   utilise  as   many  charge   states  for   a  simulation   -  convergence   occurs  as   in
\autoref{fig:transmon-sweep-number-of-charge-states}.

\begin{figure}[h]
  \centering \includegraphics[height=13cm]{transmon-simulations/transmon-sweep-number-of-charge-states}
  \caption{\small Including more charge states  in the simulation results in a more accurate  simulation - at some point
    the      energies      stop      changing,      at       which      point      the      simulation      is      good
    enough.\label{fig:transmon-sweep-number-of-charge-states}}
\end{figure}

\subsubsection{Maintaining anharmonicity}
\label{sec:maint-anharm}

The   robustness   to   charge   noise,   that   is    always   present   in   electrical   systems,   demonstrated   in
\autoref{sec:effect-n_textext} is countered by the vanishing assymetry between the transitions

\begin{equation}\label{eq:transmon-assymetry}
  \alpha = \frac{E_{21} - E_{10}}{E_{10}}.
\end{equation}

\noindent   Increasing   $E_J/E_C$   will   lead   to  the   domination   of   $\cos\left(\hat{\blue{\phi}}\right)$   in
$E_C{\left(\hat{\red{N}}-N_\text{ext}\right)^2}- E_J\cos\left(\hat{\blue{\phi}}\right)$.  The state of the system can be
viewed as  a particle  in a periodic  potential, becoming  exceedingly localised in  the potential  minima.  \red{Insert
  derivation from Bader}.

\noindent Just compare the anharmonicity for different values of $E_C$ in \autoref{fig:transmon-anharmonicity}.

\begin{figure}[h]
  \centering \includegraphics[height=9cm]{transmon-simulations/transmon-anharmonicity}
  \caption{\small Anharmonicity vanishes when $E_{C}/E_{J0} << 1$\label{fig:transmon-anharmonicity}}
\end{figure}

\subsection{Charge dispersion}
\label{sec:charge-dispersion}

The transmon must be robust  against charge variations (maximise $E_J/E_C$) so that charge  variations in the circuit do
not affect the energy levels.   \red{Bader derivation} goes through how deep decoupled states in the  well it is best to
assume that the energy levels  of individual wells are known, but coupling between  these ``wells'' perturbs the initial
energies.

\begin{figure}[h]
  \centering \inkfig{8cm}{wavefunction_transmon}
  \caption{\small View system as a  particle in a periodic potential.  It has very little Kinetic  energy, so it will be
    localised in the minima.\label{fig:wavefunction_transmon}}
\end{figure}

The wavefunction is suppressed by  the potential barrier.  The higher the relative size  of the barrier (high $E_J/E_C$)
the higher the decoupling and the less

\subsection{Summary for transmon}
\label{sec:summary-transmon}

\begin{figure}[h]
  \centering
  \includegraphics[height=12cm]{2020-09-05_(cooper-pair-box-and-transmon)/transmon-anharmonicity-and-charge-dispersion}
  \caption{\small We want to  maximise the anharmonicity $\alpha$ (have it in the  blue region) and minimise sensitivity
    to charge ($\varepsilon_m = \frac{E_m(N_{ext}=0.5) -  E_m(N_{ext}=0)}{E_{10}}$) (green region).  This is achieved by
    having $5 \le  E_{J0}/E_C \le 70$ We need  to use many charge states in  order to get a good  approximation (50 used
    here, and it is still only an approximation)\label{fig:transmon-anharmonicity-and-charge-dispersion}}
\end{figure}

The addition  of a shunt capacitance,  $C_{\text{transmon}}$, reduces $E_{C}$  - energy associated with  the capacitance
circuit     elements,     which     has     two     counter-opposing    effects     that     are     demonstrated     in
\autoref{fig:transmon-anharmonicity-and-charge-dispersion}
\begin{itemize}
\item   Robustness  of   the   system  against   charge   noise,  which   occurs   for  $\mathbf{E_{J0}/E_C>5}$   \hfill
  \autoref{sec:effect-n_textext};
\item Vanishing  of the anharmonicity  required for addressing  individual transitions -  \red{\textbf{the anharmonicity
      needs to be at least larger than the spectrum width of the 0-1 transition.  This will typically be 200MHz, meaning
      that  for  a  spacing  of  \iunit{5-20}{GHz}  one  will  need  $\alpha  >  0.04$}}  \cite{Kjaergaard_2020}  \hfill
  \autoref{sec:maint-anharm}.

  Now as per \autoref{tab:conversion2}, increasing $C_{transmon}$ is equivalent to a particle in the potential acquiring
  more mass - it becomes localised in one of the potential minima, in which case we approximate cosine with a parabollic
  potential and according to \cite{Koch_2007} find that

\begin{equation}\label{eq:transmon-anharmonicity}
  \alpha = \frac{-E_{C}}{\omega_{10}} = -\frac{1}{\sqrt{8E_{J0}/E_C} - 1} \quad \Rightarrow \quad \frac{E_{J0}}{E_C} = \frac{1}{8}\left( 1 - \frac{1}{\alpha} \right)^2,
\end{equation}

\noindent so for $\alpha>0.04$ one would need $E_{J0}/E_C < 70$:
\end{itemize}

\begin{framed}\noindent
  \noindent We need to find a compromise that fulfills the following criteria:

  \begin{itemize}
  \item Maintain anharmonicity $\iabs{\alpha} \ge 0.04 $ \hfill \red{$E_{J0}/E_C < 70$};
  \item Reduce the charge dispersion $\varepsilon_m < 0.0001$ \hfill \red{$E_{J0}/E_C > 5$};
  \item  We need  the transition  energy to  be within  the window  that our  laboratory equipment  can register  \hfill
    \red{\iunit{5-20}{GHz}};
  \end{itemize}
\end{framed}

\subsubsection{Choosing ratio}
Let   us   choose   $E_C$   and   $E_{J0}$   to    fulfil   the   above   criteria.    Hard-coding   some   values,   in
\autoref{fig:transmon-optimal-parameters}, we get a selection of $E_C$ and  $E_{J0}$ energies to use, which we will keep
a log of in \autoref{tab:transmon-fabrication-parameters}.  In order to  use consistent cross sizes (simpler to draw) we
will choose the $E_C=1GHz$ for a transmon length of $L_{\text{transmon}}$

\begin{figure}[h]
  \centering
  \includegraphics[height=4cm]{2020-09-05_(cooper-pair-box-and-transmon)/selecting-EJ0-to-fall-in-range_EC=5GHz}%
  \includegraphics[height=4cm]{2020-09-05_(cooper-pair-box-and-transmon)/selecting-EJ0-to-fall-in-range_EC=2GHz}%
  \includegraphics[height=4cm]{2020-09-05_(cooper-pair-box-and-transmon)/selecting-EJ0-to-fall-in-range_EC=1GHz}%
  \caption{\small Optimal parameters for transmon.  Left to right  we increase $E_C$ and within each plot display values
    for varying $E_{J0}$, Red region shows the window of devies.   Solid lines show $\alpha \ge 0.04$.  We will choose the
    bigger    transmon    (suppressing    $E_C$)    as    it    allows    for    a    broader    range    of    $E_{J0}$
    values.\label{fig:transmon-optimal-parameters}}
\end{figure}

\begin{framed}\noindent
  $\blue{E_C(L_{\text{transmon}})}$   and   $\red{E_{J0}(N_{sq})}$  where   the   calibration   curves  are   given   in
  \autoref{fig:EJ0-EC-selection} - this completes \autoref{tab:transmon-fabrication-parameters}.
  \begin{equation}
    \left\{
      \begin{aligned}
        L_\text{gate} & = 15\mum\\
        C_{\text{gate}} & = L_\text{gate} \times 10^{-10} = \iunit{1.5}{fF} - \text{negligible}\\
        2S_{\text{transmon}} & = 10\mum \\
        \blue{C_{\text{transmon}}} & = 4\times\left( L_{\text{transmon}} - 2S_{\text{transmon}} \right) \times 10^{-10} \\
        A_{JJ} &  = N_{sq} \times 100\times100nm^{2} \\
        C_J & = \frac{\varepsilon\varepsilon_0A_{JJ}}{d} \quad \text{
          \iunit{d=2}{nm}, $\varepsilon = 10$}\\
        C_{\Sigma} & = \blue{C_\text{transmon}} + C_g + C_J \approx \blue{C_\text{transmon}}\\
        E_c& = \frac{e^2}{2C_\Sigma}\\
        \red{E_{J0}} & = \frac{R_q}{R_{\square}/\red{N_{sq}}}\frac{\Delta(0)}{2} \quad R_q=6.43\kOhm \quad R_{\square} = 3.57\,\text{k}\Omega (1.84\,\text{k}\Omega before) \text{ for } 100 \times 100\,\text{nm}^2\\
      \end{aligned}\right.
  \end{equation}
\end{framed}

\begin{figure}[h]
  \centering%
  \includegraphics[height=6cm]{2020-09-05_(cooper-pair-box-and-transmon)/EJ0-selection}%
  \includegraphics[height=6cm]{2020-09-05_(cooper-pair-box-and-transmon)/EC-selection}
  \caption{\small More JJ  squares mean smaller resistance  and higher critical current  $\Rightarrow E_{J0}$ increases.
    Large    transmon    size    means    less     charge    concentration    $\Rightarrow    E_{C}$    decreases.     @
    $E_C(L_{\text{transmon}}=350) = 0.59\,$GHz\label{fig:EJ0-EC-selection}}
\end{figure}

 \begin{table}[h]
   \centering
   \caption{Example           $E_C$          and           $E_{J0}$           values          for           fabrication.
     $2S_{\text{transmon}}=24\mum$ \label{tab:transmon-fabrication-parameters}}
   \begin{tabular}{|c|c|c|c|c|c|c|}
     \hline
     $E_{C}$ (GHz) & $E_{J0}$ (GHz) & Ratio &  $L_{\text{transmon}} (\mum)$ & $N_{sq}$ & Square side (nm) & Resistance/RT (k$\Omega$)\\\hline
     0.85 &               10 &    11.8 &              251 &                  0.12 &              34.6 &                    15.33/12.50  \\
     0.85 &               17 &    20.0 &              251 &                  0.21 &              45.8 &                    9.67/7.89  \\
     0.85 &               40 &    47.1 &              251 &                  0.49 &               70. &                    3.75/3.06     \\
     0.85 &               80 &    94.1 &              251 &                  0.97 &              98.5 &                    1.90/1.55  \\
     1 &               20 &    20.0 &              217 &                  0.24 &              49.0 &                    7.67/6.25  \\
     0.6 &              100 &   166.7 &              345 &                  1.22 &              110. &                    1.508/1.23 \\
     0.6 &              120 &   200.0 &              345 &                  1.46 &              121. &                    1.26/1.03 \\\hline
   \end{tabular}
 \end{table}

 \noindent An example of the resulting structure is given in table \autoref{tab:transmon-fabrication-parameters}.

 \begin{figure}[h]
   \centering                     \includegraphics[width=0.3\textwidth]{2020-09-22_teresa_xmon/2020-09-22_teresa_xmon-1}
   \includegraphics[width=0.3\textwidth]{2020-09-22_teresa_xmon/2020-09-22_teresa_xmon-2}
   \includegraphics[width=0.3\textwidth]{2020-09-22_teresa_xmon/2020-09-22_teresa_xmon-3}
 \end{figure}

 \noindent


 \newpage
 \subsection{Transmon 2-level approximation}
 \label{sec:transmon-2-level}


 \noindent Normally  $ N_\text{ext} $  is biased at  a sweet spot  \red{close to $  - 1/2 $}.   Then the only  number of
 electrons   $n$   on   out   island   that   will   give   a  low   energy   (due   to   the   energy   dispersion   in
 Fig.~\ref{fig:cp_box_energy_charge}) is either  0 or -1.  The other level  will be far separated. Then we  take out the
 Hamiltonian

 \red{\begin{equation}
     \begin{aligned}
       \mathcal{H}_\text{0 or -1} & = \begin{pmatrix}
         E_C(-1-N_\text{ext})^2 & -E_J/2\\
         -E_J/2 & E_C(N_\text{ext})^2\\
       \end{pmatrix}\\
     \end{aligned}
   \end{equation}}

 \noindent And we have a two level system just  as in the previous lecture Eq.\eqref{l1-finalEVal}.  Redefining the zero
 point energy to be in the middle of the diagonal terms

 \begin{equation}
   \begin{aligned}
     \mathcal{H} & = \begin{pmatrix}
       -\epsilon/2 & -E_J/2\\
       -E_J/2 & \epsilon/2\\
     \end{pmatrix}\\
     \epsilon/2    =   \frac{\text{energy    diff}}{2}   &    =   \frac{E_CN_\text{ext}^2-E_C(1+N_\text{ext})^2}{2}    =
     -\frac{E_C}{2}\big(1+2N_\text{ext}\big)
   \end{aligned}
 \end{equation}

\begin{framed}\noindent
  This will have solutions

   \begin{equation}
     E = \pm \frac{\Delta E}{2}, \qquad \ket{\psi}_0 = \begin{pmatrix}
       \cos(\theta/2) \\ \sin(\theta/2)
     \end{pmatrix},  \qquad  \ket{\psi}_1  = \begin{pmatrix}  \sin(\theta/2)  \\
       \cos(\theta/2)
     \end{pmatrix}, \Delta E = \sqrt{\epsilon^2+\Delta^2}
   \end{equation}
 \end{framed}

 \subsection{Charge energy dominates: $ E_C/E_J $:}
 \begin{itemize}
 \item High charge noise - gate voltage changes, affects the energy levels severely
 \item High anharmonicity - quadratic $ n $ dependance dominates, allowing individual addressing of the levels;
 \end{itemize}

 \newpage
 \subsection{\red{Flux energy dominates: $ E_C/E_J << 1 $}}
 If flux is the important variable, then we chall work with the flux basis using the wavefunctions $ \psi(\phi) $

 \begin{framed}\noindent

   \begin{equation}
     \begin{aligned}
       &\blue{\hat{\varphi}} \qquad\\
       &\red{\hat{N}} = -i\frac{d}{d\blue{\varphi}}
     \end{aligned}
   \end{equation}

 \end{framed}
 \noindent and we rewrite

 \begin{equation}\label{eq:transmon-in-phase-basis}
   \begin{aligned}
     \mathcal{H} & = E_C{\left(\hat{\red{N}}-N_\text{ext}\right)^2}- E_J\cos\left(\hat{\blue{\varphi}}\right)\\
     & = E_C\left(-i\frac{d}{d\varphi} - n_g\right)^2 - E_J\cos(\varphi).
   \end{aligned}
 \end{equation}

 \subsubsection{Numerical solution}
 \label{sec:transmon-flux-numerical-solution}

 \begin{framed}\noindent
   We dice up the state of the system into individual values at discretised positions, with as step of $ \Delta \delta $
   in order to solve
   \begin{equation}
     \mathcal{H}\Psi(\varphi) = E\Psi(\varphi)
   \end{equation}

\[
  \Psi(\varphi) \rightarrow \begin{pmatrix}
    \Psi(0)\\
    \Psi(\Delta \delta)\\
    \Psi(2\Delta \delta)\\\vdots
    \\
    \Psi(n\Delta \delta)
  \end{pmatrix}
  \equiv
  \begin{pmatrix}
    w_0\\
    w_1\\
    w_2\\
    \vdots\\
    w_{n}
  \end{pmatrix}
\]

\noindent and solve for this wavefunction made up of a lot of small contributions at different phases.
\end{framed}

\begin{figure}[h]
  \centering \includegraphics[height=4cm]{dicedUp}
\end{figure}

\begin{equation}
  \frac{d}{d\varphi}\left(\omega_i\right) = \frac{w_{i+1} - w_{i-1}}{2\Delta\delta}
\end{equation}

\noindent We rewrite \autoref{eq:transmon-in-phase-basis} for $w_1$ for example:
\begin{equation}
  \begin{aligned}
    & E_C\left[\left(\frac{w_{i+1}-w_{i-1}}{2\Delta\delta} \right)^{2} + 2in_{g}\frac{w_{i+1}-w_{i-1}}{2\Delta\delta} + n_{g}^2w_1  \right] - E_J\cos(\delta)w_1 = Ew_1 \\
    &  \frac{E_C}{4\Delta\delta^{2}}\left[ w_{i+1}^2-w_{i-1}^2  - 2w_{i+1}w_{i-1}+  4in_{g}\Delta\delta(w_{i+1}-w_{i-1})
    \right] + U(w_{i})w_i = Ew_{i}
  \end{aligned}
\end{equation}

\noindent where $U(w_n) = E_Cn_{g}^2-E_J\cos(n\times\delta)$

\begin{equation}
  \begin{aligned}
    & \left[-\frac{E_C}{\Delta\delta^2}\left[\omega_{2}+\omega_{0}-2\omega_{1}\right] + U(w_1)\right]\omega_1 = Ew_1\\
    & \left[-\frac{E_C}{\Delta\delta^2}\left[\omega_{3}+\omega_{1}-2\omega_{2}\right] + U(\omega_2)\right]\omega_2 = Ew_2\\
    & \cdots
  \end{aligned}
\end{equation}

\noindent where we defined

\begin{framed}\noindent

  \begin{equation}\label{key}
    \begin{pmatrix}
      2\frac{E_c}{\Delta\delta^2} + U(w_1) & -\frac{E_c}{\Delta\delta^2} & 0 & 0 \\
      -\frac{E_c}{\Delta\delta^2} & 2\frac{E_c}{\Delta\delta^2} + U(w_2) &   -\frac{E_c}{\Delta\delta^2} & 0\\
      0 & -\frac{E_c}{\Delta\delta^2} & 2\frac{E_c}{\Delta\delta^2} + U(w_3) &   -\frac{E_c}{\Delta\delta^2}\\
      \vdots & \vdots & \vdots & \ddots
    \end{pmatrix}
    \begin{pmatrix}
      w_1\\w_2\\w_3\\\vdots
    \end{pmatrix}
    = E \begin{pmatrix}
      w_1\\w_2\\w_3\\\vdots
    \end{pmatrix}
  \end{equation}

  \noindent where

   \begin{equation}\label{key}
     \left[
       \begin{aligned}
         \mathbf{U(\phi_J)} & = \mathbf{E_L(\phi_\text{ext} - \phi_J)^2 - E_J\cos(\phi_J)}\\
         E_c & = \frac{(2e)^2}{2C}\\
         E_L & = \frac{\Phi_0^2}{2L(2\pi)^2}\\
         E_J & = \frac{\Phi_0I_c}{2\pi} = \frac{\Phi_0}{2\pi}\frac{\pi\Delta(0)}{2eR}
       \end{aligned}
     \right.
   \end{equation}
 \end{framed}

 The eigentates can be evaluated in \verb|MatLab|:

 \red{To complete }

 \subsubsection{Periodic analogy}
 \label{sec:periodic-analogy}

 \begin{itemize}
 \item Let us compare this to Hamiltonian in a periodic potential
   \begin{equation}
     \mathcal{H}_\text{crystal} = \frac{-\hbar^2}{2m}\frac{d^2}{dx^2}+V(x)\qquad V(x+a) = V(x),
   \end{equation}

   \noindent which, according to Bloch's theorem, states that the eigenstates will be of the form
   \begin{equation}
     \psi_{kn}(x) = e^{ikx}u_{kn}(x),
   \end{equation}
   \noindent which, when plugged in will result in an effective Hamiltonian

  \begin{equation}
    \mathcal{H}_{\text{eff},k} = \frac{\hbar^2}{2m}\left(-i\frac{d}{dx}+k\right)^2 + V(x)
  \end{equation}
\item We can see this mapping between
  \begin{equation}
    \begin{aligned}
      \mathcal{H}_{\text{eff},k} & = \frac{\hbar^2}{2m}\left(-i\frac{d}{dx}+k\right)^2 + V(x)\\
      \mathcal{H} & = E_C\left(-i\frac{d}{d\phi} - n_g\right)^2 - E_J\cos(\phi),
    \end{aligned}
  \end{equation}

  \noindent so, will look for solutions of a similar form.
\item As $ E_C/E_J $ gets smaller, the potential well from the $ E_J\cos(\phi) $ gets deeper \red{\textbf{and the states
      within each well localise, and stop interacting with one  another.}}  Solving, as in the Transmon Paper, will lead
  to anharmonicity relation

  \begin{framed}\noindent
    \begin{equation} \frac{E_{12} - E_{01}}{E_{01}}\approx -(8E_J/E_C)^{-1/2},
    \end{equation}
    \noindent which decreases as we continue to increase $ E_J $.
  \end{framed}
\end{itemize}
\newpage

\section{Coupling transmon with resonator}
\red{Following is extract from Bader which explains coupling to a transmon via a resonator.}

In the following we perform the following approximation:
\begin{itemize}
\item          Consider          transmon         to          have          a          single         JJ          (using
  $E_J(\Phi_{\text{ext}}) = \ballblue{E_{J0}} \times 2|\cos(\pi\Phi_\text{ext}/\Phi_0)|$);
\end{itemize}

\begin{figure}[h]
  \centering \inkfig{8cm}{cooper_pair_box_6_cqed}
  \caption{\small Adding resonator that is interacting with the transmon\label{fig:cooper_pair_box_6_cqed}}
\end{figure}

\subsubsection{Coupling strength}
\label{sec:coupling-strength}


It will be found that the matrix element that gives the coupling strength

\begin{equation}
  g = \left( 2e\beta V_{\text{rms}}^0/\hbar \right)\left( \frac{E_{J}}{8E_{C}} \right)^{1/4}
\end{equation}

\noindent which is paradoxical since:
\begin{itemize}
\item Charge sensitivity is reduced by using the shunt capacitance of the transmon
\item So therefore a bigger C (lower $E_C$) should \textbf{reduce coupling to environmental fields}.
\end{itemize}

\begin{framed}\noindent
  \red{But this is because the transmon was considered in the  static case - where we monitor response of transmon to DC
    charge offsets.}

  But for dynamic coupling (when field  is close to the frequency of the qubit) this is  not the case.  In fact, because
  flux terms  are more dominant in  the transmon, compared to  the CPB, more charge  terms are involved in  the dynamics
  making the transmon more polarizable. (see Bader).
\end{framed}

\includepdf[pages=-]{bader_resonator_cropped.pdf}

\newpage

%%% Local Variables:
%%% mode: latex
%%% TeX-master: "all_the_notes"
%%% End:
