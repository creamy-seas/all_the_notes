\section{Changing the basis}
\begin{center}
	An operator is \textbf{unitary} if its Hermetian conjugate is its inverse i.e. $UU^{\dagger} = U^{\dagger}U = \mathbb{I}$
\end{center}

Unitary operators are 
\begin{itemize}
	\item Norm preserving;
	\item Inner product preserving,
\end{itemize}

Making then ideal for transformation of basis to

\begin{center}
	\textbf{One can always transforms from $\left\lbrace \ket{\beta_i} \right\rbrace$ to $\left\lbrace \ket{\alpha} \right\rbrace$ by $\mathbf{U\ket{\beta_i} = \ket{\alpha_i}}$.}
\end{center}

Let us find the matrix form of the operator $U$ that results in $\iket{\alpha_i} \xrightarrow{U} \iket{\beta_i}$.

We begin with 

\begin{equation}
\iket{\Psi} = \sum_jc_j\iket{\alpha_i},
\end{equation}

and apply the closure relation, using the $\left\lbrace \ket{\beta_i} \right\rbrace$ basis.

\begin{equation}
\begin{aligned}
\iket{\Psi} & = \sum_jc_j\left( \sum_k \ket{\beta_k}\bra{\beta_k} \right) \ket{\alpha_j} \\
& = \sum_k \left(\sum_jc_j \braket{\beta_k|\alpha_j} \right) \ket{\beta_k}\\
& = \sum_{j,k} c_j\red{U_{kj}\iket{\beta_k}},
\end{aligned}
\end{equation}

\noindent the result of which is the transformation of \iket{\Psi_{\left\lbrace \ket{\alpha_j} \right\rbrace}} to \iket{\Psi_{\left\lbrace \ket{\beta_j} \right\rbrace}}, by applying suitable weightings $U_kj$ in the new basis.

The elements $U_{k,j} = \braket{\beta_k|\alpha_j}$ are the matrix elements on the unitary of operator

\begin{equation}
\begin{aligned}
U & = \begin{bmatrix}
U_{00} & U_{01} & U_{02} & \cdots & U_{0n}\\
U_{10} & U_{11} & U_{12} & \cdots & U_{1n}\\
\vdots & \vdots & \vdots & \ddots & \vdots\\
U_{m0} & U_{m1} & U_{m2} & \cdots & U_{mn}
\end{bmatrix}
= \begin{bmatrix}
\braket{\beta_0|\alpha_0} & \braket{\beta_0|\alpha_1} & \braket{\beta_0|\alpha_2} & \cdots & \braket{\beta_0|\alpha_n}\\
\braket{\beta_1|\alpha_0} & \braket{\beta_1|\alpha_1} & \braket{\beta_1|\alpha_2} & \cdots & \braket{\beta_1|\alpha_n}\\
\vdots & \vdots & \vdots & \ddots & \vdots\\
\braket{\beta_m|\alpha_0} & \braket{\beta_m|\alpha_1} & \braket{\beta_m|\alpha_2} & \cdots & \braket{\beta_m|\alpha_n}\\
\end{bmatrix}
\end{aligned}
\end{equation}

Thus when one perform the unitary operation

\begin{equation}
\begin{aligned}
\red{\ket{\alpha_k} = U\iket{\beta_k}} & \equiv \red{U_{kj}\iket{\beta_k}} = 
\begin{bmatrix}
\braket{\beta_0|\alpha_0} & \braket{\beta_0|\alpha_1} & \braket{\beta_0|\alpha_2} & \cdots & \braket{\beta_0|\alpha_n}\\
\braket{\beta_1|\alpha_0} & \braket{\beta_1|\alpha_1} & \braket{\beta_1|\alpha_2} & \cdots & \braket{\beta_1|\alpha_n}\\
\vdots & \vdots & \vdots & \ddots & \vdots\\
\braket{\beta_m|\alpha_0} & \braket{\beta_m|\alpha_1} & \braket{\beta_m|\alpha_2} & \cdots & \braket{\beta_m|\alpha_n}\\
\end{bmatrix} 
\left[ \begin{matrix}
0\\0\\\vdots\\ 1\\\vdots
\end{matrix}\right] \leftarrow \text{in $\left\lbrace \ket{\beta_j} \right\rbrace$ basis}\\
& = \begin{bmatrix}
\braket{\beta_0|\alpha_k} \\\braket{\beta_1|\alpha_k}\\\vdots\\\braket{\beta_m|\alpha_k}
\end{bmatrix}\leftarrow \text{now expressed in the $\left\lbrace \ket{\alpha_j} \right\rbrace$ basis}\\
& = \sum_j \braket{\beta_j|\alpha_k} \ket{\beta_j},
\end{aligned}
\end{equation}

so one has effectively projected \iket{\alpha_k} onto \textbf{every single component} of the $\left\lbrace \ket{\beta_i} \right\rbrace$ basis. 
\newpage 

