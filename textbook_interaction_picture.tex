\section{Interaction picture \label{sec:interaction}}
Given an eigenstates of $\hat{H}_0$, we can easily compute the system evolution 

\begin{equation}
\ket{\psi(t)} = \sum_j\alpha_je^{-iE_jt/\hbar}\ket{\phi_j}.
\end{equation} 

Now to find the state for a system with a Hamiltonian 

\begin{equation}
\hat{H} = {H}_0 +V,
\end{equation}

\noindent we define the interaction picture state

\begin{equation}
\ket{\psi(t)}_I = U_0(t)^\dagger\ket{\psi(t)}_S,
\end{equation}

\noindent Substituting $\ket{\psi(t)}_S = U_0(t)\ket{\psi(t)}_I$ into the Schrödinger equation $i\hbar\frac{\partial}{\partial t}\ket{\psi(t)}_S = \hat{H}\ket{\psi(t)}_S$

\begin{equation}
\begin{aligned}
i\hbar\frac{\partial}{\partial t}\left(U_0(t)\ket{\psi(t)}_I\right) = i\hbar\left(U_0(t)\frac{\partial\ket{\psi(t)}_I}{\partial t} + \frac{-iH_0}{\hbar}U_0(t)\ket{\psi(t)}_I\right) \equiv HU_0(t)\ket{\psi(t)}_I,
\end{aligned}
\end{equation}

\noindent multiplying by $U_0^\dagger(t)$

\begin{equation}\label{eqn:interactionPictureHamiltonian}
\begin{aligned}
i\hbar\ipartial{\ket{\psi(t)}_I}{t} & = H_I\ket{\psi(t)}_I\\
& H_i = U_0^\dagger(t)\left[H-H_0\right]U_0(t) = U_0^\dagger(t)\left[V\right]U_0(t).
\end{aligned}
\end{equation}

\red{\textbf{\large Ultimately, what we have found is that to solve for $H=H_0+V$, one needs to}}
\textcolor{red}{\begin{enumerate}
		\item Find $H_I = U_0^\dagger(t)VU_0(t)$
		\item Solve the interaction Scrodinger equation $i\hbar\ipartial{\ket{\psi(t)}_I}{t} = H_I\ket{\psi(t)}_I$
		\item Find the Schroinger picture states by unwinding i.e. $\ket{\psi(t)}_S = U_0(t)\ket{\psi(t)}_I$
\end{enumerate}} 

It can be easily shown that expectation values are preserved

\begin{equation}
_I\bra{\psi}\hat{O}_I\ket{\psi}_I = _S\bra{\psi}U_0U_0^\dagger\hat{O}_SU_0U_0^\dagger\ket{\psi}_s \equiv _S\bra{\psi}\hat{O}_S\ket{\psi}_S
\end{equation}

\newpage 
\section{Interaction between light and atom \label{sec:lightAtom}}
We construct the Hamiltonian for the full system

\begin{equation}
\hat{H} = \hat{H}_\text{atom} + \hat{H}_\text{light} + \hat{H}_\text{atom - light}. 
\end{equation}

\begin{itemize}
	\item $\hat{H}_\text{atom}$ comes from the energy levels of the atom
	\begin{equation}
	\label{eqn:atomHamiltonian}
	\hat{H}_\text{atom} = \hbar\omega_a\iketbra{e}{e}\otimes\mathbb{I}_\text{light},
	\end{equation}
	
	\item $\hat{H}_\text{light}$ from Sec.\ref{subsec:singleMode} is the Hamiltonian for the photon modes
	\begin{equation}
	\label{eqn:lightHamiltonian}
	\hat{H}_\text{light} = \mathbb{I}_\text{atom} \otimes \hbar\omega_j\left(\hat{a}^\dagger\hat{a}+\frac{1}{2}\right),
	\end{equation}
	
	\item and $\hat{H}_\text{atom - light} = - \hat{E}\cdot \hat{D} $ from Sec.\ref{sec:dipole}
	\begin{equation}
	\begin{aligned}
	\hat{H}_\text{atom - light} & = - \left(\mathbf{E_{0x}}(\hat{a}+\hat{a}^\dagger)\mathbf{\red{\sin(kz)}}\right)\left(\red{\mathbf{\vec{d}}}\left(\iketbra{g}{e}+\iketbra{e}{g}\right)\right)\\
	& = \mathbf{\red{\hbar g(\vec{r})}}\left(\hat{a}+\hat{a}^\dagger\right)\left(\iketbra{g}{e}+\iketbra{e}{g}\right),
	\end{aligned}
	\end{equation}	
\end{itemize}  

\noindent where \red{$\hbar g(\vec{r}) = E_{0x}\sin(kz)\vec{d}$} is the coupling strength. Although ideally we should be writing state via the tensor product notation, we shall do this instead save space and write

\begin{equation}
\ket{g,n}=\ket{g}\otimes\ket{n}
\end{equation}

\noindent The full Hamiltonian is then

\begin{equation}
\hat{H} = \hbar\omega_a\iketbra{e}{e} + \hbar\omega_j\left(\hat{a}^\dagger\hat{a}+\frac{1}{2}\right) + {\hbar g(\vec{r})}\left(\hat{a}+\hat{a}^\dagger\right)\left(\iketbra{g}{e}+\iketbra{e}{g}\right),
\end{equation}

\noindent which we split up into two parts, to use the interaction picture, covered in Sec.\ref{sec:interaction}

\begin{equation}
\left\lbrace 
\begin{aligned}
H & = H_0+\red{V}\\
H_0 & = \hbar\omega_a\iketbra{e}{e} + \hbar\omega_j\left(\hat{a}^\dagger\hat{a}+\frac{1}{2}\right)\\
\textcolor{red}{V} & \red{=} \textcolor{red}{{\hbar g(\vec{r})}\left(\hat{a}+\hat{a}^\dagger\right)\left(\iketbra{g}{e}+\iketbra{e}{g}\right)}
\end{aligned}\right.
\label{eqn:lightAtomHamiltonians}
\end{equation}

\noindent We compute the interaction Hamiltonian

\begin{equation}
\left\lbrace
\begin{aligned}
H_I & = U_0^\dagger V U_0 \\
U_0 & = \exp\left[-i\omega_j\left(\hat{a}^\dagger\hat{a}+{1}/{2}\right)t\right]\exp\left[-i\omega_a\iketbra{e}{e}t\right]
\end{aligned}\right. \Rightarrow
\end{equation}

\begin{multline}
\Rightarrow \textcolor{red}{\hbar g(\vec{r})} \times \\
\exp\left[+i\omega_j\left(\hat{a}^\dagger\hat{a}+{1}/{2}\right)t\right]\textcolor{red}{\left(\hat{a}+\hat{a}^\dagger\right)}\exp\left[-i\omega_j\left(\hat{a}^\dagger\hat{a}+{1}/{2}\right)t\right] \times\\
\exp\left[+i\omega_a\iketbra{e}{e}t\right]\textcolor{red}{\left(\iketbra{g}{e}+\iketbra{e}{g}\right)}\exp\left[-i\omega_a\iketbra{e}{e}t\right]
\end{multline}

\noindent Then we proceed term by term
\begin{itemize}
	\item $\exp\left[+i\omega_j\left(\hat{a}^\dagger\hat{a}+{1}/{2}\right)t\right]\textcolor{red}{\hat{a}}\exp\left[-i\omega_j\left(\hat{a}^\dagger\hat{a}+{1}/{2}\right)t\right] = \exp\left[+i\omega_j\hat{a}^\dagger\hat{a}t\right]\textcolor{red}{\hat{a}}\exp\left[-i\omega_j\hat{a}^\dagger\hat{a}t\right]$
	
	The lowering operator \textcolor{red}{
		\begin{equation}
		\begin{aligned}
		\hat{a} & =
		\begin{bmatrix}
		\bra{0}\hat{a}\ket{0} & \bra{0}\hat{a}\ket{1} & \cdots\\
		\bra{1}\hat{a}\ket{0} & \bra{1}\hat{a}\ket{1} & \cdots\\
		\bra{2}\hat{a}\ket{0} & \bra{2}\hat{a}\ket{1} & \cdots\\
		\vdots & \vdots & \ddots\\
		\end{bmatrix}
		= \begin{bmatrix}
		\bra{0}0 & \bra{0}\sqrt{1}\ket{0} & \cdots\\
		\bra{1}0 & \bra{1}\sqrt{1}\ket{0} & \cdots\\
		\bra{2}0 & \bra{2}\sqrt{1}\ket{0} & \cdots\\
		\vdots & \vdots & \ddots\\	
		\end{bmatrix}
		= \begin{bmatrix}
		0 & \sqrt{1} & \cdots\\
		0 & 0 & \cdots\\
		0 & 0 & \cdots\\
		\vdots & \vdots & \ddots\\	
		\end{bmatrix}\\
		& = \sum_n\sqrt{n}\iketbra{n-1}{n}
		\end{aligned}
		\end{equation}}
	
	\noindent and thus evaluating
	
	\begin{equation}
	\begin{aligned}
	e^{+i\omega_jt\hat{a}^\dagger\hat a} & \left[\sum_n\sqrt{n}\iketbra{n-1}{n}\right]e^{+i\omega_jt\hat{a}^\dagger\hat a}\\
	& = \sum_n \sqrt{n}e^{+i\omega_jt\hat{a}^\dagger\hat a}\ket{n-1}\bra{n}e^{-i\omega_jt\hat{a}^\dagger\hat a} \qquad \leftarrow\text{ expand exponential operator}\\
	& = \sum_n \sqrt{n}e^{+i\omega_j(n-1)t}\ket{n-1}\bra{n}e^{-i\omega_jnt}\\
	& = e^{+i\omega_j(n-1)t}\left[\sum_n \sqrt{n}\ket{n-1}\bra{n}\right]e^{-i\omega_jnt}\\
	& = e^{+i\omega_j(n-1)t}\hat{a}e^{-i\omega_jnt}\\ & \blue{\mathbf{\equiv  \text{\textbf{exp}}[-i\omega_jt]\hat{a}}}
	\end{aligned}
	\end{equation}
	
	\item $\exp\left[+i\omega_j\left(\hat{a}^\dagger\hat{a}+{1}/{2}\right)t\right]\textcolor{red}{\hat{a}^\dagger}\exp\left[-i\omega_j\left(\hat{a}^\dagger\hat{a}+{1}/{2}\right)t\right] = \blue{\mathbf{\text{\textbf{exp}}[+i\omega_jt]\hat{a}^\dagger}}$
	
	\item \begin{equation}
	\begin{aligned}
	\exp\left[+i\omega_a\iketbra{e}{e}t\right]&\textcolor{red}{\iketbra{g}{e}}\exp\left[-i\omega_a\iketbra{e}{e}t\right] \\
	& = \bigg[\exp\left[+i\omega_at\left(\mathbb{I} - \iketbra{g}{g}\right)\right]\textcolor{red}{\ket{g}}\bigg]\bigg[\textcolor{red}{\bra{e}}\exp\left[-i\omega_a\iketbra{e}{e}t\right]\bigg]\\
	& = \bigg[\exp\left[+i\omega_at\left(1 - 1\right)\right]\textcolor{red}{\ket{g}}\bigg]\bigg[\textcolor{red}{\bra{e}}\exp\left[-i\omega_a1t\right]\bigg]\\
	& = \bigg[\exp\left[+i\omega_at0\right]\bigg]\textcolor{red}{\ket{g}}\textcolor{red}{\bra{e}}\bigg[\exp\left[-i\omega_at\right]\bigg]\\
	& \blue{\mathbf{\equiv \text{\textbf{exp}}\left[-i\omega_at\right]\iketbra{g}{e}}}
	\end{aligned}
	\end{equation}
	\item $\exp\left[+i\omega_a\iketbra{e}{e}t\right]\textcolor{red}{\iketbra{g}{e}}\exp\left[-i\omega_a\iketbra{e}{e}t\right] \blue{\mathbf{\equiv \text{\textbf{exp}}\left[+i\omega_at\right]\iketbra{e}{g}}} $
\end{itemize}

Combining the results

\begin{equation}
\begin{aligned}
H_I & = \hbar g(\vec{r})\bigg[\blue{\exp^{-i\omega_jt}\hat{a}}+\blue{\exp^{+i\omega_jt}\hat{a}^\dagger}\bigg]\bigg[\blue{\exp^{-i\omega_at}\iketbra{g}{e}}+\blue{\exp^{+i\omega_at}\iketbra{e}{g}}\bigg] \\
& = \hbar g(\vec{r})\bigg(\mathbf{\exp\left[-i(\omega_j-\omega_a)t\right]\ \hat{a}\iketbra{e}{g}} \quad + \quad \mathbf{\exp\left[i(\omega_j-\omega_a)t\right]\ \hat{a}^\dagger\iketbra{g}{e}} \\
& \quad + \quad \mathbf{\exp\left[-i(\omega_j+\omega_a)t\right]\ \hat{a}\iketbra{g}{e}} \quad + \quad \mathbf{\exp\left[i(\omega_j+\omega_a)t\right]\ \hat{a}^\dagger\iketbra{e}{g}}\bigg)
\end{aligned}
\label{eqn:fullHamiltonianLightAtom}
\end{equation}

\subsection{Resonance}

Looking at Eq.\eqref{eqn:fullHamiltonianLightAtom} at resonance, i.e. \red{\Large $\mathbf{\omega_a = \omega_j = \omega}$}

\begin{equation}
H_I = \hbar g(\vec{r}) \bigg(\hat{a}\iketbra{e}{g} \quad+\quad \hat{a}^\dagger\iketbra{g}{e} \quad+\quad \textcolor{red}{\exp\left[-i2\omega t\right]\hat{a}\iketbra{g}{e}} \quad+\quad \textcolor{red}{\exp\left[i2\omega t\right]\hat{a}^\dagger\iketbra{e}{g}}\bigg),
\end{equation}

\noindent The two terms in \textcolor{red}{red} are neglected in the \textbf{\large ROTATING WAVE APPROXIMATION} - this is suitable for when the oscillations are much faster than the time-scales of the most significant physical processes.

Physically speaking, we keep energy conserving processes, where excitation/de-excitation processes pair up with photon destruction/creation.

\begin{equation}
H_I = \hbar g(\vec{r}) \bigg(\hat{a}\iketbra{e}{g} \quad+\quad \hat{a}^\dagger\iketbra{g}{e} \bigg).
\label{eqn:rwa1}
\end{equation}
\vspace{6ex}

\noindent Now we solve for the initial condition $\ket{\psi(0)}_I = \ket{g,1}$. Notice that since

\begin{equation}
\label{eqn:effect}
H_I\ket{g,1} = \hbar g\ket{e,0} \qquad H_I\ket{e,0} = \hbar g\ket{g,1},
\end{equation}

\noindent which means that there is only coupling between $\ket{g,1}$ and $\ket{e,0}$ states. As proof of this, recall the evolution of the density matrix from Eq.\eqref{eqn:denMatEvolution}, $\rho(t) = \sum_{i=1}^{n} p_iU(t)\Ket{\Psi_i}\Bra{\Psi_i}U(t)^{\dagger}$, and ``zooming-in'' on

\begin{equation}
\label{eqn:resonance4}
\begin{aligned}
U(t)\ket{\Psi_i} = e^{-i\frac{\hat{H}}{\hbar}t}\ket{\Psi_i} & = \sum_k \frac{\left( -i\frac{\hat{H}}{\hbar}t\right)^k }{k!}\ket{\Psi_i}\\
& = \sum_k \frac{\left(\frac{-it}{\hbar}\right)^k }{k!}\textcolor{red}{\hat{H}^k\ket{\Psi_i}},
\end{aligned}
\end{equation}

\noindent and thus if the only $\ket{\Psi_i}$ was $\ket{g,1}$, which we showed can only transform to something proportional to $\ket{e,0}$ in the \textcolor{red}{red}, these will be the only to states in the system.

Thus the general state of the system will be

\begin{equation}
\label{eqn:resonance_GeneralEquation}
\ket{\psi(t)}_I = \alpha(t)\ket{g,1} + \beta(t)\ket{e,0}.
\end{equation}

\noindent Substituting into the Schroedinger equation, using Eq.\eqref{eqn:rwa1} and Eq.\eqref{eqn:effect},

\begin{equation}
\begin{aligned}
\ipartial{\ket{\psi}_I}{t} & = -\frac{i}{\hbar}H_I\ket{\psi}_I\\
\ipartial{\alpha}{t}\ket{g,1} + \ipartial{\beta}{t}\ket{e,0} & = -\frac{i}{\hbar} \bigg(\alpha H_I\ket{g,1} + \beta H_I\ket{e,0} \bigg)\\
\ipartial{\alpha}{t}\ket{g,1} + \ipartial{\beta}{t}\ket{e,0} & = -ig\bigg(\alpha\ket{e,0} + \beta\ket{g,1} \bigg)\\
\end{aligned}
\end{equation}

\noindent Multiplying by $\ket{g,1}$ and $\ket{e,0}$

\begin{equation}
\left\lbrace
\begin{aligned}
\ipartial{\alpha}{t} & = -ig\beta\\
\ipartial{\beta}{t} & = -ig\alpha\\
\end{aligned}\right.
\Rightarrow \left\lbrace
\begin{aligned}
\ipartial{\alpha^2}{t} & = -ig\ipartial{\beta}{t} = -ig(-ig\alpha)=-g^2\alpha\\
\ipartial{\beta}{t^2} & = -ig\ipartial{\alpha}{t} = -ig(-ig\beta)=-g^2\beta\\
\end{aligned}\right.
\end{equation}

\noindent And solving for the initial conditions $\alpha(0) = 1, \beta(0) = 0$

\begin{equation}
\left\lbrace
\begin{aligned}
\alpha(t) & = \exp[igt]\\
\beta(t) & = \exp[igt]
\end{aligned}\right.
\Rightarrow
\left\lbrace
\begin{aligned}
\alpha(t) & = \cos(gt)\\
\beta(t) & = i\sin(gt)
\end{aligned}\right.,
\end{equation}

\noindent resulting in 

\begin{equation}
\begin{aligned}
\ket{\psi(t)}_I & = \cos(gt)\ket{g,1}+i\sin(gt)\ket{e,0}\\
\ket{\psi(t)}_S & = \cos(gt)U_0(t)\ket{g,1}+i\sin(gt)U_0(t)\ket{e,0}\\
& = \cos(gt)\exp[-it \hat{H}_0/\hbar]\ket{g,1}+i\sin(gt)\exp[-it \hat{H}_0/\hbar]\ket{e,0}.
\end{aligned}
\end{equation}

\noindent Using Eq.\eqref{eqn:lightAtomHamiltonians}

\begin{equation}
\begin{aligned}
\hat{H}_0\ket{g}\ket{1} & = \bigg[\hbar\omega\iketbra{e}{e}\otimes\mathbb{I}_\text{light} + \mathbb{I}_\text{atom}\otimes \hbar\omega\left(\hat{a}^\dagger\hat{a}+\frac{1}{2}\right)\bigg]\ket{g}\otimes\ket{1}  \\
& = \hbar\omega\iketbra{e}{e}\ket{g} \otimes \ket{1} + \ket{g}\otimes \hbar\omega\left(\hat{n}+\frac{1}{2}\right)\ket{1}\\
& = 0  \ket{g} \otimes\ket{1} + \ket{g} \otimes \hbar\omega(1+1/2)\ket{1}\\
& = \left(0+3\hbar\omega/2)\right)\ket{g,1} = \frac{3\hbar\omega}{2}\ket{g,1}\\
\hat{H}_0\ket{e}\ket{0} & = (\hbar\omega + \hbar\omega/2) = \frac{3\hbar\omega}{2}\ket{e,0}
\end{aligned}
\end{equation}

\noindent one gets a constant phase factor

\textcolor{red}{\begin{equation}
	\ket{\psi(t)}_s = \exp[-i3\omega t/2]\bigg[\cos(gt)\ket{g,1}+i\sin(gt)\ket{e,0}\bigg]
	\end{equation}}

\noindent For any arbitrary $n$, it can be shown that

\textcolor{red}{\begin{equation}
	\ket{\psi(t)}_I = \cos(\sqrt{n}gt)\ket{g,n}+i\sin(\sqrt{n}gt)\ket{e,n-1}
	\end{equation}}  

\newpage
