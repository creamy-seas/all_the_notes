
\section{Twin Flux qubit paper}
\iframe{We can apply a field to a meta-materials (material formed by assembling together large numbers of artificial atoms), to change the ground state of the system and supress tranmission of field.

  The ground state changes from 
  \begin{equation}\label{key}
  	\iket{g} = \iket{0} \ira \iket{g} = \iket{\pi}
  \end{equation}
  \noindent \red{which redially changes the transmission properties of the system.}
 }

The idea is to create circuits known as quantum - metamaterials. Effectively the `atoms' forming the circuit are macroscopic artificial atoms, and electromagnetic waves passing through it, would interact with these atoms in a controllable way. Nowadays the most common meta-atom in the quantum regime are resonator-qubit systems.

Qubits in a loop act as a very good meta-material, due to strong coupling between qubits and the external radiation The cental junction, $ \alpha $, is shared between the loops. 

\ipicCaption{5cm}{doubleLoop}{\label{fig:doubleLoop}}
For the two loop qubit, one has two fluxes in two loops of the circuit shown in Fig.\ref{fig:doubleLoop}. Formally there are 5 different phases available across the 5 different Josephson junction, but from flux quantisation in both of the loops, and same external phases, we get:

\begin{equation}
\begin{aligned}
\phi_{10}+\phi_{21} - \phi_{20} + \phi_\text{ext} = 2\pi n\\
- \phi_{32} - \phi_{30} + \phi_{20} + \phi_\text{ext} = 2\pi n\\
\end{aligned}
\end{equation}

\noindent From this we evaluate the Hamiltonian for the system as a function of the nodal phases $ \phi = (\phi_{10}, \phi_{20}), \phi_{30} $ and charges $ q = (q_{0}, q_{1}, q_{3}) $ (supplementary notes derives it from the Lagrangian for the circuit, but end results it the same)

\begin{equation}
\begin{aligned}
	\mathcal{H} = (\frac{(2e)^2}{2}q\hat{C}^{-1}q^{T}) + E_J\bigg(4 + \alpha & - \alpha\cos(\phi_{20}) -\cos(\phi_{10}) -\cos(\phi_{30}) \\& - \cos(\phi_{20} - \phi_{10} - \phi_{\text{ext}}) - \cos(\phi_{20} - \phi_{30} + \phi_{\text{ext}})   \bigg),
\end{aligned}
\end{equation}

\noindent and setting the phase to zero on the grounded plane:

\begin{equation}
\begin{aligned}
\mathcal{H} = (\frac{(2e)^2}{2}q\hat{C}^{-1}q^{T}) + E_J\bigg(4 + \alpha & - \alpha\cos(\phi_{2}) -\cos(\phi_{1}) -\cos(\phi_{3}) \\& - \cos(\phi_{2} - \phi_{1} - \phi_{\text{ext}}) - \cos(\phi_{2} - \phi_{3} + \phi_{\text{ext}})   \bigg),
\end{aligned}
\end{equation}

\red{\iframe{Design parameters where $ C = 5.2\, $fF, $ \alpha = 0.72 $, \iunit{E_J = 39}{GHz}}}

\grey{For evaluation of the capacitance matrix, $ \hat{C} $, we need to consider the charge on the difference islands, assuming zero voltage on the ground section:

 \begin{equation}\label{key}
 	\begin{pmatrix}
 		q_1\\q_2\\q_3
 	\end{pmatrix}
 	= 
 	\begin{pmatrix}
 		CV_{10} - CV_{21}\\
 		 \alpha CV_{20} + CV_{21} + CV_{23} \\
 		 CV_{30} - CV_{23}
 	\end{pmatrix}
 	= \begin{pmatrix}
 		CV_{1} - CV_{2} + CV_{1}\\
 		\alpha CV_{2} +2CV_{2} - CV_{1} - CV_{3}\\
 		CV_{3} - CV_{2} + CV_{3}
 	\end{pmatrix}
 	= C\begin{pmatrix}
 		2 & -1 & 0\\
 		-1 & 2 + \alpha & -1\\
 		0 & -1 & 2
 	\end{pmatrix}
 	\imatrixthree{V_1}{V_2}{V_3}
 \end{equation}
}

\ipicCaption{4cm}{twinqubit_1}{Transition from the ground energy state to the next excited one. We get a peak, instead of the usual dip. Notice how this is very flat \ra it is resistance to charge noise. In the centerl, we see a valley where we have $ \iket{\pi}_{gs} $. Outside its $ \iket{0}_{gs} $.}

 \newpage
 \subsection{Finer look at the system\label{sec:dipole_finer_look}}
  \iframe{Because we apply the same external phases to the loops, there is no current across the central junction meaning that
  	
  \[
  	I = I_c\sin(\phi) = 0 \iRa \phi_{2} = \ialigned{0\\\pi}
  \]
  
  \red{This is always true!} Thus, we can consider two Hamiltonians, depending on the ground state of the system, which we label \iket{0} and \iket{\pi} to reffer to $ \phi_2 $:
  \begin{itemize}
  	\item \iket{0}
  	\begin{equation}\label{key}
  		\mathcal{H} = (\frac{(2e)^2}{2}q\hat{C}^{-1}q^{T}) + E_J\bigg(4 -\cos(\phi_{1}) -\cos(\phi_{3}) - \cos(\phi_{1} + \phi_{\text{ext}}) - \cos(\phi_{3} - \phi_{\text{ext}})   \bigg);
  	\end{equation}
  	\item \iket{\pi}
  	\begin{equation}\label{key}
  	\mathcal{H} = (\frac{(2e)^2}{2}q\hat{C}^{-1}q^{T}) + E_J\bigg(4 + 2\alpha -\cos(\phi_{1}) -\cos(\phi_{3}) + \cos(\phi_{1} + \phi_{\text{ext}}) + \cos(\phi_{3} - \phi_{\text{ext}})   \bigg);
  	\end{equation}
  	
  \end{itemize}
}

When we actually measure the transmission,we get two strange regions of completely different transmission. At the critical field $ \Phi_c $ (and its dual analogue $ \Phi_0 - \Phi_c $):
\begin{enumerate}
	\item \textbf{Alpha junction:} phsae jumps from one allowed value to other $ \phi_{20} = 0 \ira \pi $;
%	\item \textbf{Other junctions e.g:} $ \phi_{30} + \phi_{23} = 2\pi n - \phi_\text{ext} \ira (2n-1)\pi - \phi_\text{ext}$;
	\item \textbf{The ground state} goes from the \iket{0} \ira \iket{\pi}. \red{The ground state changes - not the system state!}
\end{enumerate}

\ipicCaption{5cm}{twinqubit_2}{When the external field surpasses a certain value, the phase across the central junction jumps 0 \ra $ \pi $, as it becomes energetically favourable.\label{fig:twin_1}}

 \noindent Notice the nice flatness of the band in the central figure (immune to external magnetic field variantions) and the flipped energy spectrum, in comparisson to regular qubits
 
 At the degeneracy point, we have $ \phi_\text{ext} = \pi $ so 
 
 \[
 	\ialigned{
 		\phi_\text{ext} & = \pi;\\
 		\phi_{20} & = 0/\pi;\\
 		\phi_{10} & = \phi_{21} \quad \text{assumed};\\
 		& \phi_{10} +\phi_{21} - \phi_{20} + \phi_\text{ext} = 2\pi n
 	} \ira \text{all phases are 0 or $ \pi $ so no current}.
 \]
 
 \subsection{Electromagnetically induced transparency}
 From Fig.~\ref{fig:twin_1}, se see that there is a region of uniform tranmission, $ \pm75 $ in the magnetic field. \red{This region has electromagnetically induced transparency, where the incoming waves undergo no change in amplitude at all.} Previous papers required 3 levels, which we split to control the absorbtion of the probe field, but in this case \red{we use external field to control transmission.}
 
 \begin{center}
 	\parbox{0.4\linewidth}{EIT create dark states or cause Rabi splitting to prevent transmission} \ira \parbox{0.4\linewidth}{Chnage the ground state \iket{0} \ira \iket{\pi} to change the transmissin profile}
 \end{center}
 
 \subsection{Solving for tranmission}
  Theoretically predicting the transmission comes down to solving the wave equation for charge distribution:
  
  \begin{equation}\label{key}
  	c^2\difffrac{^2Q}{x^2} {- \difffrac{^2Q}{t^2}} \red{\blue{- \gamma\difffrac{Q}{t}} + \sum \beta\left\lbrace Q_n\right\rbrace\delta(x - x_n)},
  \end{equation}
  
  \noindent where we have a \red{source term from the individual dipole qubits positioned at various $ x_n = $} and \blue{damping term}. 
  
  Then, periodic conditions lead to the Kronig-Penney model, decoupling the equation into single ones for different atoms, after which the transmission for different ones can be found. 
  
%What is interesting is to plot these minima as a function as the external fluxes, where $  \phi_\text{ext 1} $ and $  \phi_\text{ext 2} $ lie along the two axes and use colour to represent the position of the minima. One expects there to be squares of stability around integer flux quanta, which are separated by thin lines. Moving from one of these energy minima to another is equivalent to the movement of flux in and out of the loops.

\subsection{Phases for the different flux qubit states}
 Let us find the minimum of the potential energy of the system:
 
 \begin{equation}\label{key}
 	U = E_J(4+\alpha -\alpha\cos(\phi_0) - \cos(\phi_1) - \cos(\phi_2) - \cos(\phi_2 - \phi_0 + \phi) - \cos(\phi_1 - \phi_0 - \phi).
 \end{equation}
 
 \noindent This means equating to zero the derivatives of the energy with respect to the different phases. It turns out that there are 4 solutions:

 \begin{itemize}
 	\item $ E_A $ One when the phase on the central junction is \textbf{0};
 	\item $ E_B $ Phase on central junction is $ \mathbf{\pi} $;
 	\item Degenerate solutions, when there is a `double potential well' in this phase space.
 \end{itemize}

 \ipicCaption{5cm}{twin3}{Look here now. As the field is increased, \red{the central junction phase jumps to $ \pi $ as it is energetically favourable.} The phase accumulated by the outer junctions, jumps by $ \pi/2 $ when this occurs.}

 To find the critial field, we equate the energies of the two different ground states:
 
 \begin{equation}\label{key}
 	E_A \equiv E_B \ra \phi = \arcsin\left(1-\frac{\alpha^2}{4}\right).
 \end{equation}
 
 \noindent We onyl get the 'twin' ground state, when the $ \alpha $ parameter is sufficiently large. This can be seen in the diagram of the energy of the ground state:
 
 \ipicCaption{6.5cm}{twin4}{Here we plot the energies of the different ground states. Usually we will be either in the $ A $ or $ B $ states, depending on the phase across the central junction.}

 \newpage