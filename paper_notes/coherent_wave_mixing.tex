\section{Coherent wave mixing on a single three-level system}
  Driving a three level atom with different 2-driving field configurations. The following discussion assumes driving fields $ \Omega_{32} $, $ \Omega_{31} $, that are almost in resonance with the atomic transitions \iket{2}\lra\iket{3} and \iket{1}\lra\iket{3}, as shown in Fig.~\ref{fig:transmissionLine}b. The treatment of systems under the \iket{1}\lra\iket{2} and \iket{1}\lra\iket{3} drive combination follows an analogous argument, the result for which will be stated. The two cases are distinguished with subscripts $ a $ and $ b $ respectively.
  
  The Hamiltonian for a Three-Level atom, in a corresponding basis of eigenstates \iket{1}, \iket{2}, \iket{3}, by which the three levels are labelled, can be written as
  
  \begin{equation}
  \mathcal{H}_{\text{atom}} = \begin{pmatrix}
  E_1 & 0 & 0\\0& E_2 & 0 \\0&0&E_3
  \end{pmatrix}=\begin{pmatrix}
  E_3-\hbar\omega_{31} & 0 & 0\\0& E_3-\hbar\omega_{32} & 0 \\0&0&E_3
  \end{pmatrix},
  \label{rwaAtomicHamil}
  \end{equation}
  
  \noindent where $ \hbar\omega_{31} = E_3-E_1 $ and $ \hbar\omega_{32} = E_3-E_2$. The interaction term Eq.~\eqref{theoIntFinal} for the \iket{2}\lra\iket{3}, \iket{1}\lra\iket{3}, driving arrangement will be written in matrix form as
  
  \begin{equation}
  \mathcal{H}_{\text{int}} = -\hbar\Omega_{31}\cos(\omega_{31}^{d}t) \begin{pmatrix}
  0 & 0 & 1\\0&0&0\\1&0&0
  \end{pmatrix}-\hbar\Omega_{32}\cos(\omega_{32}^{d}t) \begin{pmatrix}
  0 & 0 & 0\\0&0&1\\0&1&0
  \end{pmatrix},
  \label{rwaDriveHamiltonian}
  \end{equation}
  
  \noindent where the non-zero matrix elements indicate the states coupled by the fields. The total Hamiltonian, $ \mathcal{H}_{a} = \mathcal{H}_{\text{atom}}+\mathcal{H}_{\text{int}}$, is a sum of Eq.~\eqref{rwaAtomicHamil}, \eqref{rwaDriveHamiltonian}, written for convenience with complex exponentials,
  
  {\small \begin{equation}
  	\begin{aligned}
  	\mathcal{H}_{a}\ & {= \begin{pmatrix}
  		E_3-\hbar(\omega_{31}^{d}-\delta\omega_{31}) & 0 & -\frac{\hbar\Omega_{31}}{2}\bigg(e^{i\omega^{\text{d}}_{31}t}+e^{-i\omega^{\text{d}}_{31}t}\bigg)  \\   	0 & E_3-\hbar(\omega_{32}^{d}-\delta\omega_{32})  & -\frac{\hbar\Omega_{32}}{2}\bigg(e^{i\omega^{\text{d}}_{32}t}+e^{-i\omega^{\text{d}}_{32}t}\bigg)  \\   	-\frac{\hbar\Omega_{31}}{2}\bigg(e^{i\omega^{\text{d}}_{31}t}+e^{-i\omega^{\text{d}}_{31}t}\bigg)  & -\frac{\hbar\Omega_{32}}{2}\bigg(e^{i\omega^{\text{d}}_{32}t}+e^{-i\omega^{\text{d}}_{32}t}\bigg) & E_3 \\
  		\end{pmatrix}}\\& \qquad\qquad\qquad\qquad\qquad\qquad = \begin{pmatrix}
  	H_{11} & H_{12} & H_{13} \\   	H_{21} & H_{22} & H_{23} \\   	H_{31} & H_{32} & H_{33} \\
  	\end{pmatrix},
  	\end{aligned}
  	\label{rwaTotalHamiltonian}
  	\end{equation}}
  
  \noindent where $  \delta\omega_{31} = \omega_{31} - \omega^{\text{d}}_{31}, \delta\omega_{32} = \omega_{32} - \omega^{\text{d}}_{32}$ represent the detunings of the driving fields from the resonant frequencies of the atom. The Hamiltonian Eq.~\eqref{rwaTotalHamiltonian} governs the time evolution of the system state 
  
  \begin{equation}
  \ket{\varPsi} = c_1\ket{1}+c_2\ket{2}+c_3\ket{3},
  \end{equation} 
  
  \noindent through the standard \schrodinger equation, which takes on the matrix form
  
  \begin{equation}
  \begin{pmatrix}
  H_{11} & H_{12} & H_{13} \\   	H_{21} & H_{22} & H_{23} \\   	H_{31} & H_{32} & H_{33} \\
  \end{pmatrix}\begin{pmatrix}
  c_1\\c_2\\c_3
  \end{pmatrix} = i\hbar\begin{pmatrix}
  \dot{c_1} \\ \dot{c_2}\\\dot{c_3}
  \end{pmatrix}.
  \label{rwaTotalHamilMatrix}
  \end{equation}
  
  \noindent By applying a time dependent transformation
  
  \begin{equation}
  \label{eqn:InteractionTransformation}
  \widetilde{c_i} = e^{i\phi_{i}(t)}c_i,
  \end{equation}
  
  \noindent where $ i=1,2,3 $, one can rewrite Eq.~\eqref{rwaTotalHamilMatrix} as
  
  \begin{equation}
  \begin{pmatrix}
  H_{11}-\hbar\dot{\phi}_1 & H_{12}e^{i\phi_{12}} & H_{13}e^{i\phi_{13}} \\  H_{21}e^{i\phi{21}} & H_{22}-\hbar\dot{\phi}_2 & H_{23}e^{i\phi{23}} \\   	H_{31}e^{i\phi{31}} & H_{32}e^{i\phi{32}} & H_{33}-\hbar\dot{\phi}_3 \\
  \end{pmatrix}\begin{pmatrix}
  \widetilde{c_1}\\\widetilde{c_2}\\\widetilde{c_3}
  \end{pmatrix} = i\hbar\begin{pmatrix}
  \dot{\widetilde{c_1}} \\ \dot{\widetilde{c_2}}\\\dot{\widetilde{c_3}}
  \end{pmatrix},
  \label{rawMatrixAfterTransformation}
  \end{equation}
  
  \noindent where $ e^{i\phi_{kj}} = e^{i(\phi_k-\phi_j)} $. Setting 
  
  \begin{equation}
  \phi_1 = \bigg(\frac{E_3}{\hbar}- \omega_{31}^d\bigg)t;\ \phi_2=\bigg(\frac{E_3}{\hbar}- \omega_{32}^d\bigg)t;\ \phi_3 = \frac{E_3}{\hbar}t,
  \label{rwaTransfomration}
  \end{equation}
  
  \noindent which effectively rotates the state vector components at the natural atomic evolution and detuning frequencies, \footnote{The full unitary transformation applied:
  	
  	$\qquad\qquad\quad U(t) = \exp\big[\frac{it}{\hbar}\left[({E_1}- \hbar\delta\omega_{31})\ketbra{1}{1}+({E_2}- \hbar\delta\omega_{32}\ketbra{2}{2})+E_{3}\ketbra{3}{3}\right]\big] $} will give a simple expression for the Hamiltonian
  
  \begin{equation}
  \widetilde{\mathcal{H}}_{a} = \begin{pmatrix}
  \hbar\delta\omega_{31} & 0 & -\frac{\hbar\Omega_{31}}{2}\\  0 & \hbar\delta\omega_{32} & -\frac{\hbar\Omega_{32}}{2} \\   	-\frac{\hbar\Omega_{31}}{2} & -\frac{\hbar\Omega_{32}}{2} & 0 \\
  \end{pmatrix} + \begin{pmatrix}
  0 & 0 & -\frac{\hbar\Omega_{31}}{2}e^{-i2\omega^{d}_{31}t}\\  0 & 0 & -\frac{\hbar\Omega_{32}}{2}e^{-i2\omega^{d}_{32}t}  \\   	-\frac{\hbar\Omega_{31}}{2}e^{-i2\omega^{d}_{31}t} & -\frac{\hbar\Omega_{32}}{2}e^{-i2\omega^{d}_{32}t} & 0 \\
  \end{pmatrix}.
  \label{rawTransformedFinal}
  \end{equation}
  
  \noindent In the rotating-wave-approximation, one ignores the contribution from the fast rotating $ 2\omega^{d}_{31} $, $ 2\omega^{d}_{32} $ in the second term, since their oscillations will be averaged out at the time-scales of significant processes - physically they correspond to energy non-conserving processes \cite{ioChunHoi}. One is left with the following results for the two driving cases in Fig.~\ref{fig:transmissionLine}
  
  \begin{equation}
  \label{rwaHamitlonianApprox}
  \widetilde{\mathcal{H}}_{a} = -\frac{\hbar}{2}\begin{pmatrix}
  -2\delta\omega_{31} & 0 & \Omega_{31}\\  0 & -2\delta\omega_{32} & \Omega_{32} \\   	\Omega_{31} & \Omega_{32} & 0
  \end{pmatrix};\qquad\widetilde{\mathcal{H}}_{b} = -\frac{\hbar}{2}\begin{pmatrix}
  0 & \Omega_{21} & \Omega_{31}\\  \Omega_{21} & 2\delta\omega_{21} & 0 \\   	\Omega_{31} & 0 & 2\delta\omega_{31}\end{pmatrix}.
  \end{equation}
  
  \noindent For various $ \delta\omega_{ij} \approxeq 0$, certain energy levels in the rotated frame become degenerate, shown in Fig.~\ref{theoRotation}b for a two level system. The off-diagonal terms induce a splitting $ \hbar\Omega $ between the eigenstates of the system,\footnote{For a Two-Level \iket{0}, \iket{1}, system,
  	
  	$\qquad\quad \mathcal{H} = \hbar\big(\begin{smallmatrix}0 & -\Omega/2\\-\Omega/2&0\end{smallmatrix}\big) $ has eigenenergies $ \pm\hbar\Omega $ and eigenstates $ \big(\iket{0}\mp\iket{1}\big)/\sqrt{2} $.} lifting this degeneracy. In the original unrotated frame, these Rabi splittings persist, resulting in a richer energy level structure as in Fig.~\ref{theoRotation}.
  
  \begin{figure}
  	\ifigure{4cm}{rotwave}
  	\caption{\textbf{The Rabi splitting of the energies in a Two-Level system.} A driving field with angular frequency $ \omega_{ij}^{d} $ and Rabi amplitude $ \Omega_{ij} $ in resonance with the atomic transition~$ \omega_{ij} $, couples the states \iket{i} and \iket{j}. In the rotated frame, the eigenstates $ \big(\iket{i}~\pm~\iket{j}\big)/\sqrt{2} $ are separated by an energy $ \hbar\Omega_{ij} $. Retuning to the original frame will map the Rabi splitting onto the original levels.}
  	\label{theoRotation}
  \end{figure}
  
  The general form of a Thee-Level state, is given by the density matrix \cite{quantumOptics}
  
  \begin{equation}
  \rho = \begin{pmatrix}
  1-\rho_{22}-\rho_{33} & \rho_{12} & \rho_{13}\\
  \rho_{21} & \rho_{22} & \rho_{23}\\
  \rho_{31} & \rho_{32} & \rho_{33}\\
  \end{pmatrix},
  \label{rwaDensity}
  \end{equation} 
  
  \noindent where the diagonal elements $ 1-\rho_{22}-\rho_{33}, \rho_{22}, \rho_{33} $, are the probabilities of observing the system in the \iket{1}, \iket{2}, \iket{3}, states, and the off-diagonal elements are related to coherence in the system.  The evolution of the state Eq.~\eqref{rwaDensity} is engrained in the Markovian master equation \cite{zoller}
  
  \begin{equation}
  \dot{\rho} = -\frac{i}{\hbar}\big[\mathcal{H},\rho\big]+\mathcal{L}\big[\rho\big],
  \label{rwaMarkovian}
  \end{equation}
  
  \noindent in which the first term accounts for the coherent evolution under the system Hamiltonian Eq.~\eqref{rwaHamitlonianApprox}, and the second Linbland term
  
  \begin{equation}
  \label{linLinTerm}
  \begin{aligned}
  \mathcal{L} & = \begin{pmatrix}
  \Gamma_{21}\rho_{22} + \Gamma_{31}\rho_{33} & -\gamma_{12} & -\gamma_{13}\\
  -\gamma_{21} & -\Gamma_{21}\rho_{22} + \Gamma_{32}\rho_{33} & -\gamma_{23}\\
  -\gamma_{31} & -\gamma_{32} & -\Gamma_{31}\rho_{33} + \Gamma_{31}\rho_{33}\\
  \end{pmatrix},
  \end{aligned}
  \end{equation}
  
  \noindent takes into account the relaxation and dephasing processes in the system. A comprehensive study of the origin of the Linbland term Eq.~\eqref{linLinTerm} is given by Ithier \cite{ithier}, the relevant outcomes of which are presented in Sec.~\ref{subsubsec:Decoherence}. One solves the Markovian master equation Eq.~\eqref{rwaMarkovian} for the stationary state, $ \dot{\rho}=0 $, valid in the case of a continuous drive, to ultimately determine all of the $ \rho_{ij} $ and fully characterise the system.
  
  The atom emits a field:
  
  \[
  	V^{ji}(x,t) = i\frac{\hbar\Gamma_{ji}}{\phi_{ji}}\iaverage{\sigma_{ji}}e^{i\left(k_{ji}\iabs{x}-\omega_{ij}t\right)}.
  \]
  
  \noindent By monitoring this emission, we shall fund that it results in a peak in the emission power spectrum of:
  \[
  	S(\omega) = \frac{\hbar\omega\Gamma_{21}}{2}\iaverage{\sigma_{21}(\omega)}^2.
  \]
  
  \noindent The power or photon rate are then simply the voltage squared over the impedance of the tranmission line:
  
  
  \[
  	\text{Power} = \frac{V_{21}^2}{2Z} \quad\quad \text{Photon rate}, \nu^{21} = \frac{V_{21}^2}{2Z\hbar\omega}
  \]
  
  \ipicCaption{5cm}{Coherentwavemixingsinglethreelevelsystem}{The measured photon rate}.
  
  Sweeping both the driving and probing field detunings, we arrive at images such as this. Different features correspond to different relaxation rates:
  
  \ipicCaption{15cm}{Coherentwavemixingsinglethreelevelsystem2}{\textbf{Bottom Left - Top Right:} determined by dephasing $ \gamma_{23} $; \textbf{Horizontal:} $ \gamma_{13} $; \textbf{Vertical:} $ \gamma_{12} $.}
  
  \newpage
 