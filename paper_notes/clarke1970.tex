% -*- TeX-master: "../paper_notes.tex" -*-

\section{The Josephson effect $ e/h $ \cite{clarke1970}}
\iframe{ Explanation of how the Josephson effect comes about:
  \begin{itemize}
  \item There are two equations important for the Josephson junction:
    \begin{equation}\label{key}
      \begin{aligned}
        E & = \frac{\Phi_0}{2\pi}(1-cos(\Delta\phi))\\
        j & = j_c\sin(\Delta\phi).
      \end{aligned}
    \end{equation}
 		
  \item \textbf{\red{What we control is the current, $ j $:}}
 		
    \red{\begin{center} Increase $ j $ \ra Increase $ \delta\phi $ (to $ \pi/2 $) \ra decrease barrier
        coupling energy $ E $.
      \end{center}} Beyond the critical current, you generate a potential accross the junction - you
    are now either emitting or absorbing a photon with each tunneling event.
 	
    \ipic{3cm}{josephson_effecti}
 		
  \item To derive the above equations, we look at the conjugate variables \icommutation{n}{\phi} =
    -i. It is well known, that the time evolution of an operator $ \hat{n} $ is given by
    \begin{equation}\label{key}
      \begin{aligned}
        i\hbar\difffrac{\hat{n}}{t} & = \icommutation{\mathcal{H}}{\hat{n}}+ \grey{\difffrac{\hat{n}}{t}} \equiv i\difffrac{\mathcal{H}}{\phi} \\
        i\hbar\difffrac{\hat{\phi}}{t} & = \icommutation{\mathcal{H}}{\hat{\phi}}+ \grey{\difffrac{\hat{\phi}}{t}} \equiv -i\difffrac{\mathcal{H}}{\hat{n}} \\
      \end{aligned}
    \end{equation}
 		
    \noindent where we allow the commutator to operate on a system state to evalute the right-hand
    side.

  \item Taking the expectation values:
 		
    \begin{equation}\label{key}
      \begin{aligned}
        \red{j} & = 2e\difffrac{\iaverage{\hat{n}}}{t} = \frac{1}{\hbar}\difffrac{E}{\Delta\phi}\\
        \difffrac{\iaverage{\Delta\phi}}{t} & = \frac{1}{\hbar} \difffrac{E}{\hat{n}} \equiv \frac{2}{\hbar}\Delta\mu
      \end{aligned}
    \end{equation}
 		
    \noindent \red{\textbf{Where by definition $ \difffrac{E}{n} =2\Delta\mu$ for the transfer of a Cooper
        pair.}}
 		
  \item This above is so fundamental, that it does not depend on what type of junction is used. The
    rest of the paper talks about how this makes it a perfect quantum standard.
  \end{itemize}
 }