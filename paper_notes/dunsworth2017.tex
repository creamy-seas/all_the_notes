% -*- TeX-master: "../paper_notes.tex" -*- -*- TeX-master: "../paper_notes.tex" -*-

\section{Capacitive loss induced during JJ fabrication}
\label{sec:capac-loss-induc}

\begin{framed}\noindent
  The goal  is to  improve the time  $ T_{1} $  of fabricated  qubits.  They are  usually much
  shorter than the time  constant of the resonators they attach to, and  thus are the limiting
  factor.

  When making a JJ, you go through the process of
  \begin{enumerate}
  \item Rinse wafer in sonic acetone, and rinse in deionized water;
  \item Dip wafer in HF to remove oxide layer;
  \item Blow in nitrogen and deposit 100\,nm of Al;
  \item \red{Optical lithography}  to define resonators \ra develop resist  and develop Al not
    covered by resist with etching solution;
  \item \red{EBL lithography} for JJ \ira  \red{\textbf{before depositing ion mill the surface
        to etch away AlOx on resonator, in order to make good contact between it and the JJ}};
  \item This last process leaves behind a dielectric, which causes a lot of losses (microwaves
    loose energy in these regions)
  \end{enumerate}
\end{framed}
