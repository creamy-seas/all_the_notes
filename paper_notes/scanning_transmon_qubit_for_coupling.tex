\section{A scanning transmon qubit for strong coupling circuit quantum electrodynamics}
%  \begin{itemize}
  	 Excite voltage modes in a resonator, and scan a qubit above it. Using the standard qubit-resonator interaction as in Ch.~\ref{sec:CavityQunatumElectrodynamics}, the resonator energy level will split $ \propto g $, the coupling strength \red{as the qubit interacts with the resonator}
%  \end{itemize}

  \ipic{5cm}{cavityPic2}.
  
  The resonator spectrum is mapped out while chaning the position of the qubit, and the coupling strength oscillates as the qubit coupled to the nodes and antinodes of the resonator standing waves. We move in two directions - the x and y direction.
  
  \ipic{4cm}{scanningTransmon_1}
  \ipic{4cm}{scanningTransmon_2}
  
  \iframe{The coupling strength:
  	\[
  		g(x,y,z) = 2\sqrt{\frac{2Z_c}{h}}\red{\bigg[\sin(\pi\frac{\Delta x}{L})\bigg]}\blue{\bigg[\beta(y,z)\bigg]}\bigg[n_{01}(y,z,\nu_r)\bigg]
  	\]
  	
  	\noindent where we have \begin{itemize}
  		\item \red{Sinusoidal term, as the voltage standing wave has nodes and antinodes along its length};
  		\item \blue{Voltage drops across the transmon capacitors / voltage drop across resonator to ground place (fraction).}
  		\item The transition matrix for the qubit $ \iket{0}, \iket{§} $ at a certain frequency, found from the transmon hamiltonian:
  		
  		\[
  			4E_c\hat{n}^2 - E_j\cos(\phi),
  		\]
  		
  		\noindent with charging energy evalauted as $ e^2/2C_t $, and $ C_t $ is the capacitance of the two parrallel sets of capacitors:
  		
  		\ipic{4cm}{scan_cap}
  		
  		From the diagram
  		
  		\[
  			C_T = C_{AB} + \frac{1}{\frac{1}{C_{Ag}}+\frac{1}{C_{Bg}}} + \frac{1}{\frac{1}{C_{Ap}}+\frac{1}{C_{Bp}}}
  		\]
  	\end{itemize}
	}

\newpage