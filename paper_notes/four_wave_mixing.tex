% -*- TeX-master: "../paper_notes.tex" -*-

\newpage\section{Teresas thesis}
\subsection{Four-wave mixing}
From Olayas notes, when we subject  and atom to a wave with
detuning $ \delta\omega $, the Hamiltonian is
  
\begin{equation}
  \begin{aligned}
    H_I & = \hbar g(\vec{r})\bigg[\blue{\exp^{-i\omega_jt}\hat{a}}+\blue{\exp^{+i\omega_jt}\hat{a}^\dagger}\bigg]\bigg[\blue{\exp^{-i\omega_at}\ketbra{g}{e}}+\blue{\exp^{+i\omega_at}\ketbra{e}{g}}\bigg] \\
    & = \hbar g(\vec{r})\bigg(\mathbf{\exp\left[-i(\omega_j-\omega_a)t\right]\ \hat{a}\ketbra{e}{g}} \quad + \quad \mathbf{\exp\left[i(\omega_j-\omega_a)t\right]\ \hat{a}^\dagger\ketbra{g}{e}} \\
    &   \quad    +   \quad   \mathbf{\exp\left[-i(\omega_j+\omega_a)t\right]\
      \hat{a}\ketbra{g}{e}}          \quad         +          \quad
    \mathbf{\exp\left[i(\omega_j+\omega_a)t\right]\
      \hat{a}^\dagger\ketbra{e}{g}}\bigg)
  \end{aligned}
  \label{eqn:fullHamiltonianLightAtom}
\end{equation}
  
\noindent   Keeping  only   energy  conserving   terms  and
introducing operators:
  
  \begin{equation}\label{key}
    \mathcal{H} = \hbar g\bigg(a\idagger\sigma^{-} e^{i\delta\omega t} + a\sigma^{+} e^{-i\delta\omega t}\bigg).
  \end{equation}
  
  \noindent   Putting  into   the   unitary  operator   and
  simplifying, one can arrive at
  
  {\footnotesize\begin{equation}\label{eqn:teresa_1} U(t) =
      \sum_{n=0}^{\infty}\bigg[\red{\cos(\eta\sqrt{n})\ketbra{n}{n}s^{-}s^{+}}
      +     {\cos(\eta\sqrt{n+1})\ketbra{n}{n}s^{+}s^{-}}    -
      \red{\ketbra{n-1}{n}s^{+}\sin|(\eta\sqrt{n})}+\ketbra{n+1}{n}s^{-}\sin(\eta\sqrt{n+1})\bigg]
    \end{equation}}
  
  \noindent where $ \eta = gt $ and \iframe{
    \begin{equation}\label{key}
      \ialigned{
        s^{+} &= \sigma^{+}e^{-i\delta\omega t}\\
        s^{-} &= \sigma^{-}e^{+i\delta\omega t}\\
      }
    \end{equation}
  }

  \noindent For  the case  when we  have an  initial ground
  state, in a strong coherent photon field:
	
  \begin{equation}\label{key}
    \begin{aligned}
      &\ket{g}\\
      &\ket{\alpha}     =     e^{-\alpha^2/2}(\ket{0}     +
      \alpha\ket{1} + ...)
    \end{aligned}\ra \ket{\psi} = \ket{g,\alpha}
  \end{equation}
	
  \noindent       only        \red{red       terms       in
    Eq.~\eqref{eqn:teresa_1}}  terms  will  contribute,  as
  others will anhilate the \iket{g} state:
	
  \begin{equation}\label{key}
    \begin{aligned}
      \sum_n\cos(\eta\sqrt{n})\ketbra{n}{n}\red{\ket{\alpha}}s^{-}s^{+}\ket{g} &+ \sin|(\eta\sqrt{n})\ketbra{n-1}{n}\red{\ket{\alpha}}s^{+}\ket{g}\\
      &    \approx    \cos(\sqrt{n}\eta)\ket{g,\alpha}    -
      e^{-i\delta\omega t}\sin(\eta\sqrt{n})\ket{\alpha'}
    \end{aligned}
  \end{equation}
	
  \noindent    where    the     second    coherent    state
  $  \ket{\alpha'}  $ onlydiffers  by  the  absence of  the
  \iket{0} term. \red{As it is so small, we can approximate
    \iket{\alpha} = \iket{\alpha'}}.
	
  \subsection{The pulse pattern}
  \begin{enumerate}
  \item We have  an initial ground state  atom \iket{g} and
    coherent field \iket{\alpha}.
		
		\begin{equation}\label{key}
                  \ket{\psi} = \iket{g,\alpha}.
		\end{equation}
		
              \item  We  subject  it to  a  detuned  pulse,
                which, from above, will result in the state
		
		\begin{equation}\label{key}
                  \ket{\psi} = \cos(\frac{\theta}{2})\iket{g,\alpha} - e^{-i\delta\omega t}\sin(\frac{\theta}{2})\iket{e,\alpha'}.
		\end{equation}
		
              \item After switching  off the driving field,
                we get put the field state to 0, leaving
		
		\begin{equation}\label{eqn:teresa_3}
                  \ket{\psi} = \bigg[\cos(\frac{\theta}{2})\iket{g} - e^{-i\delta\omega t}\sin(\frac{\theta}{2})\iket{e}\bigg]\otimes \iket{0}.
		\end{equation}
		
		\noindent  \red{where  we have  managed  to
                  acquire a  phase based  on the  length of
                  the incident pulse.}
		
              \item  In order  to  perform  readout of  the
                atomic state, we need to convert it back to
                a photon state. This  is done by applying a
                $         \pi/2          $-pulse         of
                Eq.~\eqref{eqn:teresa_1}, for the case when
                $  \eta =  \pi/2 $.   Because there  are no
                photon    in    the   field    (\red{unlike
                  previously,  we did  not have  a coherent
                  photon  field  \iket{\alpha} set  up  }),
                only  $  \iket{n}   =  \iket{0}  $  remain,
                leading to a simplified
		
		\begin{equation}\label{key}
                  U_{ap} = \ketbra{0}{0}\sigma^{-}\sigma^{+} + e^{i\delta\omega t}\ketbra{1}{0}\sigma^{-}
		\end{equation}
		
		\noindent and total evolution to
		
		\begin{equation}\label{eqn:teresa_2}
                  \ket{\psi} = \ket{g}\otimes\bigg[\cos(\frac{\theta}{2}\ket{0} - \sin(\frac{\theta}{2})\ket{1})\bigg].
		\end{equation}
		
              \item   \iframe{Compating   the   expectation
                  values:
		
		\begin{equation}\label{eqn:teresa_4}
                  \ialigned{
                    Eq.~\eqref{eqn:teresa_2} & s^{+} = \sigma^{+}e^{-i\delta\omega t} & \iaverage{s^{+}} = -\frac{i}{2}\sin(\theta) \\
                    Eq.~\eqref{eqn:teresa_3} & b^{+} = \ketbra{1}{0} & \iaverage{b^{+}} = -\frac{1}{2}\sin(\theta) \\
                  }
		\end{equation}
		\noindent  meaning  that   we  can  convert
                atomic superposition to single photon field
                at any time by doing
		
		\begin{equation}\label{key}
                  s^{+} \ra ib^{+}\qquad s^{-} \ra -ib^{-}.
		\end{equation}
		\noindent The device converts
		
		\begin{equation}\label{key}
                  \iket{\alpha} \ra \iket{\beta} = cos(\frac{\theta}{2})\ket{0} - \sin(\frac{\theta}{2})\ket{1}
		\end{equation}
		
		\noindent and can be  though of as having a
                Hamiltonian
		
		\begin{equation}\label{key}
                  \mathcal{H} = i\hbar g\bigg[b^{+}a - b^{-}a\idagger\bigg].
		\end{equation}
              }
            \end{enumerate}

	\subsubsection{Single off resonant field}
        We  can automatically  use Eq.~\eqref{eqn:teresa_4}
        to evaluate  \iaverage{s^{+}} for the case  when we
        use  a classical  diriving  field. In  such a  case
        $  \theta =  gt/2  \equiv \Omega  t$  will vary  as
        during          usual           driving          in
        Eq.~\eqref{eqn:drivingExample}
	 
	 \begin{equation}\label{key}
           \theta = gt e^{i\delta\omega t} + gte^{-i\delta\omega t}
	 \end{equation}
	 
	 \noindent   which   eventually    leads   into   a
         composition  of   Bessel  function   at  different
         frequencies
         $        (2k+1)\delta\omega       t        \propto
         J_{2k+1}\cdots.$. However  this will  only explain
         classical  wave mixing,  where there  is an  equal
         amount of components either side of the drive. See
         the figure below.

	\subsection{Two off resonant fields PULSED OPERATION}
        With two fields, we  need to redo above procedures,
        but  with  the  Hamiltonian for  2  detuned  fields
        ($ s $ replaced by $ b $ operators):
	 
	 \begin{equation}\label{key}
           \mathcal{H} = i\hbar g(b^{+}_{-}a_{-} - b^{-}_{-}a\idagger_{-} + b_{+}^{+}a_{+} - b_{+}^{-}a\idagger_{+}).
	 \end{equation}

         \noindent \iframe{This time  when deriving unitary
           evolution  to get  the state  of the  system, we
           shall be getting mushed up terms of the form
	
	\begin{equation}\label{key}
          b_{m}^{+}b_{j}^{-}b_{p}^{+}a_{m}\idagger a_{j}a_{p}^{+},\qquad m,j,p = \pm 1.
	\end{equation}	
	
	\noindent  To interpret  them,  recall that  $ b  $
        operators follow the same as $ s $ operators:
	
	\begin{equation}\label{key}
          b_{m}^{+}b_{j}^{-}b_{p}^{+} \equiv s_{m}^{+}s_{j}^{-}s_{p}^{+} = \sigma^{+}\sigma^{-}\sigma^{+}e^{-i(m-j+p)\delta\omega t} = \sigma^{+}\red{e^{-i(m-j+p)\delta\omega t}}.
	\end{equation}
	
	
	\noindent  After   solving  the   master  equation,
        finding         the          output         voltage
        $  V\propto \iaverage{b}  $, the  \red{red exponent
          term  provides a  non-zero contribution}  for the
        $   (m    -   j    +   p)\delta\omega    t   \equiv
        (2k+1)\delta\omega t$ order bessel function.
	
	\begin{equation}\label{key}
          \iaverage{b^{+}_{2k+1}} \propto J_{2k+1}(2\theta)\iaverage{\text{$ a\idagger_{-}a^{+}\cdots $}}
	\end{equation}}
	 
      \iframe{      Different     series      of     photon
        creation/anhialation   processes  will   result  in
        different Bessel oscillations in $ \theta $.  These
        Bessel functions  will be emitted at  the different
        $ \omega_0  + (2k+1)\delta\omega $  frequencies.  }
      The voltage at that frequency:
	
	\begin{equation}\label{eqn:teresa_5}
          V_{2k+1}=\frac{\hbar\Gamma_1}{q_p}\iaverage{b_{2k+1}}
	\end{equation}
	
	\noindent will thus depend  on the number of photon
        processes that create this frequency.
	
	\ipicCaption{6cm}{sidePeak1}{Here     we    monitor
          voltages for different toration angle}
	
 \subsubsection{Driving power $ \Omega $. STEADY STATE}
 The    amplitude     of    the    side     component    in
 Eq.~\eqref{eqn:teresa_5} is a  function of the interaction
 time $ \theta $ and number of interacting photons.
 \[
   V_{2k+1}(\theta(t),
   k)=\frac{\hbar\Gamma_1}{q_p}\iaverage{b_{2k+1}}.
 \]
  
 \noindent However, to find the  dependance on the power of
 the driving, we need to take a slightly different approach
 for spectral decomposition. Here we
  
 \begin{enumerate}
 \item Do not care about the number of photon processes;
 \item Allow the amplitude of the drive to vary.
 \end{enumerate}

 Starting with the system driven by two waves:
  
  \begin{equation}\label{key}
    \mathcal{H} = -\frac{\hbar\omega_0}{2}\sigma_z - \bigg[\hbar\Omega_-\cos(\omega t  - \delta\omega t) + \hbar\Omega_+\cos(\omega t  + \delta\omega t)\bigg]\sigma_x.
  \end{equation}
  
  \noindent Entering  the rotating  frame, and  solving the
  Master equation,  will give us \isigmaminus,  which leads
  to          a         voltage          emitted         at
  $     V      =     i\frac{\hbar\Gamma_1}{q_0}\isigmaminus
  $. Separating the voltage  out into terms with individual
  $ \cos(\omega_0  t + (2k+1)\delta\omega t)  $ components,
  we get the intensity at the different frequencies:
  
  \begin{equation}\label{key}
    \text{Power} \propto \iabsSquared{V_{2k+1}}(\Omega_{\pm}) 
  \end{equation}
  
  \noindent which is experimentally measured by chaning the
  intensity of the driving:
  
  \ipicCaption{6cm}{sidePeak2}{Chaning  intensity   of  the
    drives, changes one of the frequency components.}
  
  \iframe{\red{Let us point out  the important different in
      the two  derivations. Right  now, we  just considered
      the strength  of the  driving, by solving  the Master
      equeation  and  identiying  different  compoents  for
      \textbf{a steady state driving}}
  
    \blue{But previously, we were considering pulses, which
      determined the  size of $  \theta $, and lead  to the
      Bessel oscillations.}  }

 \subsubsection{Single photon processes}
 \red{\textbf{For  the first  time, we  shall consider  the
     case when we have time separated pulses (up to now, we
     had  pulses arriving  at the  same time,  and creating
     even  amounts   of  symmetrical   components).}}   The
 important  operators are  the $  b^{\pm}_{m} $  operators,
 which create/anhialate ($ \pm $) the zero-one photon field
 at frequecy $ \omega_0 + m\delta\omega $. There are strict
 relations
  
  \begin{equation}\label{key}
    \ialigned{
      b^{-}_{p}b^{+}_{m} &= \ket{0}_{m-p}\bra{0} & \text{0 photon field is converted to 0 field at a different frequency}\\
      b^{\pm}b^{\pm} &= 0 & \text{cannot excite atom twice}.
    }
  \end{equation}
  
  \iframe{ As  a reminder,  the field created  at frequency
    $    \omega_0+(m-j+p)\delta\omega   $    comes    around   from    the
    $ b $ part of terms in the unitary expansion
  
  \begin{equation}\label{key}
    b_{m}^{+}b_{j}^{-}b_{p}^{+}a_{m}\idagger a_{j}a_{p}\idagger,
  \end{equation}
  
  \noindent  Evaluating  $  V\propto \iaverage{b^{+}}  $  of  the
  system state that follows this evolution, and decomposing
  into Bessel  functions. There  will be a  non-zero Bessel
  function                                               at
  $ (2k+1)\delta\omega + \omega_0 =  (m-j+p)\delta\omega + \omega_0$ associated with this
  term.
  
  Generally, the $ a $ terms will be of the form
  
  \begin{equation}\label{key}
    (a_+a_-\idagger)^ka_+,
  \end{equation}
  
  \noindent  absorbing $  k+1  $  of $  \omega_+  $ photons  and
  emitting  $  k $  of  $  \omega_-  $  photons.  The  -/+  will
  alternate, from the way that the expansion writes out.  }

Sending in one photon at frequency $ \omega_- $ prepares a state
(as in Eq.\eqref{eqn:teresa_2}) of
\[
  \ket{\beta}_- = B\bigg(\ket{0} + \beta\ket{1}\bigg).
\]
  
\noindent  \red{The  $ \omega_-  $  photon!}  Then, a  pulse  at
$ \omega_+ $  is sent in, and  since only one $ \omega_-  $ photon is
kept   in   the   atom,   the   result   is   emission   at
$ 2\omega_+  - \omega- $  from the  only $ \omega_-  $ term in  th unitary
expansion:
  
  \[
    a_+a\idagger_-a_+
  \]
  
  \ipicCaption{18cm}{sideBadn_3}{Preparing the  atom in one
    state, will mean that incoming  photons at $ \omega_+ $ will
    interact     with     only      one     $     \omega_-     $
    photon. \red{\textbf{There will only be 1 peak near the
        original frequency $ \omega_0 $}.}}


\newpage