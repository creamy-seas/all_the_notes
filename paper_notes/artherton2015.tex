% -*- TeX-master: "../paper_notes.tex" -*-

\newpage \section{Cheshire Cat  States} \iframe{It was shown that there  was nothing `quantum'
  about  the statistics  of light  measured in  two beams  of an  interferometer. A  classical
  explanation would explain the changes in phase and interference.
 
\begin{center}
  \red{\textbf{There is no sign  of light being a quantum object -  i.e. having a wavefunction
      and  collapsing upon  point of  measurements. In  fact, the  undergraduate test  for the
      Mach-Zender interferometer can also be interpreted classically}}
\end{center}}
\noindent  The overarching  idea  is that  for  polarised photons  sent into  two  arms of  an
interferometer:
\begin{itemize}
\item A measurement of \textit{polarisation} in 1 arm, predefines the state in the other;
\item A measurement of \textit{existence} in arm-2, predefines the state in the other.
\end{itemize}
 
\ipic{6cm}{artherton2015}
  
\begin{enumerate}
\item A Wollaston prism sends  $ H $-polarised photons into $ L $  and $ V $-polarised photons
  into  $ R  $.  \grey{(The  half wave  plate balances  out the  powers of  $ H  $ and  $ V  $
    polarisation in the original beam)}:
 	
 	\begin{equation}\label{chesirecat1}
          \ket{\Psi}_i = \frac{1}{\sqrt{2}}\kbordermatrix{&\\ 
            \Ket{H,L} & 1 \\
            \Ket{H,R} & 0 \\
            \Ket{V,L} & 0 \\
            \Ket{V,R} & 1 
          }.
 	\end{equation}
      \item The non-polarising beam splitter (NPBS) acts  only on the subspace of \iket{L} and
        \iket{R} (the spatial modes), casting into the \iket{1}, \iket{2} subspace
 	
 	\begin{equation}\label{chesirecat2}
          \hat{BS} = \frac{1}{\sqrt{2}}\kbordermatrix{
            & \iket{L} & \iket{R}\\
            \bra{1} & 1 &  -1\\
            \bra{2} & 1 & 1
          }
          \iratext{in full subpsace $ \otimes\mathbb{I}_{H,V} $}
          \frac{1}{\sqrt{2}}\kbordermatrix{
            & \iket{H, L} & \iket{H, R} & \iket{V, L} & \iket{V, R}\\
            \bra{H, 1} & 1 & 0 & -1 & 0\\
            \bra{H, 2} & 0 & 1 & 0 & -1 \\
            \bra{V, 1} & 1 & 0 & 1 & 0 \\
            \bra{V, 2} & 0 & 1 & 0 & 1 \\
          }
 	\end{equation}
 	
      \item Thus at the output the state after NPBS is (unnormalised)
 	\begin{equation}\label{chesirecat3}
          \ket{\Psi}_F = \hat{BS}\ket{\Psi_i} = \kbordermatrix{
            & \\
            \ket{H, 1} & 1 \\
            \ket{H, 2} & -1 \\
            \ket{V, 1} & 1 \\
            \ket{V, 2} & 1
          }
 	\end{equation}
 	
      \item A \textbf{Horizontal }polariszer before detector  1, will `trace out' any \iket{V}
        components:
 	\begin{equation}\label{chesirecat4}
          \hat{PH}\ket{\Psi}_F = \left[\ketbra{H}{H}\right]\ket{\Psi}_F = \frac{1}{\sqrt{2}}\ket{H}(\iket{1} + \iket{2}).
 	\end{equation}
      \end{enumerate}

 
      \newpage
