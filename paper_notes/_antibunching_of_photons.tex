% -*- TeX-master: "../paper_notes.tex" -*-

\section{Antibunching of microwave frequency photons observed incorrelation measurements using
  linear detectors\label{sec:antibunching}}
\begin{enumerate}
\item Transmon qubit has energy of \iunit{8.0}{GHz}.
\item Apply a pulse with phase $ \phi $ and duration $ \Omega t = \theta $ to get a superposition:
 	
 	\[
          \ket{\psi} = \cos(\theta/2)\idown+\sin(\theta/2)\iup e^{i\phi}.
 	\]
      \item  The  transmon is  placed  near  a resonator  with  \iunit{6.7}{GHz}  and the  two
        \red{\textbf{are   momenterally   brought   into   resonance  by   applying   a   bias
            flux}}. According to resonator-qubit  interaction (Jaynes cummin Hamiltonian), the
        Hamiltonian for the system:
 	\[\begin{aligned}
            \mathcal{H}_{\text{middle}}
            & = \kbordermatrix{&\ket{\uparrow,N} & \ket{\downarrow,N+1}\\
              \bra{\uparrow,N} & \blue{\hbar\omega_r(N+\frac{1}{2}) + \hbar\Delta} & \red{g_0\sqrt{N+1}}\\
              \bra{\downarrow,N+1}  &   \red{g_0\sqrt{N+1}}  &   \blue{\hbar\omega_r(N+\frac{1}{2})  -   \hbar\Delta}}  \quad
            \text{E}_{\pm} = \hbar\omega_r(N+\frac{1}{2}) \pm \frac{E_\text{coupled}}{2}
          \end{aligned}
 	\]
 	
 	\noindent so at resonance, we have unitary evolution:
 	\[
          \begin{aligned}
            U(t) & = \exp\big[-i(N+\frac{1}{2})\omega_r\mathbb{I} - i\frac{g_0\sqrt{N+1}}{\hbar}\sigma_x\big]\\
            &     =    e^{i\phi_\text{const}}\bigg[\cos(\frac{g_0\sqrt{N+1}}{\hbar}t)\mathbb{I}     +
            i\sin(\frac{g_0\sqrt{N+1}}{\hbar}t)\sigma_x\bigg]
          \end{aligned}
 	\]
 	
 	\noindent so  the state will evolve  (top row is  \iket{\uparrow, N}, bottom row  is \iket{\downarrow,
          N+1})
 	\[
          U(t) \ket{\psi} = e^{i\phi_\text{const}}\imatrix{\cos(\alpha t)}{i\sin(\alpha t)}{i\sin(\alpha t)}{\cos(\alpha
            t)}\imatrixcol{\cos(\theta/2)}{e^{i\phi}\sin(\theta/2)}.
 	\]
 	
 	By choosing a time  such that $ \alpha t =  \pi $ we get the state  transferred to the cavity
        (resonator) which we can read out.
 	\[
          \begin{aligned}
            U(\pi/\alpha)\ket{\psi} = \cos(\theta/2)\iket{\uparrow, N} & + e^{i\phi}\sin(\theta/2)\iket{\downarrow, N+1} \ira (N = 0)\\
            \red{\ket{\text{outut field}}} &= \red{\cos(\theta/2)\ket{0} + \sin(\theta/2)e^{i\phi}\ket{1}}
          \end{aligned}
 	\]
 	
 	\red{Now the cavity, \iket{0}, \iket{1}, is storing the state of the qubit, so that we
          can read it out!}
      \end{enumerate}
      \iframe{The state that we are reading out from the cavity is
 	\begin{equation}\label{eqn:bunching_1}
          \ket{\psi} =  \imatrixcol{\cos(\theta/2)}{e^{i\phi}\sin(\theta/2)} \Rightarrow   				\rho = \frac{1}{2}\imatrix{1+{\cos(\theta)}{}}{e^{-i\phi}\sin(\theta)}{e^{i\phi}\sin(\theta)}{1-\cos(\theta)}
 	\end{equation}
        % \cos\left(\frac{\theta}{2}\right)\ket{0} +
        % e^{i\phi}\sin\left(\frac{\theta}{2}\right)\ket{1} =
      }

      \subsection{Measurements}
      \begin{enumerate}
      \item Field emitted from source is split \hfill $ E_{e,f}(t) $;
      \item Amplified\hfill $ E_{e,f}^{(+)}(t) $;
      \item Noise added \hfill $ E_{e,f}^{(+)}(t) + N_{e,f}(t) $;
      \item At this point, the `useful' signal' is  mixed in with the freqeucny of the qubit -
        \red{we want to extract this envelope by hetrodyning}:
 	\begin{equation}\label{key}
          E_{e,f}^{(+)}(t) + N_{e,f}(t) \ira \red{S_{e,f}(t)}e^{-i\omega t} \xrightarrow{hetrodyne} \red{S_{e,f}(t)}
 	\end{equation}
 	\begin{figure}[h]
          \begin{center}
            \includegraphics[height=5cm]{antibunch_1}
          \end{center}
          \caption{Once we have extracted the envelope, its  phsae can either be fixed or jump
            around. \label{fig:bunching_1}}
 	\end{figure}
      \item As seen from the phase image above, \red{the complex envelope $ S_{e,f}(t) $} will
        have real and imaginary components (quadratures).
      \end{enumerate}
      So 4 combinations of outputs could be measured
      \begin{table}[h]
        \centering
        \begin{tabular}{|c|c|}
          \hline $ \re{S_e(t)} $ & $ \im{S_e(t)} $\\
          $ \re{S_f(t)} $ & $ \im{S_f(t)} $\\\hline
        \end{tabular}
      \end{table}
  
      % \begin{equation}\label{key}
      %  	\text{Signal} \propto \iaverage{\hat{a}(t)}
      % \end{equation}
  
 \subsection{Direct amplitude measurements}
 \red{As will be shown  in another paper}, when we measure one of  the quadratures, in any one
 of the arms
  
  \begin{equation}\label{key}
    \re{S(t)} \propto \iaverage{\hat{a}(t)}
  \end{equation}
  
  \noindent \red{\textbf{we will  be probing the expectation values of  the cavity anhialation
      operator.}} The way to think about it is, that in our \iket{0},\iket{1} basis

  \red{equation commented out}
  % \begin{equation}\label{eqn:bunching_4}
  %   \ialigned{
  %     \hat{a} & = \ketbra{0}{1}\\
  %     \iaverage{\hat{a}} & = \trace\lbrace\hat{a}\rho\rbrace\\
  %     \rho &= \frac{1}{2}\imatrix{1+{\cos(\theta)}{}}{e^{-i\phi}\sin(\theta)}{e^{i\phi}\sin(\theta)}{1-\cos(\theta)}
  %   } \ira \red{Re{S(t)} \propto \iaverage{\hat{a}} = \frac{1}{2}\cos(\phi)\sin(\theta)}.
  % \end{equation}
	
  \iframe{\textbf{So measuring the quadratue of the field} we are are probing
    \begin{equation}\label{key}
      \re{S(t)} \propto \iaverage{\hat{a}} = \frac{1}{2}\cos(\phi)\sin(\theta)
    \end{equation}
    In the  measurements we  see a  direct mapping of  this pattern,  as we  prepare different
    `amounts' of the superposed state (\red{corresponding  to the off diagonal elements of the
      density matrix}.)  \ipic{8cm}{anti_coherent}
    \begin{itemize}
    \item We  see the $ \sin(\theta)  $ dependance.  Essentially, when  we rotate the state  of the
      qubit,  the phase  $  \theta  $ of  the  qubit changes.   At  polar extrema  ($  \theta  = 0,  \pi$,
      corresponding to  \iket{0},\iket{1}) there is no  phase information \red{as there  is no
        projection onto the equatorial plane!} This  results in no quadrature signal, as every
      single time we perform the measurement, the phase of the signal will jump randomly.
    \item The decay we see, is associated with the decay time of the cavity mode
      \begin{equation}\label{key}
        T = \frac{Q}{\omega}.
      \end{equation} 
      \noindent Once  the state  Eq.\eqref{eqn:bunching_1} is  created, it  wil remain  in the
      cavity and slowly leak out. The `off diagonal' terms of the state will dephase, and thus
      the signal will exponentially decay.
    \end{itemize}
  }
  \newpage \subsection{Full power measurements} In the  above section we measured the coherent
  part of the signal, by measuring direct voltage $ \re{S(t)} $.
  
  What if we  now measure power, by taking the  absolute square of the signal? We  can use the
  input-output theory approach to  evaluate $ \iaverage{V^2(\omega)} $ and from  it the total power
  $ 1/Z\int\iaverage{V^2(\omega)}d\omega $ (done in the March summary report) to get
  
  \begin{equation}\label{eqn:dynamics1}
    \ialigned{\text{Total power} &= \hbar\omega\Gamma_1  \frac{1-{\isigmaz}_{}}{2}\\
      \text{Incoherent power} &= \hbar\omega\Gamma_1  \bigg[\frac{1-{\isigmaz}_{}}{2}-\isigmaplus\isigmaminus\bigg]
    }
  \end{equation}
  
  \noindent \iframe{And  getting the  absolute squared  value of this  signal will  this total
    power emitted from the qubit out for us
    \begin{equation}\label{eqn:bunching_2}
      \iaverage{S(t)^{*}S(t)} \propto \frac{1-\isigmaz}{2} = \frac{1 - \cos(\theta)}{2}.
    \end{equation}}

  \red{As  we  shall  see,  it  very  much resembles  what  is  measured  during  cross  power
    measurements.}
  
  \subsection{Cross power - \textit{photon number in the line}}
  To  evaluate cross  power, we  mutliple out  the  signals in  the two  arms (taking  complex
  conjugate because  that is the  way the dot product  between operators is  defined.) \red{As
    shall be shown in another paper}
  
  \begin{equation}\label{key}
    \iaverage{S_e^{*}(t) S_{f}(t)} \propto \iaverage{a^{\idagger}(t)a(t)}
  \end{equation}
  
  \noindent Now, operator $ a\idagger a = \hat{N} $ is the photon number operator which gives

  \begin{equation}\label{key}
    \ialigned{
      &\text{Eq.~\eqref{eqn:bunching_1}}\qquad                \rho                 =
      \frac{1}{2}\imatrix{1+\cos(\theta)}{e^{-i\phi}\sin(\theta)}{e^{i\phi}\sin(\theta)}{1-\cos(\theta)} \\
      & \iaverage{\hat{O}} = \trace{a}\\%\hat{O}\rho
      & \hat{N} = 0\ketbra{0}{0} + 1\ketbra{1}{1}
    } \ira \iaverage{\hat{a}^{\dagger}\hat{a}} = \frac{1-\cos(\theta)}{2} = \sin^2(\theta/2).
  \end{equation}
  
  This is  the signal  observed in  the graphs  \iframe{ \textbf{The  cross power  between the
      channels gives the photon number \iket{1} in the line}
    \begin{equation}\label{eqn:bunch_6}
      \iaverage{S_e^{*}(t) S_{f}(t)} \propto \iaverage{\hat{a}\idagger\hat{a}} = \sin^2(\theta/2)
    \end{equation}
    \ipic{8cm}{anti_incoherent}
    \begin{itemize}
    \item We see that  $ \sin^2(\theta/2) $ dependance, where it depends on  how excited we preapre
      the atom (peak is at $ \pi $);
    \item We see the same cavity decay time as for direct amplitude measurements.
    \end{itemize}
    \red{\underline{How  in  the  world  is  this   different  from  the  signal  we  have  in
        Eq.~\eqref{eqn:bunching_2}}? Well, it is to do with noise added by the amplifiers.
  	
  	\begin{equation}\label{key}
          \ialigned{
            \iaverage{S_e^{*}(t) S_{e}(t)} & \text{Noise of amplifier}, N_e(t)\\
            \iaverage{S_e^{*}(t) S_{f}(t)} & \text{Cross-noise of the noise in the two different amplifiers}, N_{e,f}(t)
          }
  	\end{equation}
  	\noindent     They    found     that     the    $     N_e(t)     \equiv    10.6\,$K     and
        $ N_{e,f}(t) \equiv \iunit{80}{mK}$, so there was much less noise being added!  } }
    % \begin{figure}
    %   \begin{center}
    %     \input{testLatex}
    %   \end{center}
    % \end{figure}
 
 \subsection{Cross correlation}
 Now measures signals with a time shift of $ \tau $ and integrates over the whole signal:
  
  \begin{equation}\label{eqn:bunching3}
    \Gamma^{(1)} = \int\iaverage{S_{e}^{*}(t)S_{f}(t+\tau)}dt - \Gamma^{(1)}_{ss} \propto \red{\int\iaverage{a\idagger(t)a(t+\tau)}dt},
  \end{equation}
  \noindent where we remove background correlation  $ \Gamma^{(1)}_{ss} $ which is measured between
  the photon pulses.
  
  \begin{itemize}
  \item \iket{0}: There are no photons in the ouput line, so no correlation signal;
  \item \iket{1}:  Photons emitted at  different times will  have absolutely NO  coherence, as
    there is no projection onto  the equatorial plane. Reffering to Fig.~\ref{fig:bunching_1},
    we say that  the phase is not defined, and  so averaging over all events will  result in a
    cancelation effect.
  \item \isuperposition{+}: Each  time photons will be  emitted with the same  phase, and thus
    multipling
    \begin{equation}\label{key}
      S_{e}^{*}(t)S_{f}(t+\tau),
    \end{equation}
    \noindent will accumulate a non zero value.
    \begin{itemize}
    \item At $ t = 0 $, we have the cross power result $ \iaverage{a\idagger a} $, that we had
      in Eq.~\eqref{eqn:bunch_6}, and we see this oscillation in $ \sin^2(\theta/2) $;
    \item At $  t \neq 0$, both $ S_e^{*}(t)  $ and $ S_{f}(t+\tau) $ will  have a repetitive value,
      both in phase and amplitude. This means that in Eq.~\eqref{eqn:bunching3}
      \begin{equation}\label{}
        \int\iaverage{S_{e}^{*}(t)S_{f}(t+\tau)}dt \ra \int \iaverage{S_e^{*}(t)}\iaverage{S_{f}(t+\tau)}dt,
      \end{equation}
  	
      \noindent which means that from Eq.~\eqref{eqn:bunching_4}
  	
  	\begin{equation}\label{key}
          \red{\re{S(t)} \propto \iaverage{\hat{a}} = \frac{1}{2}\cos(\phi)\sin(\theta)},
  	\end{equation}
  	
  	\noindent the signal will be $ \red{\propto \frac{1}{4}\sin^2(\theta)} $.
  	
      \end{itemize}
    \end{itemize}

    \ipicCaption{6cm}{bunch_2}{In the  superposed state,  the periodic  photons will  have the
      same   defined  phase,   and  thus   will  not   cancel  out,   when  we   multiply  out
      $ S_e(t)^{*}S_f(t+\tau) $ many times.\label{fig:bunch4}}  \iframe{\red{Just to repeat - the
        beam splitter does  NOT split the individual photon, but  splits its wavefunction into
        arms $ e  $ and $ f $.  The  correlation function is probing the direct  signal in the
        two arms}}.
	
 \subsection{The $ \Gamma^{(2)} $ function}
 Finally, the experiment was extended to measure
  
  \begin{equation}\label{key}
    \Gamma^{(2)}(\tau) = \int \iaverage{S_{e}^{*}(t)S_{f}(t)S_{e}^{*}(t+\tau)S_{f}^{*}(t+\tau)}dt,
  \end{equation}
  
  \noindent known  as the  second order  correlation function. Essentially  it is  probing the
  correlation between the power in the two arms of the system:
  
  \begin{itemize}
  \item At  $ \tau  = 0  $ we have  $ \iabsSquared{S_{e}(t)}\iabsSquared{S_{f}(t)}  $, so  we are
    measuring the average product  of the powers in the two arms of  the interferometer. For a
    single photon  source the  photon cannot split  into both  arms at the  same time,  so the
    average will be zero;
  \item  At $  \tau  \neq 0$,  we can  be  comparing the  powers  shifted by  the photon  repitition
    frequency, so this will not be zero.
  	
    \ipicCaption{8cm}{bunching_4}{We see a dip  at $ \tau = 0 $, since a  single photon cannot be
      in to arms at the same time - on average  one will be zero, and the other will be large.
      We also  notice the  $ \propto  \sin^4(\theta/2)$, since we  are preparing  the \iket{1}  states as
      Fig.~\ref{fig:bunch4}.}  \red{How is this different from the above measurement, where we
      also measured the signal  in both of the arms. Here we simply  squared it. Then why does
      $ \tau = 0 $ suddenly vanish?}
  \end{itemize}
%%%%%%%%%%%%%%%%%%%%%%%%%%%%%%%%%%%%%%%%%%%%%%%%%

\newpage