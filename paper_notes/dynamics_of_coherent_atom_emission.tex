% -*- TeX-master: "../paper_notes.tex" -*-

\section{Dynamics of coherent and incoherent Emission from an Artificial Atom in 1D space}
\begin{center}
  \textit{The focus lies  on the dynamics of the  Bloch sphere and how we  can represent state
    evolution}      \red{\textbf{Note     that      in     the      paper     they      define
      $ \isigmaz = \rho_{11} - \rho{00} $, giving changes in ngative signs.}}
\end{center}

 \subsection{Introduction}
 The flux  qubit atom  (Ch.~\ref{sec:resonanceFluorescence}) is irradiated  with a  weak field
 corresponding to a single photon. The atom emits:
  
 \begin{itemize}
 \item Total radiation \hfill $ \propto \frac{1-\isigmaz}{2} $ \red{i.e. the atomic population};
 \item Incoherent radiation \hfill $ \propto \isigmaplusminus $;
 \item Coherent radiation\hfill $ \propto \frac{1-\isigmaz}{2} - \isigmaplusminus$;
 \end{itemize}

 \iframe{All three  can be probed  by measuring emission at  the transition frequency  used to
   excite the atom i.e. \textbf{picking up the coherent radiation on the equator}.}
  
 \subsection{Setting the problem out}
 We construct the Hamiltonian for the full system

  \begin{equation}
    \hat{H} = \hbar\omega_a\ketbra{e}{e} + \hbar\omega_j\left(\hat{a}^\dagger\hat{a}+\frac{1}{2}\right) + {\hbar \Omega}\left(\hat{a}+\hat{a}^\dagger\right)\left(\ketbra{g}{e}+\ketbra{e}{g}\right),
  \end{equation}
  
  \noindent which we split up into two parts, to use the interaction picture

  \begin{equation}
    \left\lbrace 
      \begin{aligned}
        H & = H_0+\red{V}\\
        H_0 & = \hbar\omega_a\ketbra{e}{e} + \hbar\omega_j\left(\hat{a}^\dagger\hat{a}+\frac{1}{2}\right)\\
        \textcolor{red}{V}              &              \red{=}              \textcolor{red}{{\hbar
            \Omega}\left(\hat{a}+\hat{a}^\dagger\right)\left(\ketbra{g}{e}+\ketbra{e}{g}\right)}
      \end{aligned}\right.
    \label{eqn:lightAtomHamiltonians}
  \end{equation}
  
  \noindent We compute the interaction Hamiltonian
  
  \begin{equation}
    \left\lbrace
      \begin{aligned}
        H_I & = U_0^\dagger V U_0 \\
        U_0                                         &                                        =
        \exp\left[-i\omega_j\left(\hat{a}^\dagger\hat{a}+{1}/{2}\right)t\right]\exp\left[-i\omega_a\ketbra{e}{e}t\right]
      \end{aligned}\right. \Rightarrow
  \end{equation}
  
  \begin{multline}
    \Rightarrow \textcolor{red}{\hbar \Omega} \times \\
    \exp\left[+i\omega_j\left(\hat{a}^\dagger\hat{a}+{1}/{2}\right)t\right]\textcolor{red}{\left(\hat{a}+\hat{a}^\dagger\right)}\exp\left[-i\omega_j\left(\hat{a}^\dagger\hat{a}+{1}/{2}\right)t\right] \times\\
    \exp\left[+i\omega_a\ketbra{e}{e}t\right]\textcolor{red}{\left(\ketbra{g}{e}+\ketbra{e}{g}\right)}\exp\left[-i\omega_a\ketbra{e}{e}t\right]
  \end{multline}
  
  \noindent Then we proceed term by term
  \begin{itemize}
  \item
    $\exp\left[+i\omega_j\left(\hat{a}^\dagger\hat{a}+{1}/{2}\right)t\right]\textcolor{red}{\hat{a}}\exp\left[-i\omega_j\left(\hat{a}^\dagger\hat{a}+{1}/{2}\right)t\right]
    =
    \exp\left[+i\omega_j\hat{a}^\dagger\hat{a}t\right]\textcolor{red}{\hat{a}}\exp\left[-i\omega_j\hat{a}^\dagger\hat{a}t\right]$
  	
    The lowering operator \textcolor{red}{
      \begin{equation}
        \begin{aligned}
          \hat{a} & =
          \begin{bmatrix}
            \bra{0}\hat{a}\ket{0} & \bra{0}\hat{a}\ket{1} & \cdots\\
            \bra{1}\hat{a}\ket{0} & \bra{1}\hat{a}\ket{1} & \cdots\\
            \bra{2}\hat{a}\ket{0} & \bra{2}\hat{a}\ket{1} & \cdots\\
            \vdots & \vdots & \ddots\\
          \end{bmatrix}
          = \begin{bmatrix}
            \bra{0}0 & \bra{0}\sqrt{1}\ket{0} & \cdots\\
            \bra{1}0 & \bra{1}\sqrt{1}\ket{0} & \cdots\\
            \bra{2}0 & \bra{2}\sqrt{1}\ket{0} & \cdots\\
            \vdots & \vdots & \ddots\\
          \end{bmatrix}
          = \begin{bmatrix}
            0 & \sqrt{1} & \cdots\\
            0 & 0 & \cdots\\
            0 & 0 & \cdots\\
            \vdots & \vdots & \ddots\\
          \end{bmatrix}\\
          & = \sum_n\sqrt{n}\ketbra{n-1}{n}
        \end{aligned}
      \end{equation}}
  	
    \noindent and thus evaluating
  	
    \begin{equation}
      \begin{aligned}
  	e^{+i\omega_jt\hat{a}^\dagger\hat a} & \left[\sum_n\sqrt{n}\ketbra{n-1}{n}\right]e^{+i\omega_jt\hat{a}^\dagger\hat a}\\
  	& = \sum_n \sqrt{n}e^{+i\omega_jt\hat{a}^\dagger\hat a}\ket{n-1}\bra{n}e^{-i\omega_jt\hat{a}^\dagger\hat a} \qquad \leftarrow\text{ expand exponential operator}\\
  	& = \sum_n \sqrt{n}e^{+i\omega_j(n-1)t}\ket{n-1}\bra{n}e^{-i\omega_jnt}\\
  	& = e^{+i\omega_j(n-1)t}\left[\sum_n \sqrt{n}\ket{n-1}\bra{n}\right]e^{-i\omega_jnt}\\
  	&     =    e^{+i\omega_j(n-1)t}\hat{a}e^{-i\omega_jnt}\\     &    \blue{\mathbf{\equiv
            \text{\textbf{exp}}[-i\omega_jt]\hat{a}}}
      \end{aligned}
    \end{equation}
  	
  \item
    $\exp\left[+i\omega_j\left(\hat{a}^\dagger\hat{a}+{1}/{2}\right)t\right]\textcolor{red}{\hat{a}^\dagger}\exp\left[-i\omega_j\left(\hat{a}^\dagger\hat{a}+{1}/{2}\right)t\right]
    = \blue{\mathbf{\text{\textbf{exp}}[+i\omega_jt]\hat{a}^\dagger}}$
  	
  \item \begin{equation}
      \begin{aligned}
  	\exp\left[+i\omega_a\ketbra{e}{e}t\right]&\textcolor{red}{\ketbra{g}{e}}\exp\left[-i\omega_a\ketbra{e}{e}t\right] \\
  	& = \bigg[\exp\left[+i\omega_at\left(\mathbb{I} - \ketbra{g}{g}\right)\right]\textcolor{red}{\ket{g}}\bigg]\bigg[\textcolor{red}{\bra{e}}\exp\left[-i\omega_a\ketbra{e}{e}t\right]\bigg]\\
  	& = \bigg[\exp\left[+i\omega_at\left(1 - 1\right)\right]\textcolor{red}{\ket{g}}\bigg]\bigg[\textcolor{red}{\bra{e}}\exp\left[-i\omega_a1t\right]\bigg]\\
  	& = \bigg[\exp\left[+i\omega_at0\right]\bigg]\textcolor{red}{\ket{g}}\textcolor{red}{\bra{e}}\bigg[\exp\left[-i\omega_at\right]\bigg]\\
  	& \blue{\mathbf{\equiv \text{\textbf{exp}}\left[-i\omega_at\right]\ketbra{g}{e}}}
      \end{aligned}
    \end{equation}
  \item
    $\exp\left[+i\omega_a\ketbra{e}{e}t\right]\textcolor{red}{\ketbra{g}{e}}\exp\left[-i\omega_a\ketbra{e}{e}t\right]
    \blue{\mathbf{\equiv \text{\textbf{exp}}\left[+i\omega_at\right]\ketbra{e}{g}}} $
  \end{itemize}
  
  Combining the results
  
  \begin{equation}
    \begin{aligned}
      H_I & = \hbar \Omega\bigg[\blue{\exp^{-i\omega_jt}\hat{a}}+\blue{\exp^{+i\omega_jt}\hat{a}^\dagger}\bigg]\bigg[\blue{\exp^{-i\omega_at}\ketbra{g}{e}}+\blue{\exp^{+i\omega_at}\ketbra{e}{g}}\bigg] \\
      & = \hbar \Omega\bigg(\mathbf{\exp\left[-i(\omega_j-\omega_a)t\right]\ \hat{a}\ketbra{e}{g}} \quad + \quad \mathbf{\exp\left[i(\omega_j-\omega_a)t\right]\ \hat{a}^\dagger\ketbra{g}{e}} \\
      & \quad  + \quad  \mathbf{\exp\left[-i(\omega_j+\omega_a)t\right]\ \hat{a}\ketbra{g}{e}}
      \quad          +          \quad          \mathbf{\exp\left[i(\omega_j+\omega_a)t\right]\
        \hat{a}^\dagger\ketbra{e}{g}}\bigg)
    \end{aligned}
    \label{eqn:fullHamiltonianLightAtom}
  \end{equation}
  
  \iframe{and using
    \begin{itemize}
    \item $ \omega_j - \omega_a = \delta\phi$;
    \item Negelct the fast rotating terms $ \omega_j+\omega_a $;
    \item Ignore the resonator operators $ a, a\idagger $;
    \item Use the fact that the coupling strength
      \[
        \phi_p I_0/2 \equiv \hbar\Omega
      \]
    \end{itemize}
  
    \begin{equation}\label{key}
      \begin{aligned}
        \mathcal{H} & = -\hbar\Omega\big[\ketbra{e}{g}e^{i\phi}+\ketbra{g}{e}e^{-i\phi}\big]\\
        & = -\hbar\Omega\big[\sigma^{+}e^{i\phi}+\sigma^{-}e^{-i\phi}\big]\\
        & = -\hbar\Omega\big[(\sigma_x - i\sigma_y)/2 e^{i\phi}+(\sigma_x + i\sigma_y)/2e^{-i\phi}\big]\\
        & \red{= -\hbar\Omega\big[\sigma_x\cos(\varphi)-\sigma_y\sin(\varphi)\big]}
      \end{aligned}
    \end{equation}
  }
 
 
  \subsection{Dynamics with Pauli Matrices}
  Summarising up to this point, we have argued  for the appearance of the Linbland term in the
  Master  equation  to  account  for  decoherence  and  relaxation  processes  in  the  system
  \red{{\large  \begin{equation}\label{Totalequation} \mathcal{L}  = \imx{\Gamma_1\rho_{11}  -
          \Gamma^{ex}\rho_{00}}{-\Gamma_2\rho_{01}}{-\Gamma_2\rho_{10}}{\Gamma^{ex}\rho_{00}-\Gamma_1\rho_{11}};\quad
        \dot{\rho} = -\frac{i}{\hbar}\big[\mathcal{H},\rho\big]+\mathcal{L},
      \end{equation}}}
  
  \red{\iframe{
      \begin{itemize}
      \item $ \Gamma_1 $ \hfill relaxation of the z component;
      \item $ \Gamma_2 = \gamma + \Gamma_{1}/2 $\hfill relaxation in the x-y plane.
      \end{itemize}
    }}
  
  \noindent and shown a few useful expectation  values, that allow one to express the dynamics
  of the system
  
  \noindent Lets compute the evolution of these expectation values
  
  \begin{equation}\label{evolution}
    \ialigned{\difffrac{\iaverage{\sigma_j}}{t}&=\itrace{\sigma_j\difffrac{\rho}{t}} = \itrace{-\frac{i}{\hbar}\sigma_j\big(\mathcal{H}\rho - \rho\mathcal{H}\big)+\sigma_j\mathcal{L}}\\
      \mathcal{H}&  = \frac{\hbar\Omega}{2}\bigg(\sigma_x\cos(\phi)-\sigma_y\sin(\phi)\bigg),}
  \end{equation}
  
  \noindent and evaluating for all the matrices
  
  \begin{equation}\label{pauliDeriv}
    \begin{aligned}
      \difffrac{\iaverage{\sigma_x}}{t}&=-i\frac{\Omega}{2}\itrace{\cancel{\big(\sigma_x\sigma_x\rho-\sigma_x\rho\sigma_x\big)}\cos(\phi) + \big(\sigma_x\sigma_y\rho-\sigma_x\rho\sigma_y\big)\sin(\phi)} + \itrace{\sigma_x\mathcal{L}}\\
      & = \Omega\iaverage{\sigma_z}\sin(\phi)-\Gamma_2\iaverage{\sigma_x}\\
      \difffrac{\iaverage{\sigma_y}}{t}& = \Omega\iaverage{\sigma_z}\sin(\phi)-\Gamma_2\iaverage{\sigma_y}\\
      \difffrac{\iaverage{\sigma_z}}{t}& = -\Omega\big(\iaverage{\sigma_x}\cos(\phi)+\iaverage{\sigma_y}\sin(\phi)\big)-\Gamma_1\iaverage{\sigma_z}+\Gamma_1\\
    \end{aligned}
  \end{equation}
  
  \noindent or in more compact form
  
  \iframe{\begin{equation}\label{pauliEv}                        \difffrac{}{t}\begin{pmatrix}
        \iaverage{\sigma_x}\\\iaverage{\sigma_y}\\\iaverage{\sigma_z}
      \end{pmatrix} = \begin{pmatrix}
        -\Gamma_2&0&\Omega\sin(\phi)\\0&-\Gamma_2&\Omega\cos(\phi)\\-\Omega\sin(\phi)&-\Omega\cos(\phi)&-\Gamma_1\\
      \end{pmatrix}\begin{pmatrix}
        \iaverage{\sigma_x}\\\iaverage{\sigma_y}\\\iaverage{\sigma_z}
      \end{pmatrix}+\begin{pmatrix} 0\\0\\\Gamma_1
      \end{pmatrix}\iright\red{\difffrac{\vec{\iaverage{\sigma}}}{t} =
        B\vec{\iaverage{\sigma}}+\vec{b}.}
    \end{equation}
  
    \red{Where $ \phi  $ controls the axis  of rotation.  As we see  $ \varphi = 0  $ leads to
      evolution of \isigmay and \isigmaz (rotation abou the x-axis):
  	
      \textbf{Either way -  the \isigmaz will be driving  its own decay via $  \Gamma_1 $, the
        equatorial will  be driving their decay  via $ \Gamma_2  $ and the off  diagonal terms
        make the decays affect each other} }

    The trial  solution is $ \sigma  =e^{B t}\vec{A} $, where  $ \vec{A} $ is  determined from
    initial conditions. Subbing it in yields:
    \[
      e^{Bt}\dot{\vec{A}}  =  \vec{b}  \qquad\lra\qquad \dot{\vec{A}}  =  e^{-Bt}\vec{b}  \iRa
      \vec{A}(t) = -B\ipow{-1}e^{-Bt}\vec{b}+\vec{c},
    \]
  
    \noindent yielding
    \begin{equation}\begin{aligned}
  	\vec{\iaverage{\sigma}}   &=    -\mathbf{-B}^{-1}\vec{b}+e^{\mathbf{B}t}\vec{c}   \\&=
        -\mathbf{B}^{-1}\vec{b}+e^{\mathbf{-B}t}\big(\iaverage{\sigma_0}                     +
        \mathbf{B}^{-1}\vec{b}\big)\end{aligned}
    \end{equation}
  
    In the starionary state, $ \dot{\sigma} \equiv 0 $ the state is:
    \begin{equation}
      \vec{\iaverage{\sigma}} = -\mathbf{B}^{-1}\vec{b}
    \end{equation}
  }

  \noindent The dynamics  of this system are  the exact same as  for a spin 1/2  particle in a
  magnetic  field.  The  different  components of  the  vector  \isigma can  be  plotted on  a
  sphere. Note  that unlike the  Bloch sphere used  previously, these vectors  do \textbf{not}
  have to be on the surface. The various states $ \vec{\iaverage{\sigma}} $ correspond to
  
  \iframe{ \red{{\footnotesize
        \[
          \ket{\psi}           =          \cos\left(\frac{\theta}{2}\right)\ket{0}           +
          e^{i\phi}\sin\left(\frac{\theta}{2}\right)\ket{1}                                  =
          \imatrixcol{\cos(\theta/2)}{e^{i\phi}\sin(\theta/2)}     \Rightarrow      \rho     =
          \frac{1}{2}\imatrix{1+{\cos(\theta)}{}}{e^{-i\phi}\sin(\theta)}{e^{i\phi}\sin(\theta)}{1-\cos(\theta)}
        \]
        \[
          \begin{aligned}
            \rho_{00} & = \frac{\iaverage{\sigma_z}+1}{2}&& \green{= \frac{1+\cos(\theta)}{2}}\\
            \rho_{01} & =\frac{\iaverage{\sigma_x}-i\iaverage{\sigma_y}}{2}  && \green{= \frac{e^{-i\phi}\sin(\theta)}{2}} && \green{= \iaverage{\sigma_{+}}}\\
            \rho_{10} & = \frac{\iaverage{\sigma_x}+i\iaverage{\sigma_y}}{2} && \green{= \frac{e^{i\phi}\sin(\theta)}{2}} && \green{= \iaverage{\sigma_{-}}}\\
            \rho_{11} & = \frac{1-\iaverage{\sigma_z}}{2} && \green{= \frac{1-\cos(\theta)}{2}}\\
          \end{aligned}\]
        \[
          \begin{aligned}
            \iaverage{\sigma_x} & =\rho_{01}+\rho_{10} && \\
            \iaverage{\sigma_y}   &  =   i\rho_{01}-i\rho_{10}   &&\\  \iaverage{\sigma_z}   &
            =\rho_{00}-\rho_{11} &&
          \end{aligned}\iRa \green{\vec{\iaverage{\sigma}} =
            \ithreeMatrix{\cos(\varphi)\sin(\vartheta)}{\sin(\varphi)\sin(\vartheta)}{\cos(\vartheta)}}
        \]}} \green{Green means in pure state i.e. can be represented by a wavefunction} }
  
  \begin{align}
    \rho_{00}=1 \quad(\red{\theta = 0}) & \iright \isigma = \ithreeMatrix{0}{0}{1}\\
    \rho_{11}=1 \quad(\red{\theta = \pi})& \iright \isigma = \ithreeMatrix{0}{0}{-1}\\
    \rho_{00}=\rho_{11}\quad (\red{\theta = \pi/2})&\iright\isigma = \ithreeMatrix{\cos(\phi)}{\sin(\phi)}{0}
  \end{align}
  
  \begin{figure}[h]
    \ifigure{7cm}{sphere}
  \end{figure}

 \subsection{Emission from the atom}
 The atom will be emitting according to Olegs 2nd and 7th Feb talks:
  
  \[
    I^{\mp}(x,t)  =  \frac{\hbar\Gamma_{1}}{\varphi_p}i{\isigmaplusminus}e^{ik\iabs{x}-i\omega
      t},
  \]
  
  \noindent   which  we   get   from   the  expectation   value   of   the  current   operator
  $ \hat{I} = \frac{\hbar\Gamma_1}{\phi_p}i\sigma^{\pm} $.
  
  \iframe{     Note,    how     this     emission    is     related    to     \isigmaplusminus
    $ \propto \isigmax \mp i\isigmay$ the projection of the pseudo spin onto the x-y plane. In
    the limiting case of the pure state(generally it could be smaller)
    \[
      \begin{aligned}
        \isigmax &= \sin(\theta)\cos(\varphi)\\
        \isigmay &= \sin(\theta)\sin(\varphi);
      \end{aligned}
    \]
    \noindent it becomes  evident that in order to have  large emission, \isigmaplusminus, you
    need $ \theta = \pi/2 $ i.e. have the state in the equatorial plane.  }

 \subsection{Pulse configuration}
 We subject the atom to three kind of pulses:
 \begin{itemize}
 \item Pulse $ P $ for preparation of the state (e.g. \piulse);
 \item Pulse $ M $ for state manipulation;
 \item Readout pulse $ R $, which has a time $ \Delta t >> T_1 $ so that the atom has the time
   to udnergo full relaxation.
 \end{itemize}

 During readout, when we are no longer driving after time $ t_R $, Eq.~\eqref{pauliEv} will be
 simplified to:
   
   \[
     \difffrac{\vec{\iaverage{\sigma}}}{t} = \begin{pmatrix}
       -\Gamma_2&0&0\\0&-\Gamma_2&0\\0&0&-\Gamma_1\\
     \end{pmatrix}\begin{pmatrix}
       \iaverage{\sigma_x}\\\iaverage{\sigma_y}\\\iaverage{\sigma_z}
     \end{pmatrix}+\begin{pmatrix} 0\\0\\+\Gamma_1
     \end{pmatrix}\iRa
     \ialigned{
       \isigmax(t) & = \iaverage{\sigma_x(t_R)}e^{-i\Gamma_{2}(t-t_R)}\\
       \isigmay(t) & = \iaverage{\sigma_y(t_R)}e^{-i\Gamma_{2}(t-t_R)}\\
       \isigmaz(t) & = 1 - (1-\iaverage{\sigma_z(t_R)})e^{-\Gamma_1(t-t_R)} }
   \]
   \noindent and so \red{\iframe{ \textbf{Expectation values after drive is switched off}
       \begin{equation}\label{key}
         \ialigned{\isigmaminus &= \frac{\isigmax+i\isigmay}{2}&= \iaverage{\sigma^{-}(t_R)}e^{-i\Gamma_{2}(t-t_R)}\\
           \isigmaz & &= 1-(1-\iaverage{\sigma_z(t_R)}) e^{-\Gamma_1(t-t_R)} \text{  (this ensures decay to \isigmaz = 1)}
         }
       \end{equation}}}
  
  \subsection{Coherent and incoherent emission}
  Though in the article we use current operators, here I examined analogoues voltage ones
  \begin{itemize}
  \item The voltage operator is defined as:
  
  \begin{equation}\label{feb22018}
    \hat{V}^{+} = i\frac{\hbar\Gamma_1}{\phi}\sigma^{-},
  \end{equation}
  
  \noindent the `-' coming from  the fact that the atom must relax in  order for voltage to be
  produced. The average produced field would be
  
  \[
    \iaverage{\hat{V}^{+}} = i\frac{\hbar\Gamma_1}{\phi}\iaverage{\sigma^{-}},
  \]
  
\item Now, the  power resulting from this  voltage, which is effectively  noise at frequencies
  $ \omega $ as the atom relaxes spontaneously, is given by the correlation function:
  
  \begin{equation}\label{feb22018:1}
    \begin{aligned}
      \iaverage{V^2(\omega)} = & \frac{1}{2\pi}\int_{-\infty}^{\infty}\iaverage{\hat{V}^{-}(0)\hat{V}^{+}(\tau)}e^{i\omega \tau}d\tau\\
      =                                                                                      &
      \frac{\hbar^2\Gamma_1^{2}}{\phi^2}\frac{1}{2\pi}\int_{-\infty}^{\infty}\iaverage{\sigma_{+}(0)\sigma_{-}(\tau)}e^{i\omega
        \tau}d\tau
    \end{aligned}
  \end{equation}
  
\item   \red{Using    a   trick   in    Olegs   book,   one    can   find   that    the   term
    $ \iaverage{\sigma_{+}(0)\sigma_{-}(\tau)} $ can be decomposed as:
    \begin{itemize}
    \item \textbf{Total sum is} \[ \frac{1-{\isigmaz}}{2}. \]
    \item \textbf{The coherent part is} \[ \iaverage{\sigma_{+}}\iaverage{\sigma_{-}}. \]
    \item \textbf{Therefore the incoherent part must be the difference between the two}:
      \[ \frac{1-{\isigmaz}}{2} - \iaverage{\sigma_{+}}\iaverage{\sigma_{-}}. \]
    \end{itemize}
  }
  
\item Finding the total  emitted power, by integrating over the  full frequency range \red{and
    assuming that  we are dealing with  stationary states (ss)  that would form in  the system
    when averaging:}
  
  \begin{equation}\label{feb220183}
    \begin{aligned}
      \text{Power}_\text{total} &= \frac{1}{Z}\int 	\iaverage{V^2(\omega)}  d\phi\\
      & = \frac{1}{Z}\int \frac{\hbar^2\Gamma_1^{2}}{\phi^2}\frac{1}{2\pi}\int_{-\infty}^{\infty} \frac{1-{\isigmaz}_{}}{2} e^{i\omega \tau}d\tau   d\phi\\
      & = \frac{\hbar^2\Gamma_1^2}{Z\phi^2} \frac{1-{\isigmaz}_{}}{2} \int\frac{1}{2\pi}\int e^{i\omega\tau}d\tau d\phi\\
      & \text{(Using the fact that the integral over the delta function is just 0 and } \Gamma_1 = \frac{\hbar\omega\phi^2Z}{\hbar^2})\\
      & = \hbar\omega\Gamma_1 \frac{1-{\isigmaz}_{}}{2}
    \end{aligned}
  \end{equation}
\end{itemize}

Thus, the formulas we have for emission are \iframe{
  \begin{equation}\label{eqn:dynamics1}
    \ialigned{\isigmaminus &= \frac{\isigmax+i\isigmay}{2}= &\iaverage{\sigma^{-}(t_R)}e^{-\Gamma_{2}(t-t_R)}\\
      \isigmaz & = &1-(1-\iaverage{\sigma_z(t_R)}) e^{-\Gamma_1(t-t_R)} \\\\
      \text{Coherent power} & \propto \isigmaminus\\
      \text{Total power} &= \hbar\omega\Gamma_1  \frac{1-{\isigmaz}_{}}{2} =& \frac{\hbar\omega\Gamma_1}{2}(1-\iaverage{\sigma_z(t_R)}) e^{-\Gamma_1(t-t_R)}
    }
  \end{equation}
}
  
  \subsection{Catching coherent and incoherent emission}
  Practically   we   are  measuring   the   relaxation   over   a   readout  pulse   time   of
  $ T_R  \equiv \Delta  T_R $,  which means  that the average  signal we  measure will  be the
  average     of     Eq.~\eqref{eqn:dynamics1},     which    we     get     by     integration
  ($ \frac{1}{T_R}\int_{t_R}^{t_R+T_R}Ae^{-gt}dt = \frac{A}{gT_R} $)
  
  \ipic{4cm}{dynamics_pulses1} \iframe{
    \begin{equation}\label{eqn:dynamics2}
      \ialigned{\isigmaminus e^{-\Gamma_{2}(t-t_R)} & \ra \frac{\iaverage{\sigma^{-}(t_R)}}{T_R\Gamma_2}\\
        &&\text{Coherent from Eq. \eqref{feb22018}} & = \frac{\hbar\Gamma_1}{\phi_p}\isigmaminus = \frac{\hbar\Gamma_1}{\phi_p\Gamma_2T_R}\iaverage{\sigma^{-}(t_R)}\\
        (1 - \iaverage{\sigma_z(t_R)})e^{-\Gamma_1(t-t_R)} & = \frac{1-\iaverage{\sigma_z(t_R)}}{\Gamma_1T_R} \\
        &&\text{Total} &= \frac{\hbar\omega}{2T_R}(1-\iaverage{\sigma_z(t_R)})\\
        &&\text{Forward direction} &= \frac{\hbar\omega}{4T_R}(1-\iaverage{\sigma_z(t_R)})\\
      }
    \end{equation}
  }
  
  
  When it comes to evaluating noise, the coherent emission from the atom will have a strength:
  \[
    \frac{\hbar\omega}{\Gamma_1},
  \]
  settinga limit of the bandwidth we can use to observe the signal due to noise $ k_bT $
   
   \[
     \Delta f k_bT<< \hbar\omega\Gamma_1.
   \]
   
  \subsection{Oscillation measurements}
  Now begin driving the system
  \begin{equation}\label{eqn:blochDriving}
    \difffrac{}{t}\begin{pmatrix}
      \iaverage{\sigma_x}\\\iaverage{\sigma_y}\\\iaverage{\sigma_z}
    \end{pmatrix} = \begin{pmatrix}
      -\Gamma_2&0&\Omega\sin(\phi)\\0&-\Gamma_2&\Omega\cos(\phi)\\-\Omega\sin(\phi)&-\Omega\cos(\phi)&-\Gamma_1\\
    \end{pmatrix}\begin{pmatrix}
      \iaverage{\sigma_x}\\\iaverage{\sigma_y}\\\iaverage{\sigma_z}
    \end{pmatrix}+\begin{pmatrix} 0\\0\\\Gamma_1
    \end{pmatrix}.
  \end{equation}
   
  \noindent For long driving about the x-axis (rotation in the y-z plane), $ \phi = 0 $:
  \begin{equation}
    \difffrac{}{t}\begin{pmatrix}
      \iaverage{\sigma_y}\\\iaverage{\sigma_z}
    \end{pmatrix} = \begin{pmatrix}
      -\Gamma_2&\Omega\\-\Omega&-\Gamma_1\\
    \end{pmatrix}\begin{pmatrix}
      \iaverage{\sigma_y}\\\iaverage{\sigma_z}
    \end{pmatrix}+\begin{pmatrix} 0\\\Gamma_1
    \end{pmatrix}.
  \end{equation}
   
  I went and plotted \isigmay vs \isigmaz in mathematica and also the oscillations. They agree
  with what was observed.
   
  \begin{itemize}
  \item   Coherent   signal  $   \propto   \isigmaminus   $  was   picked   up   by  the   VNA
    ($ \frac{\hbar\Gamma_1}{\phi_p\Gamma_2T_R}\iaverage{\sigma^{-}(t_R)} $)
  \item   Incoherent   $   \propto   1-\sigma_z   $   measured   by   5\,MHz   bandwidth   SPA
    ($ \frac{\hbar\omega}{4T_R}(1-\iaverage{\sigma_z(t_R)}) $).
  \end{itemize}
   
  \ipicsmall{8cm}{dynamics_simulation} \ipicsmall{8cm}{dynamics_real}
   
    
  Moreover, by driving  the atom for a long  time will results in incoherent  emission peak on
  the SPA that can  be used to find out the relaxation rate  $ \Gamma_1 $ \red{\textbf{ASK WHY
      ITS GAMMA 2 IN THE GRAPH}} \ipicCaption{5cm}{dynamics_peak}{The incoherent emission peak
    gives the relaxation rate $ \Gamma_1 $}.
  \newpage
 
  \subsection{Driving in the Schrodinger picture}
  \begin{equation}\label{app2}
    \mathcal{H} = -\frac{\hbar\omega_0}{2}\sigma_z-\hbar\Omega\cos(\omega_0 t)\sigma_x
  \end{equation}
  
  \noindent for which we shall try the unitary transformation
  
  \begin{equation}\label{app2Try}
    U(t) = \exp\left[-i\frac{\omega_0 t}{2}\sigma_z\right]
  \end{equation}  
  
  \noindent resulting in the Hamiltonian
  
  \begin{equation}\label{app2New}
    \begin{aligned}
      \mathcal{H'} & = U\mathcal{H}U^{\dagger} - i\hbar U\dot{U}^{\dagger}\\
      & = -\frac{\hbar\omega}{2}e^{-i\omega_0t/2\sigma_z}\sigma_ze^{+i\omega_0t/2\sigma_z}-\hbar\Omega\frac{e^{i\omega t}+e^{-i\omega t}}{2}e^{-i\omega_0t/2\sigma_z}\sigma_xe^{i\omega_0t/2\sigma_z}- i\hbar e^{-i\omega_0t/2\sigma_z}\bigg(i\frac{\omega}{2}\sigma_z\bigg)e^{i\omega_0t/2\sigma_z}\\
      & = -\frac{\hbar\Omega}{2}\bigg(e^{i\omega t}+e^{-i\omega t}\bigg)e^{-i\omega_0t/2\sigma_z}\red{e^{(-1)i\omega_0t/2\sigma_z}\sigma_x}\\
      & = -\frac{\hbar\Omega}{2}\bigg(e^{i\omega t}+e^{-i\omega t}\bigg){e^{-i\omega_0t\sigma_z}\sigma_x}\\
      &    =-\frac{\hbar\Omega}{2}\bigg(e^{i\omega   t}+e^{-i\omega    t}\bigg)\begin{pmatrix}
        e^{-i\omega_0t}&0\\0&e^{+i\omega_0t}
      \end{pmatrix}\begin{pmatrix}
        0&1\\1&0
      \end{pmatrix}\\
      &    =-\frac{\hbar\Omega}{2}\bigg(e^{i\omega   t}+e^{-i\omega    t}\bigg)\begin{pmatrix}
        0&e^{i\omega_0t}\\e^{-i\omega_0t}&0
      \end{pmatrix}
      \\
      & =-\frac{\hbar\Omega}{2}\begin{pmatrix} 0&1+e^{2i\omega_0t}\\1+e^{-2i\omega_0t}&0
      \end{pmatrix}
      \\
      & \approx -\frac{\hbar\Omega}{2}\begin{pmatrix} 0&1\\1&0
      \end{pmatrix}
      \\
      & \approx -\frac{\hbar\Omega}{2}\sigma_x
    \end{aligned}
  \end{equation}
  
  \noindent  where we  have  applied the  RWA  where  we neglect  fast  rotating terms  (which
  correspond  to non  conserved energy  processes). \iframe{Physically  what we  have done  in
    entered       the      frame       of       the       initial      qubit       Hamiltonian
    ($ \mathcal{H} = -\frac{\hbar\omega}{2}\sigma_z $), implicitly taking into account the raw
    evolution to concentrate only on the driving field contribution.
  	
  	\begin{center}
          Coupling of levels via radiation = RWA.
        \end{center}}
  
      Now the evolution of the state
  
  \begin{equation}\label{app2Evv}
    U(t) = e^{-i\mathcal{H'}/\hbar t} = e^{i\Omega t/2\sigma_x} = \cos\left(\frac{\Omega t}{2}\right)\mathbb{I} + i\sin\left(\frac{\Omega t}{2}\right)\sigma_x = \imatrix{\cos(\Omega t/2)}{i\sin(\Omega t/2)}{i\sin(\Omega t/2)}{\cos(\Omega t/2)}
  \end{equation}
  
  \noindent gives according to Eq.\eqref{uniPauli}
  
  \begin{equation}\label{app2State}
    \ket{\Psi} = U\ket{0} = \cos(\frac{\Omega t}{2})\ket{0}+e^{i\pi/2}\sin(\frac{\Omega t}{2})\ket{1}.
  \end{equation}


 \subsection{Stationary state}
 After long  driving I am  apprantely told {\LARGE \textbf{and  i don't understand  where this
     comes from}} that the stationary state is
 \[
   \isigmay = \frac{\Gamma_1 \Omega}{\Gamma_1\Gamma_2+\Omega^2}
 \]
   
 \noindent which will saturate at long times in in the figure below:
   
 \ipicCaption{4cm}{dynamics_2}{Stationary  state  is achieved  for  long  times for  different
   driving power. The decay strength $ \propto \Gamma_1/2+\Gamma_2/2 $, to to dephasing (y) and relaxation (z)
   which the state feels equally as it rotates.}
   
 Then two things are done:
 \begin{enumerate}
 \item $ \mathbf{\Gamma_2} $  First we measure the dephasing in the equatorial  plane by applying a
   $ \pi/2$-pulse and measuring coherent emission. Decay has a rate $ \Gamma_2 $
   	
   \ipic{4cm}{dynamics_3}
   	
   From Eq.~\eqref{app2Evv} the first $ \pi/2$-pulse will create a state
   \[
     \iket{0}
     \iratext{$                                                             \frac{1}{\sqrt{2}}
       $\imatrix{1}{i}{i}{1}}\frac{\iket{0}+i\iket{1}}{\sqrt{2}}                             \equiv
     \frac{1}{2}\imatrix{1}{-i}{i}{1}\equiv\imatrixthree{0}{1}{0}\grey{=\imatrixthree{\cos(\phi)}{\sin(\phi)}{0}}
   \]
   	
   \noindent then we wait over time $ \Delta t\isub{RP} $ over which time its convenient to work in
   the  Bloch sphere  representation and  assume no  driving for  Eq.~\eqref{eqn:blochDriving}
   ($ \Omega $ = 0)
   	
   	 \begin{equation}\label{eqn:dynamics_relax}
           \difffrac{\vec{\iaverage{\sigma}}}{t} = \begin{pmatrix}
             -\Gamma_2&0&0\\0&-\Gamma_2&0\\0&0&-\Gamma_1\\
           \end{pmatrix}\begin{pmatrix}
             \iaverage{\sigma_x}\\\iaverage{\sigma_y}\\\iaverage{\sigma_z}
           \end{pmatrix}+\begin{pmatrix} 0\\0\\+\Gamma_1
           \end{pmatrix}\iRa
           \ialigned{
             \isigmax(t) & = \iaverage{\sigma_x(t_R)}e^{-i\Gamma_{2}(t-t_R)}\\
             \isigmay(t) & = \iaverage{\sigma_y(t_R)}e^{-i\Gamma_{2}(t-t_R)}\\
             \isigmaz(t) & = 1-(1-\iaverage{\sigma_z(t_R)})e^{-\Gamma_1(t-t_R)}
           }
         \end{equation}
   	
         and         so        during         readout         we        are         measuring:
         \iframe{\[ \isigmay = e^{-i\Gamma_2(t-\Delta t\isub{TR})}
           \]}
       \item $  \mathbf{\Gamma_1} $ Then  allow relaxation, and  project onto equatorial  plane for
         readout of coherent emission.  \ipic{4cm}{dynamics_4}
   	
   	   	\[
                  \iket{0}
                  \iratext{\imatrix{0}{i}{i}{0}}i\iket{1}\equiv\imatrix{0}{0}{0}{1}\equiv\imatrixthree{0}{0}{-1}
                \]
   	
                \noindent and according to Eq.~\eqref{eqn:dynamics_relax}
   	
   	\[
          \isigmaz= 1 -2e^{-\Gamma_1(t-\Delta t\isub{TR})} \iRa \ialigned{\rho_{00} &= \frac{1+\isigmaz}{2} &= 1-e^{-\Gamma_1(t-\Delta t_{TR})}\\
            \rho_{11} &= \frac{1-\isigmaz}{2} &= e^{-\Gamma_1(t-\Delta t_{TR})}}.
   	\]
   	
   	\noindent And finally projecting to the coherent state for readout:
   	
   	\[
          \imatrix{1-e^{-\Gamma_1(t-\Delta               t_{TR})}}{0}{0}{e^{-\Gamma_1(t-\Delta
              t_{TR})}}\iratext{$                                           \frac{1}{\sqrt{2}}
            $\imatrix{1}{i}{i}{1}}\frac{1}{\sqrt{2}}\imatrix{1-e^{-\Gamma_1(t-\Delta
              t_{TR})}}{ie^{-\Gamma_1(t-\Delta             t_{TR})}}{i(1-e^{-\Gamma_1(t-\Delta
              t_{TR})})}{e^{-\Gamma_1(t-\Delta t_{TR})}}
   	\]
   	
   	\noindent and the coherent part that we are measuring:
   	
   	\[
          \isigmay = \Re(i\isigmaminus)=i\rho_{10} = -e^{-\Gamma_1(t-t\isub{TR})}
   	\]
   	
   	
      \end{enumerate}

 \subsection{Dynamics and regression theorem}
 The power output as  a function of freqeuncy takes on the form  derived earlier (in the paper
 they use current, I used voltage):
  
 \[
   \begin{aligned}
     \iaverage{V^2(\omega)} = & \frac{1}{2\pi}\int_{-\infty}^{\infty}\iaverage{\hat{V}^{-}(0)\hat{V}^{+}(\tau)}e^{i\omega \tau}d\tau\\
     =                                                                                       &
     \frac{\hbar^2\Gamma_1^{2}}{\phi^2}\frac{1}{2\pi}\int_{-\infty}^{\infty}\iaverage{\sigma_{+}(0)\sigma_{-}(\tau)}e^{i\omega
       \tau}d\tau
   \end{aligned}
 \]
  
  
\newpage