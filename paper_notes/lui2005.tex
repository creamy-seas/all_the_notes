\section{Optical selection rules \cite{liu2005}}
 \iframe{\noindent In a 3-JJ qubit at the degeneracy point of $ n\Phi_0, n\in\mathbb{Z} $
 	 \begin{itemize}
 	 	\item The eigenstates of the Hamiltonian
 	 	
 	 	\begin{equation}\label{key}
 	 		\mathcal{H} = \frac{P_p^2}{2M_p} + \frac{P_m^2}{2M_m} + U(\phi_p,\phi_m),
 	 	\end{equation}
 	 	
 	 	\noindent where $ M_p = 2C_J(\Phi_0/2\pi)^2 $, $ M_m = M_p(1+2\alpha) $ and 
 	 	
 	 	\begin{equation}\label{key}
 	 		U(\phi_p, \phi_m) = 2E_J(1-\cos(\phi_p)\cos(\phi_m) + \alpha E_J(1-\cos(2\pi f + 2\phi_m)),
 	 	\end{equation}
 	 	
 	 	\noindent with $ \phi_m = (\phi_1-\phi_2)/2 $ and $ \phi_p = (\phi_1 + \phi_2)/2 $ \textbf{\red{are well defined to be \iket{i}, \iket{j}}}.
	
 	\item \textbf{However the interaction part due to a driving magnetic field is an odd function of the phases
 		\begin{equation}\label{key}
 			\begin{aligned}
	 			\mathcal{H}_I & = \red{j}\blue{\Phi_a(t)}\\
	 			& = \red{\frac{2\pi}{\Phi_0}E_J\alpha\sin\left(2\pi f + \phi_1 +\phi_2\right)}\blue{\iabs{\Phi_a}\cos(\omega_{ij}t)}
 			\end{aligned}
 		\end{equation}
 	}

 	\item Therefore the transition (transition is saying that if the system was in a given state \iket{j}, following the action of the interaction Hamiltonian, how much of the state would transfer to \iket{i})
 	\begin{equation}\label{key}
 		t_{ij} = \bra{i}\mathcal{H}_i\ket{j},
 	\end{equation}
 	
 	\noindent between the defined states \iket{i}, \iket{j} and the odd interaction $ \mathcal{H}_I $ will evaluate to zero \textbf{unless the states have opposite parity.}
 	
 	\item If you detune from degeneracy point, all transitions are acceptable. The paper looks at how you need to drive the system to get different transitions.
 \end{itemize}
}
