% -*- TeX-master: "../paper_notes.tex" -*-
\section{Coherent     quantum     dynamics     of     a     superconducting     flux     qubit
  \cite{chiorescu2003}\cite{chiorescu2004}}
\iframe{\noindent Proposal to  readout current states by using a  SQUID inductively coupled to
  the qubit system.

  It is also possible to use the SQUID as a harmonic oscillator that is coupled to the system.
}
 
 \subsection{Fabircation}
 Thee 3JJ,  where one is smaller  than the other  two by a  factor of 0.8. The  typical mutual
 inductance
 
 \[ 
   M = 9\mu H
 \]
  
 \ipicCaption{4cm}{flux_circuit}{Two resistors are  used to remove leakages in  the lines. The
   qubit is symmetrical, to avoid exciting the SQUID. }
  
 \begin{itemize}
 \item MW are fed in to interact with the qubit;
 \item $  I_b $ passses through  the qubit to  the ground (shown  in diagram). It may  put the
   SQUID in a resistive mode;
 \item $ V_{out} $  will measure any potential difference due to the  qubit being in resistive
   mode, relative to the same ground as above.
 \end{itemize}

 \subsection{Readout}
 \begin{enumerate}
 \item The SQUID  shares one of its branches with  the qubit, so the system has  a high mutual
   inductance.
 \item  \red{The flux  inside a  SQUID  is quantisaed,  and  when the  qubits current  creates
     additional flux, the SQUID will respond by generating its own persistent current so that
     \[
       \Phi_0 \equiv \Phi_\text{qubit}(i_\text{qubit}) + \Phi_\text{SQUID}(I_\text{SQUID})
     \]}
 \item Send in short 50\,ns. The SQUIDs persistent current, $ I_\text{SQUID} $, (determined by
   the state of  the qubit as mentioned above)  will add onto the bias  pulse applied, putting
   the device in the resistive state:
   \[
     I_\text{branch} = I_\text{SQUID} + i_\text{bias} \iRa V_\text{out} = \ialigned{I_\text{branch}R &&\text{if $ I_\text{branch} > i_\text{critical} $}\\
       0 &&\text{if $ I_\text{branch} < i_\text{critical} $}} \]
 \end{enumerate}

 Therefore from start to finish, the effect is:
 \[
   \text{Qubit   state}   \ira   i_\text{qubit}  \ira   \Phi_\text{qubit}   \ira   i_\text{SQUID}
   \xrightarrow{i_\text{bias}} i_\text{branch} \ira V_\text{out}.
 \]
  
 \iframe{
   \begin{itemize}
   \item \noindent State monitoring is just a case of varying $ i_\text{bias} $ until we get a
     50\% switching rate of the voltage. Thus $ i_\text{bias} $ probes the state of the system
     (e.g.  bigger  $ i_\text{bias}  $  means  that $  I_\text{SQUID}  $  is small  and  hence
     $ i_\text{qubit} $ is large, giving the state of the qubit).
   \item  $ i_\text{bias}  \equiv i_\text{background}  $ is  chosen so  that the  SQUID has  a 50\%
     probability of showing a voltage.
   \item $  i_\text{bias} $ will  actually change the  phase of the qubit  when we read  out -
     \red{this  is  good,  because at  the  degeneracy  point  there  is an  equal  amount  of
       circulation in both directions,  so the expectation value of current  in 0. By applying
       this shift before readout, we get a measurable current going, allowing $ V_\text{out} $
       to be more responsive.}
   \end{itemize}}

 \ipicCaption{6cm}{coherent_quantum_1}{\red{\textbf{The shift  of the  bias current is  due to
       the induced phase that we talked about.}}}
  
 \subsection{Ramsey Fringes}
 Here we apply a $ \pi/2$-pulse to put the atom in superposition, allow \iket{0} and \iket{1} to
 evolve  separately,  flip  themwith  \piulse  and   project  back  onto  the  z-axis  with  a
 $ \pi/2$-pulse .
  
 In effect, think of the \piulse as reversing the noise-driven diffusion of the qubit phase at
 the midpoint in  time of the free evolution.  Depending on the position of the  pi pulse, the
 \iket{0} and \iket{1} compobnents will have different dephasing
  
 \ipic{4cm}{coherent_quantum_2}
  
 Dephasing is linked to fluctuating energy levels.
  
 \subsection{Rabi oscillations}
 \ipic{5cm}{chiorescu2004_1}
    
 Apply a driving  field, to induce Rabi oscillations between  states \iket{00}\lra\iket{11} or
 \iket{10}\lra\iket{01}. The Rabi osicllations are really  weak, and a strong driving power is
 required for \iket{00}\lra\iket{11}, which shifts the Rabi frequency.
  
 \ipicCaption{5cm}{chiorescu2004_2}{Rabi  oscillations for  two different  transitions in  the
   system. Note the supressed transition after a $ \pi $ pulse in the left hand side figure - we
   return to the ground state and thus cannot reach state \iket{10}.}
 \newpage
