% -*- TeX-master: "../paper_notes.tex" -*-
\newpage
\section{Tutorial   to   improve   student   understanding   of   single   photon
  experiment\label{sec:tutorial}} The tutorial  gives a quick run  through of how
one  can use  beam  splitters to  create  interference in  with  a single  photon
source. \red{Whenever one  creates interference between the two  paths, he looses
  information about which  arm the photon was  in (and it's always  50/50, due to
  even  splitting  by  the  first  beam  splitter)  but  gain  insight  into  the
  accumulated phase difference between the two arms}.
   
The states  of the photon are  characterised as \iket{U}, \iket{L}  (upper lower)
\iket{H} and \iket{V} (horizontal, vertical). The basis of state is thus
\[
  \kbordermatrix{& \iket{U,V} & \iket{U,H} & \iket{L,V} & \iket{L,H}\\
    \Bra{U,V} & 0 & 0 & 0& 0\\
    \Bra{U,H} & 0 & 0 & 0& 0\\
    \Bra{L,V} & 0 & 0 & 0& 0\\
    \Bra{L,H} & 0 & 0 & 0& 0 }
\]
   
\iframe{The operators are
  \begin{equation}\label{photontutorial1}
    \begin{aligned}
      \text{BS1} & = \frac{1}{\sqrt{2}}\begin{pmatrix}-1&1\\1&1\end{pmatrix}\\
      \text{BS2} & = \frac{1}{\sqrt{2}}\begin{pmatrix}1&-1\\1&1\end{pmatrix}\\
      \text{M} & = \begin{pmatrix}-1&0\\0&-1\end{pmatrix}\\
      \text{PS} & = \begin{pmatrix}e^{i\phi}&0\\0&1\end{pmatrix}
    \end{aligned}
  \end{equation}
}
\begin{itemize}
\item \textbf{No  second beam splitter} \ipicCaption{6cm}{tut_2}{After  the first
    beam splitter, the state will be  a mixed superpositions, and the mirrors and
    phase shifter  will only change  the phase of the  top left and  bottom right
    elements.  The  detecors  project  this state  with  equal  probability,  and
    therefore 0.5 of the  time photon flies in from the bottom,  and 0.5 from the
    top arms. But phase gets washed out!}
  \[
    \imatrixcol{1}{0}  \iratext{\text{BS1}}  \frac{1}{\sqrt{2}}\imatrixcol{-1}{1}
    \iratext{\text{M}}
    \frac{1}{\sqrt{2}}\imatrixcol{1}{-1}\iratext{PS}\iket{\text{Final}}         =
    \frac{1}{\sqrt{2}}\imatrixcol{e^{i\phi}}{-1}.
  \]
   	
  \iframe{\red{Now,  D1  respond   to  the  photon  from   direction  \iket{U}  =
      \imatrixcol{1}{0}.} The probability of a photon collapsing to this state is
    the square of the cross-product:
    \[
      \Bra{U}\iket{\text{Final}}^2 = \frac{1}{2}.
    \]
    \textbf{So the detector picks up, which arm the photon was travelling in (the
      state collapses), and it will be 50/50 in both arms.}  }
\item  \textbf{Standard  mixing   experiment}  \ipicCaption{6cm}{tut_1}{Here  the
    second beamsplitter  will combine the state  in both arms using  off diagonal
    matrix elements.  This creates interference  and dependence  on $ \phi  $} Here,
  before measuring with detectors, we have a second beam splitter:
  \[
    \frac{1}{\sqrt{2}}\imatrixcol{e^{i\phi}}{-1}
    \iratext{\text{BS2}}\frac{1}{2}\imatrixcol{e^{i\phi}+1}{e^{i\phi}-1},
  \]
  \noindent which changes the direction  of photon propagation - the wavefunction
  is spread between both detectors.  \iframe{Upon  measurement by D1, we have the
    collapse the wavefunction to
    \[
      \Bra{U}\iket{\text{Final}}^2 = \frac{1+\cos(\phi)}{2},
    \]
  }
  \noindent \textbf{so the second beamsplitter was able to combine the two states
    in the  two arms, loosing information  about the photon origin,  but allowing
    interference phenomena to be observed.}
\item \textbf{Adding polarisation} \ipicCaption{6cm}{tut_3}{With polarisation, we
    need  to treat  the 4-basis  vector system.  The detectors  are sensitive  to
    polarisation, and  thus can  identif which  arm the  photon came  from. Going
    through  the  4-basis  vector  system,  we find  that  phase  information  is
    cancelled out - we have  two non-interacting horizontal and vertical systems,
    and matrix operations do not create a cross over between these terms.}
  \noindent With  polarisation, we need  to work  in the 4-state  basis. \red{The
    polarises act  as intermediate detectors,  collapsing the state  and stopping
    interference  from taking  place}. Effectively,  the polarizer  matrices look
  like:
   	
   	\[
          \kbordermatrix{& \iket{U,V} & \iket{U,H} & \iket{L,V} & \iket{L,H}\\
            \Bra{U,V} & 1 & 0 & 0& 0\\
            \Bra{U,H} & 0 & 1 & 0& 0\\
            \Bra{L,V} & 0 & 0 & 1& 0\\
            \Bra{L,H} & 0 & 0 & 0& 0 },
   	\]
   	
   	\noindent removing a  single state of the system.  Following through with
        calculation, one arrives at:
   	\[
          \iket{\text{Final}}                                                   =
          \frac{1}{2\sqrt{2}}\begin{pmatrix}1\\e^{i\phi}\\-1\\e^{i\phi}\end{pmatrix}.
   	\]
  	\noindent D1  can measure both  \iket{UH} and \iket{UV}, so  combigin the
        two proabilitites $ 1/8+1/8 = 1/4 $  of registering a photon at D1. There
        is no  phase infromation as  we have no  cross terms (the  horizontal and
        vertical subspaces,  corresponding to the  upper and lower arms,  are not
        being mixed) \ra \red{A horizontally  polaised photon MUST HAVE COME FROM
          \iket{U},  so we  have  origin information,  and  no infromation  about
          phase.}   \ipicCaption{6cm}{tut_4}{With  this  final  beamsplitter,  we
          create a cross-over between the  UP and LOWER states (previosuly, LOWER
          state  were  always  VERTICAL,  but  now  its  a  mix),  and  so  phase
          information  can  be  kept.}   We  apply the  matrix  which  mixes  the
        horizontal  and vertical  polarisation elements,  allowing for  the phase
        between the two arms (directly corresponding to the two polarisations) to
        interact:
   	\[
          \frac{1}{2}\kbordermatrix{& \iket{U,V} & \iket{U,H} & \iket{L,V} & \iket{L,H}\\
            \Bra{U,V} & 1 & 1 & 0& 0\\
            \Bra{U,H} & 1 & 1 & 0& 0\\
            \Bra{L,V} & 0 & 0 & 0& 0\\
            \Bra{L,H} & 0 & 0 & 0& 0 },
   	\]
      \end{itemize}
   
      \iframe{\underline{Common errors}
   	\begin{itemize}
        \item The photons \textbf{do not split}  into two photons after the first
          beam splitter;
        \item A photon is \textbf{not just a partcle!}
        \item  A  single  photon behaves  as  a  wave  in  the MZI,  being  in  a
          superposition  of states  \iket{U} and  \iket{L} before  measurement is
          performed;
        \item  If we  measured the  state  along arms,  the state  of the  system
          collapses! There is no chance for interference;
        \item Measurement of polarisation collapses the state of the photon, such
          that if a detector detects  a horizontal (or vertical polarisation), if
          must have come from  U (or L) and thus WPI is  known, and no interfence
          can take place.
   	\end{itemize}
	}
\newpage