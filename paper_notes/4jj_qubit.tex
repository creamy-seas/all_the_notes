% -*- TeX-master: "../paper_notes.tex" -*-
\newpage
\section{4-JJ persistent current Qubit \cite{mooij1999}}
 \iframe{Proposal of a 4-JJ setup, where two of the JJ's form a SQUID.
	\begin{itemize}
		\item\textbf{Covered in \"Flux Qubit 4JJ\"};
		\item Potential landscape forms with localised minima corresponding to persistent current states:
		
		\ipic{6cm}{flux_4jj_2}
		
		\item \textbf{Current line 1: }Coupling SQUID with a current line, we can bias it and change the height of the potential, governing which state we are cross-coupling most strongly;
		\item \textbf{Current line 2: } Couples to the main junction, causing evolution of the system state;
		\item Both currents induce a field that causes an energy perturbation, resulting in 
		\begin{equation}\label{key}
		\mathcal{H} = \Delta E(80\delta_1 + 42\delta_2)\sigma_z - (9.2\delta_1 + 8.3\delta_2).
		\end{equation}
		\noindent Coefficients are found from how the fluxes affect the barrier heights.
		\item Hamiltonian different for the $ T_{in} $ and $ T_{out} $ directions.
	\end{itemize} 
 }
