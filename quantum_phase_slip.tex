\newpage

\section{Coherent Quantum Phase Slip (CQPS) \cite{mooij2005} \label{sec:qps}}

\subsection{Literature}
\begin{framed}\noindent
  \begin{itemize}
  \item \textbf{Trivia}
    \begin{enumerate}
    \item Dual to  the JJ effect, where instead of  coherent transfer of charge,
      we have coherent transfer of flux
    \item Each phase slip event releases an energy $I\Phi_0$.
    \item In \cite{mooij2005} there is  a simplfying assumption that CQPS ``just
      happens'' with $E_s/h = 100\,\text{GHz}$;
    \item Writes  must have \textbf{large kinetic  inductance} and \textbf{small
        capacitance}. This will give impedance
      \begin{equation}
        Z_{c} = \sqrt{\frac{L}{C}}
      \end{equation}

      \noindent       \red{and       for        frequent       phase       slips
        \,$Z_{c} > R_q = \frac{h}{4e^2}$\,}.

    \item Just like in JJ, the CQPS  has trapped voltage in potential minima for
      voltages that do not exceed
      \begin{equation}
        V_{C} = \frac{2\pi E_{S}}{2e}
      \end{equation}
    \item Need  to have large  normal state resistivity  in order to  have large
      impedance when in superconducting state - \red{large inductance}
    \item  We will  observe when  the flux  is localised  i.e. weak  phase slip,
      \,$E_{L}>> E_s$\,
    \item Quantum phase slip process is a tunneling one:
      \begin{equation}
        E_{S} = E_0\exp(-\kappa \bar{\omega})
      \end{equation}

      \noindent where $\bar{\omega}$ is the line  width and $\kappa$ is the characteristic
      width at  which the  wire becomes  a one-dimensional  channel \textbf{with
        large quantum fluctuations possible}
    \end{enumerate}
  \item \textbf{Experiment}
    \begin{enumerate}
    \item   Fabricate  with   strongly  disordered   superconductors  near   the
      superconductor-insulator transition  (Niobium Nitride for example)  - this
      will prevent gapless excitations \cite{Astafiev_2012}.
    \item Fabricate several thin superconducting bridges with meanders;
    \item  Measure the  DC voltage,  $ V  $, acorss  then as  a function  of the
      applied gate voltage, $ V_c $, from underneath the chip;
    \item Some samples have big oscillations in $ V $ with gate voltage - Single
      Electron Transistor regime,  where nanowire is compoased  of grains, which
      feel a Coulomb Blockade;
    \item  Quantum Phase  Slip possible  in other  samples, where  there are  no
      grains in the wire.
    \end{enumerate}
  \end{itemize}
\end{framed}

Initially quantum phase slips were a result of thermal activation in thin wires.

\subsection{Cooper box duality}
The system  is dual  to the cooper  pair box system,  where the  charging energy
creates a parabollic dependence, and the Josephson coupling energy $ E_J $ mixes
states lifting degeneracy as in Sec~\ref{sec:cooper_pair_box}:

 \begin{equation}\label{key}
   \mathcal{H} = \red{E_c(\hat{n} - n_g)^2} - \blue{\frac{E_J}{2}\left[\iketbra{n+1}{n} + \iketbra{n}{n+1}\right]}.
 \end{equation}

 \noindent  In the  qunatum phase  slip,  we do  the  same thing,  but now  with
 inductor-component  energy,  $  E_L  = \frac{\Phi_0^2}{2L}  $,  is  parabolic  and
 degeneracy is lifted by  the coupling, $ E_s $, due to  the tunneling of fluxes
 (we do not specify yet the exact derivation of this energy):

 \begin{equation}\label{eq:hamiltonian-cqps-standalone}
   \mathcal{H} = \red{E_L(\hat{n} - f)^2} - \blue{\frac{E_s}{2}\left[\iketbra{n+1}{n} + \iketbra{n}{n+1}\right]},
 \end{equation}

 \noindent with $ f = \Phi/\Phi_0 $.

 \begin{framed}\noindent
   \noindent The discussion them stems off into how the Cooper pair tunneling is
   dual  to the  phase  slip, how  one  gets similar  steps  in voltage  current
   characterisitics.  There is a lot of math, which I don't want to get into.
 \end{framed}

 In both cases, a truncation of  the Hamiltonian results in states (solving like
 in             \autoref{sec:firstLecture})              and             setting
 $\delta\Phi = \Phi_{\text{ext}}-  \left( N + \frac{1}{2}  \right)\Phi_0$ as the
 deviation from a degeneracy point between states \iket{N} and \iket{N + 1}:

 {\scriptsize \begin{equation} \mathcal{H} = \begin{pmatrix}
       \frac{\Phi_0^2}{2L}\left( N - \frac{\Phi_{\text{ext}}}{\Phi_0} \right) & -\frac{E_s}{2} \\
       -\frac{E_s}{2}     &     \frac{\Phi_0^2}{2L}\left(     N    +     1     -
         \frac{\Phi_{\text{ext}}}{\Phi_0} \right)
     \end{pmatrix} =
     \begin{pmatrix}
       \frac{\Phi_0}{2L}\left( \Phi_0N - \Phi_{\text{ext}}\right) & -\frac{E_s}{2} \\
       -\frac{E_s}{2}   &    \frac{\Phi_0}{2L}\left(   \Phi_0N   +    \Phi_0   -
         \Phi_{\text{ext}}\right)
     \end{pmatrix} =
     \begin{pmatrix}
       -\delta\Phi I_p  + \frac{\Phi_{0}I_p}{2}  & -\frac{E_s}{2} \\
       -\frac{E_s}{2} & +\delta\Phi I_p + \frac{\Phi_{0}I_p}{2}
     \end{pmatrix}
   \end{equation}}

 \noindent we get

 \begin{equation}
   \begin{aligned}
     \left\lbrace\begin{aligned}    \mathcal{H}   &    =   \mathbf{-\frac{\Delta
             E}{2}\begin{pmatrix}  \cos(\theta)  & \sin(\theta)\\\sin(\theta)  &
             -\cos(\theta)
           \end{pmatrix}}\\
         \Delta E & = \sqrt{\epsilon^2+\Delta^2}\\
         \tan(\theta) & = \frac{\Delta}{\epsilon}
       \end{aligned}\right.
   \end{aligned}
 \end{equation}

 \noindent      in     which      case     $\Delta      \equiv     E_s$      and
 $\epsilon =  2I_p\delta\Phi =  \frac{\Phi_{0}}{L_{k}}\delta\Phi$ is  the energy
 difference between neighbouring states (neighbours on the diagonal).



 \subsection{Mooij 2006 paper \cite{mooij2005}}
 \label{sec:mooij-2006-paper}

 \begin{equation}
   \begin{aligned}
     E_{L} & \equiv E_C\\
     E_{J} & \equiv E_s \\
     n_{g} & \equiv f
   \end{aligned}
 \end{equation}

 \noindent


\begin{figure}[h]
  \centering
  \includegraphics[height=7cm]{quantum-phase-slip/mooij_2005_energy_graphs}
  \caption{\small The energy spectra are dual for  the two systems - from. In a)
    $E_C  >> E_J$  so that  charge  is the  relevant  quantum number  and in  b)
    $E_L     >>      E_s$     so      that     phase     is      the     quantum
    number. \cite{mooij2005}. \label{fig:mooij_2005_energy_graphs}}
\end{figure}

\subsection{Derivation}
\label{sec:derivation}

There is  a theory written in  the book \textbf{Devoret: Quantum  Fluctuation in
  electrical circuits} about how \textbf{any  linear circuit can be presented by
  an equivalent circuit with a frequency-dependent resistor}.

\begin{itemize}
\item Current biased: Resistor is in parallel \hfill Phase across junction
\item Voltage biased: Resistor is in series \hfill No of CP transfered.
\end{itemize}

\begin{figure}[h]
  \centering \includegraphics[height=7cm]{quantum-phase-slip/mooij_2005_biases}
  \caption{\small \textbf{JJ on left} vs  \textbf{CQPS on right}.  The capacitor
    is  replaced  with  an  inductor  to make  phase  the  new  defined  quantum
    variable. \label{fig:mooij_2005_biases}}
\end{figure}

The main transformation applied that links the following hamiltonians (we simply
match the inductive and QPS terms from \autoref{eq:hamiltonian-cqps-standalone})
and we add the necessary 2$\pi$ factors
\begin{equation}
  \begin{aligned}
    \mathcal{H}_{JJ} & = E_C\hat{q}^2 - E_J\cos\hat{\varphi} \\
    \mathcal{H}_{CQPS}     &    =     E_L\frac{\hat{\varphi}^2}{(2\pi)^{2}}    -
    E_S\cos(2\pi\hat{q})
  \end{aligned}
\end{equation}

\noindent This requires the following set of transformations:

\begin{itemize}
\item $\hat{q} \rightarrow - \frac{\varphi}{2\pi}$
\item $\hat{\varphi} \rightarrow 2\pi q$
\item $E_s \rightarrow E_J$
\item $E_L \rightarrow E_C$
\item $I \Leftrightarrow \frac{V}{R_{q}}$
\item $Y(\omega) \Leftrightarrow \frac{Z(\omega)}{R_q^2}$
\end{itemize}

\subsection{Inverse shapiro steps}
\label{sec:inverse-shap-steps}

Now if we take  the aforementioned mappings and apply them  to the equation used
to model a typical JJ

\begin{figure}[h]
  \centering \inkfig{8cm}{typical_jj_setup}
  \caption{\small Modelling a JJ\label{fig:typical_jj_setup}}
\end{figure}

\begin{equation}
  I(t) = \red{I_c\sin\varphi} + \frac{\Phi_{0}}{2\pi}\left[ \green{C \frac{d^2\varphi}{dt^{2}}} + \blue{\frac{1}{R}\frac{d\varphi}{dt}} \right]
\end{equation}

\noindent which  leads to  Shapiro steps.   \red{Add explanation  from metrology
  notes}.

Upon subbing in
\begin{equation}
  V(t) = \red{V_c\sin(2\pi q)} + 2e\left[ \green{L \frac{d^2q}{dt^{2}}} + \blue{R\frac{dq}{dt}} \right]
\end{equation}

\noindent where we now deal with the voltage drops across the CQPS (in becomes a
linear circuit).  \textbf{Valid  when the charge is the  relevant quantum number
  (just  like for  JJ phase  is the  quantum  number as  we are  localised in  a
  potential minimum)}

\subsection{Plasma Frequency}
\label{sec:plasma-frequency}

The JJ can be viewed as an inductance  with a shunt capacitor due to it's finite
size   (inductance   actually   depends   on    the   phase   across   the   JJ,
$L_{\text{kin} = \frac{\Phi_{0}}{2\pi I_c\cos(\varphi_0}}$):

\begin{equation}
  \left\{
    \begin{aligned}
      Z & = Z_{c} + Z_L \\
      Z_C & = \frac{1}{j\omega C}\\
      Z_L & = j\omega L
    \end{aligned}
  \right. \Rightarrow Z = \frac{L/C}{j \left( \omega L - \frac{1}{\omega
        C}
    \right)} \Rightarrow \omega_{\text{res}} = \frac{1}{\sqrt{LC}}.
\end{equation}


\noindent
\begin{figure}[h]
  \centering \inkfig{3cm}{lc_resonance}
  \caption{\small    LC     cirucit    that    is    set     up    inside    the
    JJ\label{fig:lc_resonance}}
\end{figure}

\subsection{What sets the energies}
\label{sec:what-sets-energies}

\begin{minipage}{0.45\textwidth}
  \begin{framed}\noindent
    \textbf{JJ}
    \begin{enumerate}
    \item  $E_J  =  \frac{R_{q}}{R_{\square}/N_{sq}}\frac{\Delta(0)}{2}$  \hfill
      $E_J \propto N_{sq}$
    \item $E_C = \frac{e^2}{2C_{\Sigma}}$ \hfill $E_C \propto \frac{1}{Area}$
    \end{enumerate}
  \end{framed}
\end{minipage}
\begin{minipage}{0.45\textwidth}
  \begin{framed}\noindent
    \textbf{CQPS}
    \begin{enumerate}
    \item          $E_L          =          \frac{\Phi_0^2}{2L}$          \hfill
      $E_L \propto \frac{1}{\text{Wire Length}}$
    \item \hfill $E_S \propto \text{Wire length}$
    \end{enumerate}
  \end{framed}
\end{minipage}

\subsection{Standard of voltage}
\label{sec:standard-voltage}

Just like JJ has a damping factor (\autoref{eq:jj-damping}) so does the CQPS

\begin{equation}
  Q_{\text{QPS}}^2 = \beta_L = 2\pi \frac{V_{c}}{2e}\frac{L}{R^{2}} = 2\pi^{2}\frac{E_{s}}{E_{L}}\left( \frac{R_{q}}{R} \right)^2.
\end{equation}

\newpage Too much damping is bad, as on the washboard potential is gives rise to
hysteresis, and  while $E_s$  and $E_L$  are fixed  in order  to operate  in the
voltage localisation regime, it is possible to increase R to 60\,k$\Omega$.

\subsection{Coherent quantum phase slip 2012 \cite{Astafiev_2012}}
\label{sec:coher-quant-phase}

Oleg drove the CQPS system in the two level approximation by a resonant drive.

\begin{framed}\noindent
  The Hamiltonian near a degeneracy point can be written

  \begin{equation}
    \begin{aligned}
      \mathcal{H}        &\approx        -\frac{\varepsilon}{2}\sigma_z        -
      \frac{\Delta}{2}\sigma_x = \begin{pmatrix}
        -\varepsilon/2  &  -\Delta/2 \\
        -\Delta/2 & \varepsilon/2
      \end{pmatrix} \qquad \red{\text{when } \Phi\approx(N+\frac{1}{2})\Phi_0} \\
      \varepsilon & = \frac{\Phi_0}{L}\delta\Phi = 2I_p\delta\Phi  \qquad\qquad \text{controlled by bias flux, and yes, there is a factor of 2 - not an error!}\\
      \Delta & = E_s \qquad\qquad\text{Degeneracy  lifted by the coupling of the
        flux states}
    \end{aligned}
  \end{equation}

  \noindent which can be rotated to give

  \begin{equation}
    \mathcal{H'} = -\frac{\Delta E}{2}\sigma_z \qquad \Delta E = \sqrt{\varepsilon^2 + \Delta^2}
  \end{equation}

  \noindent     Now    we     applying     a     driving    field     (labelling
  $\omega_0=\Delta E/\hbar$).   From \autoref{sec:dipole_coupling}  the coupling
  $\hbar\Omega=                  \left|I_{mw}\right|MI_p\zeta                  =
  MI_p\left|I_{mw}\right|\frac{E_s}{\Delta E}$

    \begin{equation}
      \mathcal{H}'' = -\frac{\hbar\omega_0}{2}\sigma_z-\hbar\Omega\cos(\omega_0 t)\sigma_x,
    \end{equation}

    \noindent         and        applying         a        rotating         wave
    $U(t)  =   \exp\left[-i\frac{\omega_0  t}{2}\sigma_z\right]$   and  removing
    fast-oscillating terms

    \begin{equation}
      \mathcal{H}''' = -\frac{\hbar\Omega}{2}\sigma_x
    \end{equation}

    \noindent            which            evolves           the            state
    $U(t)  =  e^{-i\mathcal{H}'''/\hbar  t}  =  e^{i\Omega  t/2\sigma_x}$  which
    evolves

    \begin{equation}
      \ket{\Psi} = U\ket{0} = \cos(\frac{\Omega t}{2})\ket{0}+e^{i\pi/2}\sin(\frac{\Omega t}{2})\ket{1}.
    \end{equation}

    \noindent \red{Since $h\Omega \ne 0$ only  is $E_s \ne 0$ this will directly
      show CQPS if one finds that the field interacts with the system.}
  \end{framed}

  \subsubsection{Choosing material}
  \label{sec:choosing-material}

  CQPS  occurs   due  to   fluctuations  in  the   order  parameters   $\Psi$  (
  $\left|\Psi\right|^2$ is  proportional to number of  CP.)  \textbf{Suprisingly
    to see CQPS, we  need to have a low CP concentration to  have a big Ginsburg
    parameter}

  \begin{equation}
    G_{i} = \frac{1}{\nu \Delta \xi^3}
  \end{equation}

  \noindent where $\nu=$DOS,  $\Delta$ is the sc gap and  $\xi$ is the coherence
  length.    To   get    small   $\xi$   we   need   to   be    close   to   the
  superconductor-insulator transition at which point

  \begin{equation}
    \text{Cohrence length } \xi \approx \xi_{localisation}
  \end{equation}

  \noindent  \textbf{An  alternative explanation}  is  to  limit the  number  of
  conductive channels

  \begin{equation}
    N_{ch} = \frac{R_{K}}{R}, \quad R_K=\frac{h}{e^{2}}
  \end{equation}

  \noindent  since $E_s  \propto e^{-aN_{ch}}$,  so we  need a  highly resistive
  material.

  \begin{framed}\noindent
    Materials that fit are NbN and TiN where $\xi = 30\,\text{nm}$.

    At around 2.7\,K the sheet inductance is

    \begin{equation}
      \begin{aligned}
        R_{sq} & = 1.7\,k\Omega \\
        L_{sq} & = \frac{\hbar R_{sq}}{\pi\Delta} \approx 0.7\,\text{nH}
      \end{aligned}
    \end{equation}

    \noindent
  \end{framed}

  Another  great way  to get  material parameters  is done  by \textbf{Peltonen}
  \cite{Peltonen_2018}:
  \begin{itemize}
  \item Measure resonator whose frequency:
    \begin{equation}
      \left\{\begin{aligned}
          f_{n} &= \frac{nv}{2L}, \quad n \in \mathbb{Z} \\
          v & = \frac{1}{\sqrt{lc}}
        \end{aligned}\right.
    \end{equation}

  \item Estimate $c_{l}\approx 7\times10^{-10}\,\text{F/m}$ and thus deduce $l$.
  \end{itemize}

  \subsubsection{Fitting of results}
  \label{sec:fitting-results}

  Results are measured  using two-tone spectroscopy and  fitting performed using
  $\Delta E = \sqrt{(2I_p\delta\Phi)^2 + E_s^2}$ to get

  \begin{itemize}
  \item E$_s$ = 10\,GHz \hfill $\zeta = 10\,\text{nm}$
  \item L$_k$ = 42\,nH (compared to prediction from BCS of 23\,nH)
  \item \textbf{Proof that  there is a linear inductance in  the system (not the
      non  linear one  as for  JJ) by  two-tone data  with different  excitation
      paths}

    \begin{figure}[h]
      \centering
      \includegraphics[height=7cm]{2012_astafiev_coherent-quantum-phase-slip/fitting-cqps-transition}
      \caption{\small Energy to excite the system comes from either \blue{second
          tone}, \green{second  tone + first  tone}, \red{two second tone  + one
          first tone}.  In contrast, the prediction  using a JJ RF-SQUID is very
        different. \label{fig:fitting-cqps-transition}}
    \end{figure}

  \end{itemize}

  \noindent It is not identical to  BCS prediction due to exponential dependence
  of the sample on material parameters.

  \subsection{Evidence  for  coherent  quantum phase-slips  across  a  Josephson
    junction array \cite{Manucharyan_2012}}
  \label{sec:evid-coher-quant}

  \begin{framed}\noindent
    \begin{itemize}
    \item A phase slip destroys superconducting order in a narrow wire
    \item There will  be shifting of energies, turning  energy-level shifts into
      linewidths.
    \item Gradient of order parameter  $\Psi$, gives the non-dissipative current
      (supercurrent) \textbf{that is constant along the wire}.
    \item Thinner wire == more phase slips
    \item Phase slip can occur by  $\pm 2\pi$ and preserve the single-valuedness
      of the macroscopic wave function $\Psi$.  Summing up the total phase slips
      that can occur anywhere $m = \sum_im_i $.
    \item Phase slips  do not have to be \textbf{dissapative}  - multiple CPS in
      different location can interefe with each other.
    \item A phase slip dissipates and  energy $I\Phi_{0}$, where $I$ is the bias
      current.
    \item Phase  slip energy can  vary \textbf{by  a lot} so  it may be  hard to
      catch it within the laboratory ranges.
    \end{itemize}

    \begin{framed}\noindent
      \textbf{So in this  experiment, they take a long chain  of JJ (to minimize
        effect of  lumped circuits}. Phase  slips will change the  inductance of
      this chain, which will be registered by a fluxonium qubit.
      \begin{itemize}
      \item  \textbf{Fluxonium  qubit} is  single  JJ  shunted  by an  array  of
        JJ. Readout by coupling it to a resonator
      \item Large $E_{J}/E_C$ mean that phase fluctuations are small - not a lot
        of phase slips.
      \end{itemize}
    \end{framed}
  \end{framed}

  \subsection{Coherent     flux     tunneling      through     NbN     nanowires
    \cite{Peltonen_2013}}
  \label{sec:coher-flux-tunn}
  \begin{framed}\noindent
    \begin{itemize}
    \item  Tunneling rate  depends  exponentially on  the  number of  conducting
      channels $N_{ch}$
    \item For the following
      \begin{equation}
        E_{S} = E_0\exp(-\kappa \bar{\omega})
      \end{equation}
      \noindent one can get

      \begin{equation}
        \begin{aligned}
          E_0 & = \frac{\Delta}{\xi^2} \frac{\left( h/4e^2 \right)}{R_{\square}}l\bar{\omega} \\
          \kappa & = a \frac{\left( h/4e^2 \right)}{R_{\square}}\frac{1}{\xi}
        \end{aligned}
      \end{equation}

      \noindent            where           $l\approx            500\,\text{nm}$,
      $\Delta \approx 1.6\,\text{meV}$, $\xi=4\,\text{nm}$.
    \item BCS theory fails in disordered material
    \end{itemize}
  \end{framed}

  \subsection{Coherent dynamics  and decoherence in a  superconducting weak link
    \cite{Peltonen_2016}}
  \label{sec:coher-dynam-decoh}

  \begin{framed}\noindent
    \begin{itemize}
    \item NbN thickness of $2-3\,\text{nm}$
    \item  First  deposition  creates  the  large  grounds  planes  \ira  second
      deposition using negative  resist \textbf{calixarene} and \textbf{reactive
        ion etching} etches away the pieces of NbN that are uncovered.
    \item  \textbf{Do  not remove  negative  resist  afterwards} -  it  protects
      against ageing.
    \item  Two tone  spectroscopy  to measure  energies -  \textbf{Nont-resonant
        coupling between  qubit and resonator} allows  one to map out  the qubit
      energies through changes in the resonator
    \item Fitting  of $\sqrt{\left(  2I_p\delta\Phi \right)^2  + E_s^2}$  to the
      spectrum allows one to find:
      \begin{itemize}
      \item $I_{p} \approx 40\,nA$
      \item $L_K = \frac{\Phi_0}{2I_{p}} \approx 25\,nH$
      \end{itemize}
      \begin{center}
        \includegraphics[height=4cm]{2016_peltonen_coherent-dynamics-and-decoherence-in-a-superconducting-weak-link.pdf/spectroscopy}
      \end{center}

    \item \textbf{Stark shift} (\autoref{sec:stark-shift}) tells us that
      \begin{equation}
        U\mathcal{H}U^{\dagger} \approx \hbar\omega_ra^{\dag}a + \left[\frac{\hbar\Omega}{2} + \blue{\frac{g^{2}}{\Delta}\left( a^{\dagger}a + \frac{1}{2} \right)} \right]\sigma_z
      \end{equation}

      \noindent depending  on number of  photon in resonator, the  atom's energy
      will shift
    \item When big  powers are applied, the \textbf{Lorentzian  dip} will deform
      into a Gaussian blur. At low power the FWHM=26\,MHz gives a dephasing rate
      (total                                                          dephasing)
      $\Gamma_2 = \pi \times \text{FWHM} = 8\times10^{7}\,\text{per second}$
    \item Authors  show that as  magnetic field is  moved from $\delta\Phi  = 0$
      relaxation rate $\Gamma_1$ stays constant, but decoherence rate $\Gamma_2$
      increases \textbf{due  to low  frequency flux  fluctuations into  the flux
        degree of freedom}
    \end{itemize}
  \end{framed}


  %%% Local Variables:
  %%% mode: latex
  %%% TeX-master: "all_the_notes"
  %%% End:
