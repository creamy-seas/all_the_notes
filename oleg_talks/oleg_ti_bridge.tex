\section{Ti-Bridge}
 \ipic{5cm}{ti_bridge}

 \begin{enumerate}
 	\item The idea is that a small titanium bridge on the end of an aluminium line has kinetic inductance given by the normal resistance as a result of material disorder
 
 \begin{equation}\label{eqn:tiBridge_1}
 	L = \frac{\hbar R}{\Delta \pi}.
 \end{equation}
 
 
 \item If this piece of titanium is placed on the end of a line, then an incoming probe wave will reflect off it, and acquire a phase 
 
 \[
 	\varphi(L),
 \]
 
 \noindent that depends on the kinetic inductance. \red{This phase change is what we want to measure!}
 
 \item Now, it is found that the inductance $ L $ from Eq.~\eqref{eqn:tiBridge_1}, will have a rate of change 
 
 \begin{equation}\label{eqn:tiBridge_2}
 	\frac{\delta L}{L} \propto \frac{1}{N},
 \end{equation}
 
 \noindent where $ N $ is the number of Cooper pairs on the titanium island. $ N\sim10-100 $ achieved on a $ 50\times100\,$nm\ipow{2} piece.
 
 \item We can change the number of cooper pairs on the island by appying radiation to the titanium. The energy needed to excite/relax quasiparticles (electrons) by $ 2\Delta  = hf$ onto the level where they can form/destroy cooper pairs can be derived from the critical temperature of titanium
 
 \[
 	f = \frac{2\times 1.76 T_c}{h},
 \]
 
 \noindent the 2 factor coming from the fact that we want to cross the full superconducting gap $ 2\Delta $.
 \end{enumerate}


 \iframe{\paragraph{Procedure}
	
	\begin{enumerate}
		\item Apply weak radiation $ f_e \approx\iunit{10}{GHz} \ge 2\Delta$ (to not heat up sample) to create/destroy cooper pairs, $ N $ \ra and change inducatnce in Eq.~\eqref{eqn:tiBridge_2}, $ L $;
		\item Measure the phase of the probing wave $ f_p << f_e $ that depends on the inductance $ \varphi(L) $.
	\end{enumerate}
	\red{In order to see the photoresponse in $ f_p $, what we do is modulate the excitation signal $ f_e $ with a carrier wave of 1\,MHz:}
	\ipic{4cm}{beat_1}
	\red{This will lead to Copper Pairs excitations being suppresed at a 1\,MHz frequency, and the probing wave ($ f_p\approx \iunit{2}{GHz} $)will therefore also have a beating at 1\,MHz in response to this (phase $ \varphi(L) $ is igonred):}
	
	\[
		\text{Probing}(t) = \sin(\omega_p t)\sin(\omega_{mod}t) \equiv \frac{1}{2}\left[\cos((\omega_p - \omega_{mod})t) - \cos((\omega_p + \omega_{mod})t) \right]
	\]
	
	\noindent which will result in sideopeaks either side of $ f_p $ @ $ f_p \pm \iunit{1}{MHz} $.
}
	\ipicCaption{3cm}{side_peak}{The probing wave will also have a modulation of \iunit{1}{MHz}, so two sidepeaks will form.}
	
	Beyond a certain modulation, $ f_{mod} $, the characteristic time for an excitation to relax back to Cooper Pairs $ \approx \iunit{10\,\mu}{s} $ means that such a relaxation doesn't have time to occur \ra \red{\textbf{In order to see modulation of the probe signal, $ f_p $, we need cosntant switching between the N = 99 and N=100 states. If not, then $ f_p $ will react to only N=99, and reflect at the same frequency.{\Huge If, however, N=99\lra N=100 changes in the time domain, then in frequency domain we will get these sideband components} Thus we can probe relaxation time of the Cooper pairs by seeing when these peak dissapear}}.
	
	So beyond a certain $ f_{crit} $ these side peaks should dissappear. Thus we can meaured the life of the excitation
	
	\[
		t_{CP} \approx \frac{1}{f_{crit}}.
	\]
	
  \subsection{Amplitude modulation}
  \ipic{7cm}{beat_7}
   So lets go through amplitude modulation step by step:
   \begin{enumerate}
   	\item Send in a modulated generator signal 
   	\[
   		\sin(\omega_e t)*\sin(\omega_{mod} t),
   	\]
   	\noindent which has two frequnecy components
   	\[
   		\red{F\lbrace\text{GEN}\rbrace = f_e\pm f_{mod}}.
   	\]
   	\ipic{4cm}{beat_2}
   	\item Send in VNA signal for probing into the system at a frequency 
   	\[
   		\red{F\lbrace\text{VNA}\rbrace = f_p.}
   	\]
   	\item Combine and send to the system. The cooper pairs are going to be excited by the excitation pulse, with the modulation frequency
   	
   	\ipic{5cm}{beat_3}
   	
   	\[
   	\red{F\lbrace\text{CP}\rbrace = f_{mod}.}
   	\]
   	\item The probing wave is going to be reflected and the bridge, unergoing phase/amplitude modulation in the time domain, by the number of CP on the island.
   	
   	Above we saw that the number of CP will have a frequency $ f_{mod} $. Thus the output signal will have 2 frequencies
   	
   	\ipic{4cm}{beat_5}
   	\[
   	\red{F\lbrace\text{PROBE}\rbrace = f_p\pm f_{mod}.}
   	\]
   	\item At the output we tune the SPA to measure @ $ f_p \pm f_{mod}$ to catch the amplitude of the probing signal on one of the sidebands and plot a heat map:
   	\ipic{4cm}{beat_6}
   \end{enumerate}

% \subsection{Phase modulation}
%The setup is the same as for ampltiude modulation up to step 4. At this point we will indirectly probe the phase by looking at the quadratures of the signal.
%
%We use a Marki Mixer that will give two qudratures at frequencies $ f_p \pm f_{mod} $:
%\ipic{5cm}{beat_8}
%
%\[
%	\ialigned{
%		& I \propto \re{\text{Signal}} \propto \cos(\phi)\\
%		& Q \propto \im{\text{Signal}} \propto \sin(\phi)\\
%	}
%\]

 \subsection{Time resolved measurements}
  \ipic{10cm}{beat_10}
  
  \noindent A train of excitation pulses was applied to excite the cooper pairs, and a VNA `bucket' was used for readout:
  \begin{itemize}
  	\item Generator 15\,GHz, 3\,dBm \ra mixed with Keysight pulses \ra creates excitation pulses;
  	\item VNA 3\,GHz, -25\,dBm \ra mixed with Keysight pulses \ra create readout pulses;
  	\item The VNA accumulates pulses to plot \red{\textbf{the scattered signal as a function of the delay after excitation, $ \tau $.}}
  \end{itemize}
	$ T = 20\mu$s have excitation length of 2$ \mu $s and read length of $ 200 $ns.
  \ipic{5cm}{beat_11}
  
 
\newpage
