\section{Representing 2-level system as a spin}
\label{spin12}
We call the magnetic dipole the quantity

\begin{equation}\label{eqn:gqmMoment}
  \vec{\mu} = IA,
\end{equation}

\noindent  the product  of the  current around  a loop,  by the  area is
encompasses. It points perpendicular to the plane of the loop.

When placed  inside a uniform  magnetic field,  the torque on  the setup
will      be      (can      be     proven      by      geometry      and
$\vec{F} = l\cdot\ \vec{I}\times \vec{B}$

  \begin{equation}
    \tau = \vec{\mu}\times\vec{B},
  \end{equation}

  \noindent which  attempts to  rotate the moment  to the  lowest energy
  configuration,  when  the  magnetic  moment falls  in  line  with  the
  magnetic field. The energy of a magnetic moment we measure relative to
  the case  when the magnetic moment  is $\perp$ to the  magnetic field,
  and taking the integral $\int\tau d\theta$, one gets the energy

  \begin{equation}
    U = -\vec{\mu} \cdot \vec{B}.
  \end{equation}

  \noindent Evaluating the moment for a simple electron in orbit like in
  \autoref{eqn:gqmMoment}

  \begin{equation}
    \mu = \big[-|e|\frac{v}{2\pi r}\big]\pi r^2 = -\frac{|e|}{2}rv,
  \end{equation}

  \noindent and remembering that $mrv = L$ - the angular momentum

  \begin{equation}
    \vec{\mu}= -\frac{|e|}{2m}\vec{L} \qquad \qquad U = -\frac{|e|}{2m}\vec{L}.\vec{B}.
  \end{equation}

  \noindent  Following  commutation   relations  for  angular  momentum,
  follows the quantisation  of angular momentum into  ladder like states
  (we guess the  way that $L_z$ and $L^2$ act  on their eigenstates. The
  general direction  that this is  done for is the  z-direction, whereby
  $L_z$ and  $\hat{L}^2$ both become  quantised. In momentum  space this
  means that  for a  fixed angular  momentum in  one direction,  we know
  nothing of the other two components.

  The Stern-Gerlach experiment on measuring the deflection of electrons,
  resulted  in the  speculation that  there  was an  additional type  of
  `hidden' momentum - the spin. By analogy the magnetic moment from spin
  is

  \begin{equation}
    \vec{\mu}_s = -g_e\frac{|e|}{2m_e}\vec{S}.
    \label{eqn:gqmSpin}
  \end{equation}

  \noindent Using similar arguments, one finds two eigenvectors

  \begin{equation}
    \ket{s=1/2,m=+1/2} = \begin{bmatrix}
      1\\0
    \end{bmatrix}\qquad \ket{s=1/2,m=-1/2} = \begin{bmatrix}
      0 \\1
    \end{bmatrix},
  \end{equation}

  \noindent  which  are  expressed  in their  eigenbasis,  that  form  a
  complete set (as long as we don't introduce other degrees of freedom).


  Writing out the way spin operators affect these eigenstates (i.e. when
  we measure spin)

  \begin{equation}
    \begin{aligned}
      \hat{S}_z\ket{\uparrow} = +\frac{\hbar}{2}\ket{\uparrow} & \hat{S}_z\ket{\downarrow} = -\frac{\hbar}{2}\ket{\downarrow}\\
      \hat{S}^2\ket{\uparrow} = \frac{3}{4}\hbar^2\ket{\uparrow} & \hat{S}^2\ket{\downarrow} = \frac{3}{4}\hbar^2\ket{\downarrow}\\
      \hat{S}_+\ket{\uparrow} = 0 &
      \hat{S}_+\ket{\downarrow} = \hbar\ket{\uparrow}\\
      \hat{S}_-\ket{\uparrow} = \hbar\ket{\downarrow} &
      \hat{S}_-\ket{\downarrow} = 0\\
    \end{aligned},
  \end{equation}

  \noindent and taking the  sandwiches $\bra{\uparrow}S\ket{\downarrow}$ to evaluate the
  matrix elements (and using $S_x = \frac{S_++S_-}{2}$ etc)

  \begin{equation}
    \begin{aligned}
      \hat{S}_z = \frac{\hbar}{2}\sigma_x  \qquad \hat{S}_y = \frac{\hbar}{2}\sigma_y
      \qquad \hat{S}_z = \frac{\hbar}{2}\sigma_z
    \end{aligned}
    \label{eqn:gqmPauli}
  \end{equation}

  \noindent  Thus  the  general  spin operator  is  found  by  combining
  \eqref{eqn:gqmSpin} and Eq. \ref{eqn:gqmPauli}

  \begin{equation}
    \hat{\mu} = -g_e\frac{|e|}{2m_e}\hat{S} = -g_e\frac{|e|}{2m_e}\frac{\hbar}{2}\begin{bmatrix}
      \sigma_x\\\sigma_y\\\sigma_z
    \end{bmatrix} = -\mu\begin{bmatrix} \sigma_x\\\sigma_y\\\sigma_z
    \end{bmatrix}
  \end{equation}

  \noindent The Hamiltonian $\mathcal{H} = \hat{\mu}.\hat{B}$ is then

  \begin{equation}
    \begin{aligned}
      \mathcal{H} & = -\mu\left(B_x\sigma_x+B_y\sigma_y+B_z\sigma_z\right)\\
      &                          \red{\mathbf{                         =
          -\frac{\hbar}{2}\bigg(\omega_x\sigma_x+\omega_y\sigma_y+\omega_z\sigma_z\bigg)}}
    \end{aligned},
  \end{equation}

  \noindent where we introduced  a convenient frequency. \textbf{Any two
    level system has  this energy spectrum, with an  offset energy}. The
  dynamics are then:
  \begin{equation}\label{eqn:evolutionBloch}
    i\hbar\frac{d\vec{\sigma}}{dt} = \mathcal{H}\vec{\sigma} \propto \vec{B}\cdot\vec{\sigma}
  \end{equation}

  Further,  the density  matrix for  an arbitrary  two level  system can
  always   be    written   in    the   form    (remember   normalisation
  $|\alpha|^2+|\beta|^2 = 1$)

  \begin{equation}
    \begin{bmatrix}
      \alpha\\\beta
    \end{bmatrix}    \qquad\Rightarrow\qquad    \begin{aligned}   \rho    &
      = \begin{pmatrix} \alpha\\\beta
      \end{pmatrix}\begin{pmatrix}
        \alpha^* & \beta^*
      \end{pmatrix}       =      \begin{pmatrix}       |\alpha|^2      &
        \alpha\beta^*\\\beta\alpha^*&|\beta|^2
      \end{pmatrix}\\
      & = \begin{pmatrix}
        |\alpha|^2 & (\re{\alpha}+i\im{\alpha})(\re{\beta}-i\im{\beta})\\ (\re{\beta}+i\im{\beta})(\re{\alpha}-i\im{\alpha}) & |\beta|^2\\
      \end{pmatrix}\\
      & = \begin{pmatrix}
        |\alpha|^2 & \re{\alpha}\re{\beta} +\im{\alpha}\im{\beta} +i(\im{\alpha}-\im{\beta})\\ \re{\alpha}\re{\beta} + \im{\alpha}\im{\beta} -i(\im{\alpha}-\im{\beta}) & |\beta|^2\\
      \end{pmatrix}\\
      & = \begin{pmatrix}
        |\alpha|^2 & X + iY\\ X - iY & 1-|\alpha|^2\\
      \end{pmatrix}\\
      & = \begin{pmatrix} \frac{1}{2} & 0\\0&\frac{1}{2}
      \end{pmatrix}  + \begin{pmatrix}  |\alpha|^2-\frac{1}{2} &  0\\0 &
        -\bigg[|\alpha|^2-\frac{1}{2}\bigg]
      \end{pmatrix}+ \begin{pmatrix} 0 & X\\X&0
      \end{pmatrix} + \begin{pmatrix} 0 & +iY\\-iY&0
      \end{pmatrix}\\
      &      \mathbf{\red{\rho     =      \frac{1}{2}\bigg(\mathbb{I}     +
          u_x\sigma_x+u_y\sigma_y+u_z\sigma_z\bigg)}}
    \end{aligned}
  \end{equation}


  \newpage
