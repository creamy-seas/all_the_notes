\section{The dipole approximation \label{sec:dipole}}
Classically, for a dipole $\vec{D} = q\vec{r}$, for example an atom and an electron orbiting at a radius $\vec{r}$. In an electric field $\vec{E}$, is gives rise to an interaction energy

\begin{equation}
  \label{}
U = - \vec{D}\cdot\vec{E}.
\end{equation}

\noindent The expression for $\vec{E}$ has already been found in Sec.\ref{subsec:singleMode} to be $E_{0x}(\hat{a}+\hat{a}^\dagger)\sin(kz)$.

As for the quantum operator for $\vec{D}$, we begin by assuming a two level atom, with a Hamiltonian

\begin{equation}
\begin{aligned}
\hat{H} = 0\iketbra{g}{g} + \hbar\omega_a\iketbra{e}{e},
\end{aligned}
\end{equation}

\noindent for a separation $\hbar\omega_a$ between the two levels. This approximation for a many level atom is good, as long as the photon mode energy $\hbar\omega_j=\hbar\omega_a$, with all other transitions off resonance. 

With such a two level system, the matrix for $\vec{D}$ will be

\begin{equation}
\begin{bmatrix}
\bra{g}\hat{D}\ket{g} & \bra{g}\hat{D}\ket{e} \\
\bra{e}\hat{D}\ket{g} & \bra{e}\hat{D}\ket{e}.
\end{bmatrix}
\end{equation}

\noindent However, from parity ($\vec{r}$ is odd),  $\bra{g}\hat{D}\ket{g}=\bra{e}\hat{D}\ket{e} \equiv 0$, and so only the off diagonal elements are non zero. We label $\bra{g}\hat{D}\ket{e} = \bra{e}\hat{D}\ket{g} \equiv \vec{d}$ resulting in

\begin{equation}
\textcolor{red}{\hat{D} =  \begin{bmatrix}
	0 & \vec{d}\\\vec{d} & 0
	\end{bmatrix} = \vec{d}\left(\iketbra{g}{e}+\iketbra{e}{g}\right).
	\label{eqn:dipole1}}
\end{equation}

\newpage
