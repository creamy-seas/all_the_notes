../../../syncFiles/latex/_preamble.tex
../../../syncFiles/latex/_phd_command_file.tex
\newcommand{\mytitlepage}[3]{%titlepage command 
	\begin{titlepage}
		\centering\vspace*{0.6in}\bgroup
		\Huge\bfseries #1 \par
		\egroup\vspace{0.5in}\bgroup
		\Huge #2\\[0.1in]
		\egroup\vspace{0.2in}\bgroup
		\Large Ilya Antonov\\[0.08in]
		\egroup\vspace{0.2in}\bgroup
		\Large Supervisor: Prof. Oleg Astafiev
		\egroup
		\par\vspace{1.5in}
		{\LARGE\bfseries Abstract}
		\par\vspace{0.1in}\hrule\vspace{0.1in}
		\begin{minipage}{0.95\textwidth}
			#3
		\end{minipage}
		\vspace{0.1in}\hrule\vspace{0.2in}\vfill
		\today
	\end{titlepage}
}

\graphicspath{{images_major_project/diagrams_photos/}{images_major_project/data_simulations/}{images_major_project/setup/}}


\begin{document}
	\title{Will quantum computation outperform off its classical analogue?}
	\author{I. Antonov}
	\mytitlepage{PH4100}{Characterisation of transmission and emission properties of a four Josephson junction flux-degree-of-freedom artificial atom}{Over the past 8 years, a lot of work has been done with superconducting artificial atom and of particular interest has been their unparallel, strong interaction with propagating electromagnetic waves. This results in an improvement on properties demonstrated by natural atoms, and gives the ability of studying fundamentally new quantum optics phenomena, uannatiable in classical systems.
		
	This project studied interactions between a superconducting loop interrupted by four Josephson junction, acting as a Three-Level artificial atom, and different permutations of externally applied microwave fields that drove resonant transitions within the system. The dynamics of the system were considered theoretically and compared with the results of transmission and emission measurements. Decoherence parameters of the system were extracted from numerical fittings and a qualitative agreement was achieved for the emission spectrum of the atom when it is subjected to two independent microwave fields.}

\tableofcontents
% Counter for each new subsection
\newcounter{iCounter}[subsection]
\newpage
 
\section{Background and motivation\label{sec:BackgroundAndMotivation}}
 The field of superconducting devices in quantum optics arose as a result the ongoing research on quantum computers, whose progress is driven by their theoretical capability of solving classical problems in exponentially less time \cite{feynman_SimulatingPhysicsComputers,grover_FastQuantumMechanicalAlgorithmForDatabaseSearch,discreteLog}. The operational entities for these computers are quantum bits, realised by Two-Level quantum systems. There has been a range of candidates for their practical realisation and of particular interest are macroscopic superconducting qubits working on the charge, flux, phase, mix of charge and flux, degrees of freedom \cite{nakamura_CoherentControlofQuantumStates, mooij_FluxQubit, rabiOscillationsInJJQubit, chargeInsenstitiveQubit}.
 
 Whereas parameters of natural quantum systems, such as the hyperfine splitting of energy levels in calcium \cite{Leibfried:00}, is fixed, superconducting circuits can be designed with a much greater flexibility. Referred to as artificial atoms, they can be supplied with previously unattainable properties, such as coupling strengths to external photon fields being as strong as the systems characteristic transition energies \cite{Atomicphysicsandquantumoptics}.
 
 This broader range of system parameters means that one has the ability to study quantum optics phenomena that are not possible in classical optical systems, such as quantum wave mixing, dynamical Cassimir effect, amongst several others \cite{Atomicphysicsandquantumoptics}. There are also significant improvements to standard quantum optics phenomena, casing point being the photon transistor which reaches a 12\% attenuation rate in natural atom-photon system \cite{Hwang:2009aa} compared to 90\% in an artificial atom-photon system \cite{astafiev_ResonanceSingleAtom}.
 
 It is important to capitalise on these advantages of artificial atoms by attaining coherent control over artificial atom-photon interactions. There are a number of interesting experiments that can been done on this topic, which a single six month-long project could simply not to justice. Thus, in accordance with the work of the research team I was assigned to, I chose to study a Three-Level artificial atom and its interaction with microwave photons, the end goal being able to characterise the dynamics of the system and observe novel phenomena of quantum optics.
 
 The report has the following structure: Chapter~\ref{sec:Theory} is intended to give an overview of theoretical description of the work done in the project, starting with the structure of the Three-Level artificial atom, it's interactions with a microwave field and ending with how its dynamics can be measured experimentally; Chapter~\ref{sec:ExperimentalWork} covers the design and fabrication of the artificial atom and the measurement setup; Chapter~\ref{sec:Outcomes} shows the experimental results and compares with theoretical predictions made in Chapter~\ref{sec:Theory}; Chapter~\ref{sec:Conclusion} gives a conclusion and proposals for further work.
 \newpage
\section{Theoretical background\label{sec:Theory}}
 This chapter provides the theory required to interpret the experimental results in Chapter~\ref{sec:Outcomes}. Section~\ref{subsec:Atom} describes the artificial atom - the four Josephson junction loop - central to our experiment. Sec.~\ref{sec:Interactions} investigates how capacitive coupling is used to interact the artificial atom with microwave photons in a transmission line, the resultant dynamics of which are studied in Sec.~\ref{subsec:3LevelAtom}. Sec.~\ref{subsec:Scattering} looks at how this affect emissions from the artificial atom, and the experimental interpretations of these emission for two different microwave field configurations are given in Sec.~\ref{subsec:singleDrive} and Sec~\ref{subsec:DoubleDrive}.
 
 \subsection{The artificial atom\label{subsec:Atom}}
  The artificial atom in the experiment was a macroscopic superconducting loop interrupted by four Josephson junctions (JJ) \cite{coupledSuperconductors}, biased by an external magnetic flux $ \Phi $, the circuit for which is shown in Fig.~\ref{theoryCircuitModel}. Three JJs arrangements have been studied previously \cite{mooij_PersistentCurrentQubit,thesisExperimentsonSuperconductingQubits}, but more symmetrical four JJs devices are favoured in this experiment, since they demonstrate better persistent current properties, while retaining qualitatively similar energy dispersions \cite{DephasingofaSuperconducting}.
  
  On the assumption that the JJs are dissipationless, they can be modelled as having a parallel capacitance, $ C_{J} $, which gives rise to two energy scales that determine the operational regime of the atom: 
  
  \begin{itemize}
  	\item The Josephson coupling energy $ E_J $ arising from the Cooper Pair (CP) current across the potential difference of a junction;
  	\item The electrostatic Coulomb energy $ E_c=(2e)^2/2C $ from the presence of CP on the JJs. 
  \end{itemize}

  The artificial atom is operated in the flux regime, whereby the bias $ \Phi $ controls the broad energy spectrum and charge interactions manifest themselves at energy degeneracy points \cite{Atomicphysicsandquantumoptics}, by having a dominant Josephson coupling energy, $ E_J/E_c \approxeq 10 $.
    
  \begin{figure}
  	\ipic{6cm}{atomCircuit}
  	\caption{\small\textbf{The four Josephson junction artificial atom}. \textbf{a)} The junctions of the right-hand side of the loop have Josephson energies $ E_J $ and capacitances $ C_J $, and the junction on the left-hand side has a Josephson energy and capacitance $ \alpha $ times larger. An external flux $ \Phi $ biases the superconducting loop; \textbf{b)} A dissipationless Josephson junction is modelled with a parallel capacitor, which accounts for the physical geometry of the device giving it a finite capacitance.}
  	\label{theoryCircuitModel}
  \end{figure}

  The evaluation of the Hamiltonian, $ \mathcal{H} $, that governs the dynamics of the artificial atom, has been performed by Yueyin Qiu et. al \cite{FourJunctionSuperconductingCircuit}. In the derivation one writes out the Lagrangian, $ \mathcal{L} = \mathcal{T} - \mathcal{U} $, for the system that is composed of the kinetic energy term 
  	\begin{equation}
  	\mathcal{T} =  \frac{1}{2}\sum_{i}C_iV_i^2,
  	\end{equation} 
  	\noindent originating from the voltages $ V_i =  \Phi_0\dot{\phi}_i/2\pi$ across each JJ, and the potential energy term
  	
  	\begin{equation}
	  	\mathcal{U} = \sum_{i=1}^{4}E_{Ji}\big(1-\cos(\phi_i)\big),
  	\end{equation} 
  	\noindent originating from energy on each JJ, where $ \phi_i $ is the phase across the $ i^{\text{th}} $ JJ. The Hamiltonian
  
  \begin{equation}
  	\mathcal{H}(\phi_1,\phi_2,\phi_3,\Phi) = \sum_i\difffrac{\mathcal{L}}{\dot{\phi}_i}\dot{\phi}_i - \mathcal{L}(\phi_1,\phi_2,\phi_3,\Phi),
  \end{equation} 
  
  \noindent describes a particles in a thee dimensional phase space $\big( \phi_1,\phi_2,\phi_3 \big)$ with a periodic potential $ \tilde{U}(\phi_1,\phi_2,\phi_3) $, shown in Fig.~\ref{theoWells}. The fact that the fourth phase, $ \phi_4 $, does not appear in the expression is a result of flux quantisation in a superconducting loop \cite{IntroductiontoSolidStatePhysics}, which reduces the number of degrees of freedom in the system.
  
  The eigenstates of the \schrodinger equation subject to a periodic potential
  
  \begin{equation}
	  \mathcal{H} \varPsi(\vec{\phi})  = E \varPsi(\vec{\phi}),
	  \label{schro}
  \end{equation}
  
  \noindent are Bloch waves \cite{TheoryoftheBloch-waveoscillationsinsmallJosephsonjunctions}, and correspond to persistent current states in the loop. The eigenenergies form bands, the numerical results of which is presented in Fig.~\ref{theoWells}. 
  
  The non-equidistant energy separation of the level, achieved through the anharmonic potential created by the non-linear JJs \cite{ioChunHoi}, allows one to individually address each transition. The flux bias can be used to choose a desirable energy configuration for the Three-Level atom, so that the atomic transition frequencies $ \omega_{31}, \omega_{32}, \omega_{21}$, defined in Fig.~\ref{theoWells}, do not coincide with any noise in the local environment, which could stimulate unwanted transitions.
  
  \begin{figure}
	\ifigure{7cm}{didntdo}
  	\caption{\small\textbf{Energy properties of the four JJ artifical atom.} \textbf{a)} The potential, $ \tilde{U}(\phi_1,\phi_2,0)$, landscape of the JJ system for a flux bias of $ 5\Phi_0 $. It is known from solid state physics that Bloch wave states will conform to this periodicity; \textbf{b)} Numerical evaluation of the eigenenergies of Eq.~\eqref{schro} for $ \alpha=0.6 $ gives the energy spectrum of the atom as a function of the bias flux, $ \Phi $. In the experiment, the three lowers energy levels are manipulated with microwave photons (Figure is taken from the paper by  Yueyin Qiu et. al \cite{FourJunctionSuperconductingCircuit}); \textbf{c)} At a particular flux bias, one defines a Thee-Level system with transition energies $ \hbar\omega_{31},\hbar\omega_{32},\hbar\omega_{21} $ for the \iket{3}\lra\iket{1}, \iket{3}\lra\iket{2}, \iket{2}\lra\iket{1} transitions respectively.}
  	\label{theoWells}
  \end{figure}
   
  
 \subsection{Interaction with the artificial atom\label{sec:Interactions}}
 The artificial atom, detailed in Ch.\ref{sec:Theory}, has characteristic $ \iket{i} \lra \iket{j}$ transition energies, $ E_{ij} = \hbar\omega_{ij} $, that correspond to microwave frequencies, and can thus detected by its resonant absorption and reemission of microwave photons \cite{astafiev_ResonanceSingleAtom,UltimateOnChipQuantumAmplifier,ElectromagneticallyInducedTransparency}. These photons travel down a transmission line in the form of ac-voltage fields, and interact with the atom via capacitive coupling.
 
 \begin{figure}
 	\ipic{8cm}{tranLine}
 	\caption{\small \textbf{Capacitive coupling the qubit to the transmission line} is mediated through a capacitor. \textbf{a)} The transmission line delivers incident microwave photon, $ V_{L}^{+} $, and guides the photons emitted by the atom, $ V_{L}^{-}, V_{R}^{-} $. When the angular frequency of the incident field, $ \omega^{d} $, is in resonance with an atomic transition \iket{i}\lra\iket{j}, the two levels become coupled. The coupling strength is determined by the Rabi frequency of the applied field, $ \Omega_{ij} $, which in turn depends on coupling parameters such as $ C_c $; \textbf{b)} Case for a \iket{2}\lra\iket{3}, \iket{1}\lra\iket{3} drive, with Rabi frequencies $ \Omega_{32} $, $ \Omega_{31}  $, respectively. \textbf{c)} Case for a \iket{1}\lra\iket{2}, \iket{1}\lra\iket{3} drive, with Rabi frequencies $ \Omega_{21} $, $ \Omega_{31}  $, respectively.}
 	\label{fig:transmissionLine}.
 \end{figure}

  Capacitive coupling, demonstrated in Fig.~\ref{fig:transmissionLine}, results in an interaction energy \cite{ioChunHoi,thesisQuantumInformationProcessing} 
  
  \begin{equation}
	  {U}_{\text{int}} = - \mu V_{L}^{+},
	  \label{theoIntEnergy}
  \end{equation}
  
  \noindent where
  
  \begin{equation}\label{theoField}
	  V_{L}^{+}=V_0\cos(\omega^{d} t),
  \end{equation} 
  
  \noindent is the sinusoidal voltage in the transmission line, and $ \mu $ is the strength of the atom-transmission line coupling, determined by the parameters of the circuit. When the angular frequency of this drive, $ \omega^{d} $, matches one of the \iket{i}\lra\iket{j} transition frequencies, $\omega_{ij} $, the two states become coupled. This interaction is quantified in the Hamiltonian with $ \ketbra{i}{j}, \ketbra{j}{i} $, terms weighted by the interaction energy Eq.~\eqref{theoIntEnergy}
  
  \begin{equation}
  	\mathcal{H}_{\text{int}} = - \mu V_0\cos(\omega^{d} t) \bigg[\ketbra{i}{j}+\ketbra{j}{i}\bigg].
  	\label{theoInt}
  \end{equation}

  \noindent Labelling $ \omega^{d} = \omega^{d}_{ij} $, to distinguish the transition driven by the field, and defining the Rabi frequency as 
  
  \begin{equation}\label{rabiRelation}
  	\hbar\Omega_{ij} = \mu V_0,
  \end{equation}
  
  \noindent Equation~\eqref{theoInt} becomes
  
  \begin{equation}\label{theoIntFinal}
  		\mathcal{H}_{\text{int}} = - \hbar\Omega_{ij}\cos(\omega^{d}_{ij} t) \bigg[\ketbra{i}{j}+\ketbra{j}{i}\bigg].
  \end{equation}
 		 	
   \noindent The coupling constant $ \mu $, that converts the amplitude of the driving field, $ V_0 $, to the Rabi frequency, $ \Omega_{ij} $, is usually determined experimentally through Rabi oscillations \cite{rabiOscillationsInJJQubit}, or Mollow triplet measurements \cite{observationofresonancefluorescence,measurementofAutlerTownes}.
   
   A 1D transmission line is used to to guide the microwave field, since it ensures maximal matching of the electromagnetic modes incident on the atom and constrains any emissions by the atom to propagation along two directions, which is an important factor to mediate destructive interference as in Sec.~\ref{subsec:Scattering}. \cite{astafiev_ResonanceSingleAtom}.
    
 \subsection{Dynamics of a three level atom\label{subsec:3LevelAtom}}

  
  The following discussion assumes driving fields $ \Omega_{32} $, $ \Omega_{31} $, that are almost in resonance with the atomic transitions \iket{2}\lra\iket{3} and \iket{1}\lra\iket{3}, as shown in Fig.~\ref{fig:transmissionLine}b. The treatment of systems under the \iket{1}\lra\iket{2} and \iket{1}\lra\iket{3} drive combination follows an analogous argument, the result for which will be stated. The two cases are distinguished with subscripts $ a $ and $ b $ respectively.
  
  The Hamiltonian for a Three-Level atom, in a corresponding basis of eigenstates \iket{1}, \iket{2}, \iket{3}, by which the three levels are labelled, can be written as
  
  \begin{equation}
	  \mathcal{H}_{\text{atom}} = \begin{pmatrix}
	  E_1 & 0 & 0\\0& E_2 & 0 \\0&0&E_3
	  \end{pmatrix}=\begin{pmatrix}
	  E_3-\hbar\omega_{31} & 0 & 0\\0& E_3-\hbar\omega_{32} & 0 \\0&0&E_3
	  \end{pmatrix},
	  \label{rwaAtomicHamil}
  \end{equation}
  
  \noindent where $ \hbar\omega_{31} = E_3-E_1 $ and $ \hbar\omega_{32} = E_3-E_2$. The interaction term Eq.~\eqref{theoIntFinal} for the \iket{2}\lra\iket{3}, \iket{1}\lra\iket{3}, driving arrangement will be written in matrix form as
  
  \begin{equation}
  	\mathcal{H}_{\text{int}} = -\hbar\Omega_{31}\cos(\omega_{31}^{d}t) \begin{pmatrix}
  		0 & 0 & 1\\0&0&0\\1&0&0
  	\end{pmatrix}-\hbar\Omega_{32}\cos(\omega_{32}^{d}t) \begin{pmatrix}
  		0 & 0 & 0\\0&0&1\\0&1&0
  	\end{pmatrix},
  	\label{rwaDriveHamiltonian}
  \end{equation}
  
  \noindent where the non-zero matrix elements indicate the states coupled by the fields. The total Hamiltonian, $ \mathcal{H}_{a} = \mathcal{H}_{\text{atom}}+\mathcal{H}_{\text{int}}$, is a sum of Eq.~\eqref{rwaAtomicHamil}, \eqref{rwaDriveHamiltonian}, written for convenience with complex exponentials,
  
  {\small \begin{equation}
	  \begin{aligned}
		  \mathcal{H}_{a}\ & {= \begin{pmatrix}
		  E_3-\hbar(\omega_{31}^{d}-\delta\omega_{31}) & 0 & -\frac{\hbar\Omega_{31}}{2}\bigg(e^{i\omega^{\text{d}}_{31}t}+e^{-i\omega^{\text{d}}_{31}t}\bigg)  \\   	0 & E_3-\hbar(\omega_{32}^{d}-\delta\omega_{32})  & -\frac{\hbar\Omega_{32}}{2}\bigg(e^{i\omega^{\text{d}}_{32}t}+e^{-i\omega^{\text{d}}_{32}t}\bigg)  \\   	-\frac{\hbar\Omega_{31}}{2}\bigg(e^{i\omega^{\text{d}}_{31}t}+e^{-i\omega^{\text{d}}_{31}t}\bigg)  & -\frac{\hbar\Omega_{32}}{2}\bigg(e^{i\omega^{\text{d}}_{32}t}+e^{-i\omega^{\text{d}}_{32}t}\bigg) & E_3 \\
		  \end{pmatrix}}\\& \qquad\qquad\qquad\qquad\qquad\qquad = \begin{pmatrix}
		  H_{11} & H_{12} & H_{13} \\   	H_{21} & H_{22} & H_{23} \\   	H_{31} & H_{32} & H_{33} \\
		  \end{pmatrix},
	  \end{aligned}
	  \label{rwaTotalHamiltonian}
  \end{equation}}
  
  \noindent where $  \delta\omega_{31} = \omega_{31} - \omega^{\text{d}}_{31}, \delta\omega_{32} = \omega_{32} - \omega^{\text{d}}_{32}$ represent the detunings of the driving fields from the resonant frequencies of the atom. The Hamiltonian Eq.~\eqref{rwaTotalHamiltonian} governs the time evolution of the system state 
  
  \begin{equation}
	  \ket{\varPsi} = c_1\ket{1}+c_2\ket{2}+c_3\ket{3},
  \end{equation} 
  
  \noindent through the standard \schrodinger equation, which takes on the matrix form
  
  \begin{equation}
  \begin{pmatrix}
  H_{11} & H_{12} & H_{13} \\   	H_{21} & H_{22} & H_{23} \\   	H_{31} & H_{32} & H_{33} \\
  \end{pmatrix}\begin{pmatrix}
  c_1\\c_2\\c_3
  \end{pmatrix} = i\hbar\begin{pmatrix}
  \dot{c_1} \\ \dot{c_2}\\\dot{c_3}
  \end{pmatrix}.
  \label{rwaTotalHamilMatrix}
  \end{equation}
  
  \noindent By applying a time dependent transformation
  
  \begin{equation}
  \label{eqn:InteractionTransformation}
  \widetilde{c_i} = e^{i\phi_{i}(t)}c_i,
  \end{equation}
  
  \noindent where $ i=1,2,3 $, one can rewrite Eq.~\eqref{rwaTotalHamilMatrix} as
  
  \begin{equation}
	  \begin{pmatrix}
	  H_{11}-\hbar\dot{\phi}_1 & H_{12}e^{i\phi_{12}} & H_{13}e^{i\phi_{13}} \\  H_{21}e^{i\phi{21}} & H_{22}-\hbar\dot{\phi}_2 & H_{23}e^{i\phi{23}} \\   	H_{31}e^{i\phi{31}} & H_{32}e^{i\phi{32}} & H_{33}-\hbar\dot{\phi}_3 \\
	  \end{pmatrix}\begin{pmatrix}
	  \widetilde{c_1}\\\widetilde{c_2}\\\widetilde{c_3}
	  \end{pmatrix} = i\hbar\begin{pmatrix}
	  \dot{\widetilde{c_1}} \\ \dot{\widetilde{c_2}}\\\dot{\widetilde{c_3}}
	  \end{pmatrix},
	  \label{rawMatrixAfterTransformation}
  \end{equation}
  
  \noindent where $ e^{i\phi_{kj}} = e^{i(\phi_k-\phi_j)} $. Setting 
  
  \begin{equation}
  	\phi_1 = \bigg(\frac{E_3}{\hbar}- \omega_{31}^d\bigg)t;\ \phi_2=\bigg(\frac{E_3}{\hbar}- \omega_{32}^d\bigg)t;\ \phi_3 = \frac{E_3}{\hbar}t,
  	\label{rwaTransfomration}
  \end{equation}
  
  \noindent which effectively rotates the state vector components at the natural atomic evolution and detuning frequencies, \footnote{The full unitary transformation applied:
  	
  	$\qquad\qquad\quad U(t) = \exp\big[\frac{it}{\hbar}\left[({E_1}- \hbar\delta\omega_{31})\ketbra{1}{1}+({E_2}- \hbar\delta\omega_{32}\ketbra{2}{2})+E_{3}\ketbra{3}{3}\right]\big] $} will give a simple expression for the Hamiltonian
  
  \begin{equation}
	  \widetilde{\mathcal{H}}_{a} = \begin{pmatrix}
	  \hbar\delta\omega_{31} & 0 & -\frac{\hbar\Omega_{31}}{2}\\  0 & \hbar\delta\omega_{32} & -\frac{\hbar\Omega_{32}}{2} \\   	-\frac{\hbar\Omega_{31}}{2} & -\frac{\hbar\Omega_{32}}{2} & 0 \\
	  \end{pmatrix} + \begin{pmatrix}
	  0 & 0 & -\frac{\hbar\Omega_{31}}{2}e^{-i2\omega^{d}_{31}t}\\  0 & 0 & -\frac{\hbar\Omega_{32}}{2}e^{-i2\omega^{d}_{32}t}  \\   	-\frac{\hbar\Omega_{31}}{2}e^{-i2\omega^{d}_{31}t} & -\frac{\hbar\Omega_{32}}{2}e^{-i2\omega^{d}_{32}t} & 0 \\
	  \end{pmatrix}.
	  \label{rawTransformedFinal}
  \end{equation}
  
  \noindent In the rotating-wave-approximation, one ignores the contribution from the fast rotating $ 2\omega^{d}_{31} $, $ 2\omega^{d}_{32} $ in the second term, since their oscillations will be averaged out at the time-scales of significant processes - physically they correspond to energy non-conserving processes \cite{ioChunHoi}. One is left with the following results for the two driving cases in Fig.~\ref{fig:transmissionLine}
  
  \begin{equation}
  \label{rwaHamitlonianApprox}
  \widetilde{\mathcal{H}}_{a} = -\frac{\hbar}{2}\begin{pmatrix}
  -2\delta\omega_{31} & 0 & \Omega_{31}\\  0 & -2\delta\omega_{32} & \Omega_{32} \\   	\Omega_{31} & \Omega_{32} & 0
  \end{pmatrix};\qquad\widetilde{\mathcal{H}}_{b} = -\frac{\hbar}{2}\begin{pmatrix}
  0 & \Omega_{21} & \Omega_{31}\\  \Omega_{21} & 2\delta\omega_{21} & 0 \\   	\Omega_{31} & 0 & 2\delta\omega_{31}\end{pmatrix}.
  \end{equation}
  
  \noindent For various $ \delta\omega_{ij} \approxeq 0$, certain energy levels in the rotated frame become degenerate, shown in Fig.~\ref{theoRotation}b for a two level system. The off-diagonal terms induce a splitting $ \hbar\Omega $ between the eigenstates of the system,\footnote{For a Two-Level \iket{0}, \iket{1}, system,
  
  $\qquad\quad \mathcal{H} = \hbar\big(\begin{smallmatrix}0 & -\Omega/2\\-\Omega/2&0\end{smallmatrix}\big) $ has eigenenergies $ \pm\hbar\Omega $ and eigenstates $ \big(\iket{0}\mp\iket{1}\big)/\sqrt{2} $.} lifting this degeneracy. In the original unrotated frame, these Rabi splittings persist, resulting in a richer energy level structure as in Fig.~\ref{theoRotation}.
  
  \begin{figure}
  	\ifigure{4cm}{rotwave}
  	\caption{\textbf{The Rabi splitting of the energies in a Two-Level system.} A driving field with angular frequency $ \omega_{ij}^{d} $ and Rabi amplitude $ \Omega_{ij} $ in resonance with the atomic transition~$ \omega_{ij} $, couples the states \iket{i} and \iket{j}. In the rotated frame, the eigenstates $ \big(\iket{i}~\pm~\iket{j}\big)/\sqrt{2} $ are separated by an energy $ \hbar\Omega_{ij} $. Retuning to the original frame will map the Rabi splitting onto the original levels.}
  	\label{theoRotation}
  \end{figure}
  
  The general form of a Thee-Level state, is given by the density matrix \cite{quantumOptics}
  
  \begin{equation}
	  \rho = \begin{pmatrix}
	  1-\rho_{22}-\rho_{33} & \rho_{12} & \rho_{13}\\
	  \rho_{21} & \rho_{22} & \rho_{23}\\
	  \rho_{31} & \rho_{32} & \rho_{33}\\
	  \end{pmatrix},
	  \label{rwaDensity}
  \end{equation} 
  
  \noindent where the diagonal elements $ 1-\rho_{22}-\rho_{33}, \rho_{22}, \rho_{33} $, are the probabilities of observing the system in the \iket{1}, \iket{2}, \iket{3}, states, and the off-diagonal elements are related to coherence in the system.  The evolution of the state Eq.~\eqref{rwaDensity} is engrained in the Markovian master equation \cite{zoller}
  
  \begin{equation}
  	\dot{\rho} = -\frac{i}{\hbar}\big[\mathcal{H},\rho\big]+\mathcal{L}\big[\rho\big],
  	\label{rwaMarkovian}
  \end{equation}
  
  \noindent in which the first term accounts for the coherent evolution under the system Hamiltonian Eq.~\eqref{rwaHamitlonianApprox}, and the second Linbland term
    
  \begin{equation}
	  \label{linLinTerm}
	  \begin{aligned}
	  \mathcal{L} & = \begin{pmatrix}
	  \Gamma_{21}\rho_{22} + \Gamma_{31}\rho_{33} & -\gamma_{12} & -\gamma_{13}\\
	  -\gamma_{21} & -\Gamma_{21}\rho_{22} + \Gamma_{32}\rho_{33} & -\gamma_{23}\\
	  -\gamma_{31} & -\gamma_{32} & -\Gamma_{31}\rho_{33} + \Gamma_{31}\rho_{33}\\
	  \end{pmatrix},
	  \end{aligned}
  \end{equation}
  
  \noindent takes into account the relaxation and dephasing processes in the system. A comprehensive study of the origin of the Linbland term Eq.~\eqref{linLinTerm} is given by Ithier \cite{ithier}, the relevant outcomes of which are presented in Sec.~\ref{subsubsec:Decoherence}. One solves the Markovian master equation Eq.~\eqref{rwaMarkovian} for the stationary state, $ \dot{\rho}=0 $, valid in the case of a continuous drive, to ultimately determine all of the $ \rho_{ij} $ and fully characterise the system.
    
 \subsubsection{Decoherence in the artificial atom \label{subsubsec:Decoherence}}
   The macroscopic artificial atom is going to suffer from decoherence - the uncontrollable and unaccountable evolution of a state which is not incorporated into the systems Hamiltonian, $ \mathcal{H} $. Two mechanism are responsible for this \cite {ithier,QuantumstateengineeringwithJosephsonjunctiondevices}.
   
   \paragraph{Spontaneous relaxation-excitation processes} lead to a population change of the diagonal elements in the density matrix Eq.~\eqref{rwaDensity}. The characteristic rate 
   	
	\begin{equation}\label{theoGamma1}
	   \Gamma_1=\Gamma_{\uparrow}+\Gamma_{\downarrow},
	\end{equation}
	   
	\noindent depends on the excitation, $ \Gamma_{\uparrow} $, and relaxation, $ \Gamma_{\downarrow} $, rates between two levels with an energy difference $ \Delta E $. In the limit of low temperatures, $ kT<< \Delta E $, $ \Gamma_1\approxeq\Gamma_{\downarrow} $, and the probability of observing the system in the excited (ground) state, $ \rho_e $ ($ \rho_g $), exponentially decreases (increases) independently of the systems Hamiltonian. The diagonal elements of the density matrix will decay (grow) with rates
	
	\begin{equation}
	   \dot{\rho}_{e} = -\Gamma_1\rho_{e};\qquad \dot{\rho}_{g} = +\Gamma_1\rho_{g},
	\end{equation}
	   
	\noindent the effect on which is taken into account by the diagonal elements of the Linbland term Eq.~\eqref{linLinTerm}.
	
	\paragraph{Pure dephasing} due to the broadening of the energy levels as they fluctuate under external noise, constitutes to another dephasing mechanism. Fig.~\ref{deph-slope} shows how a fluctuating magnetic field can induce such broadening.

	The oscillation of the corresponding transition frequency, $ \omega_{ij}(t) =E_{ij}(t)/\hbar$, about a central value of $ \omega_{ij}^{0} $, lead to a diffused phase difference, $ \phi_{ij}=\int_0^{t}\omega_{ij}(t')dt  = \omega_{ij}^{0}t+\Delta\phi_{ij}(t)$, between levels \iket{i}, \iket{j}, after a time $ t $. At some point, phase coherence is destroyed by the ever increasing $ \Delta\phi_{ij}(t) $ term, and the time $ T_{\phi} $  of when this occurs determines the pure dephasing rate
	
	\begin{equation}\label{theoDephasing}
		\Gamma_{\phi} = \frac{1}{T_{\phi}}.
	\end{equation}
	
	\begin{figure}
		\ipic{6cm}{broaden}
		\caption{\small \textbf{Broadening of energy levels due to a fluctuating magnetic flux} for the Three-Level artificial atom. A shift of $ \Delta\Phi $ will change the separation $ \Delta E $ of the two levels. As a consequence, Rabi oscillations mediated by an external drive will no longer be coherent. The state of the system becomes less and less pure, as information about its state is effectively lost.}
		\label{deph-slope}
	\end{figure}
	
   The relaxation, $ \Gamma_1 $, and dephasing, $ \Gamma_{\phi} $, rates from  Eq.~\eqref{theoGamma1}, \eqref{theoDephasing}, collectively define the decoherence rate
   
   \begin{equation}
 		\Gamma_2 = \frac{\Gamma_1}{2}+\Gamma_{\Phi},
   \end{equation}
   
   \noindent of each individual level. For multi-level systems this rate is also referred to as the total broadening of each level and labelled by $ \lambda_i$. For the Three-level atom, i=1,2,3, the relaxation processes depicted in Fig.~\ref{theoBroadening} result in values of
 
   \begin{equation}
  	\begin{aligned}
  	\lambda_3&\equiv \Gamma_{2}^{(3)} = \frac{\Gamma_{31}+\Gamma_{32}}{2}+\Gamma_{\phi}^{(3)}\\
  	\lambda_2&\equiv\Gamma_{2}^{(2)} = \frac{\Gamma_{21}}{2}+\Gamma_{\phi}^{(2)}\\
  	\lambda_1&\equiv\Gamma_{2}^{(1)} = \Gamma_{\phi}^{(1)},
  	\end{aligned}
   \end{equation}
  
   \noindent where $ (i) $ specify the level, \iket{i}, that an incoherent process refers to. The decay rate, $ \gamma_{ij}=\gamma_{ji} $, of the off diagonal matrix element $ \rho_{ij} $, will depend on the broadenings $ \lambda_i $, $ \lambda_j $,
  
   \begin{equation}\label{decoherence}
 	 \begin{aligned}
  		\gamma_{32} & =\lambda_3+\lambda_2=\frac{\Gamma_{31}+\Gamma_{21}+\Gamma_{32}}{2}+\Gamma_{\phi,32}\\
  		\gamma_{31} & =\frac{\Gamma_{31}+\Gamma_{32}}{2}+\Gamma_{\phi,31}\\
  		\gamma_{21} & = \frac{\Gamma_{21}}{2}+\Gamma_{\phi,21},
  	\end{aligned}
   \end{equation}
   
   \noindent where $ \Gamma_{\phi,ij} = \Gamma_{\phi}^{(i)}+\Gamma_{\phi}^{(j)} $ is the pure dephasing rate associated with \iket{i}\lra\iket{j} transition. The decoherence rates Eq.~\eqref{decoherence} appear as off-diagonal elements in the Linbland term Eq.~\eqref{linLinTerm}. 
   
  
   \begin{figure}
   	\ifigure{7cm}{hideRabi}
   	\caption{\small \textbf{Effective broadening of energy levels due to relaxation and pure dephasing.} $ \Gamma_{ij} $ signifies the \iket{i}\ra\iket{j} relaxation rate, and $ \Gamma_{\phi}^{(i)} $ the pure dephasing rate of level \iket{i}. These incoherent processes broaden the \iket{i} level by $ \lambda_i $. If the broadening is larger than the Rabi frequency of an applied drive, $ \lambda_i>\Omega$, then it is possible that the Rabi splitting from Sec.~\ref{subsec:3LevelAtom} will not be revealed.}
   	\label{theoBroadening}
   \end{figure}
   
 \subsection{Emission by the atom\label{subsec:Scattering}}  
  The input-output theory \cite{zoller} shows that the average field emitted by an artificial atom to an open transmission line is
  
  \begin{equation}\label{theoEmission}
  	\iexpectation{V_{\text{sc}}} =  i\sqrt{\frac{\Gamma_{ij}}{2}}\iexpectation{\sigma_{ji}},
  \end{equation}

  \noindent where $ \iexpectation{\sigma_{ji}}=\text{Tr}\left\lbrace \ketbra{j}{i}\rho \right\rbrace = \rho_{ij} $, and $ \Gamma_{ij} $ is the \iket{i}\lra\iket{j} relaxation rate. The angular frequency of the emission is $ \omega_{ij} $ \cite{UltimateOnChipQuantumAmplifier}. One labels the incoming (+) and outgoing (-) fields, with respect to the artificial atom, as in Fig.~\ref{fig:transmissionLine}:
  
  \begin{itemize}
  	\item $ V_{\text{L}}^{+} $ the incident field, in resonance with the \iket{i}\lra\iket{j} transition and defined by Eq.~\eqref{theoField};
  	\item $ V_{\text{L}}^{-} = V_{\text{L}}^{+}+V_{\text{sc}}$ the transmitted field, which is a combination of the incident and emitted fields\footnote{The artificial atom, whose dynamics, Eq.~\eqref{rwaHamitlonianApprox}, are determined by the incident field, is treated as a stand-alone quantum system that emits a field $ V_{\text{sc}} $. This resultant field is a sum of the incident, $V_{L}^{+} $, and emitted, $ V_{\text{sc}} $, fields, created by two distinct objects in the quantum system.};
  	\item $ V_{\text{R}}^{-} = - V_{\text{sc}}$ the reflected field, only composed of emission by the artificial atom.
  \end{itemize}

  The transmission, $ t $, and reflection, $ r $, coefficients are defined as
  
  \begin{equation}\label{theoRefTran}
  	\begin{aligned}
	  	t & = \frac{\iexpectation{V_{\text{L}}^{-}}}{\iexpectation{V_{\text{L}}^{+}}} = 1+\frac{i\Gamma_{ij}}{\iexpectation{V_{\text{L}}^{+}}\sqrt{2\Gamma_{ij}}}\rho_{ij};\\
	  	r & = \frac{\iexpectation{V_{\text{L}}^{-}}}{\iexpectation{V_{\text{L}}^{+}}} = 1-t;
  	\end{aligned}
  \end{equation}
  
  \noindent where $\iexpectation{V} = \frac{1}{T}\int_{0}^{T}V(t)dt$ is the average voltage over a normalisation time $ T $. Since the Rabi frequency, $ \Omega $, is related to the drive amplitude, $ \iexpectation{V_{\text{L}}^{+}}  $, \cite{BreakdownoftheCrossKerrSchemeforPhotonCounting}
  
  \begin{equation}
   \Omega_{ij}=\iexpectation{V_{\text{L}}^{+}}\sqrt{2\Gamma_{ij}},
  \end{equation}
  
  \noindent Equation~\eqref{theoEmission}, \eqref{theoRefTran}, reduce to
  
  \begin{equation}\label{theoCoeff}
  	\begin{aligned}
	  	t & = 1 + i\frac{\Gamma_{ij}}{\Omega_{ij}}\rho_{ij};\\
	  	r & = 1-t.
  	\end{aligned}
  \end{equation}
  
  \noindent The coefficients of Eq.~\eqref{theoCoeff} apply to coherent emission, when angular frequencies of the driving and emitted fields coincide, $ \omega^{d}_{ij} \approxeq\omega_{ij} $, and interference between the onset and emitted waves occur.  	
   
 \subsection{Single drive configuration\label{subsec:singleDrive}}
  When a single drive couples two levels \iket{i}, \iket{j}, the non-interacting third level can be traced out from Eq.~\eqref{rawTransformedFinal}-\eqref{linLinTerm}. The procedure of solving the Master equation, and determining~$ \rho $, was done with \texttt{Mathematica}. The $ \rho_{21} $, $ \rho_{31} $ coefficients obtained, for respective \iket{1}\lra\iket{2}, \iket{1}\lra\iket{3}, drives, upon substitution into Eq.~\eqref{theoCoeff} give
  
  \begin{equation}
  r_{21}=\frac{\Gamma_{21}}{2\gamma_{21}}\frac{1+i\delta\omega_{21}/\gamma_{21}}{1+(\delta\omega_{21}/\gamma_{21})^2+\Omega_{21}^2/\Gamma_{21}\gamma_{21}}; \quad r_{31}=\frac{\Gamma_{31}}{2\gamma_{31}}\frac{1+i\delta\omega_{31}/\gamma_{31}}{1+(\delta\omega_{31}/\gamma_{31})^2+\Omega_{31}^2/\Gamma_{31}\gamma_{31}},
  \label{singleReflectance}
  \end{equation}
  
  Figure~\ref{singleDriveReflection} shows the real components of $ r_{ij} $ as function of the detuning of the driving field from the atomic transition, $ \delta\omega_{ij} = \delta\omega_{ij}^{d}-\delta\omega_{ij}$. The reflectance peak corresponds to the case when the atom relaxes to the ground state, and emits a photons that is coherent~\footnote{Of the same frequency.} with the incident field, but shifted by a phase of $ \pi $.\footnote{Signified by the $ i $ factor in Eq.~\eqref{theoCoeff}.} Destructive interference occurs with the incident field and the wave is fully reflected.
  
  \begin{figure}
  	\ipic{9cm}{transmission_purely_theoretical}
  	\caption{\small \textbf{Simulations of elastic scattering of an incident microwave in an arbitrary $ \mathbf{\ket{i}, \ket{j}}, $ system}. Shown are the real part of the reflection coefficient, $ \re{r_{ij}} $, evaluated with Eq.~\eqref{singleReflectance} for $ \Gamma_{ij}=50, \Gamma_{\phi,ij}=5 $, for a range of driving powers, $ \Omega_{ij} $. When the incident field is on resonance with the atomic transition, $ \delta\omega_{ij}=0  $, the reflectance curve exhibits a peak, as emission from the artificial atom undergo destructive interference with the incident wave. The weaker the driving power, the stronger this interference becomes.}
  	\label{singleDriveReflection}
  \end{figure}

  The sharpness of the central features diminishes for larger driving amplitudes, $ \Omega $. As the number of photons in the transmission line grows, the photon emitted by the atom cannot interfere with all the ones propagating down the transmission line. This photon overload causes $ t $ and $ r $ to become insensitive to atomic transitions \cite{DemonstrationofaSinglePhotonRouter}.  To saturate the peak, one applies weak drives, $ \Omega_{ij}<<\Gamma_{ij}\gamma_{ij} $ in which case Eq.~\eqref{singleReflectance} no longer depends on the driving amplitude
  
\begin{equation}
	  \begin{aligned}
	  	\re{r_{21}} = \frac{\Gamma_{21}}{2\gamma_{21}}\frac{1}{1+(\delta\omega_{21}/\gamma_{21})^2}; \quad \re{r_{31}}=\frac{\Gamma_{31}}{2\gamma_{31}}\frac{1}{1+(\delta\omega_{31}/\gamma_{31})^2},
	  \end{aligned}
	  \label{singleLorentzian}
  \end{equation}
  
  \noindent allowing on to determine the decoherence rates  $ \gamma_{ij}, \Gamma_{ij} $ from fittings to observed values.
      
 \subsection{Two drive configuration\label{subsec:DoubleDrive}}
  The state for a system, $ \rho $, under two resonant drives is found by solving the Three-Level Eq.~\eqref{rawTransformedFinal}-\eqref{linLinTerm} for the steady state, $ \dot{\rho}=0 $, which was done with \texttt{Mathematica}.
  
  As was shown in Sec.~\ref{subsec:3LevelAtom}, ramping up the power of the driving fields, $ \Omega_{ij} $, will lead to the Rabi splitting of the energy levels, shown in Fig.~\ref{theTwoDrive} for the \iket{2}\lra\iket{3}, \iket{1}\lra\iket{3}, drive combination. This splitting is probed by measuring the emission of the atom between the two undriven levels, which in accordance to Eq.~\eqref{theoEmission} will have strengths
  
  \begin{equation}\label{theEmission1}
  	  	e_a=\iexpectation{V_{a}} =  i\sqrt{\frac{\Gamma_{21}}{2}}\rho_{21};\quad e_b=\iexpectation{V_{{b}}} =  i\sqrt{\frac{\Gamma_{32}}{2}}\rho_{32},
  \end{equation}
  
  \noindent for the two drive combinations defined earlier: $ a $ for \iket{1}\lra\iket{3}, \iket{2}\lra\iket{3}; $ b $ for \iket{1}\lra\iket{3}, \iket{1}\lra\iket{2}. The emissions Eq.~\eqref{theEmission1} will be carried by fields with angular frequencies
  
  \begin{equation}\label{theoEmissionFrequency}
  	\omega_{a} = \omega_{31}^{d} - \omega_{32}^{d};\quad \omega_{b} = \omega_{31}^{d} - \omega_{21}^{d}, 
  \end{equation}
  
  \noindent i.e. at the angular frequency difference of the two driving fields.\footnote{The interaction of two fields with angular frequencies $ \omega_{ij}^{d},\omega_{ik}^{d} $, via a Three-Level atom can give rise to $ \omega_{ij}^{d}\pm\omega_{ik}^{d} $ emissions  \cite{giantCrossKerrEffect_2013}. Only $ \omega_{ij}^{d}-\omega_{ik}^{d} $ is associated with a physical \iket{j}\lra\iket{k} transitions in the system, so emission at this angular frequency will be dominant in the system.}
  
  Simulations of the emission strengths for different detunings, $ \delta\omega_{ij} $, and Rabi amplitudes, $ \Omega_{ij} $, of the drives are presented in Fig.~\ref{theTwoDrive}. Weak driving, Fig.~\ref{theTwoDrive}b, does not split the levels, and high emission is observed whenever $  \omega_{31}^{d} - \omega_{21}^{d} \equiv \omega_{32} $ which corresponds to the $ \delta\omega_{21} = \delta\omega_{31} $ line. At stronger driving, $ \lambda_i<\Omega_{ij} $, the Rabi splitting of the levels can be distinguished from their broadening due to decoherence, $ \lambda_i $. In Fig.~\ref{theTwoDrive}a, strong \iket{3}\ra\iket{2} emissions will occur at frequencies, $ \omega_{32}\pm\hbar\Omega_{31} $. Thus, for every $ \omega_{31}^{d} $, there will be two values of $ \omega_{21}^{d} $ at which the \iket{3}\ra\iket{2} emissions are peaked, corresponding to splitting along the $ \delta\omega_{21} $ axis.
  
%  \eject \pdfpagewidth = 13in \pdfpageheight=10in
  \begin{figure}
	  	\ifigure{19cm}{theoryEmission1}
	  	\caption{\textbf{Simulation of the emission by a Thee-Level atom subjected to $ \mathbf{\ket{1}\leftrightarrow\ket{3}}, \mathbf{\ket{2}\leftrightarrow\ket{3}} $, drives.} The upper row shows the structure of the energy levels for different driving strengths. The strength of the \iket{2}\ra\iket{1} emission, occurring at a frequency $ \omega_{31}^{d} - \omega_{32}^{d} $, is evaluated and plotted as a function of the driving field detunings $ \delta\omega_{32} = \omega_{32}^{d}-\omega_{32}, \delta\omega_{31} = \omega_{31}^{d}-\omega_{31} $ in the row below. \textbf{a)} Strong $ \Omega_{31} $ drive will split the \iket{1} level, resulting in splitting along the $ \delta\omega_{21} $ axis; \textbf{b)} Weak driving, $ \Omega_{31}<\lambda_1, \Omega_{32} < \lambda_2 $, will maintain the original level structure, and emission will occur at an angular frequency $ \omega_{32} $ exactly; \textbf{c)} Strong $ \Omega_{32} $ drive will split the \iket{2} level, resulting in splitting along the $ \delta\omega_{31} $ axis. \textbf{d)} Emission spectrum as a function of driving amplitude, $ \Omega_{31} $, when $ \delta\omega_{31}=0, \Omega_{32}=10 $.}
	  	\label{theTwoDrive}
  \end{figure}
%    \eject \pdfpagewidth = 10in \pdfpageheight=10in
  \newpage
\section{Experimental work\label{sec:ExperimentalWork}}
 A realisation of the system detailed in Chapter~\ref{sec:Theory} was done at Royal Holloway University of London. Th artificial atoms were made from Al on an undoped silicon substrate, and coupled to a NbN transmission line premade on the wafer. This chapter goes over the design and fabrication of the sample in Sec.~\ref{subsec:Design}, the loading of the sample and experimental setup in Sec.~\ref{subsec:Setup} and a brief description of the refrigerator system, that maintained a temperature of \iunit{13}{mK} on the sample throughout measurements, in Sec.~\ref{subsec:Cooling}.
 
 \subsection{Design and fabrication\label{subsec:Design}}
 Fig.~\ref{designDesign} shows the design and final look of the device. The ratio $ E_J/E_c \approxeq 10 $, required for the device to operate in the flux regime (see Sec.~\ref{subsec:Atom}), and transition frequencies, $ \omega_{ij} $, that fell within the working range of the laboratory instruments, were achieved by simulating the geometry of the JJs \cite{Strongcouplingofasinglephotontoasuperconductingqubitusingcircuitquantumelectrodynamics}: the area and relative areas of the junction determined $ E_c $ and $ \alpha $; physical properties of the junction, such as the superconducting gap $ \Delta $, determined $ E_J $ \cite{tunnelingBetweenSuperconductors}.
 
 The atoms were positioned between a ground pane and the transmission line, to which they were coupled with grid-like capacitors. The size of the capacitors, also simulated, affected coupling strength to the transmission line. The dimensions of the transmission line were chosen to match the standard impedance, $ {Z}_0 = 50 $, of the microwave devices used in the experiment \cite{ioChunHoi}, so as to prevent the reflections within the experimental system.

  \begin{figure}
  	\ifigure{15cm}{fabrication}
  	\caption{\small\textbf{The design and final look of the artificial atoms.} \textbf{a)} Design of the full sample, showing the transmission line (gold), artificial atoms (blue), and ground planes (green). The qubits are capacitively coupled to the transmission line via the grid-like capacitors (cyan). \textbf{b)} Each individual artificial atom was a superconducting loop interrupted by four Josephson junctions, their sizes carefully chosen to achieve a ratio $ E_J/E_c \approxeq10 $ and a desired energy dispersion. The``double" pattern is required to perform angular deposition; \textbf{c)} Image of the nanofabricated device taken on a scanning electron microscope.}
  	\label{designDesign}
  \end{figure}
 
 In considerations of efficiency, seven dissimilar artificial atoms were made, to be measured in a single loading cycle. By designing the loops with different areas, ranging from \iunit{15\text{ to }40}{$ \mu $m$ ^2 $}, the energy dispersion characteristic of each atom would be centered about different magnetic fields (corresponding to the $ \Phi_0 $ degeneracy point), allowing the individual addressing of each atom.
   
 The device was fabricated on an undoped silicon wafer, initially covered by NbN. The transmission line was cut out from this NbN surface by: 1) Depositing an aluminium mask with the transmission line contour via the standard electron-beam lithography process \cite{Electronbeamtechnologyinmicroelectronicfabrication}; 2) Plasma-etching-away any areas not covered by the mask; 3) Lifting off the aluminium mask to uncover the NbN transmission line . The artificial atoms and coupling capacitors were fabricated with Al using electron beam lithography, employing a shadow evaporation technique \cite{Fabrication} to form the Al/AlO$_x$/Al JJs. The middle AlO$ _{x} $ film is formed by injecting oxygen into the chamber in between the shadow evaporations, to oxidise the surface of the first Al layer.
   
 \subsection{Sample loading and measurement setup\label{subsec:Setup}}  
  The sample loading procedure steps is summarised in Fig.~\ref{loadingLoad}. The device is glued and wire-bonded to a Printed Circuit Board (PCB). The PCB is mounted inside a doughnut-like holder which is wound around the circumference by niobium coils. Passing a current through the windings biases the atoms in the circuit to a controlled flux $ \Phi $. The setup is shielded by an aluminium cylinder, to grounds any external signals, and placed on the base temperature (\iunit{13}{mK}) stage of the refrigerator. Such extreme temperatures are required so that the atoms operate in the superconducting regime, $ < $\iunit{1.2}{K} for Al, and to suppress excitations processes.
    
  \begin{figure}
  	\ifigure{15cm}{loading4}
  	\caption{\small\textbf{The sample loading procedure:} \textbf{a)} The chip with the sample is glued with a resin and wire bonded to the PCB. The small holes on the PCB surface prevent the formation of standing waves on the surface that would be a source of noise in the system. Highlighted are ports for the two RF lines that feed signals to and from the system; \textbf{b)} The sample holder is wound with Niobium coils, biasing the artificial atoms to a desired magnetic flux in the experiment; \textbf{c)} The whole arrangement is shielded, so that any electromagnetic noise form the outside is grounded; \textbf{d)} The whole arrangement rests on the \iunit{13}{mK} stage of the refrigerator.}
  	\label{loadingLoad}
  \end{figure}
 
  Two undistributed connections from the laboratory to the transmission line on the chip were made by a combination of: SMB cables in the laboratory; low heat conducting RF lines between the laboratory and the \iunit{4}{K} stage; superconducting niobium lines between the \iunit{4}{K} and base temperature stages. The long RF-Lines are heat-sinked on various stages of the refrigerator, which ensures that they are thermalised by the time they reach the sample. All the cabling used had an impedance of $ Z_0=50\,\Omega $.
    
  \begin{figure}
  	\ifigure{12cm}{setupDiagram}
  	\caption{\small\textbf{Measurement setup for the artificial atom}. The atom is placed in the \iunit{13}{mK} region of the multi-temperature stage refrigerator. Colour coded are the components of the system: black lines represent the RF-lines; the purple curve represents the magnetic coil; blue squares represent the attenuators; orange triangles represent amplifiers. \textbf{a)} Setup for transmission measurements, used to characterise the artificial atom in Sec.~\ref{subsec:WorkingPOint} and measure the relaxation parameters in Sec.~\ref{subsec:DirectTransmission}. A VNA measures the transmission of its own signal through the system; \textbf{b)} Setup for the two-drive configuration measurements in Sec.~\ref{subsec:TwoDriveMeas}. Driving fields, with angular frequencies $ \omega_{ij}^{d}, \omega_{kl}^{d} $, outputted by the two generators are mixed by a power divider and sent down the RF line. A VNA is programmed to measure signals emitted from the system with an angular frequency $  \omega_{ij}^{d} - \omega_{kl}^{d} $. An external trigger, not shown in the diagram, is used to keep the three devices in sync.}
  	\label{setupDiagrams}
  \end{figure}

  A high temperature \iunit{50}{dBm}\footnote{\iunit{1}{dBm}\iunit{\equiv1}{mW}$;\qquad P_\text{dBm} = 20\log_{10}\big[P_\text{mW}/\iunit{1}{mW}\big]  $} attenuator on the \iunit{50}{K} stage, and a low temperature \iunit{40}{dBm} attenuator on the \iunit{4}{K} stage worked as additional heat sinks of the inner coaxial cable in the RF lines, and down-converted the power output of laboratory devices, whose lowest possible power output was \iunit{-70}{dBm}, to the $ \approxeq\iunit{-140}{dBm} $ region required for the observation of transmission features discussed in Chapter~\ref{sec:Theory}.
  
  The output line was fitted with a circulator, that blocked noise signals leaking into the system from the output end, and two \iunit{+35}{dBm} amplifiers on the \iunit{4}{K} and room temperature stages that amplified the weak and all-important signal from the system. All of the devices had good thermal contact to refrigerator stages to reduce black-body radiation within the system, and strongly dissipating ones, such as the amplifiers, were positioned away from the base temperature stage.
   
  Two configuration are used for measurements.
  
  \paragraph{Direct transmission measurements} are made with the configuration Fig.~\ref{setupDiagrams}a, where a \texttt{Rohde \& Schwartz ZNB-20} Vector Network Analyser (VNA) is hooked up to the input and output lines of the system. A VNA works by sending in signal pulses of a known amplitude and phase, and comparing them with the signals outputted from the system \cite{vnaAnalyser}, as depicted in Fig.\ref{setupVNA}. This arrangement is used for measuring transmission and reflection features theorised in Sec.~\ref{subsec:singleDrive}.
  
  \paragraph{Two-drive measurements} are made with the configuration Fig.~\ref{setupDiagrams}b, where an \texttt{Arnitsu} power divider is used to mix two driving fields emitted by two independent \texttt{Rohde \& Schwartz SMB 100A} generators. In accordance with Sec.~\ref{subsec:DoubleDrive}, the emission from the atom has to be logged at the angular frequency difference of the two driving fields, so the angular frequency of the VNA is continuously swept to match any changes in the generators. The output port of the VNA is blanked, and the VNA only registers emissions from the atom, as a result of relaxation between the two undriven levels. A common trigger is used for three devices, to ensure synchronisation in the measurements.
  
  \begin{figure}
  	\ifigure{7cm}{vnaDesc}
  	\caption{\small\textbf{Operation of a VNA.} A pulse of microwaves with a defined amplitude, $ V_{\text{in}} $, and phase, $ \phi_{\text{in}} $, is fed into the system (red), and compared with the pulse emerging from the system (blue). Interaction with the atom will lead to: changes in the amplitude, $ V_{\text{out}} = V_{\text{in}} + \Delta V  $, of the output signal; changes in the phase, $ \phi_{\text{out}} = \phi_{\text{in}} + \Delta\phi  $, of the output signal. A VNA internally calculates the differences, $ \Delta V$, $ \Delta\phi $, which characterise the interactions that took place within the system.}
  	\label{setupVNA}
  \end{figure}
    
 \subsection{Cooling\label{subsec:Cooling}}	 
  Cooling is performed with the cryogen-free dilution refrigerator system \cite{reviewOfPulseTubeRefrigiration}. The sample is placed on the base plate, and sealed inside a vacuum can that is evacuated at room temperature to \iunit{<10^{-3}}{mBar}. Initial cooling of the base plate from room temperature \iunit{4.5}{K} is done by the Pulse Tube, which uses regenerative gas expansion of pre-cooled He$ ^{3} $ to transport heat against a temperature gradient.
	 
%  \begin{figure}
%	 	\ipic{6cm}{cool}
%	 	\caption{\small \textbf{The cooling setup to achieve persistent temperatures of \iunit{13}{mK}.} The refrigerator is evacuated by the vacuum system (A) to \iunit{10^{-6}}{mBar}; the Pulse-Tube (B) performs the initial cooling of the base stage from room temperature to $ 4K $; the the cryogen-free dilution refrigeration system (C) completes the cooling to base temperature, by cycling He$^{3}$ through a superfluid He$ ^{4} $ stage - a process which absorbs energy from the surroundings.}
%	 	\label{coolingDiagram}
%  \end{figure}
	 
  Thereafter, cooling to the base temperature is performed by the He$ ^{3} $-He$ ^{4} $ dilution refrigerator system. A He$ ^{3} $-He$ ^{4}$ mixture is liquefied by compression and Joule-Thomson expansion, filling up the mixing chamber. Pumping on the mixture in the still causes evaporative cooling, which lowers the temperature to \iunit{0.8}{K}. At this point He$ ^{4} $ undergoes a phase transition to the superfluid state and collects at the bottom of the chamber. By continuing to pump the system in a closed cycle, He$ ^{3} $ is forced to enter this superfluid phase in which it has a higher enthalpy. The energy for this process will come from the surrounding environment, resulting in further cooling of the mixing chamber. A temperature-exchange system between the outflowing-incoming He$ ^3 $ makes the cooling process more efficient.

\newpage
\section{Quantum measurements\label{sec:Outcomes}}
 This chapter summarises the results of the project, which included simulations, based on material from Chapter~\ref{sec:Theory}, and measurements, performed on the setup detailed in Chapter~\ref{sec:ExperimentalWork}. Sec~\ref{subsec:Noise} explains the way in which transmission noise was taken care of and Sec.~\ref{subsec:WorkingPOint} the steps taken to put the artificial atom into an operational regime. Sec.~\ref{subsec:DirectTransmission} demonstrates the single-drive transmission results, which are used to uncover decoherence parameters of the system. Sec.~\ref{subsec:TwoDriveMeas} demonstrates the two-drive emission results, which were the culmination of the whole project.
 \subsection{Noise calibration\label{subsec:Noise}}
  \paragraph{Transmission measurements.} Contrary to the ideal case scenario, when a signal, that doesn't not interact with the artificial atom, passes completely unchanged through the measurement system, the unbalanced attenuator-amplifier system\footnote{\iunit{-90}{dBm} on the input line, and \iunit{+35}{dBm} on the output line will automatically render any outgoing signal 100 times weaker than the incoming one.} in Fig.~\ref{setupDiagrams}, the unaccounted-for attenuations, and inevitable signal loss within the RF lines, leads to a background transmission spectrum of the form in Fig.~\ref{measNoise}a.
  
  Any artificial atom interaction features will be overlayed on top of this background spectrum, $ t_{\text{b}}(\omega) $, and to isolate them from a measured spectrum, $ t(\omega) $, a normalisation
  
  \begin{equation}\label{measNorm}
  	t'(\omega) = \frac{t(\omega)}{t_{\text{b}(\omega)}},
  \end{equation}
  
  \noindent has to be performed, the effect of which is seen in Fig.~\ref{measNoise}b. Since the background spectrum, $ t_{b}(\omega) $, is unique for each drive configuration, it had to be taken for each new set of measurements. Practically this involved jumping the magnetic field to a region where the atomic transitions were outside the working frequency range, and taking background transmission measurements there. Corrections of this form were done for all measurements in Sec.~\ref{subsec:WorkingPOint}, \ref{subsec:DirectTransmission}.
  
  \begin{figure}
  	\ipic{9cm}{calibrating_noise}
  	\caption{\small\textbf{Example of noise calibration of transmission measurements.} \textbf{a)} The background transmission spectrum, $ t_{b}(\omega) $ (red), sets the standard attenuation signal with a frequency $ \omega $. This attenuation will be present in all measurements such as $ t(\omega) $ (blue); \textbf{b)} Normalising $ t(\omega) $ by $ t_{b}(\omega) $, as done in Eq.~\eqref{measNorm}, separates out artificial atom interaction features from the otherwise indistinguishable spectrum. The example presented here is a simulation.}
  	\label{measNoise}
  \end{figure}

  \paragraph{Emission measurements} are going to be affected by persistent noise that is registered in the absence of any emissions. In the experiment, emission spectra, $ e_i$, were taken for a range of driving field detunings in order to map out a graph that could be compared with the simulations from Sec.~\ref{subsec:DoubleDrive}. Since emission features are rare, the average of these spectra would effectively be the persistent noise spectrum, which is subsequently used to normalise all the readings much in the same way as was done in Eq.~\eqref{measNorm}
  
  \begin{equation}
	  e_i' = \frac{e_i}{\sum_{i}^{N}e_i/N},
  \end{equation} 
  
  \noindent where $ i $ numbers through the $ N $ emission spectra taken in the measurement run. Corrections of this form were done for measurements in Sec.~\ref{subsec:TwoDriveMeas}.
  
 \subsection{Selecting qubit and working point\label{subsec:WorkingPOint}}
  The primary readings taken on the loaded sample were two-dimensional transmission plots, which probed the energy dispersion spectrum of the artificial atoms on the chip. Figure~\ref{theoWells}b in Sec.~\ref{subsec:Atom} illustrated the energies of the artificial atoms as a function of the external flux, $ \Phi $, biasing the superconducting loop. Since experimentally one is probing transitions, which are associated with the energy differences between the levels, the relevant spectrum for these measurements is the energy difference spectrum.
  
  Measurements were done using the setup from Fig.~\ref{setupDiagrams}a:
  
  \begin{enumerate}
  	\item The magnetic flux, $ \Phi $, was stepped by changing the current through the magnetic coil;
  	\item The VNA frequency was swept from \iunit{2\rightarrow16}{GHz}, and the transmission amplitude, $ |t| $, of the signal was logged.
  \end{enumerate}

  For each value of the magnetic flux there was a particular separation of the energy levels and correspondingly a particular set of resonant $ \omega_{32}, \omega_{31}, \omega_{21} $ frequencies. The swept VNA signal would hit these resonances, and in accordance with Eq.~\eqref{theoCoeff}, \eqref{singleReflectance} one would register dips in the transmission spectrum, which are seen as bright pixels in Fig.~\ref{measFull}. These bright pixels maps out the frequency difference spectrum.
  
  The observation of the \iket{2}\lra\iket{3} transition, which would absent at absolute zero when only the \iket{1} level is populated, is a result of thermal excitations populating level \iket{2}, and is beneficial since it allows one to fully characterise the energy spectrum of the system. 
  
  \begin{figure}
	  \ipic{10cm}{magnetic_field_spectrum}
	  \caption{\small\textbf{The 2D transmission plot of the the artificial atom as a function of the current through the magnetic coil.} The bright lines represents VNA frequencies, $ \omega $, that are resonant with internal atomic transitions. Markers show the transitions that thee three visible bands correspond to. Indicated by the cyan column in the working point of the qubit, which remained fixed for the duration of the measurements.}
	  \label{measFull}
  \end{figure}
  
  At this stage a calibration of the magnetic field was made, Fig.~\ref{measCurrent}, by equating the periodicity of the artificial atom spectra in current, to the flux quantum period $ \Phi_0 $. Next, the working point for future measurements was chosen from considerations of: parasitic resonances in the system, identified as horizontal lines on the resonance map in Fig.~\ref{measFull}; the resonant transitions $ \omega_{32}, \omega_{21}, \omega_{31} $ being within the \iunit{2-16}{GHz} range of the laboratory instruments; being off the degeneracy point, $ \delta\Phi/\Phi=0 $, to allow the \iket{1}\lra\iket{3} transition.\footnote{This requirement is attributed to the identical parities of the \iket{1},\iket{3}, levels at the degeneracy point preventing a dipole transition \cite{Atomicphysicsandquantumoptics}.}
  
  The working point chosen is demonstrated in Fig.~\ref{measFull}, and corresponded to $ \delta\Phi/\Phi_0~=~0.018$. The transition frequencies at the working point are
  
  \begin{equation}\label{measWorkingPoint}
  \begin{aligned}
  	f_{21}/2\pi &= 6.54\,\text{GHz};\quad  	{f_{32}/2\pi} = 8.30{\,\text{GHz}};\quad  	f_{31}/2\pi = 14.88\,\text{GHz},
  \end{aligned}
  \end{equation}
  
  \noindent and remained fixed for subsequent measurements.
 
 \begin{figure}
	 \ipic{8cm}{calibration_magnetic_field}
	 \caption{\small\textbf{Calibration of the magnetic field} is done by looking at the period of artificial atom transmission spectra. Red numbers enumerate the artificial atoms associated with each transmission feature. The separation, $ \Delta I $, at which the first of the spectra begins to repeat, corresponds to a flux change of $ \Delta\Phi=\Phi_0 $. From the area of the artificial atom loop, $ A $, one maps this flux change to a magnetic field change $ \Delta B = \Delta\Phi/A$, from which the proportionality constant $ g = \Delta I/\Delta B =  \iunit{1.44}{G/mA} $ is found.}
	 \label{measCurrent}
 \end{figure} 

 
 
 \subsection{Direct transmission measurements\label{subsec:DirectTransmission}}
  These measurements relate to the discussion from Sec.~\ref{subsec:singleDrive}, and were executed using the setup from Fig.~\ref{setupDiagrams}a, at the working point defined by Eq.~\eqref{measWorkingPoint}, for the \iket{1}\lra\iket{3}, \iket{1}\lra\iket{2}, transitions:
  
  \begin{enumerate}
  	\item The flux detuning, $ \delta\Phi/\Phi_0=0.018 $, remained fixed;
  	\item The VNA power, determining the Rabi frequency of the drive, $ \Omega_{ij} $, is stepped from $ -20 $ to \iunit{-50}{dBm}, with a new set of measurements taken at each power;
  	\item The VNA detuning from the resonant frequency, $ \delta\omega_{ij}/2\pi $, is swept between $ \pm\iunit{100}{MHz} $ and the transmission of the signal, $ t_{ij} $, is logged.
  	\item The transmission, $ t_{ij} $, is mapped to the more convenient reflectance, $ r_{ij}= 1-t_{ij} $, which has been shown to have the analytic expression
  	
  	\begin{equation}\label{measReflectanceAgain}
  		\re{r_{ij}} = \frac{\Gamma_{ij}}{2\gamma_{ij}}\frac{1}{1+(\delta\omega_{ij}/\gamma_{ij})^2+\Omega_{ij}^2/\Gamma_{ij}\gamma_{ij}},
  	\end{equation}
  	
  	\noindent for a \iket{i}\ra\iket{j} relaxation rate $ \Gamma_{ij} $, decoherence rate $ \gamma_{ij}=\Gamma_{ij}/2+\Gamma_{\phi,ij}\ $, and pure dephasing rate $ \Gamma_{\phi,ij} $. 
\end{enumerate}

  Fitting Eq.~\eqref{measReflectanceAgain} to the observed $ \re{r_{ij}} $ values allowed the tuning of the $ \Gamma_{ij}, \Gamma_{\phi_{ij}}, \Omega_{ij} $ parameters of the system, which are later incorporated in Sec.~\ref{subsec:TwoDriveMeas} for two-drive emission simulations. Rough fittings were performed with a \texttt{Mathematica} applet, and are used as the starting estimates for a numerical least-square fit \texttt{C++} program.\footnote{Whose files were unfortunately corrupted during the writing of the project.}
  
  Figure~\ref{measTransmissionForDifferentPowers} shows the fitting obtained for the \iket{1}\lra\iket{2} and \iket{1}\lra\iket{3} transitions. For both sets of measurements, weaker driving resulted in greater noise level, so fittings of Eq.\eqref{measReflectanceAgain} were made to the largest driving power measurements. The clarity of the \iket{1}\lra\iket{3} peaks are significantly worse that the \iket{1}\lra\iket{2} ones because the standard signal attenuation, Fig.~\ref{measNoise}a, is significantly stronger in the $ \omega_{31}/2\pi = \iunit{14.88}{GHz}$ than in the $ \omega_{21}/2\pi = \iunit{6.54}{GHz} $ region. The signal from \iket{3}\lra\iket{1} has a worse signal to noise ration than the \iket{2}\lra\iket{1} one, and is less clear as a result. Table~\ref{tabResults} summarises the best fit values extracted from the fittings.
  
  \setlength{\extrarowheight}{2mm}
  \begin{table}
  	\caption{\small\textbf{Fitted decoherence and Rabi amplitude parameters} from direct transmission measurements. }
  	\label{tabResults}
  	\begin{center}
  		\begin{tabular}{|c|c|c|c|}
  			\hline
  			\textbf{Drive} & $ \Gamma_{ij} $ (GHz) & $ \Gamma_{\phi,ij} $(GHz) & $ \Omega_{ij} $(GHz)\\\hline
  			\iket{1}\lra\iket{2} & 10.6 & 1.3 & 9.3\\
  			\iket{1}\lra\iket{3} & 28.2 & 0.4 & 46.4\\\hline
  		\end{tabular}
  	\end{center}
  \end{table}
  
  \begin{figure}
  	\ifigure{12cm}{transmission}
  	\caption{\small\textbf{Reflection spectra for the $ \mathbf{\ket{1}\leftrightarrow\ket{2}, \ket{1}\leftrightarrow\ket{3}} $, transitions.} \textbf{a-b)} The measured $\re{r_{21}}, \re{r_{31}} $, spectra as a function of the detunings $ \delta\omega_{21}/2\pi, \delta\omega_{31}/2\pi$ from the resonance frequencies $ \omega_{21}/2\pi =  \iunit{6.54}{GHz}, \omega_{31}/2\pi =  \iunit{14.88}{GHz}$. The driving power is varied from \iunit{-20}{dBm} to \iunit{-50}{dBm}; \textbf{c-d)} Result of applying least square fit of Eq.~\eqref{measReflectanceAgain} to \iunit{-20}{dBm} driving power measurements. In both cases a resonance offset was corrected by shifting $ \delta\omega_{ij}\ra\delta\omega_{ij}-\Delta\omega_{ij} $, $ \Delta\omega_{21}=\iunit{-5.40}{MHz}, \Delta\omega_{31}=\iunit{-32.36}{MHz} $. The parameter values obtained from these fittings are presented in Table~\ref{tabResults}.}
  	\label{measTransmissionForDifferentPowers}
  \end{figure}
 
 \subsection{Two-drive measurements\label{subsec:TwoDriveMeas}}
  Chapter~\ref{sec:Theory} provided the theory for simulating the results of these measurements, which were made with the system setup in Fig.~\ref{setupDiagrams}b, at the working point defined by Eq.~\eqref{measWorkingPoint} for the two drive configurations: $ a $ for \iket{1}\lra\iket{3}, \iket{2}\lra\iket{3}; $ b $ for \iket{1}\lra\iket{3}, \iket{1}\lra\iket{2}:
  
  \begin{enumerate}
  	\item The flux detuning, $ \delta\Phi/\Phi_0=0.018 $, remained fixed;
  	\item The generator driving fields powers, determining $ \Omega_{ij}, \Omega_{kl} $, were stepped from \iunit{-20}{} to \iunit{-60}{dBm}, with a new set of measurements taken for each configuration.  
  	\item The detunings of the generator driving fields from the corresponding resonances, $ \delta\omega_{ij}/2\pi, \delta\omega_{kl}/2\pi $, were swept between $ \pm\iunit{200}{MHz} $;
  	\item For each pair of detuning values, $ \delta\omega_{ij} = \omega_{ij}^{d}-\omega_{ij}, \delta\omega_{kl} = \omega_{kl}^{d}-\omega_{kl}$, the VNA was set to measure emission amplitude, $ |e| $, at the frequency difference $ (\omega_{ij}^{d}-\omega_{kl}^{d})/2\pi $, which from Eq.~\eqref{theEmission1} has the analytical expression
  	
  	\begin{equation}\label{measEmission}
  		|e|=\left| \iexpectation{V}\right| =  \sqrt{\frac{\Gamma_{_{mn}}}{2}}\left|i\rho_{mn}\right|,
  	\end{equation}
  	
  	\noindent where $ \Gamma_{mn} $ is the \iket{m}\ra\iket{n} relaxation rate between the two undriven levels and $ \rho_{mn} $ is the density matrix element found from the stationary solution of the Markovian master Eq.~\eqref{rwaMarkovian}. The relevant rates for both cases were taken from Table~\ref{tabResults}.
  \end{enumerate}

  The emission measurements of the VNA, $ \left|e'\right| $, were made relative to its output signal, whereas the simulated emissions, $ |e| $, were the absolute values, meaning the two would differ by a proportionality constant
   
  \begin{equation}\label{measContanst}
	  |e| = k|e'|,
  \end{equation}
  
  \noindent where $ k $ was related to the signal outputted by the VNA.\footnote{The signal outputted by the VNA was blanked and it didn't take part in interactions with the artificial atom. Its only relevance was that it set a scale for the emission measurements, $ |e'| $, being made.} Fitting was a case of varying the driving Rabi amplitudes, $ \Omega_{ij}, \Omega_{kl} $, in emission simulations, Eq.~\eqref{measEmission}, to qualitatively\footnote{Quantitative comparisons required the knowledge of $ k $ from Eq.~\eqref{measContanst}, which can be done with suitable calibration measurements.} match the measured $ |e'| $ spectra.
  
  Figure~\ref{meas12} and \ref{meas32} demonstrate the measured and simulated emissions for various driving field power combinations. Simulations reproduce the nature of the splitting, that occurs as as dressed states form for stronger driving powers, Fig.~\ref{theTwoDrive}, to a high degree of accuracy.

\newcommand{\lastGraph}[2]{$ \Omega_{32} = #1, \Omega_{31}=#2\,$MHz}
\newcommand{\preGraph}[2]{$ \Omega_{21} = #1, \Omega_{31}=#2\,$MHz}
  \begin{figure}
	\ifigure{13cm}{drivef31f32Numbered}
  	\caption{\small\textbf{Artificial atom $ \mathbf{\ket{2}\rightarrow\ket{1}} $ emissions under $ \mathbf{\ket{1}\leftrightarrow\iket{3}, \ket{2}\leftrightarrow\iket{3}} $ drives.} Each cell corresponds to a unique set of measurement, made at different powers, $P_{32}, P_{31} $, of the driving fields. The insets show the simulations made with Eq.~\eqref{measEmission}, where emission, $ |e| $, is given as a function driving field detunings, $ \delta\omega_{32}/2\pi, \delta\omega_{31}/2\pi $, from the resonant transitions. The detuning scales on the graphs and insets are identical. The simulated and measured emission strengths, $ |e|, |e'| $, are given in arbitrary units. \textbf{a)} \lastGraph{20}{20}; \textbf{b)} \lastGraph{80}{20}; \textbf{c)} \lastGraph{200}{20}; \textbf{d)} \lastGraph{20}{70}; \textbf{e)} \lastGraph{80}{70}; \textbf{f)} \lastGraph{200}{70}; \textbf{g)} \lastGraph{20}{210}; \textbf{h)} \lastGraph{70}{210}; \textbf{i)} \lastGraph{200}{210}.}
  	\label{meas12}
  \end{figure}

  \begin{figure}
	\ifigure{13cm}{drivef13f12}
	\caption{\small\textbf{Artificial atom $ \mathbf{\ket{3}\rightarrow\ket{2}} $ emissions under $ \mathbf{\ket{1}\leftrightarrow\iket{3}, \ket{1}\leftrightarrow\iket{2}} $ drives.} Each cell corresponds to a unique set of measurement, made at different powers, $P_{31}, P_{21} $, of the driving fields. The insets show the simulations made with Eq.~\eqref{measEmission}, where emission, $ |e| $, is given as a function driving field detunings, $ \delta\omega_{31}/2\pi, \delta\omega_{21}/2\pi $, from the resonant transitions. The detuning scales on the graphs and insets are identical. The simulated and measured emission strengths, $ |e|, |e'| $, are given in arbitrary units. \textbf{a)} \preGraph{10}{15}; \textbf{b)} \preGraph{60}{15}; \textbf{c)} \preGraph{190}{20}; \textbf{d)} \preGraph{20}{25}; \textbf{e)} \preGraph{70}{30}; \textbf{f)} \preGraph{180}{40}; \textbf{g)} \preGraph{20}{60}; \textbf{h)} \preGraph{50}{60}; \textbf{i)} \preGraph{220}{65}.}
  	\label{meas32}
  \end{figure}
 
% \begin{figure}
% 	\ipic{6cm}{\iimage}
% 	\caption{\small\textbf{Comparison of the driving powers, $ \mathbf{P_{ij}} $, set in emission measurement to the Rabi amplitudes, $ \mathbf{\Omega_{ij}} $, used in simulations.} For each set of measurements, a selection of which is shown in Fig.~\ref{meas12}, Fig.~\ref{meas32}, the driving powers underwent a conversion $ P_{ij}\rightarrow\sqrt{10\log_{10}(P_{ij})} $ to convert the driving power in dBm to a driving amplitude, which in accordance with Eq.~\eqref{rabiRelation} is proportional to the Rabi frequency $ \Omega_{ij} $. Plotted are the Rabi frequencies, $ \Omega_{ij} $, that gave the best simulation for each $ P_{ij} $ measurement, error bars indicating the visual uncertainty range. Blue sets correspond to Fig.~\ref{meas12}, red sets to Fig.~\ref{meas32}.}
% \end{figure}
 
\section{Conclusion\label{sec:Conclusion}}
 This project has investigated the interactions of a Three-Level artificial atom, based on the flux degree of freedom, with propagating microwave photons. 
 
 The nature of the photon field determines the dynamics of the system state, which is manifested by the emission of photons and gives rise traceable emission and transmission properties. Initially the system was treated in the Two-Level regime, by applying microwave fields that were resonant with a single atomic transition. These fields' transmissions were monitored to determine decoherence parameters of the system.
 
 The system was then driven in the Three-Level regime by two resonant microwave fields, and the emission from the artificial atom was measured. Simulated 2D emission plots were in qualitative agreement with the measured values, replicating splitting-features that appeared under stronger driving, as a result of the richer energy level structure that formed in the system.
 
 Further work that could be done on the system includes: measuring and simulating \iket{3}\ra\iket{1} emissions when the system is under \iket{3}\lra\iket{2}, \iket{2}\lra\iket{1}, drives, which would lead to a full account of all the emission permutations started in the project; running a microwave power calibration with either Rabi oscillation of Mollow triplet measurements, so Rabi amplitudes, $ \Omega $, in simulations could be quantitatively compared to microwave powers in measurement; perform phase coherent Thee-Level emission measurements - this would involve using the VNA signal as one of the drives, so that the phase of emitted signals can be systematically compared to it, ultimately allowing one to identify phase shifts in the artificial atom. 

\renewcommand{\abstractname}{Acknowledgements}
\begin{abstract}
	{I would like to epxress immense gratitude to Prof. Astafiev for the unparalleled theoretical support during the project, as well as to Dr. Shaikhaidarov and, soon to be doctor, Teresa H\"{o}nigl-Decrinis for all the support during my months at the laboratory.}
\end{abstract}

\newpage
% /APPENDIX CACLUATIONS
%  \begin{equation}
%\begin{pmatrix}
%0\\0
%\end{pmatrix} = i\bigg[\begin{pmatrix}
%0 & \frac{\Omega_{21}}{2}\\\frac{\Omega_{21}}{2}&\delta\omega_{21}
%\end{pmatrix},\begin{pmatrix}
%1-\rho_{22} & \rho_{12}\\\rho_{21}&\rho_{22}
%\end{pmatrix}\bigg] - \begin{pmatrix}
%\Gamma_{21}\rho_{22} & -\gamma_{12}\rho_{21}\\-\gamma_{12}\rho_{21} & - \Gamma_{21}\rho_{22}
%\end{pmatrix},
%\end{equation}
\bibliographystyle{ieeetr}
\bibliography{citations}
\end{document}