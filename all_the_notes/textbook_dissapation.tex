\section{Dissipation\label{sec:linbland1}}
 In this chapter we will ultimately look to understand the image below, which describes the dissipation processes that can occur in a two levels system
 
 \begin{figure}[h]
 	\ifigure{5cm}{cohereneUnd}
 \end{figure}

 In all of the calculation we shall use the density matrix representation in order to represent mixed states that form in a system. Important properties of density matrices are:
 
 \begin{itemize}
 	\item \textbf{Diagonal - off diagonal connection for pure state}
 	
 	\begin{equation}\label{dOffD}
 		\rho=\ketbra{\psi}{\psi} = \begin{pmatrix}
 			\alpha\\\beta
 		\end{pmatrix}\begin{pmatrix}
 			\alpha^{*}&\beta^{*}
 		\end{pmatrix} = \begin{pmatrix}
	 		|\alpha|^2&\alpha\beta^{*}\\\alpha^{*}\beta&|\beta|^2
 		\end{pmatrix} \Rightarrow \red{|\rho_{01}|^2 = \rho_{00}\rho_{11}}
 	\end{equation}
 	\item \textbf{Sum} of the diagonal elements
 	
 	\begin{equation}\label{sum}
 		\red{\rho_{00}+\rho_{11}=1}
 	\end{equation}
 	\item \textbf{Some common expectation values}
 	
 	\begin{equation}\label{expectation}
 		\begin{aligned}
	 		\iaverage{\sigma_z} &= \itrace{\begin{pmatrix}
	 			\rho_{00}&\rho_{01}\\\rho_{10}&\rho_{11}
	 			\end{pmatrix}\begin{pmatrix}
	 			1&0\\0&-1
	 			\end{pmatrix}} = \itrace{\begin{pmatrix}
	 				\rho_{00}&-\rho_{01}\\\rho_{10}&-\rho_{11}
	 			\end{pmatrix}} = \rho_{00}-\rho_{11}; \\
 			&\red{\rho_{11} = \frac{1-\iaverage{\sigma_z}}{2}};\quad \quad \red{\rho_{00} = \frac{1+\iaverage{\sigma_z}}{2}}\\
				\iaverage{\sigma_x} &=  \rho_{01}+\rho_{10}\\
				\iaverage{\sigma_y} &=  i\rho_{01}-i\rho_{10}\\
				\iaverage{\sigma_{+}} &= \frac{\iaverage{\sigma_x}+i\iaverage{\sigma_y}}{2} = \rho_{10}\\
				\iaverage{\sigma_{-}}& = \frac{\iaverage{\sigma_x}-i\iaverage{\sigma_y}}{2} = \rho_{01}\\
 		\end{aligned}
 	\end{equation}
 	
 	\item \textbf{Evolution} is governed by the Von-Neumann equation
 	
 	\begin{equation}\label{vonN}
 		i\hbar\dot{\rho} = \bigg[\mathcal{H},\rho\bigg]
 	\end{equation} 
 	
 	\noindent a simple two level system, $ \mathcal{H} = -\hbar\omega\sigma_z/2 $ giving rise to 
 	
 	\begin{equation}\label{key}
 	\begin{aligned}
 		i\hbar
 			\begin{pmatrix}
	 			\dot{\rho_{00}} & \dot{\rho_{01}}\\\dot{\rho_{10}}&\dot{\rho_{11}} 
	 		\end{pmatrix} 
	 	& = -\frac{\hbar\omega}{2}
	 		\begin{pmatrix}
	 			1 & 0\\0&-1
	 		\end{pmatrix}
	 		\idensity + \frac{\hbar\omega}{2}\idensity\iz \\
	 	& = {-i\omega}{}
	 		\begin{pmatrix}
	 			0 & -\rho_{01}\\-\rho_{10}&0
	 		\end{pmatrix}\\\\
	 	\rho_{00} = &\rho_{00}(0);\quad\rho_{11}(t) = \rho_{11}(0);\quad\rho_{01}(t) = \rho_{01}(0)e^{i\omega t};\quad\rho_{10}(t) = \rho_{10}(0)e^{i\omega t}
 	\end{aligned}
 	\end{equation} 
 \end{itemize}

 \red{This is rotation about the equator of the Bloch sphere.}
 \subsection{Application to relaxation} 
  Now, if we excite the system, we will inevitably observe a decay to the ground state - the probability of the system being in an excited states exponentially decreases
 
 \begin{equation}\label{relaxationRaw}
 	 \frac{d\rho_{11}}{dt}=-\rho_{11}\frac{1}{T} =  -\rho_{11}\Gamma_1\quad\Rightarrow\quad\rho_{11} = \rho_{11}(0)e^{-t/T}.
 \end{equation}
 
 \noindent Such a relaxation will eventually lead to the state of the system to change to a mixed state
 
 \begin{equation}\label{relax2}
 	\rho(0) = \ketbra{1}{1} = \imx{0}{0}{0}{1}_{\red{\text{pure}}} \iright \rho(t=T\ln2) =\imx{\frac{1}{2}}{0}{0}{\frac{1}{2}}_{\red{\text{mixed}}},
 \end{equation}
 
 \noindent which in terms of probabilities has the \textbf{exact same properties} as
 
 \begin{equation}\label{relax3}
 	\rho = \imx{1}{1}{1}{1}_{\red{\text{pure}}}.
 \end{equation}
 
 \noindent The state in Eq.\eqref{relax2} has lost coherence as the system underwent relaxation. The system that was initially in a pure state {when \red{$ |\rho_{01}(0)|^2=\rho_{00}(0)\rho_{11}(0) $}} to a mixed state when \red{$ |\rho_{01}(t)|^2\le\rho_{00}(t)\rho_{11}(t) $}. The off diagonal terms loose information and tend to zero. 
 
 This is accounted for by writing the Linbland term
 
 \begin{equation}
 \dot{\rho} = \frac{-i}{\hbar}\bigg[\mathcal{H},\rho\bigg] + \mathcal{L};\quad\quad\mathcal{L} = \begin{pmatrix}
 \blue{\rho_{11}\Gamma_1 }& \red{-\Gamma_2\rho_{01}}\\\red{-\Gamma_2\rho_{01}}&\blue{-\Gamma_1\rho_{11}}
 \end{pmatrix}, 
 \end{equation}
 
 \begin{itemize}
 	\item \textbf{\blue{Relaxation}} will cause $ \rho_{00} $ to increase and $ \rho_{11} $ to decrease
 	\item \textbf{\red{Pure dephasing}} - its is well known that the diagonal elements of a density matrix \textbf{must} obey
 	\begin{equation}\label{pd1}
 		|\rho_{01}|^2\le\rho_{00}\rho_{11},
 	\end{equation}
 	
 	\noindent equality being achieved for a pure state. During relaxation $ \rho_{00} $ increases and $ \rho_{11} $ decreases, but as relaxation can only decrease coherence, it is only the latter term that will effect off diagonal elements
 	
 	\begin{equation}\label{pd2}
 		\ialigned{\delta|\rho_{01}| & \le \sqrt{\rho_{00}(\rho_{11}+\delta\rho_{11})} - \sqrt{\rho_{00}\rho_{11}}\\
 		\delta\rho_{11} & = -\rho_{11}\Gamma_1dt\quad\text{from Eq.\eqref{relaxationRaw}}}\iright\ialigned{\delta|\rho_{01}|&\le\sqrt{\rho_{00}\rho_{11}}\bigg(\sqrt{1-\Gamma_1dt}-1\bigg)\\&\approx\sqrt{\rho_{00}\rho_{11}}\bigg(-\frac{\Gamma_1}{2}dt\bigg)\\&\red{ = -|\rho_{01}\frac{\Gamma_1}{2}dt}},
 	\end{equation}
 	
 	\noindent meaning that the decay of the off diagonal terms cannot be slower than $ \frac{\Gamma_1}{2} $. Taking into account other ways of diagonal terms dephasing, $ \Gamma_\phi $, one assigns 
 	
 	\begin{equation}\label{pd3}
 		\Gamma_2 = \frac{\Gamma_1}{2}+\Gamma_\phi;\quad T_2 = \frac{1}{\Gamma_2}
 	\end{equation}
 \end{itemize}

	\iframe{\red{This is a very crude argument, and possibly a coincidence. Nevertheless, the required $ \Gamma_2 $ factos is distilled.}}
 
 \subsection{Pure dephasing}
  Attention will now be direct to the $ \Gamma_2 $ term introduced in Eq.\eqref{pd3}, and more specifically its pure dephasing component $ \Gamma_\phi $. Pure dephasing is due to noise that affects the separation of the qubit energy levels. Recall from Sec.~\ref{subsec:Rabi}, that a qubit system that is subjected to a resonant driving field (whose frequency, $ \omega $, matches the energy separation, $ \hbar\omega_0 $, of the levels)
  
  \begin{equation}\label{RabiHamiltonian}
  	\mathcal{H} = -\frac{\hbar\omega_0}{2}\sigma_z-\hbar\Omega\cos(\omega_0 t)\sigma_x,
  \end{equation}
  
  \noindent will, in the RWA, evolve from an initial state \iket{0} according to
  
  \begin{equation}\label{RabuSolution}
  	\ket{\Psi} = U\ket{0} = \cos(\frac{\Omega t}{2})\ket{0}+e^{i\pi/2}\sin(\frac{\Omega t}{2})\ket{1}.
  \end{equation}
  
  \noindent Should the separation of the levels vary, $ \hbar\omega_0 \rightarrow \hbar\omega' $, then the qubit will be in resonance with another component of the driving field, $ \hbar\Omega'\cos(\omega't)\sigma_x $, meaning that the evolved state will be a superposition of the form
  
  \begin{equation}\label{RabuSolutionSuperpos}
  	\iket{\Psi} = \sum_i \alpha_i\bigg[\cos\bigg(\frac{\Omega_i t}{2}\bigg)\iket{0}+i\sin\bigg(\frac{\Omega_i t}{2}\bigg)\iket{1}\bigg],
  \end{equation}
  
  \noindent leading to a probability of observing the system in \iket{0} of
  
  \begin{equation}\label{RabiProb0}
  	P_0  = |\bra{0}\iket{\Psi}|^2 = \sum_i \alpha_i\cos^2\bigg(\frac{\Omega_it}{2}\bigg)=\quad \left\lbrace \text{ equal weights for 3 states}\right\rbrace \quad= \frac{1}{3}\bigg(\cos^2\bigg(\frac{\Omega_1t}{2}\bigg)+\cos^2\bigg(\frac{\Omega_2t}{2}\bigg)+\cos^2\bigg(\frac{\Omega_3t}{2}\bigg)\bigg),
  \end{equation}
  
  \noindent which leads to a probability 'averaging' as one monitors the system for longer times $ t $, the characteristic decay time being labelled as$ T_\phi =1/\Gamma_\phi $. The Rabi oscillations are `washed out' due to this dephasing. The bigger the fluctuations of the energy levels the stronger the washing out. If one hopes to observe any oscillations, then the Rabi frequency $ \Omega>>1/T_\phi = \Gamma_2 $ to ensure that oscillation occur before dying off.
  
  \begin{figure}
  	\ifigure{4cm}{deph1}
  	\ifigure{4cm}{deph2}
  \end{figure}
 
 \newpage
 
 \subsection{Dynamics with Pauli Matrices\label{subsec:dynamics_with_pauli}}
  Summarising up to this point, we have argued for the appearance of the Linbland term in the Master equation to account for decoherence and relaxation processes in the system 
 \red{{\large  \begin{equation}\label{Totalequation}
  	\mathcal{L} = \imx{\Gamma_1\rho_{11} - \Gamma^{ex}\rho_{00}}{-\Gamma_2\rho_{01}}{-\Gamma_2\rho_{10}}{\Gamma^{ex}\rho_{00}-\Gamma_1\rho_{11}};\quad \dot{\rho} = -\frac{i}{\hbar}\big[\mathcal{H},\rho\big]+\mathcal{L},
  \end{equation}}}
  
  \noindent and shown a few useful expectation values, that allow one to express the dynamics of the system via the expectation values of the Pauli matrices, Eq.\eqref{expectation}
  
  \red{{\large \begin{equation}\label{expectationPauli}
  	\rho_{00} = \frac{\iaverage{\sigma_z}+1}{2};\quad \rho_{01}=\frac{\iaverage{\sigma_x}-i\iaverage{\sigma_y}}{2} = \iaverage{\sigma_{+}};\quad\rho_{10}=\frac{\iaverage{\sigma_x}+i\iaverage{\sigma_y}}{2} = \iaverage{\sigma_{-}};\quad\rho_{11} = \frac{1-\iaverage{\sigma_z}}{2}.
  \end{equation}
  \begin{equation}\label{expectationPauli2}
  	\iaverage{\sigma_x}=\rho_{01}+\rho_{10};\quad\iaverage{\sigma_y} = i\rho_{01}-i\rho_{10};\quad\iaverage{\sigma_z}=\rho_{00}-\rho_{11}
  \end{equation}}}
  
  \noindent Lets compute the evolution of these expectation values
  
  \begin{equation}\label{evolution}
  	\ialigned{\difffrac{\iaverage{\sigma_j}}{t}&=\itrace{\sigma_j\difffrac{\rho}{t}} = \itrace{-\frac{i}{\hbar}\sigma_j\big(\mathcal{H}\rho - \rho\mathcal{H}\big)+\sigma_j\mathcal{L}}\\
  	\mathcal{H}&  = \frac{\hbar\Omega}{2}\bigg(\sigma_x\cos(\phi)-\sigma_y\sin(\phi)\bigg),}
  \end{equation}
  
  \noindent and evaluating for all the matrices
  
  \begin{equation}\label{pauliDeriv}
  	\begin{aligned}
  	\difffrac{\iaverage{\sigma_x}}{t}&=-i\frac{\Omega}{2}\itrace{\cancel{\big(\sigma_x\sigma_x\rho-\sigma_x\rho\sigma_x\big)}\cos(\phi) + \big(\sigma_x\sigma_y\rho-\sigma_x\rho\sigma_y\big)\sin(\phi)} + \itrace{\sigma_x\mathcal{L}}\\
  	& = \Omega\iaverage{\sigma_z}\sin(\phi)-\Gamma_2\iaverage{\sigma_x}\\
  	\difffrac{\iaverage{\sigma_y}}{t}& = \Omega\iaverage{\sigma_z}\sin(\phi)-\Gamma_2\iaverage{\sigma_y}\\
  	\difffrac{\iaverage{\sigma_y}}{t}& = -\Omega\big(\iaverage{\sigma_x}\cos(\phi)+\iaverage{\sigma_y}\sin(\phi)\big)-\Gamma_1\iaverage{\sigma_z}+\Gamma_1\\
  	\end{aligned}
  \end{equation}
  
  \noindent or in more compact form
  
  \begin{equation}\label{pauliEv}
  	\difffrac{}{t}\begin{pmatrix}
  		\iaverage{\sigma_x}\\\iaverage{\sigma_y}\\\iaverage{\sigma_z}
  	\end{pmatrix} = \begin{pmatrix}
	  	-\Gamma_2&0&\Omega\sin(\phi)\\0&-\Gamma_2&\Omega\cos(\phi)\\-\Omega\sin(\phi)&-\Omega\cos(\phi)&-\Gamma_1\\
  	\end{pmatrix}\begin{pmatrix}
  	\iaverage{\sigma_x}\\\iaverage{\sigma_y}\\\iaverage{\sigma_z}
  	\end{pmatrix}+\begin{pmatrix}
  	0\\0\\\Gamma_1
  	\end{pmatrix}\iright\red{\difffrac{\vec{\iaverage{\sigma}}}{t} = B\vec{\iaverage{\sigma}}+\vec{b}.}
  \end{equation}
  
  \noindent The dynamics of this system are the exact same as for a spin 1/2 particle in a magnetic field \textbf{B} studied in Chapter \ref{spin12} Eq.~\eqref{eqn:evolutionBloch}. The different components of the vector \isigma can be plotted on a sphere. Note that unlike the Bloch sphere used previously, these vectors do \textbf{not} have to be on the surface. The various states $ \vec{\iaverage{\sigma}} $ correspond to
  
  \begin{align}
  	\rho_{00}=1 & \iright \isigma = \ithreeMatrix{0}{0}{1}\\
  	\rho_{11}=1 & \iright \isigma = \ithreeMatrix{0}{0}{-1}\\
  	\rho_{00}=\rho_{11}&\iright\isigma = \ithreeMatrix{\cos(\phi)}{\sin(\phi)}{0}
  \end{align}
  
  \begin{figure}[h]
  	\ifigure{7cm}{sphere}
  \end{figure}

	\newpage
	
 \paragraph{Starting in the superposed state}
 
 So now let us study some dynamics, where, unless specified, we take the initial state to be $ \rho_{00}=1/2 $ i.e. on the equator where the atom has equal occupation in levels \iket{0} and \iket{1}. An general state would be
 
 \[
 	\Psi = \frac{\iket{0}+e^{i\phi}\iket{1}}{\sqrt{2}} \quad\ra\quad \rho = \frac{1}{2}\imatrix{1}{e^{-i\phi}}{e^{i\phi}}{1}
 \]
 
 \begin{itemize}
 	\item \textbf{\red{No driving} No decoherence} - the superposed state will remain superposed.
 	\[	
	 	\begin{aligned}
	 	\mathcal{H} = -\frac{\hbar\omega}{2}\sigma_z &\iRa  U = \imatrix{e^{i\omega t}}{0}{0}{e^{-i\omega t}}\\
	 	&\iRa \rho(t) = U\rho(0)U\idagger = \frac{1}{2}\imatrix{e^{i\omega t}}{e^{i(\omega t - \phi)}}{e^{-i(\omega t - \phi)}}{e^{-i\omega t}}\imatrix{e^{-i\omega t}}{0}{0}{e^{+i\omega t}}\\
	 	&\qquad \qquad\qquad\qquad\red{= \frac{1}{2}\imatrix{1}{e^{i(2\omega t - \phi)}}{e^{-i(2\omega t - \phi)}}{1}}
	 	\end{aligned}
 	\]
 	
 	\red{But when we consider the situation from by entering the rotating frame (the interaction picture), see Chapter~\ref{sec:unitary} and Ch.~\ref{sec:interaction}}
 	
 	\[
 		U_0(t) = e^{i\frac{\mathcal{H}_0}{\hbar}t},	
 	\]
 	
 	\noindent the interaction picture Hamiltonian will be of the form Eq.~\eqref{eqn:interactionPictureHamiltonian}
 	
 	\[
 		\mathcal{H}_i = U_0(t)\idagger\bigg[\mathcal{H} - \mathcal{H}_0\bigg]U_0(t) \equiv 0,
 	\]
 	
 	\noindent and so there will be no evolution in the system in this rotated frame that we work with. The expectation values, \isigmax, \isigmaz are the same as in the rotated and non-rotating frames:
 	
 	\begin{equation}
 	_I\bra{\psi}\hat{O}_I\ket{\psi}_I = _S\bra{\psi}U_0U_0^\dagger\hat{O}_SU_0U_0^\dagger\ket{\psi}_s \equiv _S\bra{\psi}\hat{O}_S\ket{\psi}_S
 	\end{equation}
 	
 	\ipic{4cm}{doD2}
 	\ipicCaption{4cm}{doD1}{On the equator the atom has equal weight in \iket{0} and \iket{1}, so \isigmaz is 0. The \isigmax is stationary.}

 	
 
 	\newpage
 	\item \textbf{\red{Driving} No decoherence, $ \Gamma_1 = 0, \Gamma_2 = 0 $}
 	
 	We evolution of the wavefunction is taken to be \red{under a drive from a resonant field}
 	
 	\begin{equation}\label{dunamicState}
 	\ket{\Psi} = \cos\big(\frac{\Omega t}{2}\big)\ket{0}+i\sin\big(\frac{\Omega t}{2}\big)\ket{1};\quad\rho=\ketbra{\Psi}{\Psi}
 	\end{equation}
 	
 	
 	\begin{equation}\label{dyn1}
 		\ialigned{
 			\rho_{11} & = \sin^2(\Omega t/2) & \iaverage{\sigma_x} & = \rho_{01}+\rho_{10} = \sin(\Omega t)\\
 			\rho_{01} & = \sin(\Omega t/2)\sin(\Omega t/2) = \frac{\sin(\Omega t)}{2} & \iaverage{\sigma_y} & = i\rho_{01}-i\rho_{10} = 0\\
 			\rho_{10} & = \sin(\Omega t/2)\sin(\Omega t/2) = \frac{\sin(\Omega t)}{2} & \iaverage{\sigma_z} & = 1- 2\rho_{00} = \cos(\Omega t)\\
 		}
 	\end{equation}
 	
 	\noindent and the point on the sphere simply rotates about the \iaverage{\sigma_y} axis. If there was a phase shift in the driving field $ \phi $ then the rotation would turn by that angle about the \iaverage{\sigma_z} axis. Overall these are the Rabi oscillations
 	
 	\begin{figure}[h]
 		\ifigure{4cm}{dyn1}
 		\ifigure{4cm}{dnO}
 	\end{figure}
 
  	\newpage
   \item \textbf{\textbf{\red{Driving} With decoherence $ \Gamma_1 = 1, \Gamma_2 = 0.5, \gamma = \Gamma_1+\Gamma_2/2 $}}, it is simpler to solve the dynamics of Eq.\eqref{pauliEv} for \iaverage{\vec{\sigma}}
   
   For the case when one starts off with state $ \rho_{00} = 1 $
   
   \begin{equation}\label{dyn2}
   \ialigned{
   	\iaverage{\sigma_x} & \approx e^{-\gamma t} \sin(\Omega t) & \rho_{00} & = \frac{\iaverage{\sigma_z}+1}{2} \approx\frac{1+e^{-\gamma t} \cos(\Omega t)}{2}\\
   	\iaverage{\sigma_y} & = 0 & \rho_{01} & = \frac{\iaverage{\sigma_x}-i\iaverage{\sigma_y}}{2} = \frac{e^{-\gamma t} \sin(\Omega t)}{2}\\
   	\iaverage{\sigma_z} & \approx e^{-\gamma t} \cos(\Omega t) & \rho_{10} & = \frac{\iaverage{\sigma_x}+i\iaverage{\sigma_y}}{2} = \frac{e^{-\gamma t} \sin(\Omega t)}{2}\\
   }
   \end{equation}
   
   \noindent In this case the two components begin to spiral in as a result of the dephasing. The central point is the stationary condition in which both \iaverage{\sigma_x} and \iaverage{\sigma_z} take on a finite value - the stationary state value. 
   
   \begin{figure}[h]
   	\ifigure{6cm}{dyn2}
   	\ifigure{5cm}{dd}
   \end{figure}

	\newpage
   \item \textbf{\red{No driving} with decoherence and relaxation} $ \rho_{00}=\rho_{11} = 1/2 $ then
   
   \begin{equation}\label{dyn21}
   	\ialigned{
   	\iaverage{\sigma_x} & = e^{-\Gamma_2 t} & \rho_{00} & = 1-\frac{e^{-\Gamma_1 t}}{2}\\
   	\iaverage{\sigma_y} & = 0 & \rho_{01} & = \frac{e^{-\Gamma_2 t}}{2} \\
   	\iaverage{\sigma_z} & = 1 - e^{-\Gamma_1 t} & \rho_{10} & = \frac{e^{-\Gamma_2 t}}{2} \\
   }
   \end{equation}
   
   The system, initially on the `equator' of the sphere, transitions to the ground state
   
     \begin{figure}[h]
   	\ifigure{6cm}{dyn21}   	
   	\ifigure{6cm}{dyn22}
   \end{figure}
   
   \newpage
   \item \textbf{\red{No driving} Only dephasing $ \Gamma_1=0, \Gamma_2 \ne 0 $} when we start off in a superposed state $ \rho_{11}=1/2\rightarrow\iaverage{\sigma_z}=0 $
   
   \begin{equation}\label{dyn3}
   \ialigned{
   	\iaverage{\sigma_x} & = e^{-\Gamma_2 t} & \rho_{00} & = \frac{1}{2}\\
   	\iaverage{\sigma_y} & = 0 & \rho_{01} & = \frac{e^{-\Gamma_2 t}}{2} \\
   	\iaverage{\sigma_z} & = 0 & \rho_{10} & = \frac{e^{-\Gamma_2 t}}{2} \\
   }
   \end{equation}
   
   The expectation value simply moves towards the origin, where $ \iaverage{\sigma_z} = 0 $ and the system becomes equally likely to exist in the excited or ground state $ \rho_{00}=\rho_{11} $. The off diagonal terms loose coherence.
   
   \begin{figure}[h]
   	\ifigure{6cm}{dyn31}
   	\ifigure{4cm}{nd1}
   \end{figure}
 \end{itemize}
  
 {\large \red{\textbf{Thus, relaxation will tend to shift \iaverage{\sigma_z} to 1, and together with pure dephasing it shortens the expectation values of the \iaverage{\sigma_x} and \iaverage{\sigma_y} directions.}}}
\newpage