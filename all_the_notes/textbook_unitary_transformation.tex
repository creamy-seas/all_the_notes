\section{Unitary transformations and the rotating frame\label{sec:unitary}}
  The Hermitian conjugate of an operator is defined as
 
 \begin{equation}\label{uniHermetian}
 U^{\dagger} = (U^{*})^{T},
 \end{equation}
 
 \noindent and the operator can be classified under two types
 
 \begin{itemize}
 	\item $ U^{\dagger}\equiv U $ a Hermitian operator;
 	\item $ U^{\dagger}\equiv U^{-1} $ is a unitary operator. 
 \end{itemize}
 
 The most important equation for unitary operators is
 
 \begin{equation}\label{uniDef}
 U^{\dagger}U=\mathbb{I},
 \end{equation}
 
 \noindent examples of which are Pauli spin matrices $ \sigma_x, \sigma_y, \sigma_z $. 
\begin{framed} 
 \red{Now we shall examine {the rotation operator}, { which rotates the state of a two levels system by an angle $ 2\alpha $ about the $ j $ axis of the Bloch sphere}}
 \end{framed}

 \red{\begin{equation}\label{uniRotation}
 \begin{aligned}
 U &= \exp\bigg[i\alpha\sigma_j\bigg] = \sum_{k}\frac{(i\alpha)^k}{k!}\sigma_j^k \\&= \sum_{k=0}\frac{{\alpha^{2k}(-1)^k}}{2k}\mathbb{I}+i\sum_{k=0}\frac{{\alpha^{2k+1}(-1)^k}}{2k+1}\sigma_j\\
 &= \cos(\alpha)\mathbb{I}+i\sin(\alpha)\sigma_j,
 \end{aligned}
 \end{equation}}
 
 \noindent where we utilise $ \sigma_j\sigma_j = \mathbb{I} $. For the three Pauli spin matrices this will read
 
 \begin{equation}\label{uniPauli}
 U(\sigma_x) = \begin{pmatrix}
 \cos{\alpha} & i\sin{\alpha}\\i\sin{\alpha}&\cos{\alpha}
 \end{pmatrix};\quad	U(\sigma_u) = \begin{pmatrix}
 \cos{\alpha} & \sin{\alpha}\\-\sin{\alpha}&\cos{\alpha}
 \end{pmatrix};\quad	U(\sigma_z) = \begin{pmatrix}
 e^{i\alpha} & 0\\0&e^{-i\alpha}
 \end{pmatrix}.
 \end{equation}
 
 \noindent Furthermore, one can double check the unitarity of the matrix:
 
 \begin{equation}\label{uniunitary}
 U^{\dagger}U =\bigg( \cos(\alpha)\mathbb{I}+i\sin(\alpha)\sigma_j\bigg)bigg( \cos(\alpha)\mathbb{I}-i\sin(\alpha)\sigma_j\bigg) = \mathbb{I}\cos^2(\alpha)+\sigma_j\sigma_j\sin^2(\alpha)\equiv\mathbb{I}.
 \end{equation}
 
 \noindent When a unitary transformation \red{$ \Psi'=U\Psi \leftrightarrow \Psi = U^{\dagger}\Psi'$} is applied, the \schrodinger equation will be modified as such
 
 \begin{equation}\label{uninewschrodinger}
 \begin{aligned}
 i\hbar\difffrac{U^{\dagger}\Psi'}{t} & = \mathcal{H}U^{\dagger}\Psi'\\
 i\hbar{U}^{\dagger}\difffrac{\Psi'}{t} + i\hbar\dot{U}^{\dagger}{\Psi'} & = \mathcal{H}U^{\dagger}\Psi' \quad\text{ \red{no time dependance of $ U $}}\\
 i\hbar\difffrac{\Psi'}{t} & = \bigg[U\mathcal{H}U^{\dagger} \red{-i\hbar U\dot{U}^{\dagger}}\bigg]\Psi'\\
 \end{aligned}
 \end{equation}
 
 \iframe{Expectation values are conserved in unitary transformations:
 	\[
 		\begin{aligned}
 			\bra{\psi}A\ket{\psi} & \equiv \left(\bra{\psi}U\right)U\idagger AU\left(U\idagger\ket{\psi}\right)\\
 			& \equiv \bra{\psi'}A'\ket{\psi'}
 		\end{aligned}
 	\]
 	\red{So expectation values of an operator remain unchanged. \textbf{This means that physical quantities can be equally well calculated in transformed frames e.g. enter the rotating frame to find \isigmax.}}
	}
 
 \noindent Also important are commutation properties between the three Pauli Matrices
 
 \begin{equation}\label{uniComm}
 	\sigma_i\sigma_j=-\sigma_j\sigma_i,
 \end{equation}
 
 \noindent which results in the following relations involving the unitary operator of rotation about the y-axis
 
 \begin{equation}\label{uniComm1}
 	\left\lbrace\begin{aligned}
	 	U_{y}\sigma_z & = \bigg(\cos(\alpha)\mathbb{I}+i\sin(\alpha)\sigma_y\bigg)\sigma_z = \sigma_z\bigg(\cos(\alpha)\mathbb{I}-i\sin(\alpha)\sigma_y\bigg) = \sigma_zU_y^{\dagger}\\
	 	U_{y}\sigma_x & = \sigma_xU_y^{\dagger}\\
	 	U_{y}\sigma_y & = \sigma_yU_y\\
	 	 	\end{aligned}\right.
 \end{equation}
 
 \subsection{Example application\label{subsec:ExampleApplication}}
  A general two levels system with energy separation $ \varepsilon $ interaction between the two states of strength $ \Delta $ can be written
  
     \begin{equation}
  \label{l1-uni}
  \begin{aligned}
  \mathcal{H} = \left(\begin{matrix}
  -\epsilon/2 & 0\\ 0 & \epsilon/2
  \end{matrix}\right) + \begin{pmatrix}
  0 & -\Delta/2\\-\Delta/2 & 0
  \end{pmatrix} & = {-\frac{\epsilon}{2}\sigma_z-\frac{\Delta}{2}\sigma_x}\\
  & = -\frac{\sqrt{\epsilon^2+\Delta^2}}{2}\left(\frac{\epsilon}{\sqrt{\epsilon^2+\Delta^2}}\sigma_z+\frac{\Delta}{\sqrt{\epsilon^2+\Delta^2}}\sigma_x\right)\\
  & = -\frac{\Delta E}{2}\left(\cos\left(\theta\right)\sigma_z+\sin\left(\theta\right)\sigma_x\right)\\
  {\Rightarrow \left\lbrace\begin{aligned}
  	\mathcal{H} & = -\frac{\Delta E}{2}\big(\sigma_z\cos(\theta)+\sigma_x\sin(\theta)\big)\\
  	\Delta E & = \sqrt{\epsilon^2+\Delta^2}\\
  	\tan(\theta) & = \frac{\Delta}{\epsilon}
  	\end{aligned}\right.} 
  \end{aligned},
  \end{equation}
  
  \begin{figure}[h]
  	\ifigure{4cm}{rot}
  \end{figure}

  \noindent Now we perform a transformation to rotate the state by $ 2\alpha = \theta $
  
  \begin{equation}\label{unitary1}
   \mathbf{U=e^{i\frac{\theta}{2}\sigma_y}} =  \cos(\alpha)\mathbb{I}+i\sin(\alpha)\sigma_y
  \end{equation}
  
  \noindent to rotate the basis in this plane. \red{\textbf{Note that the transformation uses an angle HALF of the required turn}}. The {time independent Hamiltonian} will be transformed according to Eq.\eqref{uninewschrodinger}, and evaluating using the commutation relations Eq.\eqref{uniComm1}
  
  \begin{equation}\label{uniRot}
  	\begin{aligned}
	  	\mathcal{H}' & = U\mathcal{H}U^{\dagger} = U \bigg[-\frac{\Delta E}{2}\big(\sigma_z\cos(\theta)+\sigma_x\sin(\theta)\big)\bigg]\bigg[\cos(\theta/2)\mathbb{I}\red{-}i\sin(\theta/2)\sigma_y\bigg]\\
	  	& = U \bigg[\cos(\theta/2)\mathbb{I}+i\sin(\theta/2)\sigma_y\bigg]\bigg[-\frac{\Delta E}{2}\big(\sigma_z\cos(\theta)+\sigma_x\sin(\theta)\big)\bigg]\\
	  	& = UU\mathcal{H} =-\frac{\Delta E}{2} \bigg[\cos(\theta)\mathbb{I}+i\sin(\theta)\sigma_y\bigg]\bigg[\big(\sigma_z\cos(\theta)+\sigma_x\sin(\theta)\big)\bigg]\\
	  	& = -\frac{\Delta E}{2}\bigg[\cos^2(\theta)\sigma_z+\sin(\theta)\cos(\theta)\sigma_x+i\sin(\theta)\cos(\theta)\red{\sigma_y\sigma_z}+i\sin^2(\theta)\red{\sigma_y\sigma_x}\bigg]\\
	  	& = -\frac{\Delta E}{2}\bigg[\cos^2(\theta)\sigma_z+\sin(\theta)\cos(\theta)\sigma_x+i\sin(\theta)\cos(\theta)\red{i\sigma_x}+i\sin^2(\theta)\red{-i\sigma_z}\bigg]\\
	  	& = -\frac{\Delta E}{2}\sigma_z,
  	\end{aligned}
  \end{equation}
  
  
  \noindent with eigenstates  $ \iket{\tilde{0}}, \iket{\tilde{1}} $ at energies $ -\Delta E/2, +\Delta E/2 $ respectively. Recalling that the transformation we applied was
  
  \begin{equation}\label{uniTransform}
  	\tilde{\Psi} = U\Psi \quad\Rightarrow\quad \Psi = U^{\dagger}\tilde{\Psi},
  \end{equation}
  
  \noindent in the initial eigenbasis, the two states will read
  
  \begin{equation}\label{uniInitial}
  	\begin{aligned}
	  	\iket{0}_{\text{initial}} & = U^{\dagger}\ket{\tilde{0}} = \bigg(\cos(\theta/2)\mathbb{I}+i\sin(\theta/2)\sigma_y\bigg)\begin{pmatrix}
	  	1\\0
	  	\end{pmatrix} = \begin{pmatrix}
	  	\cos(\theta/2)\\\sin(\theta/2)
	  	\end{pmatrix} \\
	  		  	\iket{1}_{\text{initial}} & = U^{\dagger}\ket{\tilde{1}} = \begin{pmatrix}
	  	-\sin(\theta/2)\\\cos(\theta/2)
	  	\end{pmatrix}.
  	\end{aligned}
  \end{equation}
  
  
  \noindent So ultimately, rotating by $ \theta/2 $ will rotate the basis so as to cancel the interaction term.
  
 \subsection{Example application 2\label{subsec:Rabi}}
  Now we have a qubit that is driven by a resonant external field, \red{this time it is not a $ \Delta $ intractions as the strength varies with time}
  
  \begin{equation}\label{app2}
  	\mathcal{H} = -\frac{\hbar\omega_0}{2}\sigma_z-\hbar\Omega\cos(\omega_0 t)\sigma_x
  \end{equation}
  
  \noindent for which we shall try the unitary transformation
  
  \begin{equation}\label{app2Try}
  	U(t) = \exp\left[-i\frac{\omega_0 t}{2}\sigma_z\right]
  \end{equation}  
  
  \noindent resulting in the Hamiltonian 
  
  \begin{equation}\label{app2New}
  	\begin{aligned}
  	\mathcal{H'} & = U\mathcal{H}U^{\dagger} - i\hbar U\dot{U}^{\dagger}\\
  	& = -\frac{\hbar\omega}{2}e^{-i\omega_0t/2\sigma_z}\sigma_ze^{+i\omega_0t/2\sigma_z}-\hbar\Omega\frac{e^{i\omega t}+e^{-i\omega t}}{2}e^{-i\omega_0t/2\sigma_z}\sigma_xe^{i\omega_0t/2\sigma_z}- i\hbar e^{-i\omega_0t/2\sigma_z}\bigg(i\frac{\omega}{2}\sigma_z\bigg)e^{i\omega_0t/2\sigma_z}\\
  	& = -\frac{\hbar\Omega}{2}\bigg(e^{i\omega t}+e^{-i\omega t}\bigg)e^{-i\omega_0t/2\sigma_z}\red{e^{(-1)i\omega_0t/2\sigma_z}\sigma_x}\\
  	& = -\frac{\hbar\Omega}{2}\bigg(e^{i\omega t}+e^{-i\omega t}\bigg){e^{-i\omega_0t\sigma_z}\sigma_x}\\
  	& =-\frac{\hbar\Omega}{2}\bigg(e^{i\omega t}+e^{-i\omega t}\bigg)\begin{pmatrix}
  	e^{-i\omega_0t}&0\\0&e^{+i\omega_0t}
  	\end{pmatrix}\begin{pmatrix}
  	0&1\\1&0
  	\end{pmatrix}\\
  	& =-\frac{\hbar\Omega}{2}\bigg(e^{i\omega t}+e^{-i\omega t}\bigg)\begin{pmatrix}
  	0&e^{i\omega_0t}\\e^{-i\omega_0t}&0
  	\end{pmatrix}
  	\\
  	& =-\frac{\hbar\Omega}{2}\begin{pmatrix}
  	0&1+e^{2i\omega_0t}\\1+e^{-2i\omega_0t}&0
  	\end{pmatrix}
  	\\
  	& \approx -\frac{\hbar\Omega}{2}\begin{pmatrix}
  	0&1\\1&0
  	\end{pmatrix}
  	\\
  	& \approx -\frac{\hbar\Omega}{2}\sigma_x
  	\end{aligned}
  \end{equation}
  
  \noindent where we have applied the RWA where we neglect fast rotating terms (which correspond to non conserved energy processes). \iframe{Physically what we have done in entered the frame of the initial qubit Hamiltonian ($ \mathcal{H} = -\frac{\hbar\omega}{2}\sigma_z $), implicitly taking into account the raw evolution to concentrate only on the driving field contribution.
  
  \begin{center}
  	Coupling of levels via radiation = RWA.
  \end{center}}

  Now the evolution of the state 
  
  \begin{equation}\label{app2Ev}
  	U(t) = e^{-i\mathcal{H'}/\hbar t} = e^{i\Omega t/2\sigma_x}
  \end{equation}
  
  \noindent gives according to Eq.\eqref{uniPauli}
  
  \begin{equation}\label{app2State}
  	\ket{\Psi} = U\ket{0} = \cos(\frac{\Omega t}{2})\ket{0}+e^{i\pi/2}\sin(\frac{\Omega t}{2})\ket{1}.
  \end{equation}
  
  \begin{figure}
  	\ifigure{2cm}{rotX}
  \end{figure}
  
  \iframe{States \iket{0}, \iket{1} of the same energy (due to the driving) interact with each other}
  \noindent \textbf{It may be interesting to observe the case when the drive is changed}
  
  \begin{equation}\label{app2NewPhase}
  	\hbar\Omega\cos(\omega t)\sigma_x \quad \rightarrow \quad \hbar\Omega\cos(\omega t+\mathbf{\phi})\sigma_x = \hbar\Omega\bigg[\cos(\omega t)\cos(\phi)-\sin(\omega t)\sin(\phi)\bigg]\sigma_x
  \end{equation}
  
  \noindent for which the procedure for the cosine part using the same unitary transformation Eq.\eqref{app2Try} gives
  
  \begin{equation}\label{app2Cos}
  	-\frac{\hbar\Omega}{2}\cos(\phi)\sigma_x,
  \end{equation}
  
  \noindent while the $ \sin(\omega t) = i(e^{i\omega t}-e^{-i\omega t})/2 $ gets
  
  \begin{equation}\label{app2Sin}
  	-\frac{\hbar\Omega}{2}\sin(\phi)\sigma_y,
  \end{equation}
  
  \noindent giving
  
  \begin{equation}\label{app2Combined}
  	\mathcal{H'} = -\frac{\hbar\Omega}{2}\bigg(\sigma_x\cos\phi+\sigma_y\sin\phi\bigg)
  \end{equation}
  
 
 \subsection{Qubit operations}
  In the previous section, the qubit system was subjected to a field that induced rotation about the x-axis. In a similar way, one can perform qubit operations about other axes 
  
  {\footnotesize \begin{table}[h]
  	\begin{center}
  		\begin{tabular}{|c|c|c|c|c|}
  			\hline \textbf{Axis} & \textbf{Field operator $ \mathcal{H} $} & \textbf{Unitary evolution}$ U=\exp\left[i\mathcal{H}t/\hbar\right] $ & $ \mathbf{t=\frac{\pi}{\Omega}} $& $ \mathbf{t=\frac{\pi}{2\Omega}} $\\
  			X&$ -\frac{\hbar\Omega}{2}\sigma_x $ & $ \begin{pmatrix}
  			\cos{\Omega t/2} & -i\sin{\Omega t/2}\\-i\sin{\Omega t/2}&\cos{\Omega t/2}
  			\end{pmatrix} $ & NOT = $\begin{pmatrix}
  			0 & -i \\-i&0
  			\end{pmatrix} $ &\\
  			Y&$ -\frac{\hbar\Omega}{2}\sigma_y $ & $ \begin{pmatrix}
  			\cos{\Omega t/2} & -\sin{\Omega t/2}\\\sin{\Omega t/2}&\cos{\Omega t/2}
  			\end{pmatrix} $ & FLIP = $\begin{pmatrix}
  			0 & -1 \\1&0
  			\end{pmatrix} $ & H = $ \frac{1}{\sqrt{2}}\begin{pmatrix}
  			1 & -1 \\1&1
  			\end{pmatrix} $\\
  			Z&$ -\frac{\hbar\Omega}{2}\sigma_z $ & $ \begin{pmatrix}
  			e^{-i\Omega t/2} & 0\\0&e^{i\Omega t/2}
  			\end{pmatrix} $&$\begin{pmatrix}
  				-i&0\\0&i
  			\end{pmatrix}$&\\\hline
  		\end{tabular}
  	\end{center}
  \end{table}}
 \newpage
