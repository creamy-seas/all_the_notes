\section{Flux qubit 4 JJ \cite{mooij1999}}
  
 \ipic{7cm}{flux_4jj_1}
  
 \begin{enumerate}
 	\item The two JJ's of SQUID come together like in Chapter~\ref{subsec:cpb_2} to create a single Josephson junction with a tuneable energy:
 	
 	\begin{equation}\label{key}
 		\begin{aligned}
 			E_{SQUID} & = \frac{\Phi_0I_c}{2\pi}\times 2|\cos(\pi\Phi_\text{2}/\Phi_0)|\\
 			& \equiv E_J \times \green{2|\cos(\pi\Phi_\text{2}/\Phi_0)|}.
 		\end{aligned}
 	\end{equation}
 	
 	\item The effective flux penetrating the full loop is 
 	
 	\begin{equation}\label{key}
 		\Phi_\text{total} = \Phi_1 + \frac{1}{2}\Phi_2.
 	\end{equation}
 	
 	\noindent One can think of the flux in the SQUID loop, $ \Phi_2 $, affecting the lower one only through one of the JJ.
 	
 	\item \red{\textbf{The potential energy}} of the terms $ 1 - \cos(\phi_{ij}) $ of the JJ's around the circuit (using the quantisation condition \purple{$ \phi_3 - \phi_1 + \phi_2 - 2\pi f_\text{total} = 2\pi n $}):
 	
 	\begin{equation}\label{key}
 		U_\text{potential} = E_J \bigg[2 + 2\alpha - \cos(\phi_1) - \cos(\phi_2) - \green{2\cos(\pi f_2)}\cos\big(\purple{\phi_1 - \phi_2 + 2\pi(f_1+\frac{1}{2}f_2)}\big) \bigg].
 	\end{equation}
 	
 	\noindent And this is what it looks like:
 	
 	\ipicCaption{8cm}{flux_4jj_2}{Reeeee, ovserve the periodicity. States $ L $ circulate in the same direction, state $ R $ in the opposite. The link between the states depends on parametesr $ \alpha $ and $ f_2 $.}
 	
 	\item \
 	\iframe{\red{If $ f_1+\frac{1}{2}f_2 = 1 $, we recover the 3JJ potential from Chapter~\ref{subsec:l33jj}.}}
 	
 	\item \textbf{\red{The kinetic energy}} is harder to evaluate.
 	
 	\begin{equation}
 	\left\lbrace \begin{aligned}
 	\vec{V} & = 2eC^{-1}\vec{n}\\
 	\vec{Q} & = 2e\vec{n}\\
 	\red{U_{\text{kinetic}}} & \red{= \frac{1}{2}Q.V}\\
 	\end{aligned}\right. \Rightarrow \red{U_{\text{kinetic}} = \frac{1}{2}\left(2e\vec{n}^{\text{T}}\right)\left(2eC^{-1}\vec{n}\right) = \frac{(2e)^2}{2}\vec{n}^{\text{T}}C^{-1}\vec{n}},
 	\end{equation}
 	
 	\noindent where $ \vec{n} $ is the charge state on the 3 islands in the system and $ C $ is the capacitance matrix for the system (confusingly, the capacitance for JJ's 1 and 2 is also given the symbol $ C $).
 	
 	\iframe{\noindent \textbf{It is common to work classical analogues}, where we would write:
 		\begin{equation}\label{key}
 			\frac{(2e)^2}{2}\vec{n}^{\text{T}}C^{-1}\vec{n} = \frac{1}{2}\vec{p}^{T}M^{-1}\vec{p}.
 		\end{equation}
 		
 		\noindent $ M $ is the mass tensor, with nornalisation value $ \hbar^2/\frac{(2e)^2}{2C} $, where we have applied $ \icommutation{x}{p}=i\hbar $ and $ \icommutation{\phi}{n} = i $ to generate the $ \hbar $.
 	}
 
  	\iframe{Depending on the direction that we are looking at tunneling across, we get two tensors:
  		\begin{itemize}
  			\item $ M_a = \frac{1}{4}M$ for the $ \phi_1-\phi_2 = 0 $ direction \hfill (across the tall barrier); 
  			\item $ M_b = M$ for the $ \phi_1+\phi_2 = 0 $ direction \hfill (across the snall barrier); 
  		\end{itemize}
  	}
  
  \item The result it two different transistions, with two different oscillation frequencies.
  
  \item \red{\textbf{Make sure barrier is high enough, to localise state, and small enough to allow sufficient tunneling to occur.}}
 \end{enumerate}
 
 