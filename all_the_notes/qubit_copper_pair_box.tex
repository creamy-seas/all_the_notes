\section{Single cooper pair box system\label{sec:cooper_pair_box}}
  The Cooper pairs are trapped on an island between a gated capacitor and a Josephson junction.
  \ipic{4cm}{cooper_pair_box}
  

 \iframe{We remember that the voltage induction law reads
 	\begin{equation}\label{key}
 		V = -\dot{\Phi}.
 	\end{equation}
	}

 
 \begin{enumerate}
 	\item Forgetting about the voltage source, we consider the following diagram, which will have a \red{\textbf{charging kinetic part}}
 	
 	\begin{equation}\label{key}
 		T = \frac{C_g}{2}\dot{\Phi_J}^2 + \frac{C_J}{2}\dot{\Phi_J}^2 = \frac{C_\Sigma}{2}\dot{\Phi_J}^2.
 	\end{equation}
 	
 	\ipic{3cm}{cooper_pair_box_1}
 	
 	\item \textbf{\red{The potential energy part}} comes from the JJs and the energy supplied by the source acting on the gate side of the gate capacitor 	
 	\begin{equation}\label{key}
 		\begin{aligned}
	 		& E_J = 1 - E_J\cos(\frac{2\pi}{\Phi_0}\Phi_J)\\
		 	& \ialigned{
		 		E_\text{gate} & = V_g \times Q_c\\
	 			Q_c & = C_g\times -\dot{\Phi_J}}
 		\end{aligned}\ira U = -E_J\cos(\frac{2\pi}{\Phi_0}\Phi_J) - V_gC_g\dot{\Phi_J}.
 	\end{equation}
 	
 	\ipic{3cm}{cooper_pair_box_2}
 	
 	\item The full \red{\textbf{Lagrangian}} now reads
 	\begin{equation}\label{key}
 		\mathcal{L} = T - U = \frac{C_\Sigma}{2}\dot{\Phi}_J^2 + E_J\cos(\frac{2\pi}{\Phi_0}\Phi_J) + V_gC_g\dot{\Phi_J}.
 	\end{equation}
 	
 	\item The \textbf{\red{conjugate momentum}}
 	\begin{equation}\label{key}
 		Q_J = \difffrac{\mathcal{L}}{\dot{\Phi}_J} = C_\Sigma\dot{\Phi}_J + V_gC_g,
 	\end{equation}
 	\noindent is just an offset of the induced charge on the capacitor due to the JJ voltage with the gate induced charge.
 	
 	\item So we arrive at the following set of variables
 	
 	\begin{equation}
 	{ \textcolor{blue}{\mathbf{x\leftrightarrow \Phi \leftrightarrow \phi} \text{ (position/flux) }}\qquad \textcolor{red}{\mathbf{p\leftrightarrow Q \leftrightarrow N \text{ (momentum/electrons) }}}}
 	\end{equation}
 	
 	\noindent with the commutation relations:
 	
 	\begin{align}
 	\left[\blue{x},\red{p}\right] & =i\hbar & \left[\blue{\Phi},\red{Q}\right] & = i\hbar & \left[\phi,N\right] & = \frac{2\pi}{\frac{h}{2e}}\left[\Phi,Q\right]\frac{1}{2e} = i\\
 	\red{\hat{p}} & = -i\hbar\ipartial{}{\blue{x}} & \hat{\red{Q}} & =-i\hbar\ipartial{}{\blue{\Phi}} & \hat{\red{N}} & =-i\ipartial{}{\blue{\phi}}
 	\end{align}
 	
 	\item\ 
 	
 	\iframe{Expressing the \red{\textbf{Hamiltonian}} in the standard fashion
 	
 	\begin{equation}\label{eqn:cpbox_final}
 		\begin{aligned}
	 		\mathcal{H} & = Q_J\dot{\Phi}_J - \mathcal{L}\\
	 		& = \frac{(Q_J - C_gV_g)^2}{2C_\Sigma} - E_J\cos(\frac{2\pi}{\Phi_0}\Phi_J)\\
	 		& = \mathbf{\red{E_C{\left(\hat{N}-N_\text{ext}\right)^2}- E_J\cos\left(\phi\right)}}\qquad E_c = \frac{(2e)^2}{2C_\Sigma}
 		\end{aligned}
 	\end{equation}

	\noindent Wel call $ N_\text{ext} = \frac{C_\Sigma V_g}{2e} $ the effective offset charge. 
}
 \end{enumerate}

% \subsection{Getting more real}
%  In reality, we will be dealing with a split Cooper Pair box
%  
%  \ipic{4cm}{cooper_pair_box_1}
%  Which will affect the energy term in the above Hamiltonian, as $ E_J $ becomes external flux dependent.
\newpage \subsection{Adding a parallel JJ\label{subsec:cpb_2}}
 \ipic{5cm}{cooper_pair_box_4}
 Adding a second JJ in parallel, will allow us to control the Josephson energy $ E_J $.
 \begin{itemize}
 	\item The two currents throught the JJ wil give a total current	\begin{equation}
 	I = I_{c1}\sin(\phi_1)+I_{c2}\sin(\phi_2),
 	\end{equation}
 	
 	\item Along with the phase quantisation $ \phi_1+\phi_2+\phi_\text{ext} = 2\pi m $ and symmetric JJs $I_{c1}=I_{c2}=I_c$ results in
 	
 	\begin{equation}
 	\begin{aligned}
 	I & = 2I_c\sin(\frac{\phi_1+\phi_2}{2})\cos(\frac{\phi_1-\phi_2}{2})) \\
 	\end{aligned}
 	\end{equation}
 	
 	\iframe{Or in more general form
 	\begin{equation}\label{key}
 		I = I_{cs}\sin(\phi)
 	\end{equation}
 	\noindent where
 	\begin{equation}\label{key}
		I_{cs} = 2I_c|\cos(\pi\Phi_\text{ext}/\Phi_0)|\qquad \phi = (\phi_1+\phi_2)/2
 	\end{equation}
	}

	\item The Josephson energy
	
	\begin{equation}
	\left\lbrace\begin{aligned}
	I &= I_{cs}\sin(\phi)\\
	V&=\dot{\Phi} \equiv \frac{\dot{\phi}}{2\pi}\Phi_0
	\end{aligned}\right. \Rightarrow U = \left\lbrace\begin{aligned}
	\int_{0}^{t}IV dt & = \int_{0}^{t}I_{cs}\sin(\phi)\frac{\Phi_0}{2\pi}\frac{d\phi}{dt}dt\\
	& = \int_{0}^{\phi} E_{Js}\sin(\phi)d\phi\\
	& \red{= E_{Js}(1-\cos(\phi)),\qquad E_{Js} = \frac{\Phi_0I_c}{2\pi}\times 2|\cos(\pi\Phi_\text{ext}/\Phi_0)|.}
	\end{aligned}\right.
	\end{equation} 
	\iframe{\textbf{Acquires an additional factor which can now be controlled
	\begin{equation}\label{key}
		2|\cos(\pi\Phi_\text{ext}/\Phi_0)|
	\end{equation}}}
 \end{itemize}

\newpage \subsection{Quantising the Hamiltonian}   
 \subsubsection{\red{Charge energy dominates}, $ E_C/E_J >> 1$}
  If charge is the important variable, then we chall work with the charge basis $ \lbrace N \rbrace $. We use the number of phase operators derived in App.~\ref{sec:charge_basis}
  
  \iframe{
  	\begin{equation}\label{key}
  	\begin{aligned}
  	e^{\pm i\hat{\blue{\phi}}} & = \sum_{n}\ketbra{n\pm 1}{n}\\
  	\hat{\red{N}} & = \sum_{n}n\ketbra{n}{n}.
  	\end{aligned}  
  	\end{equation}
  }
 
   \noindent and using the exponential form of cos, we rewrite Eq.\eqref{eqn:cpbox_final}

   \begin{equation}
	\begin{aligned}
	\mathcal{H} & = E_C{\left(\hat{\red{N}}-N_\text{ext}\right)^2}- E_J\cos\left(\hat{\blue{\phi}}\right)\\
	& = \sum_n\bigg[E_C{\left(n-N_\text{ext}\right)^2}\ketbra{n}{n}- \frac{E_J}{2}\bigg(\ketbra{n+1}{n}+\ketbra{n-1}{n}\bigg)\bigg]\\
	& = \begin{pmatrix}
		E_C(-2-N_\text{ext})^2 & -E_J/2 & 0 & 0 & 0\\
		-E_J/2 & \red{E_C(-1-N_\text{ext})^2} & \red{-E_J/2} & 0 & 0\\
		0 & \red{-E_J/2} & \red{E_C(N_\text{ext})^2} & -E_J/2 & 0\\
		0 & 0 & -E_J/2 & E_C(1-N_\text{ext})^2 & -E_J/2\\
		0 & 0& 0 & -E_J/2 & E_C(2-N_\text{ext})^2\\
	\end{pmatrix}
	\end{aligned}.
	\label{l2-subbed2}
   \end{equation}
   
    \ipicCaption{3cm}{island}{System will be in the lowest energy state, i.e. have that many electrons that minimize its energy\label{fig:cp_box_energy_charge}}  
    
   \noindent Normally $ N_\text{ext} $ is biased at a sweet spot \red{close to $ - 1/2 $}. Then the only number of electrons $n$ on out island that will give a low energy (due to the energy dispersion in Fig.~\ref{fig:cp_box_energy_charge}) is either 0 or -1. The other level will be far separated. Then we take out the Hamiltonian
   
   \red{\begin{equation}
	   \begin{aligned}
	    \mathcal{H}_\text{0 or -1} & = \begin{pmatrix}
		   E_C(-1-N_\text{ext})^2 & -E_J/2\\
		   -E_J/2 & E_C(N_\text{ext})^2\\
	   \end{pmatrix}\\
	   \end{aligned}
   \end{equation}}

   \noindent And we have a two level system just as in the previous lecture Eq.\eqref{l1-finalEVal} shown in Fig.\ref{l2-closeup}. Redefining the zero point energy to be in the middle of the diagonal terms
   
   \begin{equation}
	   \begin{aligned}
		   \mathcal{H} & = \begin{pmatrix}
		   -\epsilon/2 & -E_J/2\\
		   -E_J/2 & \epsilon/2\\
		   \end{pmatrix}\\
		   \epsilon/2 = \frac{\text{energy diff}}{2} & = \frac{E_CN_\text{ext}^2-E_C(1+N_\text{ext})^2}{2} = -\frac{E_C}{2}\big(1+2N_\text{ext}\big)
	   \end{aligned}
   \end{equation}

   \noindent \iframe{This will have solutions
   
   \begin{equation}
   E = \pm \frac{\Delta E}{2}, \qquad \ket{\psi}_0 = \begin{pmatrix}
   \cos(\theta/2) \\ \sin(\theta/2)
   \end{pmatrix}, \qquad \ket{\psi}_1 = \begin{pmatrix}
   \sin(\theta/2) \\ \cos(\theta/2)
   \end{pmatrix}, \Delta E = \sqrt{\epsilon^2+\Delta^2}
   \end{equation}}
   
  \begin{itemize}
  	\item In Fig.\ref{l3-energyspec} depicted are the energy levels for different values of the charging and coupling energies. The stronger the coupling, $ E_J $, the bigger the splitting at the degeneracy points.
  	
  	\red{There is less charge noise in the sweet spots of $ n_g = \mathbb{Z}\frac{1}{2} $, but it will still be a major source of decoherence. \textbf{There is less dependence on charge fluctuations as $ E_C/E_J $ gets smaller and the band become flatter}.}
  	
  		\ipicCaption{6cm}{cpb2}{The energies of the eigenstates for different gated charge $ N_{\text{ext}} $. The energy different at the degeneracy point grows as coupling becomes stronger.\label{l3-energyspec}}
   
  \item One can also looks the ground energy state (the one the system will most probably occupy) and find the components that make it up
  
  \[
  	\Ket{\Psi}_{\text{ground}} = \sum_N\alpha_N\ket{N},
  \]
  
  \noindent and one sees in the figures when there is no applied gate voltage, the island will have $ N=0 $ charges on it predominantly. As one increases the gate charge, the presence of $ N=1 $ increases, until it finally dominates. The pattern repeats as more and more cooper pairs tunnel, to give the lowest energy configuration.
  
  	\ipic{5cm}{cpb3}
  	\ipic{5cm}{cpb4}
  
\end{itemize}

   \iframe{\textbf{High $ E_C/E_J $:} 
	\begin{itemize}
		\item High charge noise - gate voltage changes, affects the energy levels severely
		\item High anharmonicity - quadratic $ n $ dependance dominates, allowing individual addressing of the levels;
	\end{itemize}
}

\newpage\subsubsection{\red{Flux energy dominates, $ E_C/E_J << 1 $}}
  If flux is the important variable, then we chall work with the flux basis using the wavefunctions $ \psi(\phi) $
 
 \iframe{
 	\begin{equation}\label{key}
 	\begin{aligned}
 	 		&\blue{\hat{\phi}} \qquad\\
 	 		&\red{\hat{N}} = -i\difffrac{}{\blue{\phi}}
 	\end{aligned}  
 	\end{equation}
 }
 \noindent and we rewrite
 
 \begin{equation}\label{key}
 	\begin{aligned}
 	\mathcal{H} & = E_C{\left(\hat{\red{N}}-N_\text{ext}\right)^2}- E_J\cos\left(\hat{\blue{\phi}}\right)\\
 	& = E_C\left(-i\difffrac{}{\phi} - n_g\right)^2 - E_J\cos(\phi).
 	\end{aligned}
 \end{equation}
 
 \begin{itemize}
 	\item Let us compare this to Hamiltonian in a periodic potential
 	\begin{equation}\label{key}
 		\mathcal{H}_\text{crystal} = \frac{-\hbar^2}{2m}\difffrac{^2}{x^2}+V(x)\qquad V(x+a) = V(x),
 	\end{equation}
 	
 	\noindent which, according to Bloch's theorem, states that the eigenstates will be of the form
 	\begin{equation}\label{key}
 		\psi_{kn}(x) = e^{ikx}u_{kn}(x),
 	\end{equation}
 	\noindent which, when plugged in will result in an effective Hamiltonian
 	
 	\begin{equation}\label{key}
 		\mathcal{H}_{\text{eff},k} = \frac{\hbar^2}{2m}\left(-i\difffrac{}{x}+k\right)^2 + V(x)
 	\end{equation}
 	\item We can see this mapping between
 	\begin{equation}\label{key}
 		\begin{aligned}
	 		\mathcal{H}_{\text{eff},k} & = \frac{\hbar^2}{2m}\left(-i\difffrac{}{x}+k\right)^2 + V(x)\\
	 		\mathcal{H} & = E_C\left(-i\difffrac{}{\phi} - n_g\right)^2 - E_J\cos(\phi),
 		\end{aligned}
 	\end{equation}
 	
 	\noindent so, will look for solutions of a similar form.
 	\item As $ E_C/E_J $ gets smaller, the potential well from the $ E_J\cos(\phi) $ gets deeper \red{\textbf{and the states within each well localise, and stop interacting with one another.}} Solving, as in the Transmon Paper, will lead to anharmonicity relation
 	
 	\iframe{\begin{equation}\label{key}
 		\frac{E_{12} - E_{01}}{E_{01}}\approx -(8E_J/E_C)^{-1/2},
 	\end{equation}
 	\noindent which decreases as we continue to increase $ E_J $.}
 \end{itemize}
 \newpage
 
 

