\section{Quantum Electrodynamics Formalism\label{subsec:l2-CPB}}
  
   \subsection{Superconductors}
   In superconductors CP carry charge. What happens is that the Fermi level splits into 2 bands that are $\pm\Delta$ above and below $E_F$, and each electron in the CP belongs to one of these bands. 
   
   
   {{ Phase is quantised in a sc loop
	\begin{equation}
		\begin{aligned}
			\phi = \phi_\text{ext} + 2\pi N \Leftrightarrow \Phi = \Phi_\text{ext} + \Phi_0 \\(\Phi_0 = \frac{h}{2e}, \Phi/\Phi_0=\phi/2\pi)
		\end{aligned}
	\end{equation}
   }}

   \subsection{Josephson junction}
   Now work with JJ. The states on the two sides of the JJ are $\left|\psi_0\right|e^{i\phi_1}$ and $\left|\psi_0\right|e^{i\phi_2}$. Solving the Schrodinger equation for the condensate state, when E=0
   
   \begin{equation}
   	-\frac{\hbar^2}{2m}\ipartial{^2}{x^2}\psi+U\psi = 0 \Rightarrow
   	\left\lbrace\begin{aligned}
	   	\psi & = A_0e^{-kx}+B_0e^{+kx}\\
	   	k & = \frac{\sqrt{2mU}}{\hbar}
   	\end{aligned} \right. \Rightarrow
   	\left\lbrace\begin{aligned}
   	\psi & = A\cosh(kx)+B\sinh(kx)\\
   	k & = \frac{\sqrt{2mU}}{\hbar}
   	\end{aligned} \right.
   \end{equation}
   
   \noindent Apply BC for JJ $\psi(a/2) = \left|\psi_0\right|e^{i\phi_2}$ and $\psi(-a/2) = \left|\psi_0\right|e^{i\phi_1}$ to find that
   
   \begin{align}
	   A &= \frac{\left|\psi_0\right|}{\cosh(ka/2)}\\
	   B &= \frac{\left|\psi_0\right|}{\sinh(ka/2)}.
   \end{align}
   
   \noindent {The super current is then}
   
   \begin{equation}
	   I = -\frac{i\hbar}{2m}(2e)\left[\psi^*\ipartial{\psi}{t}-\psi\ipartial{\psi^*}{t}\right] \equiv -\frac{2e\hbar}{m}\text{Im}\left[\psi^*\ipartial{\psi}{t}\right],
   \end{equation}
   
   \noindent at $x=0$ can be evaluated, as can the voltage and energy, which results in a phase dependence
   
   \begin{equation}
   \label{l2-dcac}
   I = I_c\sin(\phi_2-\phi_1); \qquad \frac{d\phi}{dt} = \frac{2e}{\hbar}V
   \end{equation}
   
   The phase across the JJ in a circuit is the phase induced by the external flux i.e.
   
   \begin{equation}
   \label{eqn:l2-phasesum}
   \phi_1+\phi_2+\ldots = \phi_\text{ext} = \frac{\Phi_\text{ext}}{\Phi_0}2\pi,
   \end{equation}
   
   \begin{equation}
	   \left\lbrace\begin{aligned}
		   I &= I_c\sin(\phi_2-\phi_1)\equiv I_c\sin(\phi)\\
		   V&=\dot{\Phi} \equiv \frac{\dot{\phi}}{2\pi}\Phi_0
	   \end{aligned}\right. \Rightarrow U = \left\lbrace\begin{aligned}
		   \int_{0}^{t}IV dt & = \int_{0}^{t}I_c\sin(\phi)\frac{\Phi_0}{2\pi}\frac{d\phi}{dt}dt\\
		   & = \int_{0}^{\phi} E_J\sin(\phi)d\phi\\
		   & \red{= E_J(1-\cos(\phi)),\qquad E_J = \frac{\Phi_0I_c}{2\pi}.}
	   \end{aligned}\right.
	   \label{l2-JJEnergy}
   \end{equation} 
   
  \begin{center}
  	Increase $ I $ \ra Increase $ \phi = \phi_2-\phi_1 $ to $ \pi/2 $ \ra Increse energy $ U $ \ra \textbf{at some point, we reach critical currrent and highest energy - beyond this limit, we will generate voltage}
  	\ipic{3cm}{josephson_effecti}
  \end{center}
   
  \paragraph{Inductance} is found by performing:
  
  \begin{itemize}
  	\item Suppose a change in current, $ \delta_I $ causes a change in phase $ \delta_1\phi $:
  	
  	\begin{equation}\label{key}
  		I_0 + \delta_I = I_c\sin(\phi_0+\delta_\phi) \iRa \delta_I = \purple{I_c\cos(\phi_0)\delta_\phi.}
  	\end{equation}
  	\item The voltage across the junction:
  	\begin{equation}\label{key}
  		V = \frac{\dot{\phi}}{2\pi}\Phi_0 = \frac{\Phi_0}{2\pi}\left(\red{\dot{\phi}_0} + \dot{\delta}_\phi\right) =  \frac{\Phi_0}{2\pi}\purple{\frac{\dot{\delta}_I}{I_c\cos(\phi_0)}}.
  	\end{equation}
  	
  	\item FInally expressing the inductance
  	
  	\begin{equation}\label{key}
  		L = V/\difffrac{I}{t} = V/\dot{\delta}_I = \frac{\Phi_0}{2\pi}\frac{1}{I_c\cos(\phi_0)}.
  	\end{equation}
  \end{itemize}
   
   \iframe{\begin{equation}\label{key}
   	L_J = \Phi/I = \frac{\Phi_0}{2\pi}\frac{1}{I_c}\frac{1}{\cos(\phi_0)}
   \end{equation}}
   
   \noindent The current is given by:
   
   \begin{equation}\label{key}
   I_cR_n = \frac{\pi\Delta(T)}{2e}\tanh\big(\frac{\Delta(T)}{2k_bT}\big),
   \end{equation}
   
   \noindent derived from BCS theory for a superconducting energy gap of $ \Delta(T) $ and normal resistance $ R_n $ of the JJ. 
   
   {
   	\paragraph{Critical current $ I_c $}
   	Taking the limit of $ T\ra0 $ we get  
   	
   	\begin{equation}\label{app:criticalCurrent}
   	I_cR_n = \frac{\pi\Delta(0)}{2e}
   	\end{equation}
   }
   
   
   \iframe{\paragraph{Josephson Energy from JJ parameters} Subbing in Eq.~\eqref{app:criticalCurrent} into Eq.~\eqref{l2-JJEnergy}:
   	
   	\begin{equation}\label{key}
   	\begin{aligned}
   	E_J & = \frac{R_q}{R_n/N_{sq}}\frac{\Delta(0)}{2}\\
   	R_n &= 18.4\,\text{k}\Omega \text{ for 100 } \times 100\,\text{nm}\ipow{2}
   	\end{aligned}
   	\end{equation}
   	
   	\noindent with $ R_q = \frac{h}{(2e)^2} $. The wider the JJ is in squares, $ N_{sq} $, the lower the resistance of the junction.
   	
   	\red{JJ resistance increases by $ \sim 10\% $ as one goes from room to cryogenic temperatures.}
   }
   
   \subsubsection{Inductance energy}
   Inductance energy derives from $ \frac{\Phi_L^2}{2L} = \frac{\Phi_0^2}{(2\pi)^22L}(\phi_\text{ext}-\phi_J)^2 = E_L(\phi_\text{ext}-\phi_J)^2 $
   
   \begin{equation}
   \begin{aligned}
   E_L & = \frac{\Phi_0^2}{(2\pi)^22L}\\
   L & \propto R_n = \iunit{1.5}{nH per 100}\times\iunit{100}{nm}^2
   \end{aligned}
   \end{equation}
   
   \subsubsection{Summary of energies}
   \begin{table}[h]
   	\centering
   	{\footnotesize 	\begin{tabular}{|c|c|p{6cm}|c|}
   			\hline\textbf{Energy} & & \textbf{Variable parameter} & \textbf{Energy ($ N_{sq}=10, N_{NbN} = 5$)}\\\hline 
   			$ E_J $ & $ \frac{R_q}{R_{\square}/N_{sq}}\frac{\Delta(0)}{2} $ & $ R_q = \frac{h}{(2e)^2} = 6.484\,\text{k}\Omega,\newline \Delta = 1.73*(k_b\times 1.3\,\text{K}) = 3.1\times10^{-23}, \newline R_\square = \iunit{18.4}{k}\Omega $ & \iunit{77.5}{GHz} \\
   			$ E_C $ & $ \frac{(2e)^2}{2CN_{sq}} $ & $ \varepsilon = 10, d = \iunit{2}{nm}, \newline A = 100\times\iunit{100}{nm}^2, \newline C = \frac{\varepsilon\varepsilon_0A}{d} = \iunit{0.5}{fF} $ & \iunit{17.4}{GHz} \\
   			$ E_L $ & $ \frac{\Phi_0^2}{(2\pi)^22LN_{NbN}} $ & $ \Phi_0 = 2\itimes{-15}\,\text{Wb}, \newline L = \iunit{1.5}{nH} $ per NbN square & \iunit{16.2}{GHz}\\\hline
   	\end{tabular}}
   \end{table}


   
   \subsection{SQUID to control $ E_J $\cite{zhu2010}}
    Two JJ in a loop give a total current
    
    \begin{equation}
	    I = I_{c1}\sin(\phi_1)+I_{c2}\sin(\phi_2),
    \end{equation}
   
    \noindent which along with the phase quantisation condition Eq.\eqref{eqn:l2-phasesum} and symmetric JJs $I_{c1}=I_{c2}=I_c$ results in
    
    \begin{equation}
	    \begin{aligned}
	    I = & I_{cs}(\Phi_\text{ext})\sin(\phi)\\
	    & I_{cs} = 2I_c|\cos(\pi\Phi_\text{ext}/\Phi_0)|\\
	    & \phi = (\phi_1+\phi_2)/2
	    \end{aligned},
    \end{equation}
    
    \noindent which has a similar form to Eq.\eqref{l2-dcac}, but now the critical current $I_{cs}$ is controlled by an external field. Taking the derivative
     
    \begin{equation}
    \begin{aligned}
    	\frac{dI}{dt}=2I_{cs}(\Phi_\text{ext})\cos(\phi)\frac{d\phi}{dt},
    \end{aligned}
    \end{equation}
    
    \noindent and expressing the voltage using Eq.\eqref{l2-dcac} and assuming $\cos(\phi)\approx1$ for small excitations

    \begin{equation}
	    V = \frac{\hbar}{2e}/\frac{d\phi}{dt} = \frac{\hbar}{4eI_{cs}}\frac{dI}{dt},
    \end{equation}
    
    \noindent from which one finds the inductance ($\Phi = LI \rightarrow L = \frac{d\Phi}{dI} = \frac{d\Phi}{dt}/\frac{dI}{dt} = V/\frac{dI}{dt}$for small excitations
    
    \begin{equation}
	    L_J(\Phi_\text{ext}) = V/\frac{dI}{dt} = \frac{\hbar}{4eI_c|\cos(\pi\Phi_\text{ext}/\Phi_0)}.
	    \label{l2:squid:inductance}
	\end{equation}
   \vspace{6ex}
   
   \subsection{Building blocks for quantum circuits.}
    Quantum circuits are build from 
    \begin{itemize}
    	\item JJs, who posses an inductance $L_J$ defined Eq.\eqref{l2:squid:inductance}. The energy comes from the flux stored by the inductor
    	\begin{equation}
    		E_J(\Phi_\text{ext}) = \frac{\Phi_0}{2L_J(\Phi_\text{ext})}
    	\end{equation}
    	\item An inductor $L$ with energy from a the flux inside the inductor (coil)
    	\begin{equation}
    		E_L = \frac{\Phi^2}{2L} 
    	\end{equation}
    	\item A capacitor with energy
    	\begin{equation}
    		E_c = \frac{Q^2}{2C}
    	\end{equation}
    \end{itemize}
   
   For the most general discussion, we compare 
  
   \setlength{\extrarowheight}{4mm}
   \begin{table}[h]
   	\caption{}
   	   	\label{tab:conversion1}
 	  \begin{center}
 	  	\begin{tabular}{|c|c|c|c|}
			  \hline
			  &\textbf{$ Q $ and $ \Phi $} & \textbf{Mechanical} & \textbf{$ V $ and $ I $} \\\hline
			  \multicolumn{4}{|c|}{$ I =\dot{Q }\qquad V=\dot{\Phi} $}\\
			  \multicolumn{4}{|c|}{$ Q=CV\qquad \Phi = LI $}\\\hline
			  \multirow{2}{*}{Kinetic} & $ \frac{Q^2}{2C} $  & $ \frac{p^2}{2m} $ & \textit{\scriptsize Inductor}\\
			   & \textit{\scriptsize Capacitor} & $ \frac{m\dot{x}^2}{2} $& $ \frac{LI^2}{2} = \frac{LC^2\dot{V}^2}{2}$ \\\hline
			  Potential & $ \frac{\Phi^2}{2L} $ & $ \frac{kx^2}{2} $ & $\frac{CV^2}{2}$ \\
			  & \textit{\scriptsize Inductor} & & \textit{\scriptsize Capacitor} \\\hline
			  $ \omega_0 $ & $  \frac{1}{\sqrt{LC}} $ &$  \sqrt{\frac{k}{m}} $ &$  \frac{1}{\sqrt{LC}} $\\\hline
			  \multirow{2}{*}{Energy} & $\frac{Q^2}{2C}+\frac{1}{2}C\omega_0^2\Phi^2$ & $\frac{p_m^2}{2m}+\frac{1}{2}m\omega_m^2x^2$ & \\
			  & & $\frac{m}{2}\left(\dot{x}^2+\omega_0^2x^2\right)$  & $ \frac{LC^2}{2}\left(\dot{V}^2+\omega_0V^2 \right)$\\\hline
			  Momentum & $ Q $ & $ p $ & $I =  C\dot{V} $\\\hline
			  Coordinate & $ \Phi $ & x & V\\\hline
			  Mass & $ C $ & $ m $ & $ LC^2 $\\\hline
			  Stiffness & $ \frac{1}{L} $ & $ k $ & C\\\hline
			  Solution &$ \Phi=Ae^{i\omega_0t}+Be^{-i\omega_0t} $ & $ x=Ae^{i\omega_0t}+Be^{-i\omega_0t} $ & $ V=Ae^{i\omega_0t}+Be^{-i\omega_0t} $\\\hline
  	  \end{tabular}
 	  \end{center}
   \end{table}
   
   \setlength{\extrarowheight}{4mm}
   \begin{table}[h]
   	\caption{}
   	\label{tab:conversion2}
   	\begin{center}
   		\begin{tabular}{|c|c|c|c|}
   			\hline
   			&\textbf{$ Q $ and $ \Phi $} & \textbf{Mechanical} & \textbf{$ V $ and $ I $} \\\hline
   			Momentum & $ Q $ & $ p $ & $I =  C\dot{V} $\\\hline
   			Coordinate & $ \Phi $ & x & V\\\hline
   			Mass & $ C $ & $ m $ & $ LC^2 $\\\hline
   			Stiffness & $ \frac{1}{L} $ & $ k $ & C\\\hline
   			\multirow{2}{*}{Energy} & $\frac{Q^2}{2C}+\frac{1}{2}C\omega_0^2\Phi^2$ & $\frac{p_m^2}{2m}+\frac{1}{2}m\omega_m^2x^2$ & \\
   			& & $\frac{m}{2}\left(\dot{x}^2+\omega_0^2x^2\right)$  & $ \frac{LC^2}{2}\left(\dot{V}^2+\omega_0V^2 \right)$\\\hline
   			\multirow{3}{*}{$ \mathcal{H} $} & \multirow{2}{*}{$ \frac{\hbar\omega_0}{2}\left(\frac{Q^2}{Q_0^2}+\frac{\Phi^2}{\Phi_0^2}\right) $} &  \multirow{2}{*}{$ \frac{\hbar\omega_0}{2}\left(\frac{\hat{p}^2}{p_0^2}+\frac{\hat{x}^2}{x_0^2}\right) $} & \multirow{2}{*}{$ \frac{\hbar\omega_0}{2}\left(\frac{\hat{I}^2}{I_0^2}+\frac{\hat{V}^2}{V_0^2}\right) $}\\
   			& & & \\
   			& $ \Phi_0=\sqrt{\frac{\hbar\omega_0}{C}},\  Q_0 = \sqrt{\frac{\hbar}{C\omega_0}} $& $ x_0 = \sqrt{\frac{\hbar}{m\omega_0}},\ p_0 = \frac{\hbar}{x_0} $ & $ V_0 = \sqrt{\frac{\hbar\omega_0}{C}},\ I_0 = \sqrt{\frac{\hbar\omega_0}{L}} $ \\\hline
   			& \parbox[c]{5cm}{$ \left[\hat{\Phi},\hat{Q}\right] = i\hbar $ since $ \Phi $ and $ Q $ are exactly $ x $ and $ p $}& $ \left[\hat{x},\hat{p}\right] = i\hbar $ & \parbox{5cm}{$ \left[\hat{\Phi},\hat{Q}\right] = \left[CV,LI\right]=i\hbar $ so $ \left[\hat{V},\hat{I}\right] = \frac{i\hbar}{LC}$}\\\hline
   			& $ \hat{\Phi} $& $ \hat{x} $ & $ \hat{V} $\\\hline
   			& $ \hat{Q}=-i\hbar\difffrac{}{\Phi} $& $ \hat{p} = -i\hbar\difffrac{}{x} $ & $ \hat{I} = -i\frac{\hbar}{LC}\difffrac{}{V} $\\\hline
   			& \multirow{2}{*}{$ a^{\dagger} = \sqrt{\frac{C\omega_0}{2\hbar}}\left(\hat{\Phi}-i\frac{\hat{Q}}{C\omega_0}\right) $} & \multirow{2}{*}{$ a^{\dagger} = {\frac{1}{\sqrt{2}}}\left(\frac{\hat{x}}{x_0}-i\frac{\hat{p}}{p_0}\right) $} & \multirow{2}{*}{$ a^{\dagger} = {\frac{1}{\sqrt{2}}}\left(\frac{\hat{V}}{V_0}-i\frac{\hat{I}}{I_0}\right) $}\\
   			& & & \\\hline
   		\end{tabular}
   	\end{center}
   \end{table}

   We see that in either case, we are working with a harmonic oscillator system, with corresponding raising and lowering operators. 
   
   {\scriptsize As a reminder, an equation with the form
		\[
			\mathcal{H} = \frac{\hbar\omega_0}{2}\left(y^2-\difffrac{^2}{y^2}\right)
		\]
		
		\noindent can be rewritten as
		
		\[
			\mathcal{H} = \hbar\omega_0\left(a^{\dagger}a+\frac{1}{2}\right)= \hbar\omega_0\left(\hat{N}+\frac{1}{2}\right)
		\]
		
		with eigenstates and eigenenenergies
		
		\[
			\ket{\Psi}_n=\ket{n} \qquad E_n= \hbar\omega_0\left(n+\frac{1}{2}\right).
		\]
		
		\noindent The raising and lowering operators act in the following way on the eigenstates
		
		\[
			\left\lbrace\begin{aligned}
			a\ket{n} & = \sqrt{n}\ket{n-1}\\
			a^{\dagger}\ket{n} &= \sqrt{n+1}\ket{n+1}\\ 
			\end{aligned}\right. \Rightarrow 
			\hat{N}\ket{n} = 	a^{\dagger} \sqrt{n}\ket{n-1} = \sqrt{n}^2\ket{n}=n\ket{n}	,
		\]
		\noindent and the matrix form, evaluated by finding the matrix coefficients $ c_{ij}=\bra{i}\hat{C}\ket{j} $ 
		
		\[
			\hat{a}=\begin{pmatrix}
				0 & \sqrt{1} & 0 & 0\\
				0 & 0 & \sqrt{2} & 0\\
				0 & 0 & 0 & \sqrt{3}\\
				0 & 0 & 0 & \ddots\\
			\end{pmatrix} \qquad
			a^{\dagger}=\begin{pmatrix}
			0 & 0 & 0 & 0\\
			\sqrt{0+1} & 0 & 0 & 0\\
			0 & \sqrt{1+1} & 0 & 0\\
			0 & 0 & \sqrt{2+1} & \ddots\\
			\end{pmatrix} \qquad
			\hat{N} =\begin{pmatrix}
			0 & 0 & 0 & 0\\
			0 & 1 & 0 & 0\\
			0 & 0 & 2 & 0\\
			0 & 0 & 0 & \ddots\\
			\end{pmatrix} 
		\]
   }
   
   For the Harmonic oscillator which the system replicates, the energy levels are evenly spaced, making it impossible to address individual states. But by replacing the inductor with a JJ
   
   \begin{equation}
   	E_L = \frac{\Phi^2}{2L} \longrightarrow U = -E_J\cos(\phi),
   \end{equation}
   
   \noindent changing the quadratic to a cosine potential.
  \newpage
