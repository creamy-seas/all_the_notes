\section{3 JJ qubit \label{subsec:l33jj}}
\iframe{ From \cite{gu2017}:
  \begin{itemize}
  \item To create a double potential well we need $ \alpha > 0.5 $ and to reduce flux noise we need
    $ \alpha < 0.7 $;
  \end{itemize}
}For  this  model we  put  3  JJ  in  a loop,  all  with  a  parallel  capacitor as  shown  in
Fig.\ref{fig:l33jja}.
  
  \begin{figure}[h]
    \ipic{4cm}{3jj}
    \label{fig:l33jj}
  \end{figure}
  
  \begin{enumerate}
  \item \textbf{\red{Potential energy}} comes from all three JJs, where we denote the phases
  
  \begin{equation}
    e^{i\phi_{01}} = e^{i(\phi_0-\phi_1)}
  \end{equation}
  
  \blue{\begin{equation}
      \begin{aligned}
        U_{\text{potential}} & = E_J(1-\cos(\phi_{01})) + \alpha E_J(1-\cos(\phi_{12})) + E_J(1-\cos(\phi_{20}))\\
        &  =  E_J\bigg[2+\alpha  -  \cos(\phi_{01})   -  \cos(\phi_{20})  -  \alpha\cos(\phi_\text{ext}-\phi_{01}  -
        \phi_{20})\bigg].
      \end{aligned}
    \end{equation}}
    
  \begin{center}
    \ipic{4cm}{3jjc} \ipic{5cm}{3jj1}  \ipic{5cm}{3jj2} \captionof{figure}{$\alpha$  determines the
      height of the  barrier in the middle of  the two wells. Larger alpha  raises the barrier
      up. $ \phi_\text{ext} $ determines the symmetry of the wells (as in the previous section).}
    \label{fig:l33jjc}
  \end{center}
  
\item \red{\textbf{The  kinetic energy}} comes  from the charging  energy. Its better  to work
  with the  capacitance matrix for the  system which links a  given charge vector $  \vec{n} $
  with a potential vector $ \vec{V} $, the circuit is summarised in Fig.\ref{fig:l33jja}
  
  \begin{equation}
    \left\lbrace \begin{aligned}
        \vec{n} & = \left(\begin{smallmatrix}
            n_1\\n_2
          \end{smallmatrix}\right)\\
        \vec{V} & =\left(\begin{smallmatrix} V_1\\V_2
          \end{smallmatrix}\right)\\
        C & = \left(\begin{smallmatrix} C_{01}+C_{12} & -C_{12}\\-C_{12} &C_{02}+C_{12}
          \end{smallmatrix}\right)\\
      \end{aligned}\right. \Rightarrow \vec{n} = \frac{C\vec{V}}{2e} \equiv \begin{bmatrix}
      C_{01}\left(V_1-0\right)   +  C_{12}\left(V_1-V_2\right)\\   C_{01}\left(V_2-0\right)  +
      C_{12}\left(V_2-V_1\right)
    \end{bmatrix},
  \end{equation}
  
  \noindent which is just the typical evaluating of charge on a capacitor from $ Q = CV $. For
  evaluating the kinetic energy from the charges
  
  \begin{equation}
    \left\lbrace \begin{aligned}
        \vec{V} & = 2eC^{-1}\vec{n}\\
        \vec{Q} & = 2e\vec{n}\\
        \red{U_{\text{kinetic}}} & \red{= \frac{1}{2}Q.V}\\
      \end{aligned}\right.  \Rightarrow \red{U_{\text{kinetic}} =
      \frac{1}{2}\left(2e\vec{n}^{\text{T}}\right)\left(2eC^{-1}\vec{n}\right)               =
      \frac{(2e)^2}{2}\vec{n}^{\text{T}}C^{-1}\vec{n}},
  \end{equation}
  
  
\item The total Hamiltonian for the 3JJ/3 capacitor system being
  
  \begin{equation}
    \label{eqn:l23JJ}
    \begin{aligned}
      \mathcal{H}   &  =   \red{U_{\text{kinetic}}}   +   \blue{U_{\text{potential}}}\\  &   =
      \red{\frac{(2e)^2}{2}\vec{n}^{\text{T}}C^{-1}\vec{n}}     +     \blue{E_J\bigg[2+\alpha     -
        \cos(\phi_{01}) - \cos(\phi_{20}) - \alpha\cos(\phi_\text{ext}-\phi_{01} - \phi_{20})\bigg]}.
    \end{aligned}
  \end{equation}
  
  For simple representation in  matrix form, we write it out in  the \textbf{charge basis}. In
  such a basis (recall  from Eq.\eqref{l2-phase} for the one island case)  we have the results
  summarised in Table \ref{tab:phaseChange}.
  
  {\begin{table}[h]
      \label{tab:phaseChange}
      \caption{Note how the phase operators increase  and decrease the islands associated to a
        given phase.}
      \begin{center}
        {\footnotesize       \begin{tabular}{|c|c|}      \hline       \textbf{One      island}
            (Sec.\ref{subsec:l2-CPB}) & Two  islands\\\hline $ \hat{N} =  \sum_N\ketbra{N}{N} $ &
            $ \frac{(2e)^2}{2}\begin{pmatrix} n_1 & n_2
            \end{pmatrix}C^{-1}\begin{pmatrix}n_1         \\        n_2\end{pmatrix}         =
            \displaystyle\sum_{N_1,N_2}\mathbf{U}(n_1,n_2)\ketbra{n_1,n_2}{n_1,n_2} $\\\hline
                               \multirow{3}{*}{$ e^{i\phi}  = \sum_N\ketbra{N+1}{N}$} & $ e^{i\phi_{01}} = e^{i(\phi_{0}-\phi_1)} =  \displaystyle\bigg[\sum_{n_1}\ketbra{n_1-1}{n_1}\bigg]\otimes \red{\bigg[\sum_{n_\text{ground}}\ketbra{n_\text{ground}+1}{n_\text{ground}}\bigg] - \text{ ignore}}$\\
                                      & $ e^{-i\phi_{01}} = e^{i(\phi_{1}-\phi_0)} =  \displaystyle\sum_{n_1}\ketbra{n_1+1}{n_1}$\\
                                      & $ e^{i\phi_{20}} = e^{i(\phi_{2}-\phi_0)} =  \displaystyle\sum_{n_2}\ketbra{n_2+1}{n_2}$\\
                               \multirow{3}{*}{$ e^{-i\phi}  = \sum_N\ketbra{N-1}{N}$} & $ e^{-i\phi_{20}} = e^{i(\phi_{0}-\phi_2)} =  \displaystyle\sum_{n_2}\ketbra{n_2-1}{n_2}$\\
                                      & $ e^{i\phi_{12}} (\equiv e^{\phi_\text{ext}-\phi_{01}-\phi_{20}}) = e^{i(\phi_1-\phi_2)} =  \displaystyle\bigg[\sum_{n_1}\ketbra{n_1+1}{n_1}\bigg]\otimes {\bigg[\sum_{n_2}\ketbra{n_2-1}{n_2}\bigg]} $\\
                                      &
                                        $      e^{-i\phi_{12}}       =      e^{i(\phi_2-\phi_1)}      =
                                        \displaystyle\bigg[\sum_{n_1}\ketbra{n_1-1}{n_1}\bigg]\otimes
                                        {\bigg[\sum_{n_2}\ketbra{n_2+1}{n_2}\bigg]} $\\\hline
                             \end{tabular}}
                         \end{center}
                       \end{table}}

                   \item  We draw  parallels to  rewrite our  new Hamiltonian  in matrix  form
                     (using exponential  form of trigonometric functions  and zeroing constant
                     offsets and  replacing the last  exponential with explicit  dependence on
                     the phase across the junction)
  
                     \begin{equation}
                       \label{eqn:l3ToMatrixForm}
                       \begin{aligned}
                         \mathcal{H}  = \red{\frac{(2e)^2}{2}\vec{n}^{\text{T}}C^{-1}\vec{n}} & - \blue{\frac{E_J}{2}\big(e^{i\phi_{01}}+e^{-i\phi_{01}}\big) - \frac{E_J}{2}\big(e^{i\phi_{20}}+e^{-i\phi_{20}}\big) - \frac{\alpha E_J}{2}\big(e^{i\phi_{12}}+e^{-i\phi_{12}}\big)}\\\\
                         \Rightarrow \sum_{n_1,n_2} \red{U(n_1,n_2)\ketbra{n_1,n_2}{n_1,n_2}} & \\
                         & \blue{ - \frac{E_J}{2}\bigg[\ketbra{n_1-1}{n_1}+\ketbra{n_1+1}{n_1}\bigg]\otimes \mathbb{I}_2}\\
                         & \green{- \frac{E_J}{2} \mathbb{I}_1\otimes\bigg[\ketbra{n_2+1}{n_2}+\ketbra{n_2-1}{n_2}\bigg]}\\
                         & \purple{- \frac{\alpha E_J}{2}\bigg[\ketbra{n_1+1}{n_1}\ketbra{n_2-1}{n_2}+\ketbra{n_1-1}{n_1}\ketbra{n_2+1}{n_2}\bigg]}\\
                         \Rightarrow \sum_{n_1,n_2} \red{U(n_1,n_2)\ketbra{n_1,n_2}{n_1,n_2}} & \\
                         & \blue{ - \frac{E_J}{2}\bigg[\ketbra{n_1-1,\mathbf{n_2}}{n_1,\mathbf{n_2}}+\ketbra{n_1+1,\mathbf{n_2}}{n_1,\mathbf{n_2}}\bigg]\ra \text{ $n_2$ constant}}\\
                         & \green{- \frac{E_J}{2} \bigg[\ketbra{\mathbf{n_1},n_2+1}{\mathbf{n_1},n_2}+\ketbra{\mathbf{n_1},n_2-1}{\mathbf{n_1},n_2}\bigg]\ra \text{ $n_1$ constant}}\\
                         &                          \purple{-                          \frac{\alpha
                             E_J}{2}\bigg[\ketbra{n_1+1,n_2-1}{n_1,n_2}+\ketbra{n_1-1,n_2+1}{n_1,n_2}\bigg]}
                       \end{aligned}
                     \end{equation}
 
                     \noindent and writing out in matrix form
  
  \begin{equation}
    \mathcal{H} = \kbordermatrix{
      & \ket{-1,0} & \ket{0,-1} & \ket{0,0} & \ket{0,1} & \ket{1,0}\\
      \bra{-1,0} &\red{ U(-1,0)} & \purple{-\frac{E_J}{2}} & \blue{-\frac{E_J}{2}} & 0 & 0\\
      \bra{0,-1} & \purple{-\frac{E_J}{2}} & \red{U(0,-1) } & \green{-\frac{E_J}{2}} & 0 & 0\\
      \bra{0,0} & \blue{-\frac{E_J}{2}} & \green{-\frac{E_J}{2}} & \red{U(0,0)} & \green{-\frac{E_J}{2}} & \blue{-\frac{E_J}{2}}\\
      \bra{0,1} & 0 & 0 & \green{-\frac{E_J}{2}} & \red{U(0,1) }& \purple{-\frac{E_J}{2}}\\
      \bra{1,0} & 0 & 0 & \blue{-\frac{E_J}{2}} & \purple{-\frac{E_J}{2}} & \red{U(1,0)}\\
    }
  \end{equation}
\end{enumerate}

  \begin{center}
    \ipic{6cm}{3jja}
    \label{fig:l33jja}
    \captionof{figure}{System of tunnelling electrons between the three junctions}
  \end{center}



  \newpage

