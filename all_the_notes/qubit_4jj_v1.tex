\section{4JJ-series Flux Qubit \cite{qui2016}}
 \iframe{This qubit behaves like a 3JJ qubit, albeit with some \red{differences}:
 	\begin{itemize}
 		\item Can be operated in the single or double potential well regime;
 		\item \red{Has a lower flux sensitivity and thus better to tune};
 		\item \red{Under certain conditions \textbf{only \iket{1}\lra\iket{2}} allowed. In other systems other transitions are also possible} - no state leakage to \iket{3}.
 \end{itemize}}

 \subsection{Derivation of the Hamiltonian}
  It's a load of bloat including:
  \begin{enumerate}
  	\item Using phase loop quantisation
  	\begin{equation}\label{key}
  		\varphi_1 + \varphi_2 + \varphi_\alpha + \varphi_\beta + 2\pi f_\text{tot} = 0;
  	\end{equation}
  	\item Kinetic energy evaluation, using emf induction $ V = \dot{\Phi} $:
  	\begin{equation}\label{key}
  		\begin{aligned}
	  		\mathcal{T} = & \frac{1}{2}\sum C_iV_i^2\\
	  		& = \frac{C}{2}\left(\frac{\Phi_0}{2\pi}\right)^2\left\lbrace\dot{\varphi}^2_1 + \dot{\varphi}^2_2 + \alpha\dot{\varphi}^2_\alpha + \beta\left[\dot{\varphi}_1 + \dot{\varphi}_2 + \dot{\varphi}_\alpha + 2\pi \dot{f}_\text{tot}\right]^2\right\rbrace;
  		\end{aligned}
  	\end{equation}
  	\item Potential energy 
  	\begin{equation}\label{key}
  		U = \sum E_{Ji}\left[1 - \cos(\varphi_i)\right].
  	\end{equation}
  	\item Inductive energy
  	\begin{equation}\label{key}
  		U_L = \frac{\Phi_0^2}{2L}\left(f_\text{tot} - f_\text{ext}\right)^2.
  	\end{equation}
  	\item Some complicated transformation that maps $ \varphi_1, \varphi_2, \varphi_\alpha \ra \varphi_+, \varphi_-, \varphi, \xi $, where
  	\begin{equation}\label{key}
  		\xi = f_\text{tot} - f_e \approx f_a(t)\ (\text{negligible flux linxed by current}),
  	\end{equation}
  	 \ipic{5cm}{flux4jj_1}
  	\noindent is the time dependent component of the magnetic field to arrive at the largrangian
  	\begin{equation}\label{key}
  		\begin{aligned}
	  		\mathcal{L} & = \mathcal{T} - U\\
	  		& =  \frac{C}{2}\left(\frac{\Phi_0}{2\pi}\right)^2\left(\dot{\varphi}^2 + \Gamma_+\dot{\varphi}_+^2 + \Gamma_-\dot{\varphi}_-^2 + \Gamma_\xi\xi^2\right)\\
	  		& - U(\varphi, \varphi_+, \varphi_-, \xi) - \frac{\Phi_0^2}{2L}\left(\xi - f_\text{a}\right)^2
  		\end{aligned}
  	\end{equation}
  	\item Finding the canonical momenta 
  	\begin{equation}\label{key}
  		P_i = \difffrac{\mathcal{L}}{\varphi_i}
  	\end{equation}
  	
  	\noindent and evaluating the Hamiltonian, $ \mathcal{H} = \sum P_i\varphi_i - \mathcal{L} $.
  	\item Splitting the Hamiltonian up into a Harmonic oscillator part
  	\begin{equation}\label{key}
  		\mathcal{H} = 4E_C\left(P^2 + \frac{P_+^2}{\Gamma_+} + \frac{P_-^2}{\Gamma_-}\right) + U(\varphi, \varphi_+, \varphi_-, \xi) + \red{\mathcal{H}_\text{osc}},
  	\end{equation}
  	
  	\noindent \red{which is always in the ground state, due to its energy being much higher than the qubit levels. Thus it can be dropped from consideration.}
  \end{enumerate}

\subsection{Bloat over}
 So now lets have a look at some implications of this Hamiltonian.
 
 \begin{enumerate}
 	\item \textbf{The potential landscape that forms} has adjustable single and double wells through $ \beta $.
 	\iframe{\red{Double well for $ \beta > 0.3 $, which is much broader than the 0.5 threshold in 3 JJ qubit.}}
 		\ipic{6cm}{flux4jj_2}
 	\item The solution of $ \mathcal{H}_0\psi(\mathbf{\varphi}) = E\psi(\mathbf{\varphi}) $ is looked for in a block wave form
 	\begin{equation}\label{key}
 		\Psi(\mathbf{\varphi}) = u(\mathbf{\varphi}) = \sum_{\mathbf{K}}a_\mathbf{K}e^{i\mathbf{K}.\mathbf{\varphi}}.
 	\end{equation}
 	
 	\item In the single well configuratio (\textbf{bottom}, $ \beta < 0.3 $) the bands are flatter and less susceptible to flux noise than in the double well (\textbf{top}, $ \beta > 0.3 $) case.
 	
 	\ipic{7cm}{flux4jj_3}
 	\iframe{Note the superior separation of the levels in both the 3JJ (left) and 4JJ (right) qubits. So energy structure is preserved, and that is good.}
 	
 	\item \iframe{\red{\textbf{The addition of a time dependent field changes the hamitlonian to give a perturbation}}:
 	\begin{equation}\label{key}
 		\begin{aligned}
 		\mathcal{H} & = \mathcal{H}_0 - I\Phi_a(t)\\
 		& I = \text{ effective current, which is a monster of an equation}\\
 		& \Phi_a(t) = \iabs{\Phi_a}\cos(\omega_at).
 		\end{aligned}
 	\end{equation}
 	
 	\noindent and we have a transition that is calculated using the eigenstates of the original Hamiltonian
 	
 	\begin{equation}\label{key}
 		t_{ij} = \bra{i}I\Phi_a\ket{j}.
 	\end{equation}}
 	\item The results are:
 	\begin{itemize}
 		\item At the degeneracy point, \iket{1}\lra\iket{3} is forbidden, and we have a $ \Xi $ energy ladder. Everywhere else it's a $ \Delta $ configuration with all transitions allowed.
 		\ipic{6cm}{flux4jj_4}
 		\item For low $ \beta $ all transitions are supressed except \iket{1}\lra\iket{2}. \textbf{\red{This makes it a perfect qubit system, with no leakage to other energy levels.}}
 		\ipic{6cm}{flux4jj_5}
 	\end{itemize}
 \end{enumerate}
 
 
