% -*- TeX-master: "../all_the_notes.tex" -*-

\section{Qubit-Resonator System \cite{cqeResonator}}
\begin{enumerate}
\item The two-level qubit with a separation between the energy levels $ \Delta E $ has the Hamiltonian

\begin{equation}
  \mathcal{H}_{a}= -\frac{\Delta E}{2}\sigma_z.
  \label{qrA}
\end{equation}

\item This qubit is placed in close proximi y with a resonator, which, as we have seen in the previous chapter, has a harmonic
  oscillator-like Hamiltonian

\begin{equation}
  \mathcal{H}_{r}={\hbar\omega}\bigg(\red{\hat{N}}+\frac{1}{2}\bigg)\qquad \qquad a^{\dagger}\ket{N}=\sqrt{N+1}\ket{N+1}; \qquad a\ket{N}=\sqrt{N}\ket{N=1}.
  \label{qrR}
\end{equation}

\noindent where for simplicity we neglect the constant energy term.

\item The general state of a system, where the qubit is in state $ n $ and the resonator in state $ N $ (i.e. $ N $ photons) will
  be

\begin{equation}
  \ket{n,N}.
\end{equation}

\noindent Qualitatively speaking, the way the qubit interacts with the resonator in one of the two ways

\begin{itemize}
\item Qubit absorbs a photon and transitions to the excited state: $ \blue{\ket{0,N+1} \rightarrow \ket{1,N}} $;
\item Qubit relaxes to ground state and releases a photon into the resonator: $ \red{\ket{1,N} \rightarrow \ket{0,N+1}} $.
\end{itemize}

This can be expressed via the interaction Hamiltonian

\begin{equation}
  \mathcal{H}_{\text{int}}=\blue{a\sigma^+}+\red{a^{\dagger}\sigma^-}.
  \label{qrInt}
\end{equation}
\end{enumerate}
Giving in total \iframe{\LARGE
  \[ \mathcal{H} =\blue{-\frac{\Delta E}{2}\sigma_z}+\blue{{\hbar\omega_r}a^\dagger a} + \red{g_0\bigg({a\sigma^+}+{a^{\dagger}\sigma^-}\bigg)},
  \]}

\ipicCaption{8cm}{cavityPic1}{Interaction between states with same energy creates the superposed states in the middle. \label{qb_res_ladder}}
  
\noindent This gives rise to energy levels depicted in Fig.\ref{qb_res_ladder} which we write out in terms of the basis states,
recalling that $ \sigma_z=\iupKetBra- \idownKetBra$, $\sigma^+=\ketbra{\uparrow}{\downarrow}$,
$\sigma^-=\ketbra{\downarrow}{\uparrow}$, $ a = \sqrt{N+1}\ketbra{N}{N+1} $, $ a^\dagger = \sqrt{N+1}\ketbra{N+1}{N} $:

\begin{equation}
  \begin{aligned}
    \mathcal{H} & = \blue{\bigg[\frac{\Delta E}{2}\big(\iupKetBra-\idownKetBra\big)\bigg]\otimes\mathbb{I}_{N}} + \blue{\mathbb{I}_{n}\otimes\bigg[\hbar\omega_r\sum_{N}N\ketbra{N}{N}\bigg]} +\\
    &\ +  \red{g_0\bigg[\ketbra{\uparrow}{\downarrow}\sqrt{N+1}\sum_{N}\ketbra{N}{N+1}+\ketbra{\downarrow}{\uparrow}\sum_{N}\sqrt{N+1}\ketbra{N+1}{N}\bigg]}\\
    & = \sum_{N}\quad\blue{\bigg(\hbar N\omega_r-\frac{\Delta E}{2}\bigg)\ketbra{\downarrow,N}{\downarrow,N}+\bigg(\hbar N\omega_r+\frac{\Delta E}{2}\bigg)\ketbra{\uparrow,N}{\uparrow,N} \leftarrow \text{ diagonal }}\\
    &\ +  \red{g_0\sqrt{N+1}\bigg[\ketbra{\uparrow,N}{\downarrow,N+1}+\ketbra{\downarrow,N+1}{\uparrow,N}\bigg] \leftarrow \text{ cross terms }}\\
  \end{aligned}
\end{equation}

\noindent and in matrix form

\begin{equation}\label{eqn:qubitCavityHamil}
  \mathcal{H} = \kbordermatrix{
    & \ket{\downarrow,N} & \ket{\uparrow,N} & \ket{\downarrow,N+1} & \ket{\uparrow,N+1} \\
    \bra{\downarrow,N} &\blue{\hbar N\omega_r-\frac{\Delta E}{2}} & 0 & 0 & 0\\
    \bra{\uparrow,N} & 0 & \blue{\hbar N\omega_r+\frac{\Delta E}{2}} & \red{g_0
      \sqrt{N+1}} & 0\\
    \bra{\downarrow,N+1} & 0 & \red{g_0\sqrt{N+1}} & \blue{\hbar (N+1)\omega_r-\frac{\Delta E}{2}} & 0\\
    \bra{\uparrow,N+1} & 0 & 0 & 0 & \blue{\hbar (N+1)\omega_r+\frac{\Delta E}{2}}.\\
  }
\end{equation}

\subsection{General solutions}
\begin{enumerate}
\item For the ground state, we take the top row, which will have the lowest energy
  \[
    \iket{\downarrow,0} = \begin{pmatrix} 1 \\0\\0\\\vdots
    \end{pmatrix}\text{ with energy } -\frac{\Delta E}{2}.
  \]
 	
\item Then diagonalising an arbitrary middle matrix (N value incremented for convenience)
 	
 	\begin{equation}\begin{aligned}
            \mathcal{H}_{\text{middle}}
            & = \kbordermatrix{&\ket{\uparrow,N} & \ket{\downarrow,N+1}\\
              \bra{\uparrow,N} & \blue{\hbar\omega_r(N+\frac{1}{2}) + \hbar\Delta} & \red{g_0\sqrt{N+1}}\\
              \bra{\downarrow,N+1} & \red{g_0\sqrt{N+1}} & \blue{\hbar\omega_r(N+\frac{1}{2}) - \hbar\Delta}}\\
            &= \blue{\hbar\omega_r(N+\frac{1}{2})\mathbb{I} +\frac{ \hbar\Delta}{2}\sigma_z} + \red{g_0\sqrt{N+1}\sigma_x}\\
            & = \blue{\hbar\omega_r(N+\frac{1}{2})\mathbb{I}} + \frac{1}{2}\sqrt{(\hbar\Delta)^2 + 4g_0^2(N+1)} \bigg(\cos(\theta)\sigma_z + \sin(\theta)\sigma_x\bigg)\\
            & = \hbar\omega_r(N+\frac{1}{2})\mathbb{I} + E_{\text{coupled}}(\cos(\theta)\sigma_z + \sin(\theta)\sigma_x)\\
            & \text{where } E_\text{coupled} = \frac{\hbar}{2}\sqrt{\Delta^2 + 4(g_0/\hbar)^2(N+1)};\qquad \tan(\theta) =
            \frac{g_0\sqrt{N+1}}{\hbar\Delta/2}.
          \end{aligned}
 	\end{equation}
 	
      \item { \noindent By applying a rotation of $ \theta/2 $ about the y-axis
 		
 		\[
                  U = \exp\big[i\frac{\theta}{2}\sigma_y\big] = \cos(\theta/2)\mathbb{I} + i\sin(\theta/2)\sigma_y,
 		\]}
 	
              \noindent we will end up with
              \iframe{\[ \mathcal{H'} = \hbar\omega_r(N+\frac{1}{2})\mathbb{I} + \frac{E_\text{coupled}}{2}\sigma_z
                  = \begin{pmatrix} \hbar\omega_r(N+\frac{1}{2}) + \frac{E_\text{coupled}}{2} & 0\\0& \hbar\omega_r(N+\frac{1}{2})
                    - \frac{E_\text{coupled}}{2}
                  \end{pmatrix}
                \]
                \begin{itemize}
 		\item \textbf{Eigenstates}:\hfill \iket{\tilde{0}}, \iket{\tilde{1}};
 		\item \textbf{Eigenenergies}: \hfill $ \hbar\omega_r(N+\frac{1}{2}) \pm \frac{E_\text{coupled}}{2} $.
                \end{itemize}
              }
 	
            \item In the original basis of \iket{\uparrow, N}, \iket{\downarrow, N} \iframe{
 		\begin{itemize}
                \item \textbf{Eigenstates:}
                  \[
                    \begin{aligned}
                      \iket{+,N} & = U^{\dagger}\ket{\tilde{1}} = \bigg(\cos(\theta/2)\mathbb{I}-i\sin(\theta/2)\sigma_y\bigg)
                      \begin{pmatrix}1\\0 \end{pmatrix} =
                      \begin{pmatrix} \cos(\theta/2)\\\sin(\theta/2)\end{pmatrix} \\
                      \iket{-,N} & = U^{\dagger}\ket{\tilde{0}} = \begin{pmatrix} -\sin(\theta/2)\\\cos(\theta/2)
                      \end{pmatrix} \equiv -\sin(\theta/2)\iket{\uparrow,N} + \cos(\theta/2)\iket{\downarrow,N+1}\\
                    \end{aligned}
                  \]
 		\item \textbf{Eigenenergies:} \hfill $ \text{Energies}_{\pm} = \hbar\omega_r(N+\frac{1}{2}) \pm
                  \frac{E_\text{coupled}}{2} $
                \end{itemize}
              }
            \end{enumerate}
 
            \ipic{5cm}{cavityPic2}
 
            \noindent The eingenstate spectrum of the \iket{\pm,N} states is ladder-like.  For a single photon, and on resonance,
            $ \Delta = 0 $ the entangled states are
 
 \[
   \iket{\pm,1} = \frac{\iket{\uparrow,0} \pm \iket{\downarrow,1}}{\sqrt{2}} \quad \text{ with energies }
   \hbar\omega_r(N+\frac{1}{2}) \pm g_0\sqrt{N+1},
 \]
 
 \noindent will flip flop between the original levels with a Rabi frequnecy $ 2g_0/2\pi $. As the atom is excited for exactly half
 of the state, with a decay rate $ \gamma $ , and the cavity is excited for the other half of the state, with a decay rate
 $ \kappa $ the net decay rate is $ \frac{\gamma+\kappa}{2} $ and to observe Rabi oscillations between the original states
 
 \[
   2g > \frac{\kappa+\gamma}{2} \qquad\qquad\text{\red{to see Rabi oscillation before decay == \textbf{strong coupling}}}
 \]
 
 \subsection{Resonant case = Dressed States} {\LARGE
   \[\hbar\omega_r \equiv \Delta E\]}
 \begin{equation}
   \mathcal{H} = \kbordermatrix{
     & \ket{0,N-1} & \ket{1,N-1} & \ket{0,N} & \ket{1,N} \\
     \bra{0,N-1} &\blue{\Delta E\big(N-\frac{3}{2}\big)} & 0 & 0 & 0\\
     \bra{1,N-1} & 0 & \blue{\Delta E\big(N-\frac{1}{2}\big)} & \red{g_0\sqrt{N}} & 0\\
     \bra{0,N} & 0 & \red{g_0\sqrt{N}} & \blue{\Delta E\big(N-\frac{1}{2}\big)} & 0\\
     \bra{1,N} & 0 & 0 & 0 & \blue{\Delta E\big(N+\frac{1}{2}\big)}\\
   },
 \end{equation}

 \noindent and a degeneracy appears between the two middle states as in Fig.\ref{qrLevel}. The middle part of the Hamiltonian is
 treated as a two levels system

\begin{equation}
  \mathcal{H}_{\text{middle}} = \begin{pmatrix}
    \blue{\Delta E\big(N-\frac{1}{2}\big)} & \red{g_0\sqrt{N}}\\
    \red{g_0\sqrt{N}} & \blue{\Delta E\big(N-\frac{1}{2}\big)}
  \end{pmatrix}\qquad\Rightarrow\qquad\kbordermatrix{
    &\ket{1,N-1}&\ket{0,N}\\
    \bra{1,N-1}& 0 & \red{g_0\sqrt{N}}\\
    \bra{0,N}& \red{g_0\sqrt{N}} & 0},
\end{equation}

\noindent which can be diagonalised with two states separated by an energy $ g_0\sqrt{N} $, also depicted in
Fig.\ref{qrLevel}. The degeneracy is lifted for every single state, and these states are known as dressed.

\begin{equation}
  \ket{\Psi} = \frac{\ket{0,N}\pm\ket{1,N-1}}{\sqrt{2}} \qquad E = \pm{g_0\sqrt{N}}.
\end{equation}

\begin{figure}
  \ipic{6cm}{qrLevel}
  \caption{Degeneracy between different photon states \label{qrLevel}. Note that relative to zero energy (for $ \ket{0,0} $) one
    is shifted by $ N\hbar\omega_r-\frac{\Delta E}{2}=\bigg(N-\frac{1}{2}\bigg)\Delta E $}
\end{figure}

Fig.\ref{qrDresssed} is intepreted as follows:

\begin{itemize}
\item We have a qubit, that is either in state \iket{0} or \iket{1}, whose energy spectrum is shown by the green lines. By
  controlling some bias e.g. gate voltage, we choose the energy difference $ \Delta E $ between the two levels.
\item We apply a \textbf{fixed} resonator drive at $ \hbar\omega_r $. This makes a copy of the original qubit states, shifted by
  $ \hbar\omega_r $. This configuration corresponds to the \blue{blue} terms in the above equations.
\item At the degeneracy point for the new qubit-resonator system, degeneracy will be lifted by the qubit-resonator interaction \ra
  we have just seen that two states $ \ket{0,N}, \ket{1,N-1} $ will interact and developed a splitting of $ 2g_0\sqrt{N} $ at the
  degeneracy point. \red{This happens when qubit is biased to a value were $ \Delta E =\hbar\omega_r$ i.e. the level separation is
    exactly equal to the drive from the resonator. For all other cases there is no splitting/mixing.}
\end{itemize}

\begin{figure}[h]
  \ipic{7cm}{d4}
  \caption{We mix a qubit (with its typical energy dispersion) with a resonator. This leads to multiple 'copies' of the qubit at
    the resonator frequency separations. These states will cross whenever $ \hbar\omega_r\equiv\Delta E $ i.e. the qubit is biased
    to a configuration which is exactly in resonance with the applied field (or analogously we tune the resonator to match the
    $ \Delta E $). As seen with the above matrix, interaction between $\ket{0,N}$ and $ \ket{1,N-1} $ will lift the degeneracy at
    this anticrossing.\label{qrDresssed}}
\end{figure}

Now let us perform measurements on this system.

\begin{itemize}
\item We bias the qubit to an arbitrary $ \Delta E $ and couple it with a resonator, with frequency $ \omega_r $.
\item Then we send a weak field probe signal, measuring its transmission as we sweep it. It will have a peak at the frequency of
  resonator $ \omega_0 $, which corresponds to the excitation of the qubit.
\item We plot the maximum of this peak as shown in Fig.\ref{qrProbe}.
\item Then we step the control parameter, to move along the qubit curve and repeat.
\item Everywhere apart from the degeneracy point, there will be a single transition - the $ \hbar\omega_0 $ one that will
  dominate.
\item However near the degeneracy point, two transitions, symmetrical about $ \omega_0 $ will be present. This is a result of the
  $ \pm g_0\sqrt{N} $ splitting that occurs at the crossing point.
\item Thus at the degeneracy point one observes two peaks for the transmission as in Fig.\ref{qrDegen2} - the two strong
  transitions at the splitting point. Generally the lower peak will be stronger - smaller energy difference in the process.
\item The width of the peak $ \gamma $ corresponds to the width of the level - the area to which one can excite to. One needs
  $ \gamma<<g_0 $ to observe such a transitions. This is fulfilled for a strong drive.
\end{itemize}

\begin{figure}
  \ipic{5cm}{probe}
  \caption{We use a weak probe field and measure its transmission curve. The peak corresponds to a favourable transition
    frequency. \label{qrProbe}}
\end{figure}

{\LARGE The level splitting
	
  \red{\begin{equation} \hbar\Omega_\text{splitting} = 2g\sqrt{N},
    \end{equation}}
	
  \noindent defines the Rabi frequency $ \Omega_\text{splitting} $ and becomes constant for $N>>1$.}

The configuration in Fig.\ref{qrDegen2} is the Mollow triplet - it is probed either by a weak field (as explained above) or
directly observed in the spectrum emitted by the atom.


\begin{figure}
  \ipic{6cm}{degen2}
  \caption{At the degeneracy point, when the resonator field coincides with the qubit transition, two types of transitions appear
    from the splitting at the degeneracy point. The splitting $ \pm g_0\sqrt{N} $ will put two peaks either side of the central
    one (which disappears - there is no longer a level to accommodate this transition). $ \gamma $ characterises the width of the
    split levels and hence the width of the peak. If the width is too big, no splitting shall be seen.\label{qrDegen2}}
\end{figure}

\begin{figure}
  \ipic{5cm}{tirplet}
  \caption{Mollow triplet is formed in the resonant case. Three transitions give rise to three peaks.\label{qrMollow}}
\end{figure}
\newpage
\subsection{Realisation}
To relise, we put the atom at the maximum voltage of a `cut out' resonator:

\ipic{4cm}{cavity_qed_1} Advantages:
\begin{itemize}
\item The coupling strength is very large because of the small sizes of the elements. The voltage between theground plane and
  resonator is 0.2\,V/m,which is 100 times stronger than for a regular cavity;
\item The geometry of the resonator fixes its frequency \ra no 1/f noise;
\item Atom will emit directly into the line, and with a high enough quality factor, losses are minimised.
\end{itemize}

\subsection{Second order effects}
Just to refresh out memory, our Hamiltonian has the following form

\begin{equation}\label{sec}
  -\frac{\Delta E}{2}\sigma_x + \hbar\omega_ra^{\dagger}a + \red{g_0\big(a\sigma^{+}+a^{+}\sigma^{-}\big)},
\end{equation}

\noindent and without the \red{interaction term} one had the following eigenstates and eigenenergies

\begin{equation}\label{secStates}
  \begin{aligned}
    &\ket{n,N}\\
    &\big(n-\frac{1}{2}\big)\Delta E + \hbar\omega N.
  \end{aligned}
\end{equation}

\noindent Now we consider second order effects, where there is a frequency shift when the qubit is in the excited state. These
second order effects will change the energies and eigenstates

\begin{equation}\label{secEi}
  E'_{n,N} = E_{n,N}^{(0)} + E_{n,N}^{(1)};\quad \ket{\Psi'_{n,N}}=\ket{\Psi^{(0)}_{n,N}}+\ket{\Psi^{(1)}_{n,N}}+\ket{\Psi^{(2)}_{n,N}}.
\end{equation} 


\begin{figure}[h]
  \ifigure{4cm}{shift} \ifigure{4cm}{sfhit1}
\end{figure}

\noindent From time independent perturbation theory, one finds that the first order energy shift term is

\begin{equation}\label{sec1st}
  E_{n,N}^{(1)} = \bra{\Psi^{(0)}_{n,N}}V\ket{\Psi^{(0)}_{n,N}} = \bra{n,N}g_0\big(a\sigma^{+}+a^{+}\sigma^{-}\big)\ket{n,N} \equiv 0,
\end{equation}

\noindent while the second order energy shift

\begin{equation}\label{sec2nd}
  E_{n,N}^{(2)} = \sum_{m,M}\frac{\left|\bra{m,M}g_0\big(a\sigma^{+}+a^{+}\sigma^{-}\big)\ket{n,N}\right|^2}{E_{n,N}-E_{m,M}}.
\end{equation}

\noindent For the two states of the qubit this evaluates to

\begin{equation}\label{sec0}
  E_{0,N}^{(2)} = |g_0|^2\sum_{m,M}\frac{\left|\bra{m,M}\big(\cancel{a\sigma_{+}}+a^{\dagger}\sigma_{-}\big)\ket{0,N}\right|^2}{E_{0,N}-E_{m,M}}= |g_0|^2\frac{\left|\bra{1,N-1}\sqrt{N}\ket{1,N-1}\right|^2}{E_{0,N}-E_{1,N-1}} = -\frac{|g_0|^2N}{\Delta E - \hbar\omega}
\end{equation}

\noindent and

\begin{equation}\label{sec1}
  E_{1,N}^{(2)} = |g_0|^2\sum_{m,M}\frac{\left|\bra{m,M}\big({a\sigma_{+}}+\cancel{a^{\dagger}\sigma_{-}}\big)\ket{1,N}\right|^2}{E_{1,N}-E_{m,M}}= |g_0|^2\frac{\left|\bra{0,N+1}\sqrt{N+1}\ket{0,N+1}\right|^2}{E_{1,N}-E_{1,N-1}} = \frac{|g_0|^2(N+1)}{\Delta E - \hbar\omega \hbar\omega}
\end{equation}

\noindent From this we evaluate the energy difference between atomic and resonator transitions:

\begin{itemize}
\item \textbf{Atomic transition} can be seen to depend on the number of photons in the resonator
	
	\begin{equation}\label{secAtom}\begin{aligned}
            E'_{1,N}-E'_{0,N} & = E_{1,N}^{(0)} + \cancel{E_{1,N}^{(1)}} + E_{1,N}^{(2)} - E_{0,N}^{(0)} - \cancel{E_{1,N}^{(1)}}
            - E_{0,N}^{(2)} \\& = \Delta E + \frac{|g_0|^2(N+1)}{\Delta E - \hbar\omega} - (-\frac{|g_0|^2N}{\Delta E -
              \hbar\omega})\\& = {\Delta E+\frac{|g_0|^2(2N+1)}{\Delta E - \hbar\omega}}
          \end{aligned}
	\end{equation}
      \item \textbf{Resonator transition} depends on the state of the qubit
	
	\begin{equation}\label{secRes}
          E'_{n,N+1}-E'_{n,N} = \left[\begin{aligned}
              & \hbar\omega  -\frac{|g_0|^2}{\Delta E -  \hbar\omega} \text{ for n=0}
              \\&  \hbar\omega + {\frac{|g_0|^2)}{\Delta E -  \hbar\omega}}\text{ for n=1}
            \end{aligned}\right.
	\end{equation}
      \end{itemize}

      \newpage

      \section{Atom - resonator coupling}
      \noindent \red{\large \textbf{Solving for the Harmonic oscillator alone}}, with a relaxation rate $ \kappa $ between the
      levels, it is convenient to write the Linbland term and the density matrix as

\begin{equation}\label{ar2}
  \mathcal{H} = \hbar\omega_ra^{\dagger}a+\hbar\Omega(a+a^{\dagger})\cos(\omega t);\quad\rho^{(r)} = \sum_{M,N=0}^{\infty}\rho_{NM}^{(r)}\ketbra{N}{M};\quad \mathcal{L}^{(r)} = \frac{\kappa}{2}\bigg(2a\rho a^{\dagger}-a^{\dagger}a\rho-\rho a^{\dagger}a\bigg),
\end{equation}

\noindent which ultimately results in the following contribution to from the Linbland term

\begin{equation}\label{ar3}
  \dot{\rho}_{NN}^{(r)} \iright \kappa(N+1)\rho_{N+1\ N+1}^{(r)} - \kappa N \rho_{NN}^{(r)}.
\end{equation}


\begin{enumerate}
\item The Hamiltonian for the resonator is transformed by moving into the corresponding rotating frame
	
	\begin{equation}\label{ar4}
          \mathcal{H'} = -\hbar\delta\omega a^{\dagger}a+\frac{\hbar\Omega}{2}(a+a^{\dagger}).
	\end{equation}
	
      \item Solving the Master equation for the continuous drive for the stationary condition and assuming that the driving is
        weak (and thus only the bottom photon levels will be occupied), the solution can be truncated to have only the \iket{0}
        and \iket{1} photon states.
	
	\begin{figure}[h]
          \ifigure{4cm}{resDen1} \ifigure{4cm}{resDen2}
	\end{figure}
	
      \item The expectation value for the field in the resonator is found using the density matrix for the stationary condition
	
	\begin{equation}\label{filedRes}
          \iaverage{V} = V_0\iaverage{a+a^{\dagger}} = V_0\itrace{(a+a^{\dagger})\rho}\approx\frac{2\Omega}{\kappa+2\delta\omega},
	\end{equation}
	
	\noindent which is a Lorentian, whose peak occurs for the case when $ \delta\omega=0 $ and the field being driven through
        the resonator is exactly in resonance with it.
      \end{enumerate}

      \noindent \red{\large \textbf{Now include the atom with the resonator system}}

      \begin{equation}\label{ar1}
        \begin{aligned}
          \mathcal{H} & ={-\frac{\Delta E}{2}\sigma_z}+{{\hbar\omega_r}a^\dagger a} + {g_0\bigg({a\sigma^+}+{a^{\dagger}\sigma^-}\bigg)}\\
          &\mathcal{L}^{(a)} = \frac{\Gamma_1}{2}\bigg(2\sigma^{-}\rho \sigma^{+}-\sigma^{+}\sigma^{-}\rho-\rho \sigma^{+}\sigma^{-}\bigg);\quad\mathcal{L}^{(r)} = \frac{\kappa}{2}\bigg(2a\rho a^{\dagger}-a^{\dagger}a\rho-\rho a^{\dagger}a\bigg)\\
          \dot{\rho} & = -\frac{i}{\hbar}\big[\mathcal{H},\rho\big]+\mathcal{L}^{(a)}+\mathcal{L}^{(r)};\quad\quad\quad \rho = \sum_{m,n=0}^{1}\sum_{M,N=0}^{\infty}\rho_{nm,NM}\ketbra{nN}{mM}\\
        \end{aligned}
      \end{equation}

      \noindent and this kind of system has been solved before, but now there are decay mechanisms also. The states decay with a
      rate

\begin{equation}\label{ardecay}
  \gamma=\frac{\kappa+\Gamma_1}{2},
\end{equation}

\noindent meaning that in order to couple the qubit and resonator, the energy exchange must occur faster than this decay time. The
energy exchange occurs at a rate $ \hbar/g_0 << 1/\Gamma_1,1/\kappa$, imposing the condition that

\begin{equation}\label{arD}
  g_0>>\hbar\Gamma_1,\hbar\kappa,
\end{equation}

\noindent in which the characteristic7c energy is higher than incoherent processes.

\begin{figure}[h]
  \ifigure{4cm}{ladderGrid}\ifigure{4cm}{ladderGrid1} \ifigure{5cm}{coupling}
\end{figure}

\newpage


