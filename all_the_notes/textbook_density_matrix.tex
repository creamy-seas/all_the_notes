\section{Background on quantum mechanics principles}
 \subsection{The density matrix}
 First we introduce the concept of the density matrix, derived fully in \cite{quantum_course}. To begin with we consider a system with probabilities $p_1$ and $p_2$ of being in state \psiOne and a \psiTwo. The expectation value for the observable $\hat{O}$ is then

\begin{equation}
\Braket{\hat{O}} = p_1\Braket{\Psi_1|\hat{O}|\Psi_1} + p_2\Braket{\Psi_2|\hat{O}|\Psi_2}, 
\end{equation}

\noindent which is some form of numerical value. Taking the trace of this value, which can be viewed as a 1-dimensional matrix, and remembering trace is a linear operator and that its cyclic invariance, one gets

\begin{equation}
\begin{aligned} 
Tr\left( \Braket{\hat{O}}\right)  = & Tr\left( p_1\Braket{\Psi_1|\hat{O}|\Psi_1}\right)  + Tr\left( p_2\Braket{\Psi_2|\hat{O}|\Psi_2}\right)  \\
= & p_1Tr\left( \hat{O}\Ket{\Psi_1}\Bra{\Psi_1}\right)  + p_2Tr\left( \hat{O}\Ket{\Psi_2}\Bra{\Psi_2}\right) \\
= & Tr\left(p_1\hat{O}\Ket{\Psi_1}\Bra{\Psi_1} + p_2 \hat{O}\Ket{\Psi_2}\Bra{\Psi_2}\right) \\
= & Tr\left( \hat{O}\rho \right),
\end{aligned}
\end{equation}

\noindent where $\rho = p_1\Ket{\Psi_1}\Bra{\Psi_1} + p_2\Ket{\Psi_2}\Bra{\Psi_2}$ is the density operator. Generalising for $n$ states,

\textcolor{red}{\begin{equation}
	\begin{aligned}
	\rho & = \sum_{i=1}^{n} p_i\Ket{\Psi_i}\Bra{\Psi_i}\\
	\textcolor{red}{\Braket{\hat{O}}} &  \textcolor{red}{= \text{Tr}\left( \hat{O}\rho\right).} 
	\end{aligned}
	\label{eqn:rawDensity}
	\end{equation}}

\noindent Note that the states do \textbf{not} need to be orthogonal. This notation is convenient when there is uncertainty in the preparation of states in our system. When the density matrix is

\begin{equation}
\rho = \Ket{\Psi}\Bra{\Psi},
\end{equation}

\noindent then it is a pure state, otherwise is is mixed. Generally one shall work in the matrix representation of the density matrix, in which case one inserts the closure relation $\sum\Ket{\phi}\Bra{\phi}$ to get

\begin{equation}
\begin{aligned}
\rho = & \sum_j\Ket{\phi_j}\Bra{\phi_j}\rho\sum_k\Ket{\phi_k}\Bra{\phi_k}\\
= & \sum_{jk}\rho_{jk} \Ket{\phi_j} \Bra{\phi_k},
\end{aligned}
\end{equation}

\noindent where $\rho{jk}=\Bra{\phi_j}\rho\Ket{\phi_k}$. In such a way, the density matrix can be written

\begin{equation}
\rho = \left( \begin{matrix}
\rho_{11} & \rho_{12} & \rho_{13} \\
\rho_{31} & \rho_{22} & \rho_{23} \\
\rho_{31} & \rho_{32} & \rho_{33} \\
\end{matrix}\right).
\end{equation}

\noindent The density matrix has several important properties.

\begin{itemize}
	\item $\rho = \rho^{\dagger}$
	\item $\rho$ has unit trace. This is seen by looking at Eqn.\eqref{eqn:rawDensity} and noting that in the basis of the given vectors, the trace must be 1, since the sum of the probabilities is 1. Trace is invariant under the basis that one chooses, so $\text{Tr}(\rho) = 1$
	\item Finally one can test for the purity of the system, by evaluating $\text{Tr}(\rho^2)$. From Eqn.\eqref{eqn:rawDensity} its is obvious that the diagonal elements will be squared. Since they are all $\le1$, summing them up will lead to
	\begin{itemize}
		\item $\text{Tr}(\rho^2) = 1$, in which case the state is pure (only one element, with $p=1$).
		\item $\text{Tr}(\rho^2) < 1$, in which case the state is mixed \textbf{and cannot be expressed via a single projector}
	\end{itemize}
	\textbf{\red{Purity does not change under unitary evolution}} (prove using cyclic invariance of trace)
\end{itemize}

An important operation involving the density matrix is the partial trace. Consider first a system in an entangled state - that is we consider two quantum systems \textit{A} and \textit{B} and observe their joint state. If this state \textbf{cannot} be factorised into distinct states for system \textit{A} and \textit{B}, then the system is entangled.

Next consider the density matrix for a compound system, which is represented in the orthonormal basis $\left\lbrace \Ket{a_j,b_k}\right\rbrace $ of the joint system

\begin{equation}
\rho = \sum_{j,k}\sum_{l,m} \rho_{j,k,l,m}\Ket{a_j,b_k}\Bra{a_l,b_m},
\end{equation}

with $\rho_{j,k,l,m} = \Bra{a_j,b_k}\rho\Ket{a_j,b_k}$. If one is dealing with an entangled states, then it is not possible to separate the state into an individual description of the two systems. What one can do, is trace out one of the system, while retaining all the information about the observable of the other. To see this, consider how one would find the expectation value of an observable in \textit{A}

\begin{equation}
\begin{aligned}
\Braket{\hat{O_A}} & = \trace\left( \hat{O}_A\otimes \mathbb{I} \rho_{AB}\right) \\
& = \sum_{jk} \Bra{a_j,b_k} \hat{O}_A\otimes \mathbb{I} \rho_{AB} \Ket{a_j,b_k}\\
& = \sum_j\Bra{a_j}\hat{O}_A\left( \sum_k\Bra{b_k}\rho_{AB}\Ket{b_k}\right) \Ket{a_j} \\
& = \sum_j\Bra{a_j}\hat{O}_A\rho_A \Ket{a_j}\\
& = \trace\left( \hat{O}_A\rho_A \right),
\end{aligned}
\end{equation}

\noindent where we defined $\rho_A = \trace_B(\rho_{AB}) = \sum_{k} \Bra{b_{k}}\rho_{AB}\Ket{b_{k}}$. This has effectively traced out/took partial trace to give a reduced state of the system. The action of the tracing out

\begin{equation}
\begin{aligned}
\textcolor{red}{\rho_A = \trace_B(\rho_{AB})} &  \textcolor{red}{= \sum_{k'} \Bra{b_{k'}}\rho_{AB}\Ket{b_{k'}}}\\
& = \sum_{jk}\sum_{lm}\sum_{k'} \Bra{b_{k'}}\rho_{j,k,l,m}\Ket{a_j,b_k}\Bra{a_l,b_m}\Ket{b_{k'}}\\
& = \sum_{jk}\sum_{lm}\sum_{k'} \rho_{j,k,l,m}\Ket{a_j}\Bra{b_{k'}}\Ket{b_k}\Bra{b_m}\Ket{b_{k'}}\Bra{a_l}\\
& = \sum_{jk}\sum_{lm}\sum_{k'} \rho_{j,k,l,m}\Ket{a_j}\Bra{a_l}\delta_{k'k}\delta_{mk'}\\
& = \sum_{j}\sum_{l}\sum_{k'} \rho_{j,k',l,k'}\Ket{a_j}\Bra{a_l}\\
& = \sum_{j}\sum_{l}\left( \sum_{k}\rho_{j,k',l,k'}\right) \Ket{a_j}\Bra{a_l},
\end{aligned}
\end{equation}

\noindent and in the last line we see that the result is a set of operators with weights $\sum_{k}\rho_{j,k',l,k'}$. The reduced state $\rho_A$ will only be pure if there was no entanglement. 

The evolution of the density matrix will be

\begin{equation}
\label{eqn:denMatEvolution}
\begin{aligned}
\rho(0) = \sum_{i=1}^{n} p_i\Ket{\Psi_i}\Bra{\Psi_i} \longrightarrow \rho(t) & = \sum_{i=1}^{n} p_iU(t)\Ket{\Psi_i}\Bra{\Psi_i}U(t)^{\dagger}\\
& =  \textcolor{red}{U(t)\rho(0)U^\dagger(t)},
\end{aligned}
\end{equation}

\noindent where $U(t) = e^{-i\frac{\hat{H}}{\hbar}t}$, which, as a reminder, can be written out in matrix form as

\begin{equation}
U(t) = e^{-i\frac{\hat{H}}{\hbar}t} = \sum_k \frac{\left( -i\frac{\hat{H}}{\hbar}t\right)^k }{k!}.
\end{equation}

\textbf{The most important question of why use it? It widens the state vector formalism to represent states prepared probabilistically, which do {\LARGE Cannot be described by a wavefunction.}}

One can get a version of the \schrodinger equation, by differentiating Eq.\eqref{eqn:denMatEvolution}

\begin{equation}
\label{eqn:vonNeuman}
\begin{aligned}
\ipartial{\rho(t)}{t} & = \dot{U}(t)\rho(0)U^\dagger(t) + {U}(t)\rho(0)\dot{U}^\dagger(t)\\
& = -\frac{i}{\hbar}\hat{H}{U}(t)\rho(0)U^\dagger(t) + \frac{i}{\hbar}\hat{H}{U}(t)\rho(0){U}^\dagger(t)\\
& =  \textcolor{red}{\frac{i}{\hbar}\ \big[\rho(t),\hat{H}\big]}
\end{aligned}
\end{equation}
\newpage  
