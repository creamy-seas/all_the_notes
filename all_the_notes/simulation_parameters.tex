\section{Energy Parameters\label{app:energy_parameters}}
\subsection{Charging energy $ E_c $}
To simulate the charging energy
 
 \begin{equation}\label{eqn:sim_1}
   E_c = \frac{(2e)^2}{2C},
 \end{equation}
 
 \noindent we need the capacitance of the JJ. We can treat the two overlapping parts of the JJ
 acting a parallel plate capacitor with
 
 \iframe{\begin{equation}\label{key} C = \frac{\varepsilon\varepsilon_0A}{d},
   \end{equation}
 
   \noindent  with the  permitivity  for  Aluminium oxide  being  $ \varepsilon  \approx  10  $ and  thickness
   $  d \approx  \iunit{2}{nm} $.   This give  s a  junction 200  $ \times  $ 800\,nm\ipow{2}  a charging
   frequency of $ E_c/\hbar \approx\iunit{19}{GHz} $.}
 
 \subsection{Josephson energy $ E_J $}
 In superconductors CP carry charge. What happens is  that the Fermi level splits into 2 bands
 that are $\pm\Delta$  above and below $E_F$, and each  electron in the CP belongs to  one of these
 bands.
  
  
 {{ Phase is quantised in a sc loop
     \begin{equation}
       \begin{aligned}
         \phi = \phi_\text{ext} + 2\pi N \Leftrightarrow \Phi = \Phi_\text{ext} + \Phi_0 \\(\Phi_0 = \frac{h}{2e}, \Phi/\Phi_0=\phi/2\pi)
       \end{aligned}
     \end{equation}
   }}
  
 Now work with  JJ. The states on the  two sides of the JJ  are $\left|\psi_0\right|e^{i\phi_1}$ and
 $\left|\psi_0\right|e^{i\phi_2}$. Solving the  Schrodinger equation for the  condensate state, when
 E=0
  
  \begin{equation}
    -\frac{\hbar^2}{2m}\iderivative{^2}{x^2}\psi+U\psi = 0 \Rightarrow
    \left\lbrace\begin{aligned}
        \psi & = A_0e^{-kx}+B_0e^{+kx}\\
        k & = \frac{\sqrt{2mU}}{\hbar}
      \end{aligned} \right. \Rightarrow \left\lbrace\begin{aligned}
        \psi & = A\cosh(kx)+B\sinh(kx)\\
        k & = \frac{\sqrt{2mU}}{\hbar}
      \end{aligned} \right.
  \end{equation}
  
  \noindent    Apply     BC    for     JJ    $\psi(a/2)    =     \left|\psi_0\right|e^{i\phi_2}$    and
  $\psi(-a/2) = \left|\psi_0\right|e^{i\phi_1}$ to find that
  
  \begin{align}
    A &= \frac{\left|\psi_0\right|}{\cosh(ka/2)}\\
    B &= \frac{\left|\psi_0\right|}{\sinh(ka/2)}.
  \end{align}
  
  \noindent {The super current is then}
  
  \begin{equation}
    I = -\frac{i\hbar}{2m}(2e)\left[\psi^*\iderivative{\psi}{t}-\psi\iderivative{\psi^*}{t}\right] \equiv -\frac{2e\hbar}{m}\text{Im}\left[\psi^*\iderivative{\psi}{t}\right],
  \end{equation}
  
  \noindent at $x=0$ can be evaluated, as can the voltage and energy, which results in a phase
  dependence
  
  \begin{equation}
    \label{l2-dcac}
    I = I_c\sin(\phi_2-\phi_1); \qquad \frac{d\phi}{dt} = \frac{2e}{\hbar}V
  \end{equation}
  
  The phase across the JJ in a circuit is the phase induced by the external flux i.e.
  
  \begin{equation}
    \label{eqn:l2-phasesum}
    \phi_1+\phi_2+\ldots = \phi_\text{ext} = \frac{\Phi_\text{ext}}{\Phi_0}2\pi,
  \end{equation}
  
  \begin{equation}
    \left\lbrace\begin{aligned}
        I &= I_c\sin(\phi_2-\phi_1)\equiv I_c\sin(\phi)\\
        V&=\dot{\Phi} \equiv \frac{\dot{\phi}}{2\pi}\Phi_0
      \end{aligned}\right. \Rightarrow U = \left\lbrace\begin{aligned}
        \int_{0}^{t}IV dt & = \int_{0}^{t}I_c\sin(\phi)\frac{\Phi_0}{2\pi}\frac{d\phi}{dt}dt\\
        & = \int_{0}^{\phi} E_J\sin(\phi)d\phi\\
        & \red{= E_J(1-\cos(\phi)),\qquad E_J = \frac{\Phi_0I_c}{2\pi}.}\ec
      \end{aligned}\right.
    \label{l2-JJEnergy}
  \end{equation} 
  
  \noindent The current is given by:
  
  \begin{equation}\label{key}
    I_cR_n = \frac{\pi\Delta(T)}{2e}\tanh\big(\frac{\Delta(T)}{2k_bT}\big),
  \end{equation}
  
  \noindent derived from  BCS theory for a superconducting  energy gap of $ \Delta(T)  $ and normal
  resistance $ R_n $ of the JJ.
  
  {
    \paragraph{Critical current $ I_c $}
    Taking the limit of $ T\ra0 $ we get
  
  	 \begin{equation}\label{app:criticalCurrent}
           I_cR_n = \frac{\pi\Delta(0)}{2e}
  	 \end{equation}
       }
 
 
       \iframe{\paragraph{Josephson     Energy    from     JJ    parameters}     Subbing    in
         Eq.~\eqref{app:criticalCurrent} into Eq.~\eqref{l2-JJEnergy}:
  	
         \begin{equation}\label{key}
           \begin{aligned}
             E_J & = \frac{R_q}{R_n/N_{sq}}\frac{\Delta(0)}{2}\\
             R_n &= 18.4\,\text{k}\Omega \text{ for 100 } \times 100\,\text{nm}\ipow{2}
           \end{aligned}
         \end{equation}
  
         \noindent  with  $ R_q  =  \frac{h}{(2e)^2}  $.  The  wider  the  JJ is  in  squares,
         $ N_{sq} $, the lower the resistance of the junction.
  
  
         The calibration graph for a $ 200 \times \iunit{800}{nm}^2 $ junctions is shown below:

         \ipic{4cm}{oxidation}
	
         \red{JJ  resistance increases  by $  \sim 10\%  $  as one  goes from  room to  cryogenic
           temperatures.}\ec }
  
  \subsection{Inductance energy}
  Inductance                      energy                     derives                      from
  $ \frac{\Phi_L^2}{2L} = \frac{\Phi_0^2}{(2\pi)^22L}(\phi_\text{ext}-\phi_J)^2 = E_L(\phi_\text{ext}-\phi_J)^2 $
   
  \begin{equation}
    \begin{aligned}
      E_L & = \frac{\Phi_0^2}{(2\pi)^22L}\\
      L & \propto R_n = \iunit{1.5}{nH per 100}\times\iunit{100}{nm}^2
    \end{aligned}
  \end{equation}
   
  \subsection{Summary of energies}
  \begin{table}[h]
    \centering  {\footnotesize   \begin{tabular}{|c|c|p{6cm}|c|}  \hline\textbf{Energy}   &  &
        \textbf{Variable parameter} & \textbf{Energy ($ N_{sq}=10, N_{NbN} = 5$)}\\\hline
                                   $ E_J $ & $ \frac{R_q}{R_{\square}/N_{sq}}\frac{\Delta(0)}{2} $ & $ R_q = \frac{h}{(2e)^2} = 6.484\,\text{k}\Omega,\newline \Delta = 1.73*(k_b\times 1.3\,\text{K}) = 3.1\times10^{-23}, \newline R_\square = \iunit{18.4}{k}\Omega $ & \iunit{77.5}{GHz} \\
                                   $ E_C $ & $ \frac{(2e)^2}{2CN_{sq}} $ & $ \varepsilon = 10, d = \iunit{2}{nm}, \newline A = 100\times\iunit{100}{nm}^2, \newline C = \frac{\varepsilon\varepsilon_0A}{d} = \iunit{0.5}{fF} $ & \iunit{17.4}{GHz} \\
                                   $   E_L   $   &   $   \frac{\Phi_0^2}{(2\pi)^22LN_{NbN}}   $   &
                                                                                 $    \Phi_0    =
                                                                                               2\itimes{-15}\,\text{Wb},
                                                                                               \newline
                                                                                               L
                                                                                               =
                                                                                               \iunit{1.5}{nH}
                                                                                               $
                                                                                               per
                                                                                               NbN
                                                                                               square
                                    & \iunit{16.2}{GHz}\\\hline
                                 \end{tabular}}
                             \end{table}

\newpage