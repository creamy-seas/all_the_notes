\section{Two level system evolution\label{sec:firstLecture}}
   To represent a qubit state, we use a Bloch sphere, and say that
   
   \begin{equation}
	   \label{eqn:bloch}
	   \ket{\psi} = \begin{cases}
		   \cos\left(\frac{\theta}{2}\right)\ket{0} + e^{i\phi}\sin\left(\frac{\theta}{2}\right)\ket{1}\\
		   \alpha\ket{0}+\beta\ket{1}
	   \end{cases},
   \end{equation}
   
   \noindent from which we can express
   
   \begin{equation}
	   \left\lbrace\begin{aligned}
		   \alpha & = \cos\left(\frac{\theta}{2}\right)\\
		   \beta & = e^{i\phi}\sin\left(\frac{\theta}{2}\right)
	   \end{aligned}\right. \Rightarrow
	   \left\lbrace\begin{aligned}
		   \left|\frac{\beta}{\alpha}\right| & =\tan\left(\frac{\theta}{2}\right)\\
		   e^{i\phi} = \frac{\beta}{\sqrt{1-\alpha^2}}\Rightarrow \phi & = i\ln\left(\frac{\sqrt{1-\alpha^2}}{\beta}\right)
	   \end{aligned}\right.
   \end{equation}
   
   \noindent Now consider a system with two wells. The state is determined by the well being occupied - \iket{0} for one well and \iket{1} for the other. Energy in each well is $E_0$.
   
   \textbf{Now if we allow tunnelling} through the barrier, the Hamiltonian of the system changes, to accommodate for the possibility of state exchange
   
   \begin{equation}
	   \label{l1-decreaseBarrier}
	   \mathcal{H} = \begin{pmatrix}
	   E_0 & 0 \\ 0 & E_0
	   \end{pmatrix} \Rightarrow
	   \mathcal{H} = \begin{pmatrix}
		   E_0 & -\Delta/2\\-\Delta/2 & E_0
	   \end{pmatrix} \overrightarrow{\text{set energies to zero}} \qquad
	   \mathcal{H} = \begin{pmatrix}
	   0 & -\Delta/2\\-\Delta/2 & 0.
	   \end{pmatrix}
   \end{equation}
   
   \noindent Solving the eigenvalue equation (det($\mathcal{H}-\lambda\mathbb{I}\equiv0$)), we find
   
   \begin{equation}
	   \label{l1-solutionFirst}
	   E = \pm\Delta/2\qquad\qquad\ket{\Psi}=\frac{\ket{0}\pm\ket{1}}{\sqrt{2}},
   \end{equation}
   
   \noindent and these two superposition states are represented diagrammatically as shown in Fig.\ref{fig:newSystem}.
   
   \begin{figure}[h]
   	\begin{center}
   		\includegraphics[height=4cm]{newSystem}
	   	\caption{\small The original states have the same energy, but the superposition ones are split. \label{fig:newSystem}}
   	\end{center}
   \end{figure}
   
   \textbf{Now we add an electric field} to the quantum wells, see Fig.\ref{fig:withField}. The energy shift from the field is $e\vec{E}\vec{d}$, where $\vec{d}$ is the distance between the two wells. We have a two level system, and we define the zero energy to be halfway between them, making the Hamiltonian
   
   \begin{equation}
   \label{l1-1}
   \mathcal{H}_{\text{no tunneling}} = -\frac{\epsilon}{2}\sigma_z = \left(\begin{matrix}
   -\epsilon/2 & 0\\ 0 & \epsilon/2
   \end{matrix}\right).
   \end{equation}
   
   \begin{figure}
   	\begin{center}
   		\includegraphics[height=3cm]{withField}
   		\caption{\small Adding an electric field, will mean that one well state is at a higher energy (higher potential energy).}
   		\label{fig:withField}
   	\end{center}
   \end{figure}

   \textbf{Now tunnelling + electric field gives us }
   
   \begin{equation}
	   \label{l1-newH}
	   \begin{aligned}
	   \mathcal{H} = \left(\begin{matrix}
	   -\epsilon/2 & 0\\ 0 & \epsilon/2
	   \end{matrix}\right) + \begin{pmatrix}
	   0 & -\Delta/2\\-\Delta/2 & 0
	   \end{pmatrix} & = {-\frac{\epsilon}{2}\sigma_z-\frac{\Delta}{2}\sigma_x}\\
	   & = -\frac{\sqrt{\epsilon^2+\Delta^2}}{2}\left(\frac{\epsilon}{\sqrt{\epsilon^2+\Delta^2}}\sigma_z+\frac{\Delta}{\sqrt{\epsilon^2+\Delta^2}}\sigma_x\right)\\
	   & = -\frac{\Delta E}{2}\left(\cos\left(\theta\right)\sigma_z+\sin\left(\theta\right)\sigma_x\right)\\
	   \red{\Rightarrow \left\lbrace\begin{aligned}
	    \mathcal{H} & = \mathbf{-\frac{\Delta E}{2}\begin{pmatrix}
	   		\cos(\theta) & \sin(\theta)\\\sin(\theta) & -\cos(\theta)
	   		\end{pmatrix}}\\
   		\Delta E & = \sqrt{\epsilon^2+\Delta^2}\\
   		\tan(\theta) & = \frac{\Delta}{\epsilon}
	   \end{aligned}\right.} 
	   \end{aligned},
   \end{equation}
   
   \noindent \red{\text{ small $\theta$ = small mixing.}}. Finding eigenvalues and eigenvectors 
   
   \begin{equation}
	   E = \pm \frac{\Delta E}{2}, \qquad \ket{\psi}_0 = \begin{pmatrix}
	    \cos(\theta/2) \\ \sin(\theta/2)
	   \end{pmatrix}, \qquad \ket{\psi}_1 = \begin{pmatrix}
		    \sin(\theta/2) \\ -\cos(\theta/2)
	   \end{pmatrix},
	   \label{l1-finalEVal}
   \end{equation}

   \noindent and we note that inside the Bloch Sphere, the eigenstates are now at an angle $\theta$ relative to the initial ``north-south'' eigenstates as seen in Fig.\ref{newEigenstates}
   
   \begin{figure}[h]
   	\begin{center}
   		\includegraphics[height=4.5cm]{newEigenstates}
   		\caption{\small The drive $\Delta$, tilts the eigenstates of the system. \red{$\tan(\theta) = \frac{\Delta}{\epsilon}$}}
   		\label{newEigenstates}
   	\end{center}
   \end{figure}
   
   \iframe{\noindent So we have gone from a purely potential system with $\mathcal{H} = -\epsilon/2\sigma_z$, and a purely kinetic system with $\mathcal{H} = -\Delta/2\sigma_x$, to a mixed one as shown in Fig.\ref{l1-combined}.}
   
   \begin{figure}[h]
   	\begin{center}
   		\includegraphics[height=7cm]{together}
   		\caption{\small In the potential system, the energies of the eigenstates changed linearly with field. At the crossing point (no field), there is a degeneracy, where both wells are at the same potential. The kinetic tunnelling case we have also seen. Together they form a system whose eigenstates are split.}
   		\label{l1-combined}
   	\end{center}
   \end{figure}
 

   	\textbf{Considering the case with no bias $\mathbf{\epsilon = 0 \Rightarrow \Delta E = \Delta, \theta = \pi/2}$:}
   	
   	\[
   	 \mathcal{H} = -\frac{\Delta}{2}
   	 \begin{pmatrix}
   		0 & 1\\ 1 & 0
   	 \end{pmatrix} = -\frac{\Delta}{2} \sigma_x.
   	\]
 
   \noindent Computing the unitary evolution of the state
   
   \begin{equation}
	   \begin{aligned}
		   U & = \exp\left[\frac{-i\mathcal{H}t}{\hbar}\right]\\
		   & = \sum_k\frac{-i\Delta/2\sigma_xt}{k!}\\
		   & \red{=
			   \cos\left(\frac{\Delta}{2\hbar}t\right)\mathbb{I} + i\sin\left(\frac{\Delta}{2\hbar}t\right)\sigma_x = \begin{pmatrix}
				   \cos\left(\frac{\Delta}{2\hbar}t\right) & i\sin\left(\frac{\Delta}{2\hbar}t\right)\\
				   i\sin\left(\frac{\Delta}{2\hbar}t\right) & \cos\left(\frac{\Delta}{2\hbar}t\right)
			   \end{pmatrix}}
	   \end{aligned},
	   \label{l1-finalRot}
   \end{equation}
   
   \noindent so 
   
   \begin{equation}
	   U\ket{0} = \begin{pmatrix}
		   \cos\left(\frac{\Delta}{2\hbar}t\right)\\
		   i\sin\left(\frac{\Delta}{2\hbar}t\right),
	   \end{pmatrix},
   \end{equation}
   
   \noindent so we can prepare any state perpendicular to the x-axis.
   
  \iframe{Thus no bias + tunneling $ \equiv $ bias + resonant field in RWA. In both cases, the interaction causes the same state evolution}
   
  \newpage 
