\section{Scattering by system \cite{Astafiev2010}}
 \iframe{
 	\begin{equation}\label{key}
 				2ikV_{sc} = i\omega \phi_p\iaverage{\sigma_{-}}.
 	\end{equation}
	}
	Now we consider a system scattering waves due changing it's state:
	
	\ipic{5cm}{scattering}
	
 \begin{enumerate}
 	\item Beggining with the telegraph equations for the voltage difference at the position of the atom:
 	\begin{equation}\label{7thFeb1}
 	\left\lbrace\begin{aligned}
 	\difffrac{\Delta V}{x} & = l\difffrac{I}{t}\\
 	\Delta V & = \iabs{V_{sc}}\left[e^{i(kx - \omega t)} - e^{i(-kx - \omega t)}\right]
 	\end{aligned}\right. \Rightarrow \difffrac{\Delta V}{x} = 2ikV_{sc}
 	\end{equation}
 	
 	\item Now, the atom is situated specific point on the line, $ x = 0 $, \red{\textbf{and only at this point is the voltage discontinous.}} Thus we incoporate delta function into Eq.~\eqref{7thFeb1}
 	\begin{equation}\label{7thFeb4}
 	\frac{d\Delta V}{dx} = 2ikV_{sc}\delta(x)
 	\end{equation}
 	
 	\item This voltage is caused by a change of the linked flux (dipole) during a transition (Eq.~\eqref{dipole_2level})
 	
 	\begin{equation}\label{key}
 		\text{small change in flux } = \vartheta_{ij}(t) = MI_p\isigmaminus e^{i\omega t}.
 	\end{equation}
 	
 	\item Let's evaluated the induced voltage due to this changing of flux
 	
 	\begin{equation}\label{7thFeb5}
 		\Delta V = \dot{\Phi} = i\omega MI_p\isigmaminus e^{i\omega t}
 	\end{equation}
 	
 	\item Integrating Eq.~\eqref{7thFeb4} and subbing in Eq.~\eqref{7thFeb5} we arrive at
 	
 	\begin{equation}\label{7thFev5}
 	2ikV_{sc} = i\omega\phi_p\iaverage{\sigma_{-}}\qquad \phi_p = MI_pe^{i\omega t}
 	\end{equation}
 	
 \end{enumerate}