% -*- TeX-master: "../all_the_notes.tex" -*-

\section{Emission from system} 
Before we begin, let us clarify certain assumptions:
\begin{itemize}
\item An atom left to itself in an excited state \textbf{will never decay};
\item Decay only  occurs due to quantum  noise. The most trivial  case, is having
  two $ Z_0 $ resistors somewhere on the tranmission line:
 	
  \ipic{3cm}{noise}
 	
\item The quantum noise \textbf{\red{which  is the variation of current squared}}
  is defined by
  \begin{equation}\label{key}
    S(w) = \red{\iaverage{j^2}} = \frac{1}{Z_0}\hbar\omega
  \end{equation}
 	
  \noindent but  only a  small part  of it  will be  affecting relaxation  of our
  $ \omega_0 $ qubit.
 	
  \ipic{4cm}{noise_spectrum}
 	
\item This  mean square current, $  \iaverage{j^2} $ will interact  with the flux
  being     created     by    the     switching     state     of    the     atom,
  $ \vartheta = MI_p\isigmaminus e^{-i\omega t} $, to give an energy
	
	\begin{equation}\label{key}
          E_\text{interaction}^2 = \iaverage{j^2}\vartheta^2 = \hbar\omega_0\frac{1}{Z_0}\vartheta^2
	\end{equation}
	
      \item The frequency  associated with this energy (and  hence the transition
        rate associated with this energy)
	
	\begin{equation}\label{key}
          \Gamma = \sqrt{\frac{E_\text{interaction}^2}{\hbar^2}} = \cdots (\text{ oleg promised to explain})
	\end{equation}
      \end{itemize}

 \subsection{As derived in Oleg's papers \cite{abdumalikov2010}\cite{Astafiev2010}}
 Here it is written, that relaxation occur due to quantum noise in the system:
 \begin{equation}\label{key}
   \Gamma_{ij} = \hbar\omega_{ij}\frac{\vartheta_{ij}^2}{\hbar^2Z_0},
 \end{equation}

 \noindent which depends on
 \begin{itemize}
 \item The energy associated with the transition\hfill $ \hbar\omega_{ij} $;
 \item      The      dipole       matrix      transition      element      \hfill
   $ \vartheta_{ij} = \bra{i}U\ket{j} \equiv MI_\text{PC}\zeta_{ij}$;
 \item The impedance of the line \hfill $ Z_0 $;
 \end{itemize}

\subsection{Atom emission}
The atom will scatter waves according to
\[
  I_\text{sc}(x,t)                                                              =
  \big[i\frac{\hbar\Gamma_{21}}{\phi_{21}}\iaverage{\sigma_{21}}\big]e^{ik\iabs{x}-\omega_{21}t},
\]

\noindent and \iaverage{\sigma_{21}} is found from  the stationary state of the Master
equation $ \dot{\rho} = 0 $. The transmission coefficient is:

\[
  \begin{aligned}
    t = & \frac{\text{Input current + scattered current}}{\text{Input current}} \ge 1\\
    & = 1 + \big[i\frac{\hbar\Gamma_{21}}{\phi_{21}}\iaverage{\sigma_{21}}\big]\red{e^{ik\iabs{x}-\omega_{21}t}}/\frac{\hbar\Omega_{21}}{\phi_{21}}\quad \red{\text{ignore}}\\
    & = 1 + i\frac{\Gamma_{21}}{\Omega_{21}}\iaverage{\sigma_{21}}
  \end{aligned}
\]

\iframe{The relaxation rate is caused by quantum noise in the 1D space:
  \[
    \Gamma_{21}                                                                 =
    \hbar\omega_{21}\bigg(\frac{MI_\text{persistent}}{\hbar}\bigg)^2\frac{1}{Z}
  \]
}

\subsection{Emission by the atom\label{subsec:Scattering}}  
\iframe{The  input-output theory  shows  that  the average  field  emitted by  an
  artificial atom to an open transmission line is

\begin{equation}\label{theoEmission}
  \iexpectation{V_{\text{sc}}} =  i\sqrt{\frac{\Gamma_{ij}}{2}}\iexpectation{\sigma_{ji}},
\end{equation}}

\noindent                                                                   where
$        \iexpectation{\sigma_{ji}}=\text{Tr}\left\lbrace       \ketbra{j}{i}\rho
\right\rbrace  = \rho_{ij}  $, and  $ \Gamma_{ij}  $ is  the \iket{i}\lra\iket{j}
relaxation rate. The angular frequency of the emission is $ \omega_{ij} $.
\begin{itemize}
\item  $   V_{\text{L}}^{+}  $  resonant  input   with  the  \iket{i}\lra\iket{j}
  transition and defined by Eq.~\eqref{theoField};
\item $ V_{\text{L}}^{-} = V_{\text{L}}^{+}+V_{\text{sc}}$ the transmitted field,
  which  is  a  combination  of  the  incident  and  emitted  fields\footnote{The
    artificial  atom,   whose  dynamics,   Eq.~\eqref{rwaHamitlonianApprox},  are
    determined by the incident field, is  treated as a stand-alone quantum system
    that emits a  field $ V_{\text{sc}} $.  This resultant field is a  sum of the
    incident, $V_{L}^{+}  $, and emitted,  $ V_{\text{sc}} $, fields,  created by
    two distinct objects in the quantum system.};
\item $ V_{\text{R}}^{-} = - V_{\text{sc}}$ the reflected field, only composed of
  emission by the artificial atom.
\end{itemize}

The transmission, $ t $, and reflection, $ r $, coefficients are defined as

\begin{equation}\label{theoRefTran}
  \begin{aligned}
    t & = \frac{\iexpectation{V_{\text{L}}^{-}}}{\iexpectation{V_{\text{L}}^{+}}} = 1+\frac{i\Gamma_{ij}}{\iexpectation{V_{\text{L}}^{+}}\sqrt{2\Gamma_{ij}}}\rho_{ij};\\
    r & = \frac{\iexpectation{V_{\text{L}}^{-}}}{\iexpectation{V_{\text{L}}^{+}}}
    = 1-t;
  \end{aligned}
\end{equation}

\noindent where $\iexpectation{V} = \frac{1}{T}\int_{0}^{T}V(t)dt$ is the average
voltage over a normalisation time $ T $. Since the Rabi frequency, $ \Omega $, is
related to the drive amplitude, $ \iexpectation{V_{\text{L}}^{+}} $,

\begin{equation}
  \Omega_{ij}=\iexpectation{V_{\text{L}}^{+}}\sqrt{2\Gamma_{ij}},
\end{equation}

\noindent Equation~\eqref{theoEmission}, \eqref{theoRefTran}, reduce to

\begin{equation}\label{theoCoeff}
  \begin{aligned}
    t & = 1 + i\frac{\Gamma_{ij}}{\Omega_{ij}}\rho_{ij};\\
    r & = 1-t.
  \end{aligned}
\end{equation}

\noindent The  coefficients of Eq.~\eqref{theoCoeff} apply  to coherent emission,
when  angular   frequencies  of   the  driving   and  emitted   fields  coincide,
$ \omega^{d}_{ij} \approxeq\omega_{ij} $, and  interference between the onset and
emitted waves occur.

  \subsection{Single drive configuration\label{subsec:singleDrive}}
  When a single drive couples  two levels \iket{i}, \iket{j}, the non-interacting
  third         level        can         be        traced         out        from
  Eq.~\eqref{rawTransformedFinal}-\eqref{linLinTerm}.  The  procedure of  solving
  the   Master   equation,   and   determining~$    \rho   $,   was   done   with
  \texttt{Mathematica}. The $  \rho_{21} $, $ \rho_{31}  $ coefficients obtained,
  for   respective  \iket{1}\lra\iket{2},   \iket{1}\lra\iket{3},  drives,   upon
  substitution into Eq.~\eqref{theoCoeff} give

\begin{equation}
  r_{21}=\frac{\Gamma_{21}}{2\gamma_{21}}\frac{1+i\delta\omega_{21}/\gamma_{21}}{1+(\delta\omega_{21}/\gamma_{21})^2+\Omega_{21}^2/\Gamma_{21}\gamma_{21}}; \quad r_{31}=\frac{\Gamma_{31}}{2\gamma_{31}}\frac{1+i\delta\omega_{31}/\gamma_{31}}{1+(\delta\omega_{31}/\gamma_{31})^2+\Omega_{31}^2/\Gamma_{31}\gamma_{31}},
  \label{singleReflectance}
\end{equation}

Figure~\ref{singleDriveReflection} shows  the real  components of  $ r_{ij}  $ as
function  of the  detuning  of  the driving  field  from  the atomic  transition,
$ \delta\omega_{ij}  = \delta\omega_{ij}^{d}-\delta\omega_{ij}$.  The reflectance
peak corresponds to the case when the atom relaxes to the ground state, and emits
a photons  that is coherent~\footnote{Of  the same frequency.} with  the incident
field, but shifted by a phase of $  \pi $.\footnote{Signified by the $ i $ factor
  in Eq.~\eqref{theoCoeff}.}  Destructive interference  occurs with  the incident
field and the wave is fully reflected.

\begin{figure}[h]
  \ipic{5cm}{transmission_purely_theoretical}
  \caption{\small  \textbf{Simulations  of  elastic  scattering  of  an  incident
      microwave in an arbitrary $ \mathbf{\ket{i}, \ket{j}}, $ system}. Shown are
    the real part of the reflection  coefficient, $ \re{r_{ij}} $, evaluated with
    Eq.~\eqref{singleReflectance} for $ \Gamma_{ij}=50, \Gamma_{\phi,ij}=5 $, for
    a range  of driving powers,  $ \Omega_{ij} $. When  the incident field  is on
    resonance  with   the  atomic   transition,  $  \delta\omega_{ij}=0   $,  the
    reflectance  curve exhibits  a peak,  as  emission from  the artificial  atom
    undergo  destructive interference  with  the incident  wave.  The weaker  the
    driving power, the stronger this interference becomes.}
  \label{singleDriveReflection}
\end{figure}

The sharpness of  the central features diminishes for  larger driving amplitudes,
$ \Omega $. As  the number of photons in the transmission  line grows, the photon
emitted  by the  atom cannot  interfere with  all the  ones propagating  down the
transmission  line. This  photon  overload causes  $  t $  and $  r  $ to  become
insensitive  to atomic  transitions.   To  saturate the  peak,  one applies  weak
drives,    $     \Omega_{ij}<<\Gamma_{ij}\gamma_{ij}    $    in     which    case
Eq.~\eqref{singleReflectance} no longer depends on the driving amplitude

\begin{equation}
  \begin{aligned}
    \re{r_{21}}                                                                 =
    \frac{\Gamma_{21}}{2\gamma_{21}}\frac{1}{1+(\delta\omega_{21}/\gamma_{21})^2};
    \quad
    \re{r_{31}}=\frac{\Gamma_{31}}{2\gamma_{31}}\frac{1}{1+(\delta\omega_{31}/\gamma_{31})^2},
  \end{aligned}
  \label{singleLorentzian}
\end{equation}

\noindent     allowing    on     to    determine     the    decoherence     rates
$ \gamma_{ij}, \Gamma_{ij} $ from fittings to observed values.


\subsection{Combining the sections above}
Emission from a system is linked to the voltage operator, $ V^{+} $.
\begin{itemize}
\item The voltage operator is defined as:
 	
 	\begin{equation}\label{feb22018}
          \hat{V}^{+} = i\frac{\hbar\Gamma_1}{\phi}\sigma^{-},
 	\end{equation}
 	
 	\noindent the `-' coming from the fact  that the atom must relax in order
        for voltage to be produced. The average produced field would be
 	
 	\[
          \iaverage{\hat{V}^{+}}                                                =
          i\frac{\hbar\Gamma_1}{\phi}\iaverage{\sigma^{-}},
 	\]
 	
      \item  Now, the  power resulting  from this  voltage, which  is effectively
        noise as the atom relaxes spontaneously, can be found:
 	
 	\begin{equation}\label{feb22018:1}
          \begin{aligned}
            \iaverage{V^2(\omega)} = & \frac{1}{2\pi}\int_{-\infty}^{\infty}\iaverage{\hat{V}^{-}(0)\hat{V}^{+}(\tau)}e^{i\omega \tau}d\tau\\
            =                                                                   &
            \frac{\hbar^2\Gamma_1^{2}}{\phi^2}\frac{1}{2\pi}\int_{-\infty}^{\infty}\iaverage{\sigma_{+}(0)\sigma_{-}(\tau)}e^{i\omega
              \tau}d\tau
          \end{aligned}
 	\end{equation}
 	
      \item  \red{Using  a trick  in  Olegs  book, one  can  find  that the  term
          $ \iaverage{\sigma_{+}(0)\sigma_{-}(\tau)} $ can be decomposed as:
          \begin{itemize}
          \item \textbf{Total sum is} \[ \frac{1+{\isigmaz}}{2}. \]
          \item          \textbf{The          coherent          part          is}
            \[ \iaverage{\sigma_{+}}\iaverage{\sigma_{-}}. \]
          \item  \textbf{Therefore the  incoherent  part must  be the  difference
              between                          the                          two}:
            \[                      \frac{1+{\isigmaz}}{2}                      -
              \iaverage{\sigma_{+}}\iaverage{\sigma_{-}}. \]
          \end{itemize}
 	}
 	
      \item  Finding  the total  emitted  power,  by  integrating over  the  full
        frequency range  \red{and assuming  that we  are dealing  with stationary
          states (ss) that would form in the system when averaging:}
 	
 	\begin{equation}\label{feb220183}
          \begin{aligned}
            \text{Power}_\text{total} &= \frac{1}{Z}\int 	\iaverage{V^2(\omega)}  d\phi\\
            & = \frac{1}{Z}\int \frac{\hbar^2\Gamma_1^{2}}{\phi^2}\frac{1}{2\pi}\int_{-\infty}^{\infty} \frac{1+{\isigmaz}_{ss}}{2} e^{i\omega \tau}d\tau   d\phi\\
            & = \frac{\hbar^2\Gamma_1^2}{Z\phi^2} \frac{1+{\isigmaz}_{ss}}{2} \int\frac{1}{2\pi}\int e^{i\omega\tau}d\tau d\phi\\
            & \text{Using the fact that the integral over the delta function is just 0 and } \Gamma_1 = \frac{\hbar\omega\phi^2Z}{\hbar^2}\\
            & = \hbar\omega\Gamma_1 \frac{1+{\isigmaz}_{ss}}{2}
          \end{aligned}
 	\end{equation}
 	
 	\noindent  Now,  because we  are  driving  continously, decoherence  will
        result in our rotation of the state  from \iket{0} to \iket{1} to form an
        intermediate value when $ \isigmaz = 0 $ and so the maximum emitted power
        from the atom will be:
 	
 	\iframe{\begin{equation}\label{key}      \text{Power}_\text{total}      =
            \frac{\hbar\omega\Gamma_1}{2}
          \end{equation}}
 	
 	
      \item Now, redoing the same, but only considering the coherent contribution
        i.e.     instead    of     $     \frac{1+{\isigmaz}_{ss}}{2}    $     use
        $ \iaverage{\sigma_{+}}\iaverage{\sigma_{-}} $ we round up at:
 	
 	\begin{equation}\label{maximalCoherentEmission}
          \begin{aligned}
            \text{Power}_\text{coherent} &= \hbar\omega\Gamma_1 \iaverage{\sigma_{+}}_{ss} \iaverage{\sigma_{-}}_{ss} = \text{ upon subbing in the obtained expectationv values }\\
            &                                                                   =
            \hbar\omega\Gamma_1\bigg(\frac{2\Gamma_1\Omega}{2\Gamma_1^2+\Omega^2}\bigg)\\&\Rightarrow
            \text{max value } = \frac{\hbar\omega\Gamma_1}{8}
          \end{aligned}
 	\end{equation}
 	
 	\begin{itemize}
        \item \red{You can  measure this power with the SPA  i.e. you measure 1/8
            of a single photon power;}
        \item  Then, you  can tune  your  VNA power  to  match this  $ 8\times  $
          coherent  signal power,  and be  supplying  exactly one  photon to  the
          system. Then there will be no  leakages, as the photon will be absorbed
          by the system, with no leftover for leaking;
 	\end{itemize}
      \item The total number of photons in the system can be found from
 	\begin{equation}\label{key}
          N = \frac{\Omega}{\Gamma_1}
 	\end{equation}
      \end{itemize}
 
      \newpage
 
      \subsection{Coherent and incoherent emission}
      \begin{enumerate}
      \item \textbf{Coherent  field}, means a  fixed phase relation with  the two
        input   fields.    For   example,    supplying   $   e^{i\omega_1t}    $   and
        $  e^{i\omega_2t}  \iright  e^{i(\omega_1+\omega_2)t +  \phi}$  where  $  \phi  $ is  a  fixed
        value. This allow further entanglement procedures;
      \item  \textbf{Emission by  qubit}  can be  characterised  as coherent  and
        incoherent emission.  One needs  to work  with \iaverage{\sigma_i}  values and
        look at the corresponding bloch sphere.
 	
 	\begin{figure}[h]
          \ifigure{7cm}{sphere}
 	\end{figure}
 	
 	Recalling that
 	
 	\red{{\large     \begin{equation}\label{expectationPauli}    \rho_{00}     =
              \frac{\iaverage{\sigma_z}+1}{2};\quad
              \rho_{01}=\frac{\iaverage{\sigma_x}-i\iaverage{\sigma_y}}{2}                   =
              \iaverage{\sigma_{+}};\quad\rho_{10}=\frac{\iaverage{\sigma_x}+i\iaverage{\sigma_y}}{2} =
              \iaverage{\sigma_{-}};
            \end{equation}
            \begin{equation}\label{expectationPauli2}
              \iaverage{\sigma_x}=\rho_{01}+\rho_{10};\quad\iaverage{\sigma_y} = i\rho_{01}-i\rho_{10};\quad\iaverage{\sigma_z}=\rho_{00}-\rho_{11}
            \end{equation}}}
 	
 	\noindent The more a  state is on the equator, the  more coherence it has
        since a superposition is formed. Recall that, so that when the ``arrrow''
        projects  fully  onto  the equator,  it  means  that  the  atom is  in  a
        superposed                           state                           e.g.
        $ \frac{\iket{0}+\iket{1}}{\sqrt{2}} \ra \imatrix{1/2}{1/2}{1/2}{1/2} \ra
        \isigma_z = 0, \isigma_x = 1$ (or some other rotation around y).
 	
 	\begin{center}
          \textbf{Coherent     emissions     $    \propto     \isigmax$\newline     Incoherent
            $ \propto\isigmaz $. \large}
 	\end{center}
 	\red{Both emissions  will occur  at the  frequency of the  qubit $  \omega_0 $
          (i.e.  energy level  separaion),  but emission  from  \isigmax will  be
          sharp, while incoherent  emission from \isigmaz will be  broad as shown
          below}
 	
 	\ipic{5cm}{emission}
      \item Now draw the parrallels between:
 	
 	\begin{itemize}
        \item The field in a resonator \hfill \ia\ and \iadagger;
        \item                        Qubit\hfill                       excitation
          $  \sigma_{+} =  \ketbra{1}{0}  = \imatrix{0}{0}{1}{0}  = (\sigma_x-i\sigma_y)/2$  and
          relaxation
          $ \sigma_{-} = \ketbra{0}{1} = \imatrix{0}{1}{0}{0} = (\sigma_x+i\sigma_y)/2$
 		
 	\end{itemize}
 	in the following way
 	
 	\begin{center}
          \begin{tabular}{|c|c|c|}
            \hline 
            \textbf{Coherent field} & $ \big(\ia + \iadagger\big) $ & $ \isigmax $ \\ 
            \hline 
            \textbf{Photon number} & $ \ia\iadagger $& \isigmaz \\ 
            \hline 
          \end{tabular} 
 	\end{center}

      \end{enumerate}
      \newpage
