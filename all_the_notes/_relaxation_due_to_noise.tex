\section{Relaxation due to noise}
%A system coupled to resonators in thermal equilibrium energy exchange depends on the temperature of these resonators. When $ kT>>\hbar\omega $ photons will be exchanged between the medium and system. When $ kT<<\hbar\omega $ the system will remain indefinitely in the ground state
 \subsection{Relaxation from the noise spectrum}

\ipic{4cm}{noise1}

\noindent In the systems above noise may come from current or voltage fluctuations. We shall analyse the latter case, in which the voltage fluctuation acts as an effective driving field

\begin{equation}\label{noise1}
\mathcal{H} = -\frac{\hbar\omega}{2}\sigma_z+\frac{\mu\delta V(\omega)}{2}\sigma_x\cos(\omega t);\quad\delta V(\omega) = \frac{1}{T}\int_{-T/2}^{T/2}\delta V(t e^{-i\omega t})dt,
\end{equation}

\noindent and after applying the RWA

\begin{equation}\label{noise2}
\mathcal{H} = \frac{\mu\delta V(\omega)}{2}\sigma_x,
\end{equation}

\noindent the evolution of a system state

\begin{equation}\label{noise3}
U(T)\iket{0} = \iket{0} - \frac{\mu\delta V(\omega)}{2}\frac{T}{\hbar}\iket{1}\iright P_1 = \frac{\mu^2T^2}{4\hbar^2}\iaverage{\delta V^2},
\end{equation}

\noindent and incorporating the spectral noise density $ S(\Omega) $ which we integrate over in the frequency range $ \Delta\omega = 2\pi/T $ (longer acquisition time means narrower spectrum of noise since we average more and more, and only low frequency remains will remain in the system.)

\begin{equation}\label{noise4}
\ialigned{\iaverage{\delta V^2}&=S(\omega)\frac{2\pi}{T}\\
	P_1 &= \frac{\mu^2T^2}{4\hbar^2}\iaverage{\delta V^2}}\Rightarrow P_1 = \frac{\pi\mu^2}{2\hbar^2}S(\omega)T \rightarrow \text{ rate of excitation } \approx \Gamma = \frac{\pi\mu^2}{2\hbar^2}S(\omega).
\end{equation}

\begin{figure}[h]
	\ifigure{4cm}{noise2}
\end{figure}

\newpage