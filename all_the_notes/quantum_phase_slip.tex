\newpage

\section{Quantum phase slip \cite{mooij2005} \label{sec:qps}}

\subsection{Literature}
 \iframe{
 	\begin{itemize}
 		\item \cite{degraf2018} \textbf{Charge control of blockade of Cooper pair tunnelling in highly disordered TiN nanowires in an inductive environment}
 		\begin{enumerate}
 			\item Fabricate several thin superconducting bridges with meanders;
 			\item Measure the DC voltage, $ V $, acorss then as a function of the applied gate voltage, $ V_c $, from underneath the chip;
 			\item Some samples have big oscillations in $ V $ with gate voltage - Single Electron Transistor regime, where nanowire is compoased of grains, which feel a Coulomb Blockade;
 			\item Quantum Phase Slip possible in other samples, where there are no grains in the wire.
 		\end{enumerate}
 	\end{itemize}
 }

Initially quantum phase slips were a result of thermal activation in thin wires.

\subsection{Cooper box duality}
 The system is dual to the cooper pair box system, where the charging energy creates a parabollic dependence, and the Josephson coupling energy $ E_J $ mixes states lifting degeneracy as in Sec~\ref{sec:cooper_pair_box}:
 
 \begin{equation}\label{key}
 	\mathcal{H} = E_c(\hat{n} - n_g)^2 - \frac{E_J}{2}\left[\ketbra{n+1}{n} + \ketbra{n}{n+1}\right].
 \end{equation}
 
 \noindent In the qunatum phase slip, we do the same thing, but now with inductor energy, $ E_L = \frac{\Phi_0^2}{2L} $, and couping, $ E_s $, due to the tunneling of fluxes (we do not specify yet the exact derivation of this energy):
 
 \begin{equation}\label{key}
 	\mathcal{H} = E_L(\hat{n} - f)^2 - \frac{E_s}{2}\left[\ketbra{n+1}{n} + \ketbra{n}{n+1}\right],
 \end{equation}
 
 \noindent with $ f = \Phi/\Phi_0 $.

 \iframe{\noindent The discussion them stems off into how the Cooper pair tunneling is dual to the phase slip, how one gets similar steps in voltage current characterisitics. There is a lot of math, which I don't want to get into.}
 
 

\newpage