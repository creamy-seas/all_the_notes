\newpage
\section{Simulating circuits \label{sec:simulating_circuits}}
 Oleg went through how one would simulate a circuit, such as the one below (in fact this is the setup for the inverse Shapiro step):
 
  \ipic{5cm}{circuit_inverse_shapiro}
 
 \noindent Constraints in this system are that we solve the current for
 \begin{itemize}
 	\item Work with $ n $ \hfill number of CP of charge $ \mathbf{2e} $;
 	\item $ V $ \hfill the DC voltage that we fix;
 	\item $f$ \hfill is the frequency of the rf-bias.
 	\item \red{Ignore that large capacitance $ C $} - it's so large that it has no effect on the whole system.
 \end{itemize}
 
 \noindent Next we write out a system of simultaneous equations that Oleg solves numerically in \verb|Mathematica| for $ n(t) $.
 
 \begin{equation}\label{key}
 	\left\lbrace\begin{aligned}
 		V & = \red{2eR\dot{n}} + \blue{\frac{2e}{C_a}n_2}\\
 		V & = \red{L\frac{d}{dt}(\dot{n}-\dot{n}_2)} + \blue{\frac{2e}{C_b}(n_3 - n_4)}  \green{- V_{rf}\cos(2\pi ft)}\\
		V & = \red{L\frac{d}{dt}(\dot{n}-\dot{n}_2)} + \blue{\frac{2e}{C_c}n_3}
 	\end{aligned}\right.
 \end{equation}
 \begin{itemize}
 		\item \red{Charge passing through resistor $ R $ and inductor, $ V = L\dot{I} $};
 		\item \blue{Potential difference across capacitors due to onset charge};
 		\item \green{RF-voltage source in action}.
 \end{itemize}
 
 \noindent What we find if oscillations in this charge number $ n_1(t) $. Evaluating $ \iaverage{I(V)} = \int_{t_1}^{t_2}2e n(t)dt $ for different bias voltages $ V $, we can plot a $ V \text{ vs } \iaverage{I(V)} $ graph, which will show Shapiro steps.
 
 \ipic{4cm}{inverseShapiro_oscillations}.