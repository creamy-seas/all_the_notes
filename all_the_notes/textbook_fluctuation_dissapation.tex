\section{Fluctuation dissapation theory from Brian Cowans Notes}\label{sec:fluctuationDissapation}
% \subsection{Summary}
% 	\begin{itemize}
% 		\item How correlation functions of voltage and current can be integrated to give the resistances in a given circuit:
% 		  	\[
% 			{R} = \frac{1}{2kT}\int\iaverage{V(0)V(t)}dt\qquad\frac{1}{R}=\frac{1}{2kT}\int\iaverage{I(0)I(t)}dt.
% 		\]
% 		\item When a system feels an effect $ B(\tau) $, it responds with $ M(t) $ according to the dynamical susceptibility $ X $ 
% 		  \ipic{5cm}{susceptibility}
% 		
% 		\iframe{\begin{equation}\label{eqn:response_2}
% 			M(t) = \int_{-\infty}^{t}X(t - \tau)B(\tau)d\tau.
% 			\end{equation}}
% 		\item For a sinusidal excitation $ B(t) = be^{-i\omega t}  $ the response is
% 		\iframe{
% 			\begin{equation}\label{eqn:response_3}
% 			\ialigned{M(t) &= be^{-i\omega t}\chi(\omega)\\ \chi(\omega) &= \int_{-\infty}^{\infty}X(\tau)e^{i\omega t}dt}.
% 			\end{equation}
% 		}
% 		
%		The real and imaginary parts of dynamical susceptibility are even and add respectively.
% 		
% 		\ipic{5cm}{susceptibility_1}
% 		\item Working in frequency domain
% 		   \iframe{
% 			Given a superposition of forces (expressed by $ b(\omega) $ in the frequency range)
% 			\begin{equation}\label{eqn:freq_2}
% 			B(t) = \frac{1}{2\pi}\int_{-\infty}^{\infty}b(\omega)e^{-i\omega t}d\omega,
% 			\end{equation}
% 			\noindent the response can be found by evaluating the response at frequency:
% 			\begin{equation}\label{eqn:freq_3}
% 			m(\omega) = \chi(\omega)b(\omega),
% 			\end{equation}
% 			\noindent and `summing' them up, which is equivalent to taking the fourier transform
% 			\begin{equation}\label{eqn:freq_4}
% 			M(t) = \frac{1}{2\pi}\int_{-\infty}^{\infty}m(\omega)e^{-i\omega t}d\omega.
% 			\end{equation}
% 		}
% 	\end{itemize}

 First and foremost, a word must be stated about the outstanding quality of the book. Chapter 5 describes correlation functions with ease and is a very recommended read.
 \subsection{Correltion function}
  The correlation function 
  \begin{equation}\label{eqn:corr1}
  	g(\tau) = \iaverage{X(0)X(\tau)} - \iaverage{X^2},
  \end{equation}
  
  \noindent measures how fast variations in a system will go away. Effectively what is happening is we are averaging the fluctuation $ X(t) $ over time from its equilibrium value $ \iaverage{X} $. Of course, a decay should slowly decay (on average, and hence that is why we take the expectation value $ \iaverage{X(t)} $). 
  
  \ipic{5cm}{g1_decay}
  
  \red{Because fluctuations are symmetrical about the equilibrium value, we need to multiiply the fluctuation by its initial value $ X(0) $ to avoid averaging out to 0. \textit{This is a very clever trick indeed.}}
  
 \subsection{Fluctuation dissapation theory}\label{subsec:fluctuationDissapation}
  This is a very general theory, applied here for current and voltage respectively. The idea is to write out the dynamics of one variable, and equate the energy found to the energy available. This contrains the value of some of the parameters.
  
  \begin{enumerate}
  	\item Write out the equation of motion:
  	\begin{equation}\label{eqn:corr_2}
  		L\frac{dI}{dt} + RI(t) = V(t) \quad (\text{Kirchoffs 2nd})\qquad\qquad C\frac{dV}{dt} + \frac{1}{R}V(t) = I(t) \quad \text{(Kirchoffs 1st)}.
  	\end{equation}
  	\item Propose a trial solution (this is called the integrating factor method, and I covered on Wikipedia):
  	\begin{equation}\label{eqn:corr_3}
  		I(t) = I(0)e^{-tR/L} + \frac{1}{L}\int_0^t e^{({u-t})\frac{R}{L}}V(u)du\qquad\qquad V(t) = V(0)e^{-t/CR} + \frac{1}{C}\int_0^t e^{({u-t})/{CR}{}}I(u)du
  	\end{equation}
  	\item Assume that the correlation of the random fluctuation is very short:
  	\begin{equation}\label{eqn:corr_4}
  		\iaverage{I(t_1)I(t_2)} = I^2\delta(t_1-t_2)\qquad\qquad \iaverage{V(t_1)V(t_2)} = V^2\delta(t_1-t_2)
  	\end{equation}
  	\item Compute the mean power depisited from Eq.~\eqref{eqn:corr_3} after a long time (so the first exponentially decaying terms go away):
  	\begin{equation}\label{key}
  	\begin{aligned}
  		\iaverage{I^2} & =\red{ \frac{1}{L}2e^{-2Rt/L}\int_0^{t}e^{uR/L} \iaverage{I(0)V(u)}du} + \blue{\frac{e^{-2Rt/L}}{L^2}\int_0^tdu\int_0^tdw\ e^{(u+v)R/L}\iaverage{V(u)V(w)}}\\
		\iaverage{V^2} & =\red{ \frac{1}{C}2e^{-2t/RC}\int_0^{t}e^{u/RC} \iaverage{V(0)I(u)}du} + \blue{\frac{e^{-2t/RC}}{C^2}\int_0^tdu\int_0^tdw\ e^{(u+v)/RC}\iaverage{I(u)I(w)}}.\\
  	\end{aligned}
  	\end{equation}
  	\noindent The \red{red} terms dissapear, because there is no correlation between current and voltage
  	\[
  		\iaverage{V(t_1)I(t_2)} = 0,
  	\]
  	\noindent while for the \blue{blue} terms, we need to use Eq.~\eqref{eqn:corr_4} to get cancelation of one of the integrals:
  	\begin{equation}\label{eqn:corr_5}
  	\begin{aligned}
  		\iaverage{I^2}& = {V^2\frac{e^{-2Rt/L}}{L^2}\int_0^tdu e^{2uR/L}} = V^2\frac{1}{2RL}(1-e^{-2Rt/L}) \iratext{t $\ra \infty$} \frac{V^2}{2RL}\\
	\iaverage{V^2} & = {I^2\frac{e^{-2t/RC}}{C^2}\int_0^tdu e^{2u/RC}} = I^2\frac{R}{2C}(1-e^{-2t/RC})\iratext{t $\ra \infty$} = \frac{I^2R}{2C}\\
  	\end{aligned}
  	\end{equation}
  	
  	\item In thermal equilibrium, a system in an environment at temperature $ T $, will have $ \frac{1}{2}k_bT $ of energy associated with each degree of freedom in the system. For circuits there is one degree of freedom (curret or voltage determines the value of the other), and so
  	
  	\begin{equation}\label{key}
  		\frac{1}{2}k_bT=\frac{1}{2}L\iaverage{I}^2 \quad(\text{inductor})\qquad\qquad \frac{1}{2}k_bT = \frac{1}{2}C\iaverage{V}^2,
  	\end{equation}
  	
  	\noindent Combined with Eq.~\eqref{eqn:corr_5} this gives
  	\begin{equation}\label{eqn:corr_6}
  	\frac{V^2}{2R} = kT\qquad \frac{I^2R}{2} = kT.
  	\end{equation}
  	
  	\item Combining Eq.~\eqref{eqn:corr_6} with Eq.~\eqref{eqn:corr_4} we wind up at the result fluctuation disspaation result:
  	\iframe{
  	\[
  		{R} = \frac{1}{2kT}\int\iaverage{V(0)V(t)}dt\qquad\frac{1}{R}=\frac{1}{2kT}\int\iaverage{I(0)I(t)}dt.
  	\]
  	\textbf{Expression of dissapation through the area under the correlations.}
  }
  \end{enumerate}  
 
 \newpage\subsection{Responses}
  \iframe{
  	In the following sections we consider the dynamic susceptibility $ \red{X} $, that transforms an input $ B $ to an output $ M $.
	}
  The response of a system, $ M$, to an external force $ B $ we treat by assuming that the system responds to the history $ B(\tau) $ via the \textbf{dynamical susceptibility} function
  \begin{equation}\label{eqn:response_1}
  	X(t-\tau).
  \end{equation}
  
  \noindent We write out the history of the forces, and use the mapping Eq.~\eqref{eqn:response_1}, to find how each of these forces manifests itself at time $ t $. 
  
  \ipicCaption{5cm}{susceptibility}{We break up the force in time, and mapping their effect using the susceptibility function, we can evaluate the response of the system at a later time.}
 
  Two assumptions are made:
  \begin{itemize}
 	\item Linearity - the effect of the different forces $ X(t-t_i)B(t_i) $ is linearly summed up to find the resultant;
 	\item Causality implies that $ t_i < t $, since the force cannot preceed its own effet.
  \end{itemize}

  \noindent to write the formula:
  
  \iframe{\begin{equation}\label{eqn:response_2}
  	M(t) = \int_{-\infty}^{t}X(t - \tau)B(\tau)d\tau.
  \end{equation}}
 
 \subsubsection{Sinusoidal Excitation}
  An excitation of the form:
  \begin{equation}\label{key}
  	B(t) = b\cos(\omega t) - ib\sin(\omega t) = be^{-i\omega t},
  \end{equation}
  
  \noindent results in the response (where we will remmeber manually that $ X(t) = 0 $ if $ t<0 $, so the limits are change from Eq.~\eqref{eqn:response_2})
  
  \begin{equation}
  	\begin{aligned}
  	M(t) &= b\int_{-\infty}^{+\infty}X(t - \tau)e^{-i\omega \tau}d\tau\quad\text{chang of var}&\\
  	& = be^{-i\omega t}\int_{-\infty}^{\infty}X(\tau)e^{ii\omega\tau}d\tau&\\
  	\end{aligned}
  \end{equation}
  
  \noindent or more compactly
  \iframe{
  	\begin{equation}\label{eqn:response_3}
  		\ialigned{M(t) &= be^{-i\omega t}\chi(\omega)\\ \chi(\omega) &= \int_{-\infty}^{\infty}X(\tau)e^{i\omega t}dt}.
  	\end{equation}
	}
	
	\noindent Because response is linear, if we split up the frequency domain susceptibility into real and imaginary components
	
	\begin{equation}\label{key}
		\chi(\omega) = \chi'(\omega) + i\chi''(\omega),
	\end{equation}
	
	\noindent then the response to
	
	\begin{equation}\label{eqn:response_4}
	  b\cos(\omega t) = \Re[be^{-i\omega t}]  =\Re[M(t)] = b\big[\red{\cos(\omega t)\chi'(\omega)} + \blue{\sin(\omega t)\chi''(\omega)}\big],
	\end{equation}
	
	\noindent with an \red{in-phase} and \blue{quadrature} components.
	
	Moreso, the unbound limits of Eq.~\eqref{eqn:response_3} mean that 
	
	\begin{equation}\label{key}
		\begin{aligned}
			\chi(\omega) & = \chi^{*}(-\omega)\\
			\chi'(\omega) = \chi'(-\omega)\\
			\chi''(\omega) = -\chi''(-\omega)\\
		\end{aligned}
	\end{equation}
	
	\noindent meaning that the real and imaginary parts of dynamical susceptibility are even and add respectively.
	
	\ipic{5cm}{susceptibility_1}
	
  \newpage\subsubsection{Frequency representation}
   \iframe{For this section we shall use definition of the FT:
   		\[
   			\ialigned{X(t) & = \frac{1}{2\pi}\int X(\omega)e^{-i\omega t}d\omega\\
   			X(\omega) & = \int X(t)e^{i\omega t}dt}
   		\]
   }
   Eq.~\eqref{eqn:response_3} describes the response to a single sinusoidal exciation. Now let us consider that the exciation is a superposition of such sinusoids :
   
   \begin{equation}\label{key}
   	be^{-i\omega t} \iRa B(t) = \frac{1}{2\pi}\int_{-\infty}^{\infty}b(\omega)e^{-i\omega t}d\omega.
   \end{equation}
   
   Defining $ m(\omega)e^{i\omega t} $ as the response to one of these sinusoids (in Eq.~\eqref{eqn:response_3}) %(\red{we redefined the $ \pm $ in the exponentials for consistency})
   \begin{equation}\label{eqn:freq_1}
   \begin{aligned}
	   {m(\omega)e^{-i\omega t} = b(\omega)e^{-i\omega t}\chi(\omega)} \iratext{integrating over all $ \omega $ + normalising} &\\\\ \
	   	   M(t) = \frac{1}{2\pi}\int_{-\infty}^{\infty}m(\omega)e^{-i\omega t}d\omega &= \int_{-\infty}^{\infty}\chi(\omega)b(\omega)e^{-i\omega t}d\omega.
   \end{aligned}
   \end{equation}
   
   \noindent Fully this will all read
   
   \red{\iframe{
   		Given a superposition of forces (expressed by $ b(\omega) $ in the frequency range)
   		\begin{equation}\label{eqn:freq_2}
 			B(t) = \frac{1}{2\pi}\int_{-\infty}^{\infty}b(\omega)e^{-i\omega t}d\omega,
   		\end{equation}
   		\noindent the response can be found by evaluating the response at frequency:
   		\begin{equation}\label{eqn:freq_3}
   			m(\omega) = \chi(\omega)b(\omega),
   		\end{equation}
   		\noindent and `summing' them up, which is equivalent to taking the fourier transform
   		\begin{equation}\label{eqn:freq_4}
   			M(t) = \frac{1}{2\pi}\int_{-\infty}^{\infty}m(\omega)e^{-i\omega t}d\omega.
   		\end{equation}
	}}
 
 \newpage \subsubsection{Step excitation}
  A step excitation is removed from the system at time $ t = 0 $
  
  \begin{equation}\label{key}
  B(\tau) = b \text{ only for } \tau < 0. 
  \end{equation}
  
  \noindent Evaluating Eq.~\eqref{eqn:response_2}
  
  \iframe{\begin{equation}\label{eqn:step_1}
  	\begin{aligned}
	  	M_\text{step}(t)  \equiv& = b\int_{-\infty}^{0}X(t-\tau)d\tau\\
	  	& = b\red{\int_{t}^{\infty} X(\tau)d\tau} = b \red{\varPhi(t)} \\
	  	& \red{\Phi(t) \text{ - resonse of system to step function}}
%	  	& = b\varPhi(t) \qquad\qquad \varPhi(t) = \int_{t}^{\infty} X(\tau)d\tau.
  	\end{aligned}
  \end{equation}}
 \subsubsection{Delta exciation}
  A delta excitation is only switched on at a certain time. 
  \begin{equation}\label{key}
  	B(\tau) = b\delta(\tau)
  \end{equation}
  \noindent The selection mechanism of this function results in
  
  \iframe{\begin{equation}\label{eqn:delta}
 	 \begin{aligned}
 	 M_\delta(t) & = b\blue{X(t)} \equiv -b\frac{d}{dt}\red{\Phi(t)}\\
 	 & \blue{X(t) \text{ - resonse of system to delta function}} %\iratext{Eq.~\eqref{eqn:step_1}} M_\delta(t) = -b\frac{d}{dt}M_\text{step}(t) \red{= - \frac{d}{dt}}
	\end{aligned}
 \end{equation}}
 
 \subsection{Summary}
 \begin{table}[h]
 	\centering
 	\caption{Refference table for responses\label{tab:responses}}
 	\begin{tabular}{|c|c|c|}
 		\hline\textbf{Excitation} & \textbf{Response} $ M(t) $ & \textbf{Special Response for b = 1}\\\hline
 		$ b\cos(\omega \tau) $ & $ b\big[{\cos(\omega t)\chi'(\omega)} + {\sin(\omega t)\chi''(\omega)}\big] $ & \\
 		$ b\theta_-(\tau) $ & $ b\int_{t}^{\infty}X(\tau)d\tau $ & $ \Phi(t) $\\
 		$ b\delta(\tau) $ & $ bX(t) $ & $ X(t) $\\\hline
 	\end{tabular}
 \end{table}
 
  \red{The delta function is minus the derivative of the step down function
   \[
   	\delta(\tau) = -\frac{d}{dt}\theta_-(\tau).
   \]
  
    \textbf{Because our system is linear, any relation between the excitation will mirror itself in the response}
 
 	 \begin{equation}\label{eqn:delta_2}
 		X(t) = -\frac{d}{dt}\Phi(t).
 	 \end{equation}}
 
\subsection{Energy}
 If we treat $ B(t) $ as a force and $ M $ as a displacement, then we can in general write that
 \begin{equation}\label{eqn:energy_1}
 B = \frac{\partial\text{Energy}}{\partial M},
 \end{equation}	
 
 \noindent and the power dissapation is then:
 
 \begin{equation}\label{key}
 P = \iaverage{\ipartial{\text{Energy}}{t}} = \iaverage{B\frac{\partial M}{\partial t}}.
 \end{equation}
 
 \noindent Using an excitation of the from for the response function $ M(t) $ we get
 
 \begin{equation}\label{key}
 \ialigned{
 	B & = b\cos(\omega t)\\
 	M(t) & = b\big[{\cos(\omega t)\chi'(\omega)} + {\sin(\omega t)\chi''(\omega)}\big]
 }
 \iRa P = \omega b^2\big[-\chi'(\omega)\iaverage{\sin(2\omega t)/2} + \chi''(\omega)\iaverage{\cos^2(\omega t)}\big]
 \end{equation}
 
 \noindent which evalutes to ($ \iaverage{\sin(2\omega t)/2} = 0 $ and $ \iaverage{\cos^2(\omega t)} = 1/2 $)
 
 
 \iframe{\begin{equation}\label{eqn:energy_2}
 	P = \frac{1}{2}b^2\omega\red{\chi''(\omega)}.
 	\end{equation}
 	\textbf{\red{Disspation that occurs is proportional to imaginary part of the susceptibility, $ \chi''(\omega) $.}}
}
 Using the definitions for working out the units of different quantities:
 \begin{equation}\label{key}
 	\ialigned{
 		[M] &= [\chi][B] &&\text{from Eq.\eqref{eqn:response_3}}\\
 		[B] & = [\text{energy}]/[M] &&\text{from Eq.~\eqref{eqn:energy_1}}
 	} \ira [\chi] = [M^2]/[\text{energy}].
 \end{equation}
 \begin{equation}\label{eqn:unit1}
 \ialigned{
 	[M] &= [X][B][\text{time}] &&\text{from Eq.\eqref{eqn:response_2}}\\
 	[B] & = [\text{energy}]/[M] &&\text{from Eq.~\eqref{eqn:energy_1}}
 } \ira [X] = [M^2]/[\text{energy}][\text{time}], [\Phi] = [M^2]/[\text{energy}].
 \end{equation}
 
 
\begin{center}
	 		\begin{tabular}{|c|c|c|}
 			\hline 
 			$ [\chi] $ & $ [X] $ & $ [\Phi] $\\
 			$ [M^2]/[\text{energy}] $&$ [M^2]/[\text{energy}][\text{time}] $ & $ [M^2]/[\text{energy}] $\\\hline
 		\end{tabular}
\end{center}
 
 \newpage \subsection{Onsager's hypothesis}
%  \textbf{Let us derive and expression for the frequency domain susceptibility $ \chi(\omega) $ from the observed correlation function of the systems response, be it voltage, current, etc}.
    \iframe{\textbf{Let us make an assumption, (that Onsager made), that after a step-like distrubrance, a system, $ M_\text{step}(t) $, regresses to its equilibrium state in the same way as it would under a random fluctuation.
	%In other words, we can study the state of Xnon-equilibrium system with an equilibrium approach. Which is understandable - at the microscopic level, there is no concept of an equilibirum state
	}}
    \begin{equation}\label{key}
    	\left\lbrace\begin{aligned}
    		M_\text{step}(t) & = \beta\iaverage{M(0)M(t)} && \text{according to Onsager, regression same as equilibium one}\\
	    	M_\text{step}(t) & = \beta\Phi(t) && \text{when we deal with step excitation as in Table.~\ref{tab:responses}}\\
	    	X(t) & = -\frac{d}{dt}\Phi(t) && \text{definition of dynamical susceptibility}
    	\end{aligned}\right.,
    \end{equation}
    
    \noindent where $ \beta $ is a constant, which for classical mechanics, and quantum treatment at high temperatures, reduces to $ \frac{1}{kT} $.
    
    The Fourier Transform (only need to do it after the reponse at t=0 has occured)
    
    \begin{equation}\label{eqn:susceptibility_from_response_0}
    	\begin{aligned}
	    	\chi(\omega) & = \int X(t)e^{i\omega t} dt\\
	    	& = -\frac{1}{kT}\int_0^\infty\iaverage{M(0)\dot{M}(t)}e^{i\omega t}dt\\
	    	& = \left.\iaverage{M(0)M(t)}e^{i\omega t} \right|_0^\infty - - \frac{1}{kT}i\omega\int_{0}^\infty\iaverage{M(0)M(t)}e^{i\omega t}dt\\
	    	& = \chi_0 + \frac{i\omega}{kT}\int_{0}^\infty\iaverage{M(0)M(t)}e^{i\omega t}dt\\
    	\end{aligned}
%    	 -\frac{1}{kT}
    \end{equation}
  
  \iframe{For a step like disturbance, the susceptibility function in the frequency domain can be evaluated using
  	\begin{equation}\label{eqn:susceptibility_from_response}
  		\chi(\omega) = \chi_0 + \frac{i\omega}{kT}\int_{0}^\infty\iaverage{M(0)M(t)}e^{i\omega t}dt
  	\end{equation}
	}
  
%  \noindent Converting to the usage of $ \Phi(t) $ instead of $ M_\text{step}(t) $, we need to add a normalisation factor $ \beta $ to keep the units from Eq.~\eqref{eqn:unit1}
%  
%  \begin{equation}\label{key}
%  \ialigned{
%  	\Phi(t) &= \beta\iaverage{M(0)M(t)}\\
%  	\Phi(t) &= -\frac{dX(t)}{dt}
%  } \ira X(t) = -\frac{1}{kT}\iaverage{M(0)\dot{M}(t)}
%  %\chi(\omega) = \int_{-\infty}^{+\infty}X(\tau)e^{i\omega t}dt = 
%  \end{equation} 
%  
%  \noindent where, without going into too much depth, $ \beta = \frac{1}{kT} $ from equipartition (derived in the last chapter of the book). The dynamical susceptibility, evaluated for after the excitation,
%  
%  \iframe{\begin{equation}\label{eqn:onsager_1}
%  \chi(\omega) = -\frac{1}{kT}\int_{0}^{\infty}\iaverage{M(0)\dot{M}(t)} %\xi_0+\frac{i\omega}{kT}\int_{0}^{\infty}\iaverage{M(0){M}(t)}e^{i\omega t}dt
%  \end{equation}
%  \red{So if we know the autocorrelation function of a systems response, we can evaluate the dynamic susceptibility,}
%  }

 \subsection{Charge and current}
  Consider the step to be voltage, $ B \lra V $, and the response a charge, $ M \lra Q $. \textbf{\red{We can monitor the charge Q}}. Like Eq.\eqref{eqn:freq_2},\eqref{eqn:freq_3}:
  
  \begin{equation}\label{key}
  \begin{aligned}
  	Q(t) & = \int X(t-\tau)V(t)d\tau\\
  	q(\omega) & = \chi(\omega)v(\omega),
  \end{aligned}
  \end{equation}
  
  \noindent \red{$ \chi(\omega) $ being the frequency dependent capacitance}. As we saw in Eq~\eqref{eqn:energy_2}, its imaginary part will lead to dissapation, $ P =\iaverage{B\difffrac{M}{t}} = \iaverage{V\difffrac{Q}{t} }= \frac{1}{2}b^2\omega\red{\chi''(\omega)}. $
  
  Above in Eq.~\eqref{eqn:susceptibility_from_response_0} we have seen how this capacitance susceptibility, $ \chi(\omega) $, can be calculated from the response, $ Q(t) $

  \begin{equation}\label{key}
  	\begin{aligned}
	  	\chi(\omega) & = -\frac{1}{kT}\int_{0}^\infty\iaverage{Q(0)\dot{Q}(t)}e^{i\omega t}dt = -\frac{1}{kT}\int_{0}^\infty\iaverage{Q(0)\green{I(t)}}\green{e^{i\omega t}}dt \\
	  	& \text{not trivial, but you don't need to do it}\\
	  	& = \frac{i\omega}{kT}\int_{0}^{\infty}\iaverage{I(0)I(t)}e^{i\omega t}dt
%	  	& = \left.\iaverage{Q(0)\green{Q(t)}}\green{e^{i\omega t}\frac{1}{i\omega}}\right|_0^\infty + \frac{1}{kT\green{i\omega}}\int_{0}^{\infty} \\
%	  	& = \left.\iaverage{M(0)M(t)}e^{i\omega t} \right|_0^\infty - - \frac{1}{kT}i\omega\int_{0}^\infty\iaverage{M(0)M(t)}e^{i\omega t}dt\\
  	\end{aligned}
%  	\frac{1}{Z(\omega)} = i\omega c(\omega) = i\omega \chi(\omega)
  	% 	 \frac{-i\omega}{kT}\int_{0}^{\infty}\iaverage{Q(0)\dot{Q}(t)}e^{i\omega t}dt.
  \end{equation}
    
  \iframe{\noindent Or in terms of the Impedance $ Z = \frac{1}{i\omega \chi(\omega)} $ (equation for capactitive impedance)
  
  \begin{equation}\label{key}
  	\frac{1}{Z(\omega)} = \frac{1}{kT}\int_{0}^{\infty}\iaverage{I(0)I(t)}e^{i\omega t}dt.
  \end{equation}
	The impedance of an object is related to the area under the current autocorrelation graph.
	
	And this is different from the $ \frac{1}{R}=\frac{1}{2kT}\int\iaverage{I(0)I(t)}dt $ expression form before, because now we are getting the impedance at a particular frequency, not just the general one. (In fact its just the above intergrated for all the frequencies).
	}  

\newpage