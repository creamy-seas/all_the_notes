\section{Rabi oscillations}
 Rabi oscillations - occur between two levels when subjected to their resonant transition frequency $ \omega_r $, and occur with a frequency of $ \Omega $ between the two levels i.e. \textbf{The Rabi frequency = Amplitude of the driving field}. In the original frame the state evolves between $ \iket{0} $ and \iket{1} in a cyclic manner. In the rotated frame, two eigenstates of the Hamiltonian are $ \big(\iket{0}\pm\iket{1}\big)/\sqrt{2} $ and the state also evolves coherently between them.
 
 This drive will also lead to the splitting of the energy levels, with a characteristic width $ \Omega $ as well. So the amplitude dictates the oscillation frequency and the splitting of the two level system.
 
 So one applies a pulse for a time $ t $ at a given power $ \Omega $, so that Rabi oscillations are driven for a time $ t $. Next one removes the drive, and measures the resulting state of the qubit (by measuring emission from the atom). The position of the final state on the Bloch sphere, determines the output signal. 
 
 Now, the data capturing device has a minimum bandwidth of 1Hz i.e. averages the signal over 1 second. We repeat the above experiment (same power and pulse length) 1 million times, and output the cumulative signal acquired. It will be proportional to the analytic probability of the qubit being in its final state.
 
 Then we elongate the length of the pulse, so that the state now finishes on a different part of the Bloch sphere. And repeat. We thus get oscillations, with Rabi period (from analytical calculations).
 
 Because of dephasing, longer pulses will mean that one acquires more and more random phases. When the pulses were short, the phases didn't have time to evolve freely and deviate from the exact value. For longer pulses, the separate measurements differ more and more in phase i.e. the integral of the different phases will get closer and closer to zero. This is the decay of the collective signal.
 
\newpage