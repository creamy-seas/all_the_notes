% -*- TeX-master: "../all_the_notes.tex" -*-

\section{Linbalnd operators covered in depth\label{sec:linbland2}}
This  section  will  cover  how  we  derive  these $  L_j  $  operators  which  were  used  in
Sec.~\ref{subsec:dynamics_with_pauli}.
\subsection{Pure dephasing\label{sec:lin_1}}
Let us consider a two level system, whose  Hamiltonian is undergoing a fluctuation in time due
to \red{movement of the energy levels}
\begin{equation}\label{eq:lin_2}
  \mathcal{H} = \imatrix{\red{\epsilon(t)}}{0}{0}{0} \iratext{eigenstates} \ialigned{\iket{0} & = \imatrixcol{0}{1}\\\iket{1} & = \imatrixcol{1}{0}}.
\end{equation}
  
\noindent Under  free evolution an  arbitrary state \iket{\Psi(0)}  = a\iket{1} +  b\iket{0} will
evolve to:
  
\begin{equation}\label{eq:lin_3}
  \begin{aligned}
    \iket{\Psi(t)} & = \prod_i U(\Delta t_i) \iket{\Psi(0)} \\
    & = \prod_i \exp\left[-i\frac{\epsilon(\Delta t_i)}{\hbar} \imatrix{1}{0}{0}{0}\right]\imatrixcol{a}{b}\\
    & = \prod_i \bigg(\mathbb{I}\imatrixcol{a}{b} + \blue{\sum_{{N = 1}}\frac{\left[\frac{-i\epsilon(\Delta t_i)}{\hbar}\right]^N}{N!} \imatrix{1}{0}{0}{0}\imatrixcol{a}{b}}\bigg)\\
    & = \prod_i \bigg(\imatrixcol{a}{b} + \blue{\exp\left[-i\frac{\epsilon(\Delta t_i)}{\hbar}\right]\imatrixcol{a}{0} - \imatrixcol{a}{0}\bigg)}\\
    & = \prod_i \imatrixcol{e^{-i\frac{\epsilon(\Delta t_i)}{\hbar}}a}{b} = \imatrixcol{e^{-i\sum_i\frac{\epsilon(\Delta t_i)}{\hbar}}a}{b} \\
    & \red{= \imatrixcol{ae^{i\varphi}}{b} \qquad \varphi = -\int_{0}^{t}\frac{\epsilon(t')}{\hbar}dt'}
  \end{aligned}
\end{equation}
 
\iframe{\noindent  Thus, free  evolution  under a  fluctuating  energy will  lead  to a  phase
  accumulation between the two states:
  \begin{equation}\label{eq:lin_4}
    \ket{\Psi} = a\red{e^{i\varphi}}\iket{1} + b\iket{0} =  \imatrixcol{a\red{e^{i\varphi}}}{b}
  \end{equation}
 		
  \noindent which we  can probe by measuring  the \isigmax\ component and  see this integrated
  phase:
 		
 		\begin{equation}\label{eqn:lin_1}
                  \isigmax = ab^{*}\red{e^{i\varphi}} + a^{*}b\red{e^{-i\varphi}} \iratext{a,b real} 2ab\cos(\varphi).
 		\end{equation}
              }
 
              \newpage
              \paragraph{Single run}  Each measurement  of $  \sigma_x $ gives  $ \pm1  $ with
              probabilities (projecting onto the respective eigensubspaces):
  
              \begin{equation}\label{eq:linrelax_8}
                \begin{aligned}
                  &P(1) = \iabsSquared{\bra{\Psi}\left(\frac{\iket{0} + \iket{1}}{\sqrt{2}}\right)} = \frac{1}{{2}}\iabsSquared{ae^{-i\varphi} + b} = \frac{1}{2}\left({a}^2 + {b}^2 + 2ab\cos(\varphi) \right)\\
                  & P(-1) = \frac{1}{2}\left(a^2 + b^2 - 2ab\cos(\varphi)\right)
                \end{aligned}
              \end{equation}
    
              \ipicCaption{3cm}{lin_1}{The  energy  of the  two  levels  fluctuates around  0.
                Integration       will       give       a      unique       phase       value,
                $ \phi  = -\int_0^t\frac{\epsilon(t')}{\hbar}dt'  $, that  determines collapse
                probability. \textbf{This is for a single run.}}
  
              \paragraph{Averaging} Averaging over the $ \pm 1  $ will allow us to recover the
              \isigmax $ \propto \cos(\varphi) $ dependence.
              \begin{itemize}
              \item  Each run  has an  independent $  \epsilon(t) $  and hence  an independent
                $ e^{i\varphi} $;
              \item    Averaging    will    thus    result   in    the    classical    average
                \iaverage{e^{i\varphi}}$ _\varphi $.
              \end{itemize}

              \iframe{\noindent And thus Eq.~\eqref{eqn:lin_1} will in practise be reading:
  
  \begin{equation}\label{eq:linrelax_10}
    \isigmax = ab^{*}\blue{\iaverage{e^{i\varphi}}_\varphi} + a^{*}b\iaverage{e^{-i\varphi}}_\varphi \iratext{a,b real} 2ab\cos(\iaverage{\varphi}_\varphi).
  \end{equation}}
  
\noindent Let us label this interference term factor
  
  \begin{equation}\label{eq:linrelax_11}
    v(t) = \blue{\iaverage{e^{i\varphi}}_\varphi},
  \end{equation}
  
  \noindent   which    is   reflective    of   the    noise   supression    of   interference:
  $ \blue{v(t)} \ra 0 $.
  
  \ipicCaption{3cm}{lin_2}{The longer we  allow the phase, $ \varphi $,  to evolve the greater
    amount  of cancellation  between the  independent  runs we  shall  get.  In  this way  and
    interference that we can infer will wash out.\label{fig:lin_2}}
  
  \noindent  \red{\paragraph{Why does  it decrease  with  time?}  Making  the assumption  that
    $ \epsilon(t) $ fluctuates around 0 we will find that %$ \varphi $ can be expanded:
  	
    % \begin{equation}\label{key}
    %             \iaverage{\varphi(t)}_\varphi           =
    %             \iaverage{\varphi(0)}_\varphi           +
    %                 %             t\iaverage{\difffrac{\varphi}{t}|_{t=0}}_\varphi
    %                 %             +
    %                 %             \frac{t^2}{2}\iaverage{\difffrac{^2\varphi}{t^2}|_{t=0}}_\varphi
    %                 %             + \cdots
    % \end{equation}
%  	
    % \noindent and because
    \begin{itemize}
    \item Initially any $ \varphi(0)  = -\int_{0}^{0}\frac{{\epsilon(t')}}{\hbar}dt' = 0 $ (no
      phase has been allowed to accumulate) and therefore
      \begin{equation}\label{key}
        \iaverage{e^{i\varphi(0)}}_\varphi = 1.
      \end{equation}
    \item Subsequently, the random evolution of the phase will lead to
  		
      \begin{equation}\label{key}
        \begin{aligned}
          \iaverage{e^{i\varphi(t)}}_\varphi & = e^{i\phi_1(t)} + e^{i\phi_2(t)} + e^{i\phi_3(t)} + \cdots\\
          & = x_1 + iy_1 + x_2 + iy_2 + x_3 + iy_3 + \cdots\\
          & = \iaverage{x} + i\iaverage{y}\\
          & = 0 + i0 = 0
        \end{aligned}
      \end{equation}
  		
      \noindent and thus perfect cancellation - any agreement between the states will persists
      only for short times.
    \end{itemize}}
 
 \subsection{Pure dephasing from partial trace\label{sec:lin_2}}
 We can  arrive at  Eq.~\eqref{eqn:lin_1} in  a similar fashion  by considering  a system-bath
 state
   
   \begin{equation}\label{key}
     \ket{\Psi_{SB}} = \ket{\Psi} \otimes \ket{\chi_B} \quad\text{where}\quad 
     \ialigned{
       \Ket{\Psi} & = a\iket{1} + b\iket{0}\\
       \ket{\chi} & = \sum_i\sqrt{w_i}\ket{\chi_j}
     } 
   \end{equation}
  
   \noindent where the bath is a weighted superposition of \iket{\chi_i} states (normalised of
   course).
  
   \iframe{\noindent We postulate, that  the state of the system will  evolve acoording to the
     state of the bath:
     \begin{equation}\label{key}
       \begin{aligned}
         U_{SB}(t)\iket{1}\ket{\chi_i} & = \red{e^{i\varphi_i(t)}}\iket{1}\ket{\chi_i}\\
         U_{SB}(t)\iket{0}\ket{\chi_i} &=\iket{0}\ket{\chi_i}\\
       \end{aligned}
     \end{equation}
     \noindent acquiring a  phase only for the  \iket{1} state and depending on  the states of
     the bath.  }

   \noindent Thus we shall get the following evolution:
   \begin{equation}\label{key}
     \begin{aligned}
       \ket{\Psi_{SB}(t)} & = aU_{SB}(t)\left[\sum_i\sqrt{w_i}\ket{1}\ket{\chi_i}\right] + bU_{SB}(t)\left[\sum_i\sqrt{w_i}\ket{0}\ket{\chi_i}\right]\\
       & = a\iket{1}\red{\sum_i\sqrt{w_i}e^{i\varphi_i(t)}\ket{\chi_i}} + b\iket{0}\blue{\sum_i\sqrt{w_i}\ket{\chi_i}}\\
       & = a\iket{1}\red{\ket{\chi_1}} + b\iket{0}\blue{\ket{\chi_0}}
     \end{aligned}
   \end{equation}
   \vspace{1cm}
  
   \noindent Taking  the experctation value of  the operator $ \mathbb{I}  \otimes \sigma_x $,
   where we only operate on the system, we get
  
   \begin{equation}\label{key}
     \begin{aligned}
       \iaverage{\sigma_x}_S & = \bra{\Psi_{SB}(t)}\mathbb{I} \otimes \sigma_x\ket{\Psi_{SB}(t)} \\
       & = ab^{*}\blue{\bra{\chi_0}}\red{\ket{\chi_1}} + a^{*}b\red{\bra{\chi_1}}\blue{\ket{\chi_0}}\\
       & = \sum_iw_iab^{*}e^{i\varphi_i(t)} + cc.\\
       & = \red{ab^{*}\iaverage{e^{i\varphi(t)}}_\varphi + cc.}
     \end{aligned}
   \end{equation}
  
   \noindent                                                                             where
   \red{$       \iaverage{e^{i\varphi(t)}}_\varphi      =       \sum_iw_ie^{i\varphi_i}      =
     \blue{\bra{\chi_0}}\red{\ket{\chi_1}}$} and we  recover the result Eq.~\eqref{eqn:lin_1}.
   See how the environment picks up the state of the system
  
 \subsection{General roundup}
 In Sec.~\ref{sec:lin_1} and \ref{sec:lin_2} we have  come to two equivalent ways of recovring
 the state of the system:
  
  \begin{equation}\label{key}
    \ket{\Psi(t)}_\varphi = a\iket{1}\red{v(t)} + b\iket{0}.
  \end{equation}
  
  \noindent where the  phase supression term $ \red{v(t)  = \iaverage{e^{i\varphi}}_\varphi} $
  is a consqeuence of:
  
  \begin{itemize}
  \item  Classical noise  process as  we average  over different  runs that  acquire different
    phases
    \begin{equation}\label{key}
      \ket{\Psi(t)}_\varphi = \red{\sum_i} a\iket{1}\red{e^{i\varphi_i(t)}} + b\iket{0}.
    \end{equation}
  \item Quantum  mechanical Interaction  with the  weighted superposition  of the  bath states
    \iket{\chi_i},
    \begin{equation}\label{key}
      \begin{aligned}
        \ket{\Psi}_\phi & = 	a\iket{1}{\sum_i\sqrt{w_i}e^{i\varphi_i(t)}\ket{\chi_i}} + b\iket{0}{\sum_i\sqrt{w_i}\ket{\chi_i}}\\
        & = \left[a\iket{1}\red{\sum_ie^{i\varphi_i(t)}} + b\iket{0}\right]\ket{\chi}
      \end{aligned}
    \end{equation}
  \end{itemize}
  
  \iframe{\noindent Let us represent the state as a density matrix
  
  \begin{equation}\label{key}
    \rho(t) = \imatrix{\iabsSquared{a}}{ab^{*}\red{v(t)}}{a^{*}b\red{v^{*}(t)}}{\iabsSquared{b}},
  \end{equation}
   
  Where we see that dephasing affects only the off-diagonal elements. Taking the purity
   
  \begin{equation}\label{key}
    \begin{aligned}
      \itrace{\rho^2} & = \iabs{a}^4 + 2\iabsSquared{a}\iabsSquared{b}\iabsSquared{v(t)} + \iabs{b}^4\\
      & = (\iabsSquared{a}+\iabsSquared{b})^2 - 2\iabsSquared{a}\iabsSquared{b} + 2\iabsSquared{a}\iabsSquared{b}\iabsSquared{v(t)}\\
      & = 1 - 2\iabsSquared{a}\iabsSquared{b}\left(1-\iabsSquared{v(t)}\right),
    \end{aligned}
  \end{equation}
   
  \noindent and  as $ v(t) \ra  0 $ the  purity goes $ 1  \ra 1/2 $  if we start from  an even
  superposition.  }
  
\noindent Let us assume that the phase suppresion  term falls as an exponential (which we kind
of show in Fig.~\ref{fig:lin_2})
  
  \begin{equation}\label{key}
    v(t) = \iaverage{e^{i\varphi(t)}}_\varphi = e^{-\Gamma_{\varphi}t}
  \end{equation}
  
  \noindent in which case we can write
  
  \begin{equation}\label{eqn:lin_3}
    \dot{\rho}_{10}(t) = -\Gamma_{\varphi}\rho_{10}(t).
  \end{equation}
  
  \noindent \textbf{We want to incorporate this into the evolution of the form}
  \begin{equation}\label{eq:linrelax_5}
    \mathbf{\mathcal{L}[\rho(0)] = \rho(t)}.
  \end{equation}
  \subsection{Relaxation in the Linbland Equation}
  However,  the  Linbalnd equation,  incorporating  Eq.~\eqref{eqn:lin_3},  cannot be  written
  arbitraraly, because the density matrix is a weighted \textbf{probability} superposition
  
  \begin{equation}\label{eq:linrelax_1}
    \rho = \sum_i\ketbra{\psi_i}{\psi_i},
  \end{equation}
  
  \noindent and  therefore cannot  have negative  eigenvalues.  If we,  for example,  take the
  decay of a state with initial off diagonal elements:
  
  \begin{equation}\label{eq:linrelax_2}
    \imatrix{\rho_{00}(0)}{\rho_{01}(0)}{\rho_{10}(0)}{\rho_{11}(0)} \ira \imatrix{0}{\rho_{01}(t)}{\rho_{10}(t)}{1} \red{\ra \text{negative eigenvalues!}}
  \end{equation}
  
  \iframe{An arbitrary operator $ \Phi $ has to be a completely positive:
    \begin{equation}\label{eq:linrelax_3}
      \begin{aligned}
        \Phi(\rho) & \ge 0\\
        \Phi : & B(\mathcal{\tilde{H}}) \ira B()\mathcal{\tilde{H}})\\
        & \text{where } \mathcal{\tilde{H}} = \mathcal{{H}} \otimes \mathcal{H'}\\
        & \text{and } \Phi \text{ acts on one product space and leaves the other untouched}
      \end{aligned}
    \end{equation}
  }

  \noindent This is fullfilled by the Kraus operator:
  
  \begin{equation}\label{eq:linrelax_4}
    \Phi(\rho) = \sum_iK_i\rho K_i\idagger\qquad \sum_iK_i\idagger K_i = \mathbb{I}.
  \end{equation}
  
  \noindent  A great  example is  unitary evolution  $ K  = U(t)  $, random  unitary evolution
  $ K_i = \sqrt{w_i}U_i(t) $ or the evolution of a system batch state:
  
  \begin{equation}\label{eq:linrelax_5}
    \begin{aligned}
      \rho(t) & = \itrace{U_{SB} \rho_S\otimes\ketbra{\chi_B(0)}{\chi_B(0)}U_{SB}\idagger}\\
      & = \sum_j\left[\bra{\chi_j}U_{SB}\iket{\chi_B(0)}\right]\rho_{S}\bra{\chi_{B}(0)U_{SB\idagger}\iket{\chi_j}}\\
      & = \sum_j K_j\rho_{S}K_j\idagger,
    \end{aligned}
  \end{equation}
 
  \noindent  where the  Kraus operators  describe  the traced  out  effect of  the Bath  state
  evolution.  The full Hamiltonian reads:
 
 \begin{equation}\label{eq:linrelax_6}
   \dot{\rho}(t) = -\frac{i}{\hbar}\icommutation{\mathcal{H}}{\rho} + \sum_j\left(R_j\rho R_j\idagger - \frac{1}{2}\rho R_j\idagger R_j - \frac{1}{2}\rho R_j\idagger R_j\right)
 \end{equation}

 
 \iframe{Usually, relaxation  operators are  of a  simple form,  describing the  operator that
   induces the  dissipative transition,  multiplied by  the square  root of  the corresponding
   rate.  \red{The  decay rates must  be much smaller than  the transition frequencies  of the
     system.}
   \begin{itemize}
   \item \textbf{Pure dephasing} due  to energy level fluctuations \red{as we  set out to show
       in Eq.~\eqref{eqn:lin_3}} is associated with the $ \sigma_z $ term
     \begin{equation}\label{eq:linrelax_7}
       R = \sqrt{\frac{\Gamma_{\varphi}}{2}}\sigma_z.
     \end{equation}
 		
     \noindent which leads to relaxation terms
 		
     \begin{equation}\label{eq:linrelax_8}
       \begin{aligned}
         &R\rho R\idagger - \frac{1}{2}R\idagger R - \frac{1}{2}\rho R\idagger R\\
         & = \Gamma_{\varphi}\imatrix{0}{-\rho_{01}}{-\rho_{10}}{0}.
       \end{aligned}
     \end{equation}
 		
     \noindent It recovers the decaying off diagonal dynamics, and that is good enough for me.
 		
   \item  \textbf{Exponential relaxation}  from excited  to  ground state,  is ossicated  with
     $ \sigma_{-} = \ketbra{0}{1}$
 		
 		\begin{equation}\label{key}
                  R = \sqrt{\Gamma}\sigma_{-}
 		\end{equation}
 		
 		\noindent giving
 		
 		\begin{equation}\label{eqn:lin_5}
                  \Gamma\imatrix{+\rho_{11}}{\red{-\frac{1}{2}\rho_{01}}}{\red{-\frac{1}{2}\rho_{10}}}{-\rho_{11}},
 		\end{equation}
 		
 		\noindent which results in non-trivial  \red{off dioagonal elements}, which is
                exactly 1/2 of the decay rate.
              \end{itemize}
 
              Combining   the  effect   of   diagonal   terms  in   Eq.~\eqref{eq:linrelax_8},
              \eqref{eqn:lin_5} we define
  
  \begin{equation}\label{key}
    \Gamma_2 = \Gamma_{\varphi} + \frac{\Gamma_1}{2}
  \end{equation}
}
\newpage

%%% Local Variables:
%%% mode: latex
%%% TeX-master: "../all_the_notes"
%%% End:
