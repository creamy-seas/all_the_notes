% -*- TeX-master: "all_the_notes.tex" -*-

\newpage
\section{Flux Qubit 4JJ-linear}
\label{sec:flux-qubit-4jj}

% Now  lets  repeat the  same  procedure,  but  with  4 JJs:  3JJ  have
% E$_{J}, C$, while the odd one out has $\alpha E_J, \alpha C$.

% \begin{figure}[h]
%   \centering \includegraphics[height=5cm]{flux4jj_linear}
%   \caption{\small One of the junctions is scaled by a factor $\alpha$
%   relative to the others \label{fig:flux4jj_linear}}
% \end{figure}


% \begin{enumerate}
% \item Capacitance system reads:
%   \begin{equation}
%     \label{eq:4jjlinear_1}
%     \vec{n} = \frac{C\vec{V}}{2e} \equiv \begin{bmatrix}
%       C\left(V_1-0\right) + C\left(V_1-V_2\right) \\
%       C\left(V_2-V_1\right) + C\left(V_2-V_3\right) \\
%       C\left(V_3-V_2\right) + \alpha C\left(V_3-0\right) \\
%     \end{bmatrix} = \iabs{C} \begin{pmatrix}
%       2 & -1 & 0\\
%       -1 & 2 & -1\\
%       0 & -1 & 1 + \alpha
%     \end{pmatrix}\begin{pmatrix}
%       V_{1}\\V_2\\V_{3}.
%     \end{pmatrix}
%   \end{equation}

%   \noindent
% \item The Total Hamiltonian, just like in Ch.~(\ref{eqn:l23JJ}), is
%   \begin{equation}
%     \label{eq:4jjlinear_2}
%     \begin{aligned}
%       \mathcal{H}      &       =      \red{U_{\text{kinetic}}}      +
%       \blue{U_{\text{potential}}}\\ & =
%       \red{\frac{(2e)^2}{2}\vec{n}^{\text{T}}C^{-1}\vec{n}} \\
%       & \qquad +  \blue{\frac{E_J}{2}\bigg[3+\alpha - \cos(\varphi_{10}) -
%       \cos(\varphi_{21})        -         \cos(\varphi_{32})        -
%       \alpha\cos(\varphi_\text{ext}-\varphi_{10}  -   \varphi_{20}  -
%       \varphi_{30})\bigg]}.
%     \end{aligned}
%   \end{equation}

%   \noindent which can be solved in  the number, $ \hat{n} $, of flux,
%   $ \vec{\varphi} $ basis.
% \end{enumerate}

% \subsection{Number basis}
% \label{sec:number-basis}

% This was what was done previously. Term by term:

% \begin{equation}
%   \label{eq:4jjlinear_3}
%   \begin{aligned}
%     &\red{\frac{(2e)^2}{2}\vec{n}^{\text{T}}C^{-1}\vec{n}}\qquad\qquad E_C \iabs{C} \sum_{n_{1},n_{2},n_3}\iketbra{n_{1},n_2,n_3}{n_{1},n_2,n_3}\vec{n}^TC^{-1}\vec{n}\\
%     & \blue{\cos(\varphi_{10}):}\qquad\qquad -\frac{E_J}{2}\bigg[\iketbra{n_1-1}{n_1}+\iketbra{n_1+1}{n_1}\bigg]\otimes \mathbb{I}_{2,}\\
%     & \blue{\cos(\varphi_{21}):}\qquad\qquad - \frac{E_J}{2}\bigg[\iketbra{n_1+1}{n_1}\iketbra{n_2-1}{n_2}+\iketbra{n_1-1}{n_1}\iketbra{n_2+1}{n_2}\bigg]\otimes\mathbb{I}_3\\
%     & \blue{\alpha\cos(\varphi_\text{ext}-\varphi_{1} - \varphi_{2} - \varphi_{3}):}\qquad\qquad -\frac{E_J}{2}\left(e^{i\varphi_{ext}}e^{-i\varphi_{1}}e^{-i\varphi_2}e^{-i\varphi_{3}} + cc.\right)\\
%     &       \qquad       =       -\frac{E_{J}}{2}       e^{i\varphi_{ext}}
%     \left(\sum_{n_{1},n_{2},n_{3}}
%       \iketbra{n_{1}-1,n_{2}-1,n_{3}-1}{n_1,n_2,n_{3}}\right) + cc.
%   \end{aligned}
% \end{equation}

% \noindent Plug this into a matrix and solve. Simple as that.


% \subsection{Voltage Dipole Transition}
% \label{sec:volt-dipole-trans}

% Transition of the system between the ground and first excited states,
% will change the voltage on island 3. The characteristic change is

% \begin{equation}
%   \label{eq:4jjlinear_3}
%   \ibra{e}\hat{V}_3 \iket{g},
% \end{equation}

% \noindent where $V_{3}$ is found from Eq.~\ref{eq:4jjlinear_1}:

% \begin{equation}
%   \label{eq:4jjlinear_4}
%   \begin{aligned}
%     \vec{V}          &          =          2eC^{-1}\vec{n}          =
%     \frac{2e}{\iabs{C}(1+3\alpha)}\begin{pmatrix}
%       1 + 2\alpha & 1 + \alpha & 1 \\
%       1 + \alpha & 2 + 2\alpha & 2\\
%       1 & 2 & 3
%     \end{pmatrix}\\
%     &\\
%     V_3             &             =             \frac{E_{C}}{\iabs{e}
%     (1+3\alpha)}\left[\hat{n}_1+2\hat{n}_{2}+3\hat{n}_{3}\right].
%   \end{aligned}

% \end{equation}

% \noindent
